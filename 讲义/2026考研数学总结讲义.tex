\makeatletter
   \def\input@path{{..}} % 搜索上层目录的 LALUbook
\makeatother

\documentclass[
    % colors = false,
    geometry = a4,
]{LALUbook}

\usepackage{mathdots}
\usepackage{booktabs} % Excel 导出的大表格
\usepackage{rotating}
\usepackage{extarrows}
\usepackage{blkarray}
\usepackage{cases}

\usepackage{float}
\usepackage{diagbox}
\usepackage{caption}

\usepackage{pgfplots}
\usetikzlibrary{cd, arrows, arrows.meta, calc, intersections, decorations.pathreplacing, patterns, decorations.markings,angles,quotes, graphs, positioning, shapes.geometric}
\pgfplotsset{compat=newest}

\usepackage[xindy, splitindex]{imakeidx}
\makeindex[
    columns=1,
    program=truexindy,
    intoc=true,
    options=-M texindy -I xelatex -C utf8,
    title={名词索引}
] % 名词索引
\makeindex[
    columns=3,
    program=truexindy,
    intoc=true,
    options=-M numeric-sort -M latex -M latex-loc-fmts -M makeindex -I xelatex -C utf8,
    name=sym,
    title={符号索引}
] % 符号索引

% 标题格式
% \chapter   正常专题
% \LUchapter 强化专题

\newcounter{LUchapter}
\newcounter{LUgreekchap}

\makeatletter
% 此处可按需增改
% \texorpdfstring 的两个参数分别显示在正文中与 PDF 书签中
\newcommand*{\@LUgreek}[1]{%
    \ifcase#1\or\texorpdfstring{$\boldsymbol{\varepsilon}$}{ε}%
    \or\texorpdfstring{$\boldsymbol{\delta}$}{δ}%
    \or\texorpdfstring{$\boldsymbol{\lambda}$}{λ}%
    \or\texorpdfstring{$\boldsymbol{\mu}$}{μ}%
    \or\texorpdfstring{$\boldsymbol{\varphi}$}{φ}%
    \or\texorpdfstring{$\boldsymbol{\theta}$}{θ}%
    \else\@ctrerr\fi%
}
\newcommand*{\LUgreek}[1]{%
    \expandafter\@LUgreek\csname c@#1\endcsname
}
\newcommand*{\LUchapsancheck}{%
\expandafter\@ifundefined{@exist@LUchapter@\arabic{chapter}.\arabic{LUgreekchap}}%
    {\setcounter{LUgreekchap}{1}}
    {\stepcounter{LUgreekchap}}
}
\newcommand*{\LUgroupsancheck}{%
\expandafter\@ifundefined{@exist@LUchapter@\arabic{chapter}}%
    {}
    {\endgroup}
}
\let\@std@chapter\chapter
\renewcommand*{\chapter}{%
    \LUgroupsancheck%
    \@std@chapter
}
\let\@std@chaptermark\chaptermark
\def\chaptermark#1{\def\@LALU@chaptername{#1}\@std@chaptermark{#1}}
\makeatother

% OPD 系列
\newcommand{\OOne}{\textcolor{blue}{\textbf{$O$ (盯住目标)}}}
\newcommand{\DOne}{\textcolor{blue}{\textbf{$D_1$ (常规操作)}}}
\newcommand{\DTwo}{\textcolor{blue}{\textbf{$D_2$ (脱胎换骨)}}}
\newcommand{\DThree}{\textcolor{blue}{\textbf{$D_3$ (移花接木)}}}
\newcommand{\DFour}{\textcolor{blue}{\textbf{$D_4$ (可圈可点)}}}
\newcommand{\DTwoOne}{\textcolor{blue}{\textbf{$D_{21}$ (观察研究对象)}}}
\newcommand{\DTwoTwo}{\textcolor{blue}{\textbf{$D_{22}$ (转换等价表述)}}}
\newcommand{\DTwoThree}{\textcolor{blue}{\textbf{$D_{23}$ (化归经典形式)}}}
\newcommand{\POne}{\textcolor{blue}{\textbf{$P_{1}$ (常规思路)}}}
\newcommand{\POneOne}{\textcolor{blue}{\textbf{$P_{11}$ (正向思路)}}}
\newcommand{\POneTwo}{\textcolor{blue}{\textbf{$P_{12}$ (反向思路)}}}
\newcommand{\POneThree}{\textcolor{blue}{\textbf{$P_{13}$ (双向思路)}}}
\newcommand{\PTwo}{\textcolor{blue}{\textbf{$P_{2}$ (反证思路)}}}
\newcommand{\PThree}{\textcolor{blue}{\textbf{$P_{3}$ (数学归纳)}}}
\newcommand{\PFour}{\textcolor{blue}{\textbf{$P_{4}$ (逆否思路)}}}
\newfontfamily\RomanSymbols{Arial Unicode MS}

\newtcbtheorem[number within=section]{detail}{细节}{laluthmstyle={red}}{det}
\newtcbtheorem[number within=section]{idea}{思路}{laluthmstyle={teal}}{ide}

% 内容总结
\newenvironment{summary}{%
    \hypersetup{bookmarksnumbered=false}%
    \titleformat{\subsection}[block]{\centering\heiti\Large}{}{1em}{}%
    \phantomsection%
    \subsection*{内容总结}%
}{}

\NewDocumentCommand{\LUchapter}{m}{%
\LUgroupsancheck
\begingroup
\LUchapsancheck
\addtocounter{chapter}{-1}
\refstepcounter{LUchapter}
\renewcommand*{\thechapter}{\arabic{chapter}\LUgreek{LUgreekchap}}
% \renewcommand*{\thechapter}{\arabic{chapter}}
\renewcommand*{\theHchapter}{LU.\arabic{LUchapter}}
\ctexset{
    chapter={format={\centering\Huge\bfseries},name={强化专题,},number={\zhnumber{\arabic{LUchapter}}}},
}
\csname @std@chapter\endcsname{#1}
\expandafter\xdef\csname @exist@LUchapter@\arabic{chapter}\endcsname{\null}
\expandafter\xdef\csname @exist@LUchapter@\arabic{chapter}.\arabic{LUgreekchap}\endcsname{\null}
}

\ctexset{
    chapter={format={\centering\Huge\bfseries},name={强化专题,},number={\zhnumber{\arabic{chapter}}}},
    % chapter={format={\centering\Huge\bfseries},name={第,讲},number=\arabic{chapter}},
    section={format={\raggedright\Large\bfseries},name={,},number={\thechapter.\arabic{section}}},
    subsection={format={\raggedright\large\bfseries},name={,},number={\thesection.\arabic{subsection}}},
    subsubsection={format={\raggedright\normalsize\bfseries},name={,},number={\thesubsection.\arabic{subsubsection}}},
}

\title{\heiti 临沂大学 2025--2026 学年 \\ 2026考研数学总结讲义}

\AtEndPreamble{\hypersetup{
    hypertexnames=true,
    pdfauthor={林子立},
    pdftitle={2026考研数学总结讲义},
}}

\begin{document}

\title{2026考研数学总结讲义}
\author{林子立}
\date{\today}
\maketitle

\songti

\pagenumbering{Roman}
\clearpage

\pdfbookmark[0]{目录}{contents}
\tableofcontents

\addtolength{\parskip}{.5em}

\mainmatter
\chapter{行列式}
\XIANchapter{余子式和代数余子式的计算}

\section{计算余子式、代数余子式的线性组合}
\chapter{矩阵运算}

\chapter{矩阵的秩}
\chapter{线性方程组}
\section{线性方程组理论总结}
\DOne
\begin{enumerate}
    \item 齐次线性方程组$Ax=0$ \DOne
    \item 非齐次线性方程组$Ax=b$ \DOne
\end{enumerate}


\section{线性方程组问题}
\begin{enumerate}
    \item 一般求解问题
    \item 公共解问题
    \item 同解问题
          \DOne+\DTwoTwo
          \begin{detail}{齐次线性方程组\DTwoTwo}{}
          \end{detail}

          \begin{detail}{非齐次线性方程组\DTwoTwo}{}
              设(\RomanSymbols Ⅰ)$A_{m\times n}x=\beta$与(Ⅱ)$B_{s\times n}x=\gamma$均有解,则

              ①(\RomanSymbols Ⅰ)与(Ⅱ)同解

              $\Leftrightarrow$②$A_{m\times n}x=0$与$B_{s\times n}x=0$同解且(\RomanSymbols Ⅰ)与(Ⅱ)有公共解

              $\Leftrightarrow$③$r\left(\begin{bmatrix}A & \beta \\ B & \gamma\end{bmatrix}\right)=r\left(\begin{bmatrix}A \\ B\end{bmatrix}\right)=r(A)=r(B)$

              $\Leftrightarrow$④$[A,\beta]$与$[B,\gamma]$的行向量组等价.
          \end{detail}
\end{enumerate}


\section{线性方程组的几何意义}



\chapter{向量组}

\section{研究具体型向量关系}

\subsection{定义法}
\subsection{求极大线性无关组}
\section{研究抽象型向量关系}
\subsection{定义法}
\subsection{综合问题}
\DOne+\DTwoThree

$$x = \eta^{*} + k_{1} \xi_{1} + \ldots + k_{n-r} \xi_{n-r}.$$
$$
A \xi_{i} = 0.
$$
$$A \eta^{*} = \beta.$$
\section{研究向量组等价}
\section{向量空间}
\subsection{概念}
\subsection{过渡矩阵}
\subsection{坐标变换}
\chapter{特征向量与特征值}
\term{求解利用} $A$的特征值与特征向量

\section{利用特征值命题}
\DOne +\DTwoTwo
\begin{enumerate}
    \item $\lambda_0$是$A$ 的特征值$\Leftrightarrow|\lambda_0E-A|=0$(建立方程求参数或证明行列式 $|\lambda_0E-A|=0$ );
          $\lambda_0$不是$A$ 的特征值$\Leftrightarrow|\lambda_0E-A|\neq0$(矩阵可逆,满秩).
    \item 若$\lambda_1,\lambda_2,\cdots,\lambda_n$是$A$的 $n$个特征值,则

          $$\begin{cases}|A|=\lambda_1\lambda_2\cdots\lambda_n\:,\\\mathrm{tr}\left(A\right)=\lambda_1+\lambda_2+\cdots+\lambda_n\:.\end{cases}$$
    \item \begin{enumerate}
              \item 记住下表
                    \begin{table}[h]
                        \centering
                        \begin{tabular}{|c|c|c|c|c|c|c|}
                            \hline
                            矩阵      & $A$       & $f(A)$       & $A^{-1}$            & $f(A^*)$               & $P^{-1}AP=B$ & $P^{-1}f(A)P=B$ \\
                            \hline
                            特征值     & $\lambda$ & $f(\lambda)$ & $\frac{1}{\lambda}$ & $\frac{|A|}{\lambda} $ & $\lambda$    & $f(\lambda)$    \\
                            \hline
                            对应的特征向量 & $\xi$     & $\xi$        & $\xi$               & $\xi$                  & $P^{-1}\xi$  & $P^{-1}\xi$     \\
                            \hline
                        \end{tabular}
                    \end{table}

                    表中$\lambda$在分母上的,设 $\lambda != 0$
                    \begin{note}{}{}
                        当$\lambda\neq0$ 时,$af(A)\pm bA^{-1}\pm cA^{.}$的特征值为$af\left(\lambda\right)\pm b\frac1\lambda\pm c\frac{|A|}\lambda$,特征向量仍为$\xi$.但
                        $f(A),A^{-1},A^*$ 与$A^T$,$B$ 的线性组合因特征向量不同,无上述规律.
                    \end{note}
              \item $\text{虽然 }A^\mathrm{T}\text{ 的特征值与 }A\text{ 相同,但特征向量不再是 }\xi\text{,要单独计算才能得出 }.$
                    \begin{note}{}{}
                        $ A^\mathrm{T}\text{ 和 }A\text{ 属于不同特征值的特征向量正交 }.$
                    \end{note}
              \item 归零原则.
                    \begin{enumerate}
                        \item 归零准则一: 设$f(x)$为多项式, 若矩阵$A$满足$f(A)=O$, $\lambda$是$A$的任一特征值, 则$\lambda$满足$f(\lambda)=0$.
                        \item 归零准则二:设$n$ 阶方阵 $A$ 的特征多项式为$f(\lambda)=|\lambda E-A|=\lambda^n+a_{n-1}\lambda^{n-1}+\cdots+a_1\lambda+a_0$,则$A$ 的
                              多项式$f(A)$为零矩阵,即$f( A) = A^{n }+ a_{n- 1}A^{n- 1}+ \cdots + a_{1}A+ a_{0}E= O$ .
                    \end{enumerate}
          \end{enumerate}
\end{enumerate}

\section{利用特征向量命题}
\DOne + \DTwoTwo
\begin{enumerate}
    \item $\xi(\neq0)$是$A$的属于$\lambda_{0}$的特征向量$\Leftrightarrow\xi$是$(\lambda_{0}E-A)x=0$的非零解.\DTwoTwo
    \item 重要结论.
          \begin{enumerate}
              \item 单根恰有$1$个线性无关的特征向量.
              \item $k$重特征值$\lambda$至多只有$k$个线性无关的特征向量($k≥2$).
              \item 若$\xi_1,\xi_2$是$A$的属于不同特征值$\lambda_1,\lambda_2$的特征向量,则$\xi_1,\xi_2$线性无关.
              \item 若$\xi_1,\xi_2$是$A$ 的属于同一特征值 $\lambda$ 的特征向量,则当 $k_1k_2\neq0$ 时,非零向量 $k_1\xi_1+k_2\xi_2$仍是 $A$ 的属于特征值$\lambda$的特征向量(常考其中一个系数(如$k_2$)等于0的情形).
              \item 若$\xi_1,\xi_2$是$A$的属于不同特征值$\lambda_1,\lambda_2$的特征向量,则当$k_1\neq0,k_2\neq0$时,$k_1\xi_1+k_2\xi_2$不是$A$的任何特征值的特征向量(常考$k_1=k_2=1$的情形 ).
              \item 若 $\xi$ 是 $A$ 的属于特征值 $\lambda_1$ 的特征向量,$\lambda_1 \neq \lambda_2$,则 $\xi$ 不是 $\lambda_2$ 的特征向量.
              \item 若$A$只有$1$个线性无关的特征向量,即$\sum_{i=1}^{m}[n-r(\lambda_{i}E-A)]=1$,$\lambda_{i}(i=1,2,\cdots,m)$是A的m个不同特征值,则只能有一个$\lambda_{k}(1\leqslant k\leqslant m)$,使$r(\lambda_{k}E-A)=n-1$,而其余$r(\lambda_{i}E-A)=n$,这与$r(\lambda_{i}E-A)<n$矛盾。故A只能有一个$\lambda_{k}$,且此$\lambda_{k}$为n重特征值.
              \item 设$n$阶矩阵$A,B$满足$AB=BA$,且$A$有$n$个互不相同的特征值,则$A$的特征向量都是$B$的特征向量.
              \item 若 $r(A) + r(B) < n$,则 $Ax = 0$,$Bx = 0$ 至少有一个公共非零解 $\xi$.
          \end{enumerate}
\end{enumerate}
\section{利用矩阵方程命题}
\DOne + \DTwoTwo + \DTwoThree

\begin{enumerate}
    \item $AB=O \Rightarrow A[\beta_1, \beta_2, \cdots, \beta_n] = [0, 0, \cdots, 0]$, 即 $A\beta_i = 0\beta_i (i=1, 2, \cdots, n)$, 若 $\beta_i$ 均为非零列向量, 则 $\beta_i$ 为 $A$ 的属于特征值 $\lambda=0$ 的特征向量.
    \item 若任意 $n$ 维列向量 $\xi (\neq 0)$ 均为 $(\lambda E - A)x = 0$ 的解, 则令 $e_1 = \begin{bmatrix} 1 \\ 0 \\ \vdots \\ 0 \end{bmatrix}$, $e_2 = \begin{bmatrix} 0 \\ 1 \\ \vdots \\ 0 \end{bmatrix}$, $\cdots$, $e_n = \begin{bmatrix} 0 \\ \vdots \\ 0 \\ 1 \end{bmatrix}$,且$\boldsymbol{B}=[\boldsymbol{e}_{1},\boldsymbol{e}_{2},\cdots,\boldsymbol{e}_{n}]$,于是$(\lambda\boldsymbol{E}-\boldsymbol{A})\boldsymbol{B}=\boldsymbol{O}$,由于$\boldsymbol{B}$可逆,因此有$\lambda\boldsymbol{E}-\boldsymbol{A}=\boldsymbol{O}$,即$\boldsymbol{A}=\lambda\boldsymbol{E}$.
    \item $AB = C \Rightarrow A[\beta_1, \beta_2, \cdots, \beta_n] = [\gamma_1, \gamma_2, \cdots, \gamma_n] = [\lambda_1\beta_1, \lambda_2\beta_2, \cdots, \lambda_n\beta_n]$, 即 $A\beta_i = \lambda_i\beta_i (i=1, 2, \cdots, n)$, 其中 $\gamma_i = \lambda_i\beta_i$, $\beta_i$ 为非零列向量, 则 $\beta_i$ 为 $A$ 的属于特征值 $\lambda_i$ 的特征向量.
    \item  $AP = PB$, $P$ 可逆 $\Rightarrow P^{-1}AP = B \Rightarrow A \sim B \Rightarrow \lambda_A = \lambda_B$.
    \item $A$ 的每行元素之和均为 $k \Rightarrow A\begin{bmatrix} 1 \\ 1 \\ \vdots \\ 1 \end{bmatrix} = k\begin{bmatrix} 1 \\ 1 \\ \vdots \\ 1 \end{bmatrix}\Rightarrow k$ 是特征值, $\begin{bmatrix} 1 \\ 1 \\ \vdots \\ 1 \end{bmatrix}$ 是 $A$ 的属于特征值 $k$ 的特征向量.
    \item 若 $A$ 可逆, $A$ 的每行元素之和均为 $k$, 则 $A^{-1}$ 的每行元素之和均为 $\frac{1}{k}$.
    \item 若 $A$ 的每行元素之和均为 $k$, 则 $A^n$ 的每行元素之和均为 $k^n$.
\end{enumerate}
\chapter{相似理论}
\section{化归相似对角化的基本局面}
\DOne+\DTwoThree

若 $n$ 阶矩阵 $A$ 有 $n$ 个线性无关的特征向量,则 $A$ 可相似对角化,且有
$$[\xi_1, \xi_2, \cdots, \xi_n]^{-1} A [\xi_1, \xi_2, \cdots, \xi_n] = \begin{bmatrix} \lambda_1 & & \\ & \lambda_2 & \\ & & \ddots \\ & & & \lambda_n \end{bmatrix},$$
牢记这个形式.

\section{用各种条件判$A$能否相似对角化}
\DOne+\DTwoTwo

$\star \star \star$

\begin{enumerate}
    \item 充要条件
          \begin{enumerate}
              \item $A$有$n$个线性无关的特征向量 $\Leftrightarrow A \sim\Lambda$
              \item $n_i = n - r(\lambda_i E - A) \Leftrightarrow A \sim \Lambda$.
          \end{enumerate}
    \item 充分条件
          \begin{enumerate}
              \item $A$是实对称矩阵$\Leftrightarrow A \sim \Lambda$.
              \item $A$有$n$个互异特征值$\Leftrightarrow A \sim \Lambda$.
              \item $A^k=E$($k$为正整数) $\Leftrightarrow A \sim \Lambda$.
              \item $A^2 - (k_1 + k_2)A + k_1 k_2 E = O$ 且 $k_1 \neq k_2 \Rightarrow A \sim \Lambda$.
              \item $r(A) = 1$ 且 $\text{tr}(A) \neq 0 \Rightarrow A \sim \Lambda$.
          \end{enumerate}
    \item 必要条件

          $A \sim \Lambda \Rightarrow r(A) =$ 非零特征值的个数 (重根按重数算).
    \item 否定条件
          \begin{enumerate}
              \item $A \neq O$, $A^k = O$ ($k$ 为大于 1 的整数) $\Rightarrow A$ 不可相似对角化.
              \item $A$ 的特征值全为 $k$, 但 $A \neq kE \Rightarrow A$ 不可相似对角化.
          \end{enumerate}
\end{enumerate}

\section{非对称矩阵$A$与实对称矩阵$A$相似对角化的异同}
\DOne+\DTwoOne

\begin{enumerate}
    \item 非对称矩阵$A$不存在正交矩阵$Q$,使其相似对角化
    \item 实对称矩阵$A$存在正交矩阵$Q$,使其相似对角化
\end{enumerate}
\section{$A$与$B$相似}
\DOne+\DTwoOne+\PFour

$\star \star \star$

\begin{enumerate}
    \item 若$A$相似于$B$,则
          \begin{enumerate}
              \item $|A| = |B|$;
              \item $r(A)=r(B)$;
              \item $tr(A)=tr(B)$;
              \item $\lambda_{A} = \lambda_{B}$ (或 |$\lambda E - A$| = |$\lambda E - B$|);
              \item 属于 $\lambda_{A}$ 的线性无关的特征向量的个数等于属于 $\lambda_{B}$ 的线性无关的特征向量的个数;
              \item $A, B$  的各阶主子式之和分别相等.
          \end{enumerate}

    \item 若$A$相似于$\Lambda$,$B$相似于$\Lambda$,则$A$相似于$B$.
    \item 若$A$相似于$B$,$B$相似于$\Lambda$,则$A$相似于$\Lambda$.
    \item $A$与$B$的相似手段的“三同一不同”.

          若 $P^{-1}AP = B$, 则 $P^{-1}f(A)P = f(B)$, $P^{-1}A^{-1}P = B^{-1}$, $P^{-1}A^{*}P = B^{*}$, 即 $f(A)$ 与 $f(B)$, $A^{-1}$ 与 $B^{-1}$, $A^{*}$ 与 $B^{*}$ 相似的手段相同, 也即 $P^{-1}[af(A) + bA^{-1} + cA^{*}]P = af(B) + bB^{-1} + cB^{*}$. 但 $A^{T}$ 与 $B^{T}$ 相似的手段与上面不同.
\end{enumerate}
\section{相似对角化的应用}
\DOne+\DTwoTwo

\begin{example}{}{}
    已知数列 $\{x_n\}$, $\{y_n\}$, $\{z_n\}$ 满足 $x_0 = -1$, $y_0 = 0$, $z_0 = 2$, 且
    $$\begin{cases}
            x_n = -2x_{n-1} + 2z_{n-1}, \\
            y_n = -2y_{n-1} - 2z_{n-1}, \\
            z_n = -6x_{n-1} - 3y_{n-1} + 3z_{n-1},
        \end{cases}$$
    记 $\alpha_n = \begin{bmatrix} x_n \\ y_n \\ z_n \end{bmatrix}$, 写出满足 $\alpha_n = A\alpha_{n-1}$ 的矩阵 $A$, 并求 $A^n$ 及 $x_n$, $y_n$, $z_n (n=1,2,\cdots)$.
\end{example}
\begin{solution}
    由题设得$\begin{bmatrix}x_n\\y_n\\z_n\end{bmatrix}=\begin{bmatrix}-2&0&2\\0&-2&-2\\-6&-3&3\end{bmatrix}\begin{bmatrix}x_{n-1}\\y_{n-1}\\z_{n-1}\end{bmatrix}$,得矩阵$A=\begin{bmatrix}-2&0&2\\0&-2&-2\\-6&-3&3\end{bmatrix}$满足$\alpha_n=A\alpha_{n-1}$.

    因为

    $|\lambda E-A|=\begin{vmatrix}\lambda+2&0&-2\\0&\lambda+2&2\\6&3&\lambda-3\end{vmatrix}=\lambda(\lambda-1)(\lambda+2)$,

    所以矩阵$A$的特征值为$\lambda_1=0$,$\lambda_2=1$,$\lambda_3=-2$.

    当$\lambda_1=0$时,解方程组$(0E-A)x=0$,得特征向量$\xi_1=\begin{bmatrix}1&-1&1\end{bmatrix}^{T}$;

    当$\lambda_2=1$时,解方程组$(E-A)x=0$,得特征向量$\xi_2=\begin{bmatrix}2&-2&3\end{bmatrix}^{T}$;

    当$\lambda_3=-2$时,解方程组$(-2E-A)x=0$,得特征向量$\xi_3=\begin{bmatrix}-1&2&0\end{bmatrix}^T$.

    令$P=[\xi_1,\xi_2,\xi_3]=\begin{bmatrix}1&2&-1\\-1&-2&2\\1&3&0\end{bmatrix}$,则$P^{-1}AP=\begin{bmatrix}0&0&0\\0&1&0\\0&0&-2\end{bmatrix}$,即$A=P\begin{bmatrix}0&0&0\\0&1&0\\0&0&-2\end{bmatrix}P^{-1}$,从而得$A^n=P\begin{bmatrix}0&0&0\\0&1&0\\0&0&-2\end{bmatrix}^nP^{-1}=\begin{bmatrix}1&2&-1\\-1&-2&2\\1&3&0\end{bmatrix}\begin{bmatrix}0&0&0\\0&1&0\\0&0&(-2)^n\end{bmatrix}\begin{bmatrix}6&3&-2\\-2&-1&1\\1&1&0\end{bmatrix}$

    $$=\begin{bmatrix}-4-(-2)^n&-2-(-2)^n&2\\4-(-2)^{n+1}&2-(-2)^{n+1}&-2\\-6&-3&3\end{bmatrix}$$.

    由递推式$\alpha_n=A\alpha_{n-1}$知$\alpha_n=A^n\alpha_0$,其中$\alpha_0=\begin{bmatrix}-1&0&2\end{bmatrix}^T$,所以

    $\alpha_n=A^n\alpha_0=\begin{bmatrix}-4-(-2)^n&-2-(-2)^n&2\\4-(-2)^{n+1}&2-(-2)^{n+1}&-2\\-6&-3&3\end{bmatrix}\begin{bmatrix}-1\\0\\2\end{bmatrix}=\begin{bmatrix}8+(-2)^n\\-8+(-2)^{n+1}\\12\end{bmatrix}$,

    故$x_n=8+(-2)^n$,$y_n=-8+(-2)^{n+1}$,$z_n=12(n=1,2,\cdots)$.
\end{solution}

\section{正交矩阵及其使用}
\DOne + \DTwoOne

$\star \star \star$

\begin{enumerate}
    \item 若$A$为正交矩阵,则
          $$A^\top A = E \Leftrightarrow A^{-1} = A^\top$$
          $$\Leftrightarrow A \text{ 由规范正交基组成 }$$
          $$\Leftrightarrow A^{\mathrm{T}}\text{是正交矩阵}$$
          $$\Leftrightarrow A^{-1}\text{是正交矩阵}$$
          $$\Leftrightarrow A^{*}\text{是正交矩阵}$$
          $$\Leftrightarrow-A\text{是正交矩阵.}$$
    \item 若$A,B$为同阶正交矩阵,则$AB$为正交矩阵,但$A+B$不一定为正交矩阵.
    \item ${\text{若 }A\text{ 为正交矩阵,则其实特征值的取值范围为}\{-1,1\}}.$
    \item 设$A$为$n$阶非零矩阵,$\begin{cases}\text{若}a_{ij}=A_{ij},\text{则}A^{\mathrm{T}}=A^{*},AA^{\mathrm{T}}=E,\text{且}\mid A\mid=1;\\\text{若}a_{ij}=-A_{ij},\text{则}A^{\mathrm{T}}=-A^*,AA^{\mathrm{T}}=E,\text{且}\mid A\mid=-1.\end{cases}$
\end{enumerate}
\chapter{二次型}
\section{$f=x^TAx$ 中$A$的表示}
\DOne+\DTwoThree

\begin{enumerate}
    \item 给出非对称矩阵 $B$,令$A=\frac{B+B^T}{2}$,则$A=A^\mathrm{T}.$
    \item 通过题设或基本变形显化出 $A.$
\end{enumerate}

\section{配方法与正交变换法的异同}

\begin{enumerate}
    \item 命题语言
          \DTwoTwo
          \begin{enumerate}
              \item 配方法

                    二次型语言:将 $f = x^T A x$ 通过配方法化为标准形,并求出可逆变换矩阵 $C$.

                    矩阵语言:求可逆矩阵 $C$,使得 $C^T A C = \Lambda$.
              \item 正交变换法

                    二次型语言:将 $f = x^T A x$ 通过正交变换法化为标准形,并求出正交矩阵 $Q$.

                    矩阵语言:求正交矩阵 $Q$,使得 $Q^T A Q = \Lambda$.
          \end{enumerate}
    \item 过程与结果的异同
          \DTwoThree

          设$f(x)=x^TAx$.
          \begin{enumerate}
              \item 配方法(可逆线性变换)

                    $x=Cy$,$C$可逆.使得$f\xlongequal{x=\boldsymbol{C}y}y^T\Lambda y$,其中$C^TAC=\Lambda$(使$A$合同于对角矩阵).
              \item 正交变换法(可逆线性变换):

                    $x=Qy$(这里的$Q$不仅可逆,还满足$Q^{-1}=Q^{T}$),使得$f\xlongequal{x=\boldsymbol{Q}y}y^T\Lambda y$,其中$Q^{T}AQ=Q^{-1}AQ=A$.
          \end{enumerate}
          二者区别:在配方法中,$c$只满足可逆,所以$c^{-1}$不一定等于$c^T$,但是在正交变换法中,变换手段$Q$满足$Q^{- 1}= Q^T$ .

          二者相同点:它们的正、负惯性指数是对应相等的.
    \item 惯性指数
          \begin{example}{}{}
              $f(x_{1},x_{2},x_{3})=-2x_{1}x_{2}-2x_{1}x_{3}+6x_{2}x_{3}$的正惯性指数为(  ).
          \end{example}
          \begin{solution}
              令$\begin{cases}x_{1}=y_{1}+y_{2},\\x_{2}=y_{1}-y_{2},\end{cases}$则

              $$f=-2y_{1}^{2}+2y_{2}^{2}+4y_{1}y_{3}-8y_{2}y_{3}
                  =-2(y_{1}-y_{3})^{2}+2(y_{2}-2y_{3})^{2}-6y_{3}^{2},$$

              再令$\begin{cases}z_{1}=y_{1}-y_{3},\\z_{2}=y_{2}-2y_{3},\end{cases}$则

              $$f=-2z_{1}^{2}+2z_{2}^{2}-6z_{3}^{2},$$

              故$f$的正惯性指数为1.
          \end{solution}
\end{enumerate}

\section{伪配方法}
\DTwoThree

“平方和式$A^2+B^2+C^2$”未必就是(拉格朗日)配方法得来的结果,故若非拉格朗日配方法,则称伪配方法.要注意伪配方法的变换矩阵是否有可逆性.
\begin{enumerate}
    \item 如果变换没有可逆性,则有可能改变表达式的几何性质,如封闭性,此时不能得出平方和式正定;
    \item 如果变换是可逆的,则平方和式正定.
\end{enumerate}


\begin{note}{}{}
    对于$f(x_{1},x_{2},x_{3})=(a_{1}x_{1}+a_{2}x_{2}+a_{3}x_{3})^{2}+(b_{1}x_{1}+b_{2}x_{2}+b_{3}x_{3})^{2}+(c_{1}x_{1}+c_{2}x_{2}+c_{3}x_{3})^{2}$的情形,可总结如下做题方法:

    令$f=0$,即$\begin{cases}a_{1}x_{1}+a_{2}x_{2}+a_{3}x_{3}=0,\\b_{1}x_{1}+b_{2}x_{2}+b_{3}x_{3}=0,\\c_{1}x_{1}+c_{2}x_{2}+c_{3}x_{3}=0,\end{cases}$计算$|\boldsymbol{A}|=\begin{vmatrix}a_{1}&a_{2}&a_{3}\\b_{1}&b_{2}&b_{3}\\c_{1}&c_{2}&c_{3}\end{vmatrix}$,若$|\boldsymbol{A}|\neq0$,则$f$正定;若$|\boldsymbol{A}|=0$,则$f$不正定.
\end{note}
\section{正交变换法的传递性}
\DOne+\DTwoThree

若$A$相似于$B$,则$B$相似于$C$,则$A$相似于$C$.这里$B$常为$\Lambda$.

\section{合同的判定与手段}
\DOne+\DTwoThree

\begin{enumerate}
    \item 同阶实对称矩阵$A,B$合同的判定

          用正、负惯性指数:$A,B$合同$\Leftrightarrow p_A=p_B,q_A=q_B$(相同的正、负惯性指数).
    \item 已知$A$,$\Lambda$($\Lambda$是对角矩阵),求可逆矩阵$C$,使得$C^TAC=\Lambda$
    \item 已知$A$,$B$($B$不是对角矩阵),求可逆矩阵$C$,使得$C^TAC=B$
\end{enumerate}
\begin{idea}{求可逆矩阵$C$,使得$C^TAC=\Lambda$}{}
    \begin{enumerate}
        \item 配方 盯着$\Lambda$的对角线元素,提出对应系数
        \item 换元
        \item 求逆
    \end{enumerate}
\end{idea}
\begin{idea}{求可逆矩阵$C$,使得$C^TAC=B$}{}
    \begin{enumerate}
        \item 对$f$配方、换元,写$D_1$
        \item 对$g$配方、换元,写$D_2$
        \item 令$D_1x=D_2y$,求$D_2^{-1}D_1$
    \end{enumerate}
\end{idea}
\section{合同与相似的异同}
\DOne+\DTwoThree

对于实对称矩阵$A$与$B$,相似必合同,反之不成立.
\begin{example}{合同与相似的异同}{合同与相似的异同}
    已知二次型
    $$f(x_{1}, x_{2}, x_{3}) = x_{1}^{2} + 2x_{2}^{2} + 2x_{3}^{2} + 2x_{1}x_{2} - 2x_{1}x_{3}$$
    $$
        g(y_{1}, y_{2}, y_{3}) = y_{1}^{2} + y_{2}^{2} + y_{3}^{2} + 2y_{2}y_{3}$$
    \begin{enumerate}
        \item 求可逆变换 $x = Py$,将 $f(x_{1}, x_{2}, x_{3})$ 化成 $g(y_{1}, y_{2}, y_{3})$.
        \item 是否存在正交变换 $x = Qy$,将 $f(x_{1}, x_{2}, x_{3})$ 化成 $g(y_{1}, y_{2}, y_{3})$?
    \end{enumerate}
\end{example}
\begin{idea}{解题思路 \ref{ex:合同与相似的异同}}{}
    \begin{enumerate}
        \item 求可逆变换用配方法
        \item 判断是否存在正交变换,如果存在必相似,使用相似的充分条件和充要条件
    \end{enumerate}
\end{idea}
\section{正定的判定与应用}
\DOne+\DTwoThree

$\star\star\star$

\begin{enumerate}
    \item 前提
          $A=A^T$($A$是实对称矩阵)
    \item 二次型$f=x^TAx$正定的充要条件 \DTwo

          $n$元二次型$f=x^{T}Ax$正定

          $\Leftrightarrow$对任意的$x\neq 0$,有$x^{T}Ax>0$(定义)

          $\Leftrightarrow A$的特征值$\lambda_{i}>0(i=1,2,\cdots,n)$

          $\Leftrightarrow f$的正惯性指数$p=n$

          $\Leftrightarrow$存在可逆矩阵$D$,使得$A=D^{T}D$

          $\Leftrightarrow A$与$E$合同

          $\Leftrightarrow A$的各阶顺序主子式均大于0.
    \item 二次型$f=x^TAx$正定的必要条件
          \begin{enumerate}
              \item $a_{ii}>0\left(i=1,2,\cdots,n\right).$
              \item $| A| > 0$.
          \end{enumerate}
    \item 重要结论
          \begin{enumerate}
              \item 若$A$正定,则$A^-1,A^{*},A^{m}(m$为正整数$),kA(k>0),C^{\mathrm{T}}AC(C$可逆 )均正定 .

              \item 若$A,B$正定,则$A+B$正定,$\begin{bmatrix}A&O\\O&B\end{bmatrix}$正定.

              \item ${\text{若}A,B}$正定,则$AB$正定的充要条件是$AB= BA$ .
          \end{enumerate}

\end{enumerate}

\section{二次型的最值}
\DOne+\DTwoTwo+\DTwoThree



\LUgroupsancheck

\makeatletter
\let\chapter\@std@chapter
\let\@std@chapter\relax
\makeatother

\backmatter
{\small
    \printindex
    \printindex[sym]
}

\end{document}
