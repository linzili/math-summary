\makeatletter
   \def\input@path{{..}} % 搜索上层目录的 LALUbook
\makeatother

\documentclass[
  UTF8
    % colors = false,
    geometry = a4,
]{LALUbook}

\usepackage{mathdots}
\usepackage{booktabs} % Excel 导出的大表格
\usepackage{rotating}
\usepackage{extarrows}
\usepackage{blkarray}
\usepackage{cases}

\usepackage{float}
\usepackage{diagbox}
\usepackage{caption}

\usepackage{pgfplots}
\usetikzlibrary{cd, arrows, arrows.meta, calc, intersections, decorations.pathreplacing, patterns, decorations.markings,angles,quotes, graphs, positioning, shapes.geometric}
\pgfplotsset{compat=newest}

\usepackage[xindy, splitindex]{imakeidx}
\makeindex[
    columns=1,
    program=truexindy,
    intoc=true,
    options=-M texindy -I xelatex -C utf8,
    title={名词索引}
] % 名词索引
\makeindex[
    columns=3,
    program=truexindy,
    intoc=true,
    options=-M numeric-sort -M latex -M latex-loc-fmts -M makeindex -I xelatex -C utf8,
    name=sym,
    title={符号索引}
] % 符号索引


% OPD 系列
\newcommand{\OOne}{\textcolor{blue}{\textbf{$O$ (盯住目标)}}}
\newcommand{\DOne}{\textcolor{blue}{\textbf{$D_1$ (常规操作)}}}
\newcommand{\DTwo}{\textcolor{blue}{\textbf{$D_2$ (脱胎换骨)}}}
\newcommand{\DThree}{\textcolor{blue}{\textbf{$D_3$ (移花接木)}}}
\newcommand{\DFour}{\textcolor{blue}{\textbf{$D_4$ (可圈可点)}}}
\newcommand{\DFourThree}{\textcolor{blue}{\textbf{$D_{43}$ (数形结合)}}}
\newcommand{\DTwoOne}{\textcolor{blue}{\textbf{$D_{21}$ (观察研究对象)}}}
\newcommand{\DTwoTwo}{\textcolor{blue}{\textbf{$D_{22}$ (转换等价表述)}}}
\newcommand{\DTwoThree}{\textcolor{blue}{\textbf{$D_{23}$ (化归经典形式)}}}
\newcommand{\POne}{\textcolor{blue}{\textbf{$P_{1}$ (常规思路)}}}
\newcommand{\POneOne}{\textcolor{blue}{\textbf{$P_{11}$ (正向思路)}}}
\newcommand{\POneTwo}{\textcolor{blue}{\textbf{$P_{12}$ (反向思路)}}}
\newcommand{\POneThree}{\textcolor{blue}{\textbf{$P_{13}$ (双向思路)}}}
\newcommand{\PTwo}{\textcolor{blue}{\textbf{$P_{2}$ (反证思路)}}}
\newcommand{\PThree}{\textcolor{blue}{\textbf{$P_{3}$ (数学归纳)}}}
\newcommand{\PFour}{\textcolor{blue}{\textbf{$P_{4}$ (逆否思路)}}}
\newfontfamily\RomanSymbols{Arial Unicode MS}

\newtcbtheorem[number within=section]{detail}{细节}{laluthmstyle={red}}{det}
\newtcbtheorem[number within=section]{idea}{思路}{laluthmstyle={teal}}{ide}

% 内容总结
\newenvironment{summary}{%
    \hypersetup{bookmarksnumbered=false}%
    \titleformat{\subsection}[block]{\centering\heiti\Large}{}{1em}{}%
    \phantomsection%
    \subsection*{内容总结}%
}{}


% 标题格式
% \chapter   正常专题
% \Wchapter 高数强化专题
% \Xchapter 线代强化专题
% \Gchapter 概率强化专题
\newcounter{GAOchapter}
\newcounter{XIANchapter}
\newcounter{GAIchapter}

\makeatletter
\newcommand*{\GAOgroupsancheck}{%
  \expandafter\@ifundefined{@exist@GAOchapter@\arabic{chapter}}%
    {}%
    {\endgroup}%
}
\newcommand*{\XIANgroupsancheck}{%
  \expandafter\@ifundefined{@exist@XIANchapter@\arabic{chapter}}%
    {}%
    {\endgroup}%
}
\newcommand*{\GAIgroupsancheck}{%
  \expandafter\@ifundefined{@exist@GAIchapter@\arabic{chapter}}%
    {}%
    {\endgroup}%
}

\let\@std@chapter\chapter
\renewcommand*{\chapter}{%
  \GAOgroupsancheck%
  \XIANgroupsancheck%
  \GAIgroupsancheck%
  \@std@chapter%
}

\newcommand{\GAOchapter}[1]{%
  \begingroup
  \addtocounter{chapter}{-1}
  \refstepcounter{GAOchapter}
  % 设置章节编号格式
  \renewcommand*{\thechapter}{\arabic{GAOchapter}}
  % 设置超链接锚点格式
  \renewcommand*{\theHchapter}{GAO.\arabic{GAOchapter}}
  
  \ctexset{
      chapter={
        format={\centering\Huge\bfseries},
        name={高数专题,},
        number={\zhnumber{\arabic{GAOchapter}}}
      },
  }
  \csname @std@chapter\endcsname{#1}
  \expandafter\xdef\csname @exist@GAOchapter@\arabic{chapter}\endcsname{\null}
}

\newcommand{\XIANchapter}[1]{%
  \begingroup
  \addtocounter{chapter}{-1}
  \refstepcounter{XIANchapter}
  % 设置章节编号格式
  \renewcommand*{\thechapter}{\arabic{XIANchapter}}
  % 设置超链接锚点格式
  \renewcommand*{\theHchapter}{XIAN.\arabic{XIANchapter}}
  
  \ctexset{
      chapter={
        format={\centering\Huge\bfseries},
        name={线代专题,},
        number={\zhnumber{\arabic{XIANchapter}}}
      },
  }
  \csname @std@chapter\endcsname{#1}
  \expandafter\xdef\csname @exist@XIANchapter@\arabic{chapter}\endcsname{\null}
}

\newcommand{\GAIchapter}[1]{%
  \begingroup
  \addtocounter{chapter}{-1}
  \refstepcounter{GAIchapter}
  % 设置章节编号格式
  \renewcommand*{\thechapter}{\arabic{GAIchapter}}
  % 设置超链接锚点格式
  \renewcommand*{\theHchapter}{GAI.\arabic{GAIchapter}}
  
  \ctexset{
      chapter={
        format={\centering\Huge\bfseries},
        name={概率专题,},
        number={\zhnumber{\arabic{GAIchapter}}}
      },
  }
  \csname @std@chapter\endcsname{#1}
  \expandafter\xdef\csname @exist@GAIchapter@\arabic{chapter}\endcsname{\null}
}

\makeatother

\ctexset{
    chapter={format={\centering\Huge\bfseries},name={强化专题,},number={\zhnumber{\arabic{chapter}}}},
    % chapter={format={\centering\Huge\bfseries},name={第,讲},number=\arabic{chapter}},
    section={format={\raggedright\Large\bfseries},name={,},number={\thechapter.\arabic{section}}},
    subsection={format={\raggedright\large\bfseries},name={,},number={\thesection.\arabic{subsection}}},
    subsubsection={format={\raggedright\normalsize\bfseries},name={,},number={\thesubsection.\arabic{subsubsection}}},
}

\title{\heiti 临沂大学 2025--2026 学年 \\ 2026考研数学总结讲义}

\AtEndPreamble{\hypersetup{
    hypertexnames=true,
    % linktocpage=true,
    pdfauthor={林子立},
    pdftitle={2026考研数学总结讲义},
}}

\begin{document}

\title{2026考研数学总结讲义}
\author{林子立}
\date{\today}
\maketitle

% \songti
\pagenumbering{Roman}
\clearpage

\pdfbookmark[0]{目录}{contents}
\tableofcontents

\addtolength{\parskip}{.5em}

\mainmatter
\GAOchapter{函数极限与连续}

\section{判定类型,做好计算}

\begin{enumerate}
    %-------------------------------------------------------
    \item \textbf{未定式整体判定} \DTwoThree

          一般地,见到以下形式的极限:
          \[
              \frac{?}{0}, \quad \frac{0}{?}, \quad \frac{\infty}{?}, \quad
              \frac{?}{\infty}, \quad ?\cdot\infty, \quad
              \infty\cdot?, \quad \infty-?, \quad \infty^{?}, \quad ?^{\infty},
          \]
          可直接判断它们分别对应的七种未定式类型:
          \[
              \frac{0}{0}, \quad \frac{\infty}{\infty}, \quad 0\cdot\infty, \quad
              \infty-\infty, \quad \infty^{0}, \quad 0^{0}, \quad 1^{\infty}.
          \]
          若题设形式不属于这七类,则通常不是未定式问题。

          %-------------------------------------------------------
    \item \textbf{未定式局部判定} \DTwoThree

          例如:
          \[
              \lim_{n\to\infty}\frac{1+x}{1+nx^{2n}}=
              \begin{cases}
                  0,   & x=\pm1, \\
                  1+x, & |x|<1,  \\
                  0,   & |x|>1.
              \end{cases}
          \]
          关键在于判定局部项 $nx^{2n}$ 的极限性质。常见局部极限总结如下:

          \[
              \lim_{n\to\infty}|x|^n=
              \begin{cases}
                  \infty, & |x|>1, \\
                  1,      & |x|=1, \\
                  0,      & |x|<1;
              \end{cases}
              \quad
              \lim_{x\to0^+}x^a=
              \begin{cases}
                  0,       & a>0, \\
                  1,       & a=0, \\
                  +\infty, & a<0;
              \end{cases}
          \]
          \[
              \lim_{n\to\infty}nx^{2n}=
              \begin{cases}
                  +\infty, & |x|\ge1, \\
                  0,       & |x|<1;
              \end{cases}
              \quad
              \lim_{n\to\infty}e^{nx}=
              \begin{cases}
                  +\infty, & x>0, \\
                  1,       & x=0, \\
                  0,       & x<0;
              \end{cases}
          \]
          \[
              \lim_{n\to\infty}n^x=
              \begin{cases}
                  +\infty, & x>0, \\
                  1,       & x=0, \\
                  0,       & x<0.
              \end{cases}
          \]

          %-------------------------------------------------------
    \item \textbf{常用的无穷小量阶的比较}

          \begin{enumerate}
              \item \textbf{普通函数型:}

                    当 $x\to0$ 时,
                    \[
                        \sin x \sim x, \quad \tan x \sim x, \quad
                        \arcsin x \sim x, \quad \arctan x \sim x,
                    \]
                    \[
                        e^x - 1 \sim x, \quad \ln(1+x) \sim x, \quad
                        \ln(x + \sqrt{1+x^2}) \sim x,
                    \]
                    \[
                        a^x - 1 = e^{x\ln a}-1 \sim x\ln a \ (a>0,a\ne1),
                        \quad 1 - \cos x \sim \tfrac{1}{2}x^2,
                    \]
                    \[
                        1 - \cos^{\alpha}x \sim \tfrac{\alpha}{2}x^2 \ (\alpha\ne0),
                        \quad (1+x)^{\alpha}-1 \sim \alpha x,
                        \quad (1+x)^x - 1 \sim x^2.
                    \]

              \item \textbf{差函数型:}

                    当 $x\to0$ 时,
                    \[
                        x - \sin x \sim \tfrac{1}{6}x^3, \quad
                        x - \arcsin x \sim -\tfrac{1}{6}x^3, \quad
                        x - \tan x \sim -\tfrac{1}{3}x^3, \quad
                        x - \arctan x \sim \tfrac{1}{3}x^3,
                    \]
                    \[
                        x - \ln(1+x) \sim \tfrac{1}{2}x^2,
                        \quad e^x - 1 - x \sim \tfrac{1}{2}x^2.
                    \]

                    可通过恒等变形“创造”差函数,如:
                    \[
                        \begin{cases}
                            x - \ln(1+\tan x) = (x - \tan x) + (\tan x - \ln(1+\tan x)); \\[4pt]
                            \sin x + \ln(1 - \sin x) = -[-\sin x - \ln(1 - \sin x)];     \\[4pt]
                            f(x) - \tan x = (f(x) - x) + (x - \tan x).
                        \end{cases}
                    \]

              \item \textbf{复合函数型:}

                    若 $f(x)\sim ax^m,\ g(x)\sim bx^n$ 且 $ab\ne0$,则
                    \[
                        f[g(x)] \sim ab^m x^{mn}.
                    \]

              \item \textbf{变上限积分型:}
                    \begin{enumerate}
                        \item 若 $f(x)\sim ax^m$,则
                              \[
                                  \int_0^x f(t)\,\mathrm{d}t \sim \int_0^x a t^m\,\mathrm{d}t.
                              \]
                        \item 若 $\lim_{x\to0}f(x)=A\ne0$, $h(x)\to0$,则
                              \[
                                  \int_0^{h(x)} f(t)\,\mathrm{d}t \sim A\,h(x).
                              \]
                    \end{enumerate}

              \item \textbf{复合+变上限积分型:}
                    \[
                        f(x)\sim ax^m,\ g(x)\sim bx^n
                        \Rightarrow
                        \int_0^{g(x)}f(t)\,\mathrm{d}t \sim \int_0^{bx^n}a t^m\,\mathrm{d}t.
                    \]

              \item \textbf{带头大哥型:}

                    若 $\alpha=o(\beta)$,则
                    \[
                        \textcircled{1}\ \alpha+\beta\sim\beta, \qquad
                        \textcircled{2}\ \alpha+\beta\text{ 与 }\beta\text{同号}, \qquad
                        \textcircled{3}\ \alpha\beta=o(\beta^2).
                    \]
          \end{enumerate}

          %-------------------------------------------------------
    \item \textbf{常用无穷大量阶的比较}

          \[
              \begin{cases}
                  \text{当 }x\to+\infty: & \ln^p x \ll x^q \ll a^x \ll x^x,        \\[4pt]
                  \text{当 }n\to\infty:  & \ln^p n \ll n^q \ll a^n \ll n! \ll n^n,
              \end{cases}\quad (p,q>0,a>1)
          \]
          因此:
          \[
              \lim_{n\to\infty}\frac{\ln^p n}{n^q}=0,\quad
              \lim_{n\to\infty}\frac{n^q}{a^n}=0,\quad
              \lim_{n\to\infty}\frac{a^n}{n!}=0,\quad
              \lim_{n\to\infty}\frac{n!}{n^n}=0.
          \]

          %-------------------------------------------------------
    \item \textbf{涉及 $\infty$ 的计算问题} \DTwoThree

          关于 $\infty-\infty$ 型,注意以下四点:

          \begin{enumerate}
              \item 若 $f(x)$ 在 $|x|$ 足够大时有定义,则
                    \[
                        \lim_{x\to\infty}f(x) \text{ 存在}
                        \iff
                        \lim_{x\to+\infty}f(x) = \lim_{x\to-\infty}f(x).
                    \]

              \item 若出现差式 $f(x)-f(x)$(如三角、对数、反三角函数差),
                    可用\textbf{拉格朗日中值定理}变形再求极限。

              \item 若出现幂次差:
                    \[
                        [f_1(x)]^{g(x)}-[f_2(x)]^{g(x)} = [f_2(x)]^{g(x)}
                        \left(\left[\frac{f_1(x)}{f_2(x)}\right]^{g(x)}-1\right),
                    \]
                    \[
                        [f(x)]^{g_1(x)}-[f(x)]^{g_2(x)} = [f(x)]^{g_2(x)}
                        \left([f(x)]^{g_1(x)-g_2(x)}-1\right).
                    \]

              \item 若见 $\lim_{x\to\infty}[f(x)-ax]$,
                    常通过\textbf{恒等变形}将其改写为乘除形式处理。
          \end{enumerate}
\end{enumerate}

\section{判定连续与间断}

\begin{enumerate}
    %---------------------------------------------------
    \item \textbf{常见备选点判定}

          常见函数的无定义点或分段点如下表所示:

          \begin{enumerate}
              \item $\displaystyle \mathrm{e}^{\frac{1}{x}} \Rightarrow x=0$ 为无定义点;
              \item $\displaystyle \frac{1}{\int_{1}^{x}|\sin t|\,\mathrm{d}t} \Rightarrow x=\pm1$ 为无定义点;
              \item $\displaystyle \frac{1}{\sin x} \Rightarrow x=k\pi \ (k=0,\pm1,\pm2,\cdots)$ 为无定义点;
              \item $\displaystyle \frac{1}{\arctan x} \Rightarrow x=0$ 为无定义点;
              \item $\displaystyle \frac{1}{\tan\!\left(x-\frac{\pi}{4}\right)},\ 0<x<2\pi
                        \Rightarrow x=\frac{\pi}{4},\frac{3\pi}{4},\frac{5\pi}{4},\frac{7\pi}{4}$ 为无定义点;
              \item $\displaystyle \frac{1}{|x|(x^2-1)} \Rightarrow x=0,\pm1$ 为无定义点;
              \item $[x] \Rightarrow x=n\ (n=0,\pm1,\pm2,\cdots)$ 为分段点;
              \item $\displaystyle |x|^{\frac{1}{(1-x)(x-2)}} \Rightarrow x=0,1,2$ 为无定义点。
          \end{enumerate}

          \vspace{0.5em}
          \textit{技巧提示:}
          若题目中出现分母、根号、对数或分段定义函数,应先检查其定义域边界与分段点,这些往往是可能的间断点。

          %---------------------------------------------------
    \item \textbf{计算三个关键值}

          判定连续性需依次计算以下三个值(或判断其是否存在):
          \[
              \lim_{x\to x_0^-}f(x), \quad
              \lim_{x\to x_0^+}f(x), \quad
              f(x_0).
          \]
          然后进行比较。

          %---------------------------------------------------
    \item \textbf{根据定义作出结论}

          若以上三者不全相等或有不存在的情况,按下列规则分类:

          \begin{enumerate}
              \item \textbf{跳跃间断点(第一类间断)}:
                    \[
                        \lim_{x\to x_0^-}f(x) \neq \lim_{x\to x_0^+}f(x),
                    \]
                    且左右极限都存在。

              \item \textbf{可去间断点}:
                    \[
                        \lim_{x\to x_0}f(x) \text{ 存在}, \quad
                        \text{但 } f(x_0)\text{ 未定义或 }\lim f(x)\neq f(x_0).
                    \]
                    若通过重新定义 $f(x_0)=\lim f(x)$ 可使函数连续。

              \item \textbf{无穷间断点}:
                    \[
                        \lim_{x\to x_0^\pm}f(x)=\pm\infty.
                    \]
                    函数趋于无穷大,图像在此处“竖直渐近”。

              \item \textbf{振荡间断点(第二类间断)}:
                    \[
                        \lim_{x\to x_0}f(x) \text{ 不存在且无穷振荡}.
                    \]
                    例如 $\sin\frac{1}{x}$ 在 $x=0$ 处。
          \end{enumerate}
\end{enumerate}
\section{研究 $x \to \cdot$ 时 $f(x)$ 的微观性态}

\begin{enumerate}
    %--------------------------
    \item \textbf{定义法} \DTwoTwo

          极限的 $\varepsilon$–$\delta$ 定义:
          \[
              \lim_{x \to x_0} f(x) = A
              \;\Leftrightarrow\;
              \forall \varepsilon > 0,\ \exists \delta > 0,\
              \text{当 } 0 < |x - x_0| < \delta \text{ 时},\ |f(x) - A| < \varepsilon.
          \]

          ✅ \textit{说明:}
          表明当 $x$ 无限接近 $x_0$ 时,$f(x)$ 可以被控制在 $A$ 的任意邻域内。

          %--------------------------
    \item \textbf{局部保号性} \DTwoTwo

          \begin{enumerate}
              \item 若 $f(x) \to A \ (x \to x_0)$ 且 $A > 0$(或 $A < 0$),
                    则存在 $\delta > 0$,使得当 $0 < |x - x_0| < \delta$ 时,
                    $f(x) > 0$(或 $f(x) < 0$)。

              \item 若在 $x_0$ 的某去心邻域内 $f(x) \ge 0$(或 $\le 0$),
                    且 $\displaystyle \lim_{x \to x_0} f(x) = A$,
                    则 $A \ge 0$(或 $A \le 0$)。
          \end{enumerate}

          ✅ \textit{应用技巧:}
          在求极限符号问题(如 $\lim f(x)/g(x)$ 是否为正)时,常结合保号性与等价无穷小判断符号。

          %--------------------------
    \item \textbf{夹逼准则(两边夹法)} \DTwoTwo+\DTwoThree

          若存在函数 $g(x)$、$h(x)$ 使:
          \begin{enumerate}
              \item $h(x) \le f(x) \le g(x)$;
              \item $\displaystyle \lim_{x \to x_0} h(x) = \lim_{x \to x_0} g(x) = A$;
          \end{enumerate}
          则 $\displaystyle \lim_{x \to x_0} f(x) = A$。

          ✅ \textit{常见应用:}
          \[
              \lim_{x \to 0} x^2 \sin\frac{1}{x} = 0, \quad
              \lim_{x \to 0} x \sin\frac{1}{x^2} = 0.
          \]

          %--------------------------
    \item \textbf{单调有界准则} \DTwoThree

          若存在 $\delta > 0$,使得:

          \begin{enumerate}
              \item $f(x)$ 在区间 $(x_0, x_0 + \delta)$ 内单调且有界,
                    则右极限 $\displaystyle \lim_{x \to x_0^+} f(x)$ 存在;
              \item $f(x)$ 在区间 $(x_0 - \delta, x_0)$ 内单调且有界,
                    则左极限 $\displaystyle \lim_{x \to x_0^-} f(x)$ 存在。
          \end{enumerate}

          同理,若 $f(x)$ 在 $(a, +\infty)$ 或 $(-\infty, b)$ 上单调有界,
          则 $\displaystyle \lim_{x \to +\infty} f(x)$ 或 $\lim_{x \to -\infty} f(x)$ 存在。

          ✅ \textit{典型函数:}
          \[
              f(x) = 1 - \frac{1}{x},\quad f(x)=\arctan x,\quad f(x)=1 - \frac{1}{2^x}.
          \]
\end{enumerate}
\GAOchapter{数列极限}
\GAOchapter{一元函数微分学的概念}

\section{微分——一阶泰勒公式}
\DTwo

当 $f'(x_0)$ 存在时,函数 $f(x)$ 在点 $x_0$ 处的微分为
\[
    d[f(x)]\big|_{x=x_0} = f'(x_0)\Delta x = f'(x_0)dx,
\]
即
\[
    f(x) - f(x_0) = f'(x_0)\Delta x + o(1)\Delta x = f'(x_0)\Delta x + o(\Delta x) \quad (\Delta x \to 0).
\]
此即为 \textbf{微分公式},亦称 \textbf{一阶泰勒公式}.

\section{导数——因变量差与自变量差的比值极限}
\DTwoTwo

函数 $f(x)$ 在点 $x_0$ 处的导数为
\[
    f'(x_0) = \lim_{\Delta x \to 0} \frac{\Delta f}{\Delta x}
    = \lim_{\Delta x \to 0} \frac{f(x_0 + \Delta x) - f(x_0)}{\Delta x}.
\]
由上式可得:
\[
    f(x_0 + \Delta x) - f(x_0)
    = f'(x_0)\Delta x + o(\Delta x), \quad \Delta x \to 0.
\]
其中 $f'(x_0)$ 为一次项系数,也称 \textbf{微分系数}.
它反映了因变量增量与自变量增量的一次项比值的极限.若该极限存在,则称 $f(x)$ 在 $x_0$ 处可导.

\section{$f(x)$ 与 $|f(x)|$ 的连续与可导关系}
\DTwoOne+\DFourThree

\begin{enumerate}
    \item 若 $f(x)$ 在 $x_0$ 处连续,则 $|f(x)|$ 在 $x_0$ 处亦连续;反之不成立.
    \item 若 $f(x)$ 在 $x_0$ 处可导,则:
          \begin{enumerate}
              \item 若 $f(x_0) \neq 0$,则 $|f(x)|$ 在 $x_0$ 处可导,且
                    \[
                        [|f(x)|]'_{x=x_0} =
                        \begin{cases}
                            f'(x_0),  & f(x_0) > 0, \\
                            -f'(x_0), & f(x_0) < 0.
                        \end{cases}
                    \]
              \item 若 $f(x_0) = 0$,则:
                    \[
                        \begin{cases}
                            f'(x_0) = 0 \Rightarrow |f(x)| \text{ 在 } x_0 \text{ 处可导,且 } [|f(x)|]'_{x=x_0}=0, \\[5pt]
                            f'(x_0) \ne 0 \Rightarrow |f(x)| \text{ 在 } x_0 \text{ 处不可导.}
                        \end{cases}
                    \]
          \end{enumerate}
\end{enumerate}

\section{导函数 $f'(x)$ 的性质总结}
\DTwoOne

\begin{enumerate}
    \item 关于 $f'(x)$ 的连续性:
          \begin{enumerate}
              \item 若导函数 $f'(x)$ 存在,则当 $f'(x)$ 在某点极限存在时,该点处导函数连续;
              \item 若导函数在某点存在,则该点 \textbf{不可能是第一类间断点};
              \item 若 $f(x)$ 可导,则 $f'(x)$ 可能连续,也可能含有振荡间断点.
          \end{enumerate}

    \item 关于 $\lim\limits_{x\to+\infty} f'(x)$:
          \begin{enumerate}
              \item 若 $\lim\limits_{x\to+\infty} f(x)$ 存在,$\lim\limits_{x\to+\infty} f'(x)$ 不一定存在.
                    例如:$f(x)=\dfrac{\sin x^2}{x}$,
                    则 $f'(x)=2\cos x^2 - \dfrac{\sin x^2}{x^2}$,其极限不存在.
              \item 若 $f(x)$ 在 $(0,+\infty)$ 可导,且 $y=f(x)$ 在 $x\to+\infty$ 时有斜渐近线,
                    $\lim\limits_{x\to+\infty} f'(x)$ 也不一定存在.
                    例如:$f(x)=x+\dfrac{\sin x^2}{x}$,
                    有斜渐近线 $y=x$,但 $f'(x)=1+2\cos x^2-\dfrac{\sin x^2}{x^2}$,其极限不存在.
          \end{enumerate}
\end{enumerate}

\section{函数在一点求导的注意事项}
\DTwoTwo
\begin{enumerate}
    \item \textbf{关于 $f'(x_0)$ 与 $f'(x)$ 的区别}
          \begin{enumerate}
              \item $f'(x_0)$ 表示函数在点 $x_0$ 处的导数;
              \item $f'(x)$ 表示用求导法则求得的导函数表达式;
              \item 若 $f'(x)$ 在 $x_0$ 处无定义,仅说明求导法则在此点不适用,
                    并不意味着 $f(x)$ 在 $x_0$ 处不可导,此时应当\textbf{使用定义求导}.
          \end{enumerate}

    \item \textbf{绝对值型函数的可导性}
          \begin{enumerate}
              \item 若 $F(x)=f(x)g(x)$,其中 $f(x)$ 在 $x_0$ 处连续但不可导,$g(x)$ 在 $x_0$ 处可导,
                    则
                    \[
                        F(x) \text{ 在 } x_0 \text{ 处可导 } \Longleftrightarrow g(x_0)=0.
                    \]
                    特别地,若 $F(x)=|x-x_0|g(x)$,且 $g(x)$ 在 $x_0$ 处可导,
                    则 $F(x)$ 在 $x_0$ 处可导 $\Longleftrightarrow g(x_0)=0$.
              \item 由 $|f(x)| = \sqrt{f^2(x)}$ 得
                    \[
                        [|f(x)|]' = [\sqrt{f^2(x)}]'
                        = \frac{1}{2\sqrt{f^2(x)}} \cdot 2f(x)f'(x)
                        = \frac{f(x)f'(x)}{|f(x)|}.
                    \]
          \end{enumerate}

    \item \textbf{分段函数在分段点处的可导性}

          \begin{example}{}{}
              下列函数中,在 $x=0$ 处不可导的是( ).

              \vspace{0.5em}
              \begin{tabular}{ll}
                  (A) & $f(x)=|x|\tan|x|$        \\[0.3em]
                  (B) & $f(x)=|x|\tan\sqrt{|x|}$ \\[0.3em]
                  (C) & $f(x)=\sqrt{\cos|x|}$    \\[0.3em]
                  (D) & $f(x)=\cos\sqrt{|x|}$
              \end{tabular}
          \end{example}

          \begin{solution}
              \begin{enumerate}
                  \item[(A)]
                        \[
                            f'(0)=\lim_{x\to0}\frac{f(x)-f(0)}{x}
                            =\lim_{x\to0}\frac{|x|\tan|x|}{x}
                            =\lim_{x\to0}\frac{x^2}{x}=0.
                        \]

                  \item[(B)]
                        \[
                            f'(0)=\lim_{x\to0}\frac{|x|\tan\sqrt{|x|}}{x}
                            =\lim_{x\to0}\left(\frac{|x|}{x}\cdot\tan\sqrt{|x|}\right)=0.
                        \]

                  \item[(C)]
                        \[
                            f'(0)=\lim_{x\to0}\frac{\sqrt{\cos|x|}-1}{x}
                            =\lim_{x\to0}\frac{\cos|x|-1}{x(\sqrt{\cos|x|}+1)}
                            =\lim_{x\to0}\frac{-\frac{1}{2}x^2}{2x}=0.
                        \]

                  \item[(D)]
                        \[
                            f'(0)=\lim_{x\to0}\frac{\cos\sqrt{|x|}-1}{x}
                            =\lim_{x\to0}\frac{-\frac{1}{2}|x|}{x},
                        \]
                        左、右极限符号相反,极限不存在,因此 $f'(0)$ 不存在.
              \end{enumerate}
              \textbf{故选:} (D).
          \end{solution}
\end{enumerate}
\GAOchapter{一元函数微分学的计算}
\GAOchapter{一元函数微分学的应用(一)——几何应用}

\section{切线、法线与截距}
\DTwoTwo

设 $y = f(x)$ 在 $x_0$ 处可导,切点为 $(x_0, y_0)$.

\begin{enumerate}
    \item \textbf{切线方程:}
          \[
              y - y_0 = f'(x_0)(x - x_0).
          \]
    \item \textbf{法线方程:}
          \[
              y - y_0 = -\frac{1}{f'(x_0)}(x - x_0).
          \]
    \item \textbf{各截距公式:}
          \[
              \begin{aligned}
                  \text{x轴切线截距:} & \quad x_0 - \frac{y_0}{f'(x_0)}, \\
                  \text{y轴切线截距:} & \quad y_0 - x_0 f'(x_0),         \\
                  \text{x轴法线截距:} & \quad x_0 + y_0 f'(x_0),         \\
                  \text{y轴法线截距:} & \quad y_0 + \frac{x_0}{f'(x_0)}.
              \end{aligned}
          \]
\end{enumerate}


\section{单调性、极值、凹凸性与拐点}
\DTwoTwo
\subsection{单调性判别}

设 $f(x)$ 在 $[a,b]$ 上连续、在 $(a,b)$ 内可导:
\begin{enumerate}
    \item 若 $f'(x) \ge 0$ 且仅在有限点取等号,则 $f(x)$ 在 $[a,b]$ 上\textbf{严格递增};
    \item 若 $f'(x) \le 0$ 且仅在有限点取等号,则 $f(x)$ 在 $[a,b]$ 上\textbf{严格递减}.
\end{enumerate}

\subsection{极值的定义与判别}

若存在 $x_0$ 的某邻域,使得
\[
    f(x) \le f(x_0) \ (\text{或 } f(x) \ge f(x_0)),
\]
则称 $x_0$ 为 $f(x)$ 的\textbf{极大值点}(或极小值点).

\textbf{常用判别:}
\begin{enumerate}
    \item 一阶导数法:$f'(x_0) = 0$;
    \item 二阶导数法:若 $f''(x_0) > 0$,则为极小值;若 $f''(x_0) < 0$,则为极大值.
\end{enumerate}

\subsection{凹凸性的定义与判别}

\begin{enumerate}
    \item 定义法:
          \[
              f\!\left(\frac{x_1+x_2}{2}\right)
              \begin{cases}
                  < \frac{f(x_1)+f(x_2)}{2}, & \text{凹函数;} \\[4pt]
                  > \frac{f(x_1)+f(x_2)}{2}, & \text{凸函数.}
              \end{cases}
          \]
    \item 导数法:
          若 $f''(x) > 0$,则函数在该区间上\textbf{凹向上(凸)};
          若 $f''(x) < 0$,则函数在该区间上\textbf{凹向下(凹)}.
\end{enumerate}

\subsection{拐点}

连续曲线的凹弧与凸弧的分界点称为\textbf{拐点}.

\textbf{判别条件:}
\[
    f''(x_0) = 0 \quad \text{且 } f''(x) \text{ 在 } x_0 \text{ 附近变号}.
\]

\subsection{重要结论总结}

\[
    \boxed{
        \begin{aligned}
            \text{有极值点}        & \Leftrightarrow f'(x) \text{ 有零点},     \\[3pt]
            \text{有拐点}         & \Leftrightarrow f''(x) \text{ 有零点且变号}, \\[3pt]
            f'(x) \text{ 无零点}  & \Rightarrow f(x) \text{ 单调性不变},        \\[3pt]
            f''(x) \text{ 无零点} & \Rightarrow f'(x) \text{ 单调性不变.}
        \end{aligned}
    }
\]

\section{渐近线}

\textbf{定义:} 曲线 $y=f(x)$ 的渐近线是指曲线无限接近的一条直线.

\begin{enumerate}
    \item \textbf{竖直渐近线:}
          若 $\displaystyle \lim_{x\to a^\pm}f(x)=\infty$,则 $x=a$ 为竖直渐近线.

    \item \textbf{水平渐近线:}
          若 $\displaystyle \lim_{x\to\infty}f(x)=A$ 或 $\lim_{x\to-\infty}f(x)=B$,
          则 $y=A$ 或 $y=B$ 为水平渐近线.

    \item \textbf{斜渐近线:}
          若 $\displaystyle \lim_{x\to\infty}\frac{f(x)}{x}=a$ 且 $\lim_{x\to\infty}[f(x)-ax]=b$,
          则 $y=ax+b$ 为斜渐近线.
\end{enumerate}

\section{最值与值域求法}
\DTwoThree
\begin{enumerate}
    \item 若 $f(x)$ 在 $[a,b]$ 上连续,则\textbf{最值只可能出现在:}
          \[
              \text{驻点、导数不存在点、区间端点}.
          \]
    \item 若 $f(x)$ 在 $(a,b)$ 内连续,且仅有一个极值点 $x_0$,
          则 $x_0$ 即为全区间的最值点.
    \item 若难以直接判断最值,可使用:
          \begin{enumerate}
              \item 平方和放缩法;
              \item 三角代换法;
              \item 单调性区间法;
              \item 取值范围不等式(如 $a^2+b^2\ge2ab$).
          \end{enumerate}
\end{enumerate}
\GAOchapter{一元函数微分学的应用(二)——中值定理、微分等式与微分不等式}
\GAOchapter{一元函数微分学的应用(三)——物理应用}

\section{寻找、建立相关变化率等式并求解}

\begin{enumerate}
    \item \textbf{运动学关系:}
          若质点的位移关于时间的函数为 $x=x(t)$,则:
          \[
              v = \frac{dx}{dt}, \quad
              a = \frac{dv}{dt} = \frac{dv}{dx}\frac{dx}{dt} = v\frac{dv}{dx}.
          \]
          (其中 $v$ 为速度,$a$ 为加速度.)

    \item \textbf{参数形式下的变化率关系:}
          若函数 $y=f(x)$ 由参数方程
          \[
              \begin{cases}
                  x = x(t), \\
                  y = y(t)
              \end{cases}
          \]
          所确定,且 $y$ 对 $t$ 的变化率与 $x$ 对 $t$ 的变化率成比例,
          即
          \[
              \frac{dy}{dt} = k a \frac{dx}{dt}, \quad (k \neq 0),
          \]
          则有
          \[
              \frac{dy}{dx} = k a.
          \]
          (常用于求相关变化率或速度比.)
\end{enumerate}

\section{根据题设写出物理量微元(微段常量化),建立等式并求解}

若题中给出“A 对时间 $t$ 的变化率与 $B$ 成正比”,则有:
\[
    \frac{dA}{dt} = kB, \quad (k \neq 0),
\]
由此建立微分方程并求解即可.

\subsection*{常见模型举例}
\begin{enumerate}
    \item \textbf{匀加速直线运动:}
          \[
              \frac{dv}{dt} = a_0 \Rightarrow v = a_0t + v_0, \quad x = \frac{1}{2}a_0t^2 + v_0t + x_0.
          \]
    \item \textbf{牛顿冷却定律:}
          物体温度 $T$ 的变化率与其与环境温度 $T_0$ 的温差成正比:
          \[
              \frac{dT}{dt} = -k(T - T_0),
          \]
          解得
          \[
              T = T_0 + Ce^{-kt}.
          \]
    \item \textbf{放射性衰变 / 电容放电:}
          数量 $N$ 的减少率与其本身成正比:
          \[
              \frac{dN}{dt} = -kN,
          \]
          解得
          \[
              N = N_0 e^{-kt}.
          \]
    \item \textbf{人口增长模型(逻辑斯蒂型):}
          若增长率受限于总容量 $M$,
          \[
              \frac{dN}{dt} = kN\left(1-\frac{N}{M}\right),
          \]
          解得
          \[
              N = \frac{M}{1+Ce^{-kt}}.
          \]
\end{enumerate}

\subsection*{小结:解题步骤}
\begin{enumerate}
    \item 根据题意确定变量间的依赖关系;
    \item 用比例、速率等物理语言写出微分方程;
    \item 若能分离变量,则两边积分;
    \item 结合初值(如 $t=0$ 时条件)求出常数;
    \item 给出解析式或函数关系.
\end{enumerate}
\section{一元函数积分学的概念与性质}
\GAOchapter{一元函数积分学的计算}

\section{恒等变形法}

通过代数变形,将被积函数化为基本积分公式中的形式,从而直接求得原函数。

常用方法:
\begin{enumerate}
    \item 配方、通分、分解因式;
    \item 利用奇偶性、对称性;
    \item 利用三角恒等变换或平方差公式;
    \item 拆分常数倍:$\displaystyle \int (a f(x) + b g(x))dx = a\int f(x)dx + b\int g(x)dx$。
\end{enumerate}

\section{第一类换元法(凑微分法)}

\[
    \int f[g(x)]g'(x)\,dx = \int f[g(x)]\,d[g(x)] = \int f(u)\,du.
\]

若被积函数是 $g(x)$ 的函数与 $g'(x)$ 的乘积,则“凑微分”后即可直接积分。

\textbf{技巧总结:}
\begin{enumerate}
    \item 常见凑法:若分母是 $x^2+1$、$1-x^2$、$x\sqrt{1-x^2}$ 等;
    \item 若被积函数可写成 $f'(x)/f(x)$,则 $\displaystyle \int \frac{f'(x)}{f(x)}dx = \ln|f(x)| + C$。
\end{enumerate}

\section{第二类换元法(代换法)}

\[
    \int f(x)\,dx = \int f[g(u)]g'(u)\,du.
\]

当被积函数复杂、含根号或分式时,可令 $x=g(u)$,使 $f(x)$ 化为较简单的 $h(u)$。

\textbf{常见换元:}
\begin{enumerate}
    \item 去根号:如 $x=a\sin\theta,\ a\tan\theta,\ a\sec\theta$;
    \item 分母变单项:如 $\sqrt[n]{\frac{ax+b}{cx+d}} = u$;
    \item 含对称结构:如 $x+\frac{1}{x}=t$;
    \item 对称区间换元:$x=a-t$、$x=\frac{1}{t}$ 等。
\end{enumerate}

\textbf{核心目标:} 通过换元使被积函数\textbf{简化为多项式或基本初等函数形式}。

\section{分部积分法}

\[
    \int u\,dv = uv - \int v\,du.
\]

\textbf{口诀:} “一降一不变,优先降幂项”。

\begin{enumerate}
    \item 若为 $\int f(x)g(x)\,dx$,选取 $u$ 为\textbf{易微分、难积分}的部分;
    \item 若为单个函数(如 $x e^x,\, x\sin x$ 等),仍可设 $dv$ 为整项的可积分部分;
    \item 常见四类应用:
          \begin{enumerate}
              \item \textbf{再现法:} 积分两边含原式,建立方程;
              \item \textbf{抵消法:} 通过两次积分消除复杂项;
              \item \textbf{递推法:} 构造 $I_n$ 递推公式;
              \item \textbf{定积分结合法:} 特殊上限下限抵消项。
          \end{enumerate}
\end{enumerate}

\section{有理函数的积分}

形如 $\displaystyle \int \frac{P_n(x)}{Q_m(x)}dx$($n<m$)的积分称为有理函数积分。

\textbf{常用步骤:}
若 $n\ge m$,先作多项式除法;若 $n<m$,再作部分分式分解。

\textbf{四类基本积分:}
\begin{enumerate}
    \item $\displaystyle \int\frac{A}{ax+b}\,dx = \frac{A}{a}\ln|ax+b| + C$;
    \item $\displaystyle \int\frac{A}{(ax+b)^k}\,dx = \frac{A}{a(1-k)}(ax+b)^{1-k} + C \quad (k>0,\ k\ne1)$;
    \item $\displaystyle \int\frac{Bx+C}{px^2+qx+r}\,dx\ (q^2-4pr<0)$:
          \[
              \text{如 } \int\frac{x+1}{x^2+x+1}dx = \frac{1}{2}\int\frac{2x+1}{x^2+x+1}dx + \frac{1}{2}\int\frac{dx}{\left(x+\frac{1}{2}\right)^2+\left(\frac{\sqrt{3}}{2}\right)^2}.
          \]
    \item $\displaystyle \int\frac{Bx+C}{(px^2+qx+r)^k}\,dx\ (q^2-4pr<0,k>0,k\ne1)$:
          \[
              \text{如 } \int\frac{x+1}{(x^2+x+1)^2}dx = \frac{1}{2}\int\frac{2x+1}{(x^2+x+1)^2}dx + \frac{1}{2}\int\frac{dx}{\left[\left(x+\frac{1}{2}\right)^2+\left(\frac{\sqrt{3}}{2}\right)^2\right]^2}.
          \]
\end{enumerate}

\section{三角有理式的积分}

形如 $\displaystyle \int R(\sin x, \cos x)\,dx$ 的积分称为三角有理式积分。

\subsection{全角换元法}
\begin{enumerate}
    \item 若 $R(-\sin x, \cos x) = -R(\sin x, \cos x)$,令 $t = \cos x$;
    \item 若 $R(\sin x, -\cos x) = -R(\sin x, \cos x)$,令 $t = \sin x$。
\end{enumerate}

\subsection{半角万能换元法}
设 $t=\tan\frac{x}{2}$,则
\[
    \sin x = \frac{2t}{1+t^2}, \quad \cos x = \frac{1-t^2}{1+t^2}, \quad dx = \frac{2\,dt}{1+t^2}.
\]
代入后化为 $t$ 的有理函数积分。

\section{定积分的计算(牛顿–莱布尼茨公式)}

若 $F'(x)=f(x)$,则:
\[
    \int_a^b f(x)\,dx = F(b)-F(a).
\]

\subsection{反常积分的牛顿–莱布尼茨公式}
\begin{enumerate}
    \item $\displaystyle \int_a^{+\infty} f(x)\,dx = \lim_{x\to+\infty}F(x)-F(a)$;
    \item $\displaystyle \int_{-\infty}^b f(x)\,dx = F(b)-\lim_{x\to-\infty}F(x)$;
    \item 若 $x=c$ 为瑕点,则
          \[
              \int_a^b f(x)\,dx = \int_a^c f(x)\,dx + \int_c^b f(x)\,dx.
          \]
\end{enumerate}

\section{变限积分函数的求导}

\begin{enumerate}
    \item \textbf{直接求导型:}
          \[
              \left[\int_{a}^{\varphi(x)} f(t)\,dt\right]' = f[\varphi(x)]\varphi'(x),\quad
              \left[\int_{\varphi_1(x)}^{\varphi_2(x)} f(t)\,dt\right]' = f[\varphi_2(x)]\varphi_2'(x) - f[\varphi_1(x)]\varphi_1'(x).
          \]
    \item \textbf{换元求导型:} 先换元,再代入求导。
    \item \textbf{拆分求导型:} 若带绝对值或分段区间,需拆分处理。
    \item \textbf{换序积分型:} 累次积分中先交换积分次序以简化计算。
\end{enumerate}

\section{分段函数的积分}

\begin{enumerate}
    \item \textbf{不定积分:} 各段分别求原函数,检查分段点连续性;
    \item \textbf{定积分:} 按区间分段后相加;
    \item \textbf{变限积分:} 随 $x$ 变化分情况讨论,$F(x)=\int_a^x f(t)dt$ 为累加函数。
\end{enumerate}

\section{几何法}

定积分具有面积意义,可快速计算几何型积分:
\[
    \int_{-a}^{a}\sqrt{a^2-x^2}\,dx = \frac{\pi a^2}{2}, \quad
    \int_{0}^{a}\sqrt{x(2a-x)}\,dx = \frac{\pi a^2}{4}.
\]
由此还可得:
\[
    \int_{0}^{2a}\sqrt{x(2a-x)}\,dx = \frac{\pi a^2}{2}, \quad
    \int_{a}^{2a}\sqrt{x(2a-x)}\,dx = \frac{\pi a^2}{4}.
\]

\textbf{应用提示:} 若被积函数可化为圆弧、椭圆或抛物线截面形式,可直接用面积公式代替积分计算。
\GAOchapter{一元函数积分学的应用(一)——几何应用}
\section{计算公式}

\begin{enumerate}
    \item \textbf{面积公式}
          \begin{enumerate}
              \item 直角坐标系下:
                    $$S = \int_{a}^{b} |f_{1}(x) - f_{2}(x)| \, \mathrm{d}x.$$
              \item 极坐标系下:
                    $$S = \frac{1}{2} \int_{\alpha}^{\beta} \left| r_{2}^{2}(\theta) - r_{1}^{2}(\theta) \right| \, \mathrm{d}\theta.$$
              \item 参数方程形式:
                    若曲线由
                    $$\begin{cases}
                            x = x(t), \\
                            y = y(t)
                        \end{cases}
                        \quad (\alpha \leqslant t \leqslant \beta)
                    $$
                    给出,则曲边梯形的面积为
                    $$
                        S = \int_{a}^{b} |y| \, \mathrm{d}x
                        = \int_{\alpha}^{\beta} |y(t) x'(t)| \, \mathrm{d}t.
                    $$
          \end{enumerate}

    \item \textbf{旋转体体积}
          \begin{enumerate}
              \item 绕 $x$ 轴旋转:
                    $$
                        V = \pi \int_{a}^{b} [f(x)]^{2} \, \mathrm{d}x.
                    $$
              \item 绕 $y$ 轴旋转:
                    $$
                        V = 2\pi \int_{a}^{b} x |f(x)| \, \mathrm{d}x.
                    $$
              \item 绕一般直线 $L_{0}$ 旋转:
                    若曲线 $L: y = f(x)$,定直线 $L_{0}: Ax + By + C = 0$,
                    且 $L_{0}$ 的任意垂线与 $L$ 至多有一个交点,则
                    $$
                        V = \frac{\pi}{(A^{2} + B^{2})^{3/2}}
                        \int_{a}^{b} [A x + B f(x) + C]^{2} |A f'(x) - B| \, \mathrm{d}x.
                    $$
                    特别地,若 $A = C = 0, B \neq 0$,则 $L_{0}$ 为 $x$ 轴,有
                    $$
                        V = \pi \int_{a}^{b} [f(x)]^{2} \, \mathrm{d}x.
                    $$
              \item 极坐标下:
                    若区域
                    $D = \{ (r, \theta) \mid 0 \le r \le r(\theta),\, \theta \in [\alpha, \beta] \subset [0, \pi] \}$,
                    则绕极轴旋转一周的体积为
                    $$
                        V = \frac{2}{3}\pi \int_{\alpha}^{\beta} [r(\theta)]^{3} \sin\theta \, \mathrm{d}\theta.
                    $$
          \end{enumerate}

    \item \textbf{平均值公式}

          若 $f(x)$ 在区间 $[a, b]$ 上连续,则其平均值为
          $$
              \overline{f} = \frac{1}{b - a} \int_{a}^{b} f(x) \, \mathrm{d}x.
          $$

    \item \textbf{平面曲线的弧长}
          \begin{enumerate}
              \item 直角坐标方程 $y = y(x)$:
                    $$
                        s = \int_{a}^{b} \sqrt{1 + [y'(x)]^{2}} \, \mathrm{d}x.
                    $$
              \item 极坐标方程 $r = r(\theta)$:
                    $$
                        s = \int_{\alpha}^{\beta} \sqrt{[r(\theta)]^{2} + [r'(\theta)]^{2}} \, \mathrm{d}\theta.
                    $$
              \item 参数方程 $\begin{cases} x = x(t), \\ y = y(t) \end{cases}$:
                    $$
                        s = \int_{\alpha}^{\beta} \sqrt{[x'(t)]^{2} + [y'(t)]^{2}} \, \mathrm{d}t.
                    $$
          \end{enumerate}

    \item \textbf{旋转曲面的面积(侧面积)}
          \begin{enumerate}
              \item 绕 $x$ 轴旋转:
                    $$
                        S = 2\pi \int_{a}^{b} |y(x)| \sqrt{1 + [y'(x)]^{2}} \, \mathrm{d}x.
                    $$
              \item 极坐标形式:
                    $$
                        S = 2\pi \int_{\alpha}^{\beta} |r(\theta)\sin\theta| \sqrt{[r(\theta)]^{2} + [r'(\theta)]^{2}} \, \mathrm{d}\theta.
                    $$
              \item 参数方程形式:
                    若
                    $$\begin{cases}
                            x = x(t), \\
                            y = y(t)
                        \end{cases}, \quad (\alpha \le t \le \beta,\, x'(t) \neq 0),
                    $$
                    则绕 $x$ 轴旋转的曲面面积为
                    $$
                        S = 2\pi \int_{\alpha}^{\beta} |y(t)| \sqrt{[x'(t)]^{2} + [y'(t)]^{2}} \, \mathrm{d}t.
                    $$
          \end{enumerate}
\end{enumerate}

\section{常见极坐标曲线及其面积、体积、弧长汇总表}
\begin{small}
    \begin{longtable}{|c|c|c|c|c|}
        \hline
        \textbf{名称}                                                              & \textbf{表达式}                     & \textbf{面积 $S$} & \textbf{绕极轴体积 $V$} & \textbf{弧长 $L$} \\ \hline
        \endfirsthead
        \hline
        \textbf{名称}                                                              & \textbf{表达式}                     & \textbf{面积 $S$} & \textbf{绕极轴体积 $V$} & \textbf{弧长 $L$} \\ \hline
        \endhead

        心形线                                                                      & $r = a(1 + \cos\theta)$          &
        $\displaystyle S = \frac{3}{2}\pi a^2$                                   &
        $\displaystyle V = \frac{5}{2}\pi^2 a^3$                                 &
        $\displaystyle L = 8a$                                                                                                                                               \\ \hline

        内缩心形线                                                                    & $r = a(1 - \cos\theta)$          &
        $\displaystyle S = \frac{3}{2}\pi a^2$                                   &
        $\displaystyle V = \frac{5}{2}\pi^2 a^3$                                 &
        $\displaystyle L = 8a$                                                                                                                                               \\ \hline

        竖直心形线                                                                    & $r = a(1 + \sin\theta)$          &
        $\displaystyle S = \frac{3}{2}\pi a^2$                                   &
        $\displaystyle V = \frac{5}{2}\pi^2 a^3$                                 &
        $\displaystyle L = 8a$                                                                                                                                               \\ \hline

        双纽线($\cos2\theta$型)                                                      & $r^2 = a^2 \cos2\theta$          &
        $\displaystyle S = a^2$                                                  &
        $\displaystyle V = \frac{4}{3}\pi a^3$                                   &
        $\displaystyle L = 4aE\!\left(\frac{1}{\sqrt2}\right)$                                                                                                               \\ \hline

        双纽线($\sin2\theta$型)                                                      & $r^2 = a^2 \sin2\theta$          &
        $\displaystyle S = a^2$                                                  &
        $\displaystyle V = \frac{4}{3}\pi a^3$                                   &
        $\displaystyle L = 4aE\!\left(\frac{1}{\sqrt2}\right)$                                                                                                               \\ \hline

        三叶玫瑰线                                                                    & $r = a\sin3\theta$               &
        $\displaystyle S = \frac{3}{4}\pi a^2$                                   &
        $\displaystyle V = \frac{3}{8}\pi^2 a^3$                                 &
        $\displaystyle L = 12aE\!\left(\frac{\sqrt{3}}{2}\right)$                                                                                                            \\ \hline

        四叶玫瑰线                                                                    & $r = a\cos2\theta$               &
        $\displaystyle S = \pi a^2$                                              &
        $\displaystyle V = \frac{4}{3}\pi^2 a^3$                                 &
        $\displaystyle L = 8aE\!\left(\frac{1}{\sqrt2}\right)$                                                                                                               \\ \hline

        阿基米德螺线                                                                   & $r = a\theta$                    &
        $\displaystyle S = \tfrac{1}{6}a^2(\theta_2^3 - \theta_1^3)$             &
        $\displaystyle V = \tfrac{\pi a^3}{2}(\theta_2^4 - \theta_1^4)$          &
        $\displaystyle L = \tfrac{a}{3}\!\left[(\theta^2+1)^{3/2}-1\right]$                                                                                                  \\ \hline

        对数螺线                                                                     & $r = ae^{b\theta}$               &
        $\displaystyle S = \tfrac{a^2}{4b}(e^{2b\theta_2} - e^{2b\theta_1})$     &
        $\displaystyle V = \tfrac{\pi a^3}{6b}(e^{3b\theta_2} - e^{3b\theta_1})$ &
        $\displaystyle L = \tfrac{a}{2b}\sqrt{1+b^2}(e^{b\theta_2} - e^{b\theta_1})$                                                                                         \\ \hline

        半圆                                                                       & $r = 2a\sin\theta$               &
        $\displaystyle S = \tfrac{1}{2}\pi a^2$                                  &
        $\displaystyle V = \tfrac{2}{3}\pi^2 a^3$                                &
        $\displaystyle L = \pi a$                                                                                                                                            \\ \hline

        星形线                                                                      & $r = a(1 - \sin\theta)$          &
        $\displaystyle S = \tfrac{3}{2}\pi a^2$                                  &
        $\displaystyle V = \tfrac{5}{2}\pi^2 a^3$                                &
        $\displaystyle L = 8a$                                                                                                                                               \\ \hline

        阿斯特罗伊德                                                                   & $r = a\sqrt{\cos2\theta}$        &
        $\displaystyle S = \tfrac{3}{8}\pi a^2$                                  &
        $\displaystyle V = \tfrac{3}{8}\pi^2 a^3$                                &
        $\displaystyle L = 6a$                                                                                                                                               \\ \hline

        伯努利双纽线                                                                   & $r^2(r^2 - a^2) = b^2 a^2$       &
        $\displaystyle S = \tfrac{1}{2}a^2\sin2\theta$                           &
        $\displaystyle V = \tfrac{\pi a^3}{3}(1 - \cos^3\theta)$                 &
        $\displaystyle L = \int \!\sqrt{r^2 + (r')^2}\, d\theta$                                                                                                             \\ \hline

        圆                                                                        & $r = a$                          &
        $\displaystyle S = \pi a^2$                                              &
        $\displaystyle V = \tfrac{4}{3}\pi^2 a^3$                                &
        $\displaystyle L = 2\pi a$                                                                                                                                           \\ \hline

        椭圆                                                                       & $\dfrac{r}{1 - e\cos\theta} = p$ &
        $\displaystyle S = \pi ab$                                               &
        $\displaystyle V = \tfrac{4}{3}\pi a^2 b$                                &
        $\displaystyle L \approx \pi[3(a+b) - \sqrt{(3a+b)(a+3b)}]$                                                                                                          \\ \hline
    \end{longtable}
\end{small}
\section{各种函数表达形式的几何量计算}

本节总结典型函数形式在求面积与弧长中的常见方法。总体思路:
\begin{itemize}
    \item 先确定函数定义区间;
    \item 求导得到 $y'$;
    \item 再根据几何量定义公式求解相应积分。
\end{itemize}

\subsection{幂函数类表达式的几何量}
\begin{example}{}{}
    设
    $$y = \lim_{n \to \infty} \frac{1 + x}{1 + n x^{2n}},$$
    求曲线 $y = y(x)$ 与 $x$ 轴及 $x = 1$ 所围成图形的面积。
\end{example}

\begin{solution}
    当 $|x| < 1$ 时,$x^{2n} \to 0$,故
    $$\frac{1 + x}{1 + n x^{2n}} \to 1 + x.$$
    当 $|x| \ge 1$ 时,$n x^{2n} \to +\infty$,故
    $$\frac{1 + x}{1 + n x^{2n}} \to 0.$$
    于是:
    $$
        y =
        \begin{cases}
            0,     & |x| \ge 1, \\[3pt]
            1 + x, & |x| < 1.
        \end{cases}
    $$
    由此可得面积:
    $$
        S = \int_{-1}^{1} (1 + x)\, \mathrm{d}x
        = \left(x + \tfrac{x^2}{2}\right)\Big|_{-1}^{1} = 2.
    $$
\end{solution}

\subsection{三角函数类表达式的几何量}
\begin{example}{}{}
    求曲线
    $$y = \int_{0}^{x} \sqrt{\cos t}\, \mathrm{d}t$$
    的全弧长。
\end{example}

\begin{solution}
    由 $y' = \sqrt{\cos x}$,且要求 $\cos x \ge 0$,即 $x \in \left[-\tfrac{\pi}{2}, \tfrac{\pi}{2}\right]$。
    则有:
    $$
        \mathrm{d}s = \sqrt{1 + (y')^2}\, \mathrm{d}x
        = \sqrt{1 + \cos x}\, \mathrm{d}x
        = \sqrt{2}\cos \frac{x}{2}\, \mathrm{d}x.
    $$
    因此:
    $$
        s = \int_{-\frac{\pi}{2}}^{\frac{\pi}{2}} \sqrt{1 + \cos x}\, \mathrm{d}x
        = 2\sqrt{2} \int_{0}^{\frac{\pi}{2}} \cos \frac{x}{2}\, \mathrm{d}x
        = 4.
    $$
\end{solution}

\subsection{对数函数类表达式的几何量}
\begin{example}{}{}
    求下列曲线的弧长:
    \begin{enumerate}
        \item $y = \ln(\cos x),\quad 0 \le x \le \tfrac{\pi}{6}$;
        \item $\ln y + 2x - \tfrac{1}{2}y^2 = 0,\quad 1 \le y \le e$。
    \end{enumerate}
\end{example}

\begin{solution}
    \begin{enumerate}
        \item 由 $y' = -\tan x$,则
              $$
                  s = \int_{0}^{\frac{\pi}{6}} \sqrt{1 + y'^2}\, \mathrm{d}x
                  = \int_{0}^{\frac{\pi}{6}} \sec x\, \mathrm{d}x
                  = \ln|\sec x + \tan x|\Big|_{0}^{\frac{\pi}{6}}
                  = \tfrac{1}{2}\ln 3.
              $$
        \item 将方程化为 $x = \tfrac{1}{4}y^2 - \tfrac{1}{2}\ln y$,则
              $$
                  x_y = \frac{1}{2}y - \frac{1}{2y}, \quad
                  s = \int_{1}^{e} \sqrt{1 + x_y^2}\, \mathrm{d}y
                  = \int_{1}^{e} \frac{1}{2}\!\left(y + \frac{1}{y}\right)\mathrm{d}y
                  = \frac{1}{4}(e^2 + 1).
              $$
    \end{enumerate}
\end{solution}

\vspace{1em}
\noindent\textbf{小结:}
\begin{itemize}
    \item 幂函数型通常需判断极限定义域;
    \item 三角函数型常通过半角公式简化;
    \item 对数函数型适合换元或视$y$为自变量积分。
\end{itemize}
\GAOchapter{一元函数积分学的应用(二)——积分等式与积分不等式}
\GAOchapter{一元函数积分学的应用(三)——物理应用}
\GAOchapter{多元函数微分学}
\GAOchapter{二重积分}

\section{和式极限与二重积分的定义}

设 $D = \{(x, y) \mid a \le x \le b,\, c \le y \le d\}$,若极限存在,则称为 $f(x,y)$ 在 $D$ 上的二重积分:
\[
    \iint_{D} f(x, y) \, d\sigma
    = \lim_{n \to \infty} \sum_{i=1}^{n} \sum_{j=1}^{n}
    f\!\left(a + \frac{b-a}{n}i,\; c + \frac{d-c}{n}j\right)
    \cdot \frac{b-a}{n} \cdot \frac{d-c}{n}.
\]

\section{交换积分次序的技巧与必要性}

\begin{itemize}
    \item 若题中给出累次积分:
          \[
              \int_{a}^{b} \left[\int_{g_1(x)}^{g_2(x)} f(x,y)\,dy\right] dx,
          \]
          当 \textbf{内层积分} 属于以下类型时,应优先考虑交换积分次序:
          \begin{enumerate}
              \item 可积但不可求积型;
              \item 计算困难型.
          \end{enumerate}

    \item 常见“可积不可求积”形式:
          \[
              \int \frac{\sin x}{x}\,dx, \quad
              \int \frac{\cos x}{x}\,dx, \quad
              \int \frac{1}{\ln x}\,dx, \quad
              \int \sin x^2\,dx, \quad
              \int e^{x^2}\,dx, \quad
              \int \frac{e^x}{x}\,dx, \quad
              \int \frac{\tan x}{x}\,dx.
          \]
          见到这些形式的被积函数时,一般都是\textbf{需要交换积分次序}的信号.
\end{itemize}

\section{积分的保号性}

\begin{enumerate}
    \item 若连续函数 $f(x,y) \ge 0$ 且不恒为 $0$,则
          \[
              \iint\limits_D f(x,y)\, d\sigma > 0.
          \]
    \item 若连续函数 $f(x,y)$ 满足:
          \[
              \forall D, \quad \iint\limits_D f(x,y)\, d\sigma = 0,
          \]
          则必有 $f(x,y) \equiv 0$.
\end{enumerate}

\section{对称性法则}

\subsection{普通对称性}
设 $D$ 与 $f(x,y)$ 同时具备一定对称性,则可大幅简化计算:

\begin{enumerate}
    \item 关于 $y$ 轴对称:
          \[
              \iint\limits_D f(x,y)\,d\sigma =
              \begin{cases}
                  2\iint\limits_{D_1} f(x,y)\,d\sigma, & f(x,y)=f(-x,y),  \\[4pt]
                  0,                                   & f(x,y)=-f(-x,y),
              \end{cases}
          \]
          其中 $D_1$ 为 $D$ 的右半部分.
    \item 关于 $x$ 轴对称:
          \[
              \iint\limits_D f(x,y)\,d\sigma =
              \begin{cases}
                  2\iint\limits_{D_1} f(x,y)\,d\sigma, & f(x,y)=f(x,-y),  \\[4pt]
                  0,                                   & f(x,y)=-f(x,-y).
              \end{cases}
          \]
    \item 关于原点对称:
          \[
              \iint\limits_D f(x,y)\,d\sigma =
              \begin{cases}
                  2\iint\limits_{D_1} f(x,y)\,d\sigma, & f(x,y)=f(-x,-y),  \\[4pt]
                  0,                                   & f(x,y)=-f(-x,-y).
              \end{cases}
          \]
    \item 关于直线 $y = x$ 对称:
          \[
              \iint\limits_D f(x,y)\,d\sigma =
              \begin{cases}
                  2\iint\limits_{D_1} f(x,y)\,d\sigma, & f(x,y)=f(y,x),  \\[4pt]
                  0,                                   & f(x,y)=-f(y,x).
              \end{cases}
          \]
\end{enumerate}

\subsection{轮换对称性}
若交换 $x$ 与 $y$ 后区域 $D$ 不变(即关于 $y=x$ 对称),则:
\[
    \iint\limits_D f(x,y)\,d\sigma = \iint\limits_D f(y,x)\,d\sigma.
\]

\section{二重积分常用结论(单位圆区域)}

\[
    \begin{aligned}
         & \iint_{x^2+y^2\le1}(x^2+y^2)\,d\sigma = \frac{\pi}{2}, \qquad
        \iint_{x^2+y^2\le1}\sqrt{x^2+y^2}\,d\sigma = \frac{2\pi}{3},                  \\[6pt]
         & \iint_{x^2+y^2\le1}\sqrt{1-(x^2+y^2)}\,d\sigma = \frac{2\pi}{3}, \qquad
        \iint_{x^2+y^2\le1}\!\!\bigl(1-\sqrt{x^2+y^2}\bigr)\,d\sigma = \frac{\pi}{3}, \\[6pt]
         & \iint_{x^2+y^2\le1}\!\!\left(\frac{x^2}{a^2}+\frac{y^2}{b^2}\right)d\sigma
        = \frac{\pi}{4}\!\left(\frac{1}{a^2}+\frac{1}{b^2}\right).
    \end{aligned}
\]


\section{二重积分的计算方法}

\subsection{直角坐标法}
\begin{enumerate}
    \item $X$ 型区域:
          \[
              \iint\limits_D f(x,y)\,d\sigma = \int_a^b \!dx \int_{\varphi_1(x)}^{\varphi_2(x)}\! f(x,y)\,dy.
          \]
    \item $Y$ 型区域:
          \[
              \iint\limits_D f(x,y)\,d\sigma = \int_c^d \!dy \int_{\psi_1(y)}^{\psi_2(y)}\! f(x,y)\,dx.
          \]
\end{enumerate}

\subsection{极坐标法}

设
\[
    \begin{cases}
        x = r\cos\theta, \\
        y = r\sin\theta,
    \end{cases}
    \quad d\sigma = r\,dr\,d\theta,
\]
则
\[
    \iint\limits_D f(x,y)\,d\sigma
    = \int_{\alpha}^{\beta}\!d\theta \int_{r_1(\theta)}^{r_2(\theta)}\! f(r\cos\theta, r\sin\theta)\,r\,dr.
\]
常见情况:
\begin{itemize}
    \item 极点 $O$ 在区域外;
    \item 极点在边界上;
    \item 极点在区域内部(常为圆或扇形区域).
\end{itemize}

\subsection{二重积分的换元法}

\begin{enumerate}
    \item \textbf{一元换元回顾:}
          \[
              \int_a^b f(x)\,dx = \int_{\alpha}^{\beta} f[\varphi(t)]\,\varphi'(t)\,dt.
          \]
    \item \textbf{二重积分换元:}
          \[
              \iint_{D_{xy}} f(x,y)\,dx\,dy
              = \iint_{D_{uv}} f[x(u,v),y(u,v)]\,
              \left|\frac{\partial(x,y)}{\partial(u,v)}\right| du\,dv,
          \]
          其中
          \[
              \frac{\partial(x,y)}{\partial(u,v)} =
              \begin{vmatrix}
                  \dfrac{\partial x}{\partial u} & \dfrac{\partial x}{\partial v} \\[4pt]
                  \dfrac{\partial y}{\partial u} & \dfrac{\partial y}{\partial v}
              \end{vmatrix} \ne 0.
          \]

    \item \textbf{极坐标换元:}
          \[
              \begin{cases}
                  x = r\cos\theta, \\
                  y = r\sin\theta,
              \end{cases}
              \quad
              \left|\dfrac{\partial(x,y)}{\partial(r,\theta)}\right|
              = r.
          \]
          因此:
          \[
              \iint_{D_{xy}} f(x,y)\,dx\,dy
              = \iint_{D_{r\theta}} f(r\cos\theta, r\sin\theta)\,r\,dr\,d\theta.
          \]
\end{enumerate}
\GAOchapter{微分方程}
\section{求解微分方程并研究解的性质}
\subsection{一阶微分方程的求解}
\DTwoThree

若题目中出现 $y'$ 或 $dy=\cdots dx$,则通常属于以下类型:

\begin{enumerate}
    % ========== 可分离变量型 ==========
    \item \textbf{可分离变量型}(或可换元化为此形式)
          \begin{enumerate}
              \item 若可写为 $y' = f(x)g(y)$,
                    则分离变量得:
                    \[
                        \frac{dy}{g(y)} = f(x)\,dx,
                        \qquad
                        \int \frac{dy}{g(y)} = \int f(x)\,dx.
                    \]

              \item 若可写为 $y' = f(ax+by+c)$,
                    令 $u=ax+by+c$,则 $u' = a + b f(u)$,
                    于是:
                    \[
                        \frac{du}{a + b f(u)} = dx,
                        \qquad
                        \int \frac{du}{a + b f(u)} = \int dx.
                    \]
          \end{enumerate}

          % ========== 齐次型 ==========
    \item \textbf{齐次型}(或可换元化为此形式)
          \begin{enumerate}
              \item 若 $y' = f\!\left(\frac{y}{x}\right)$,
                    令 $\displaystyle \frac{y}{x} = u$,即 $y=ux,\ y' = u + x u'$,
                    代入得:
                    \[
                        x\,\frac{du}{dx} + u = f(u)
                        \quad\Rightarrow\quad
                        \frac{du}{f(u)-u} = \frac{dx}{x}.
                    \]

              \item 若 $\displaystyle \frac{1}{y'} = f\!\left(\frac{x}{y}\right)$,
                    令 $\displaystyle \frac{x}{y} = u$,即 $x=uy,\ x' = u + y u'$,
                    代入得:
                    \[
                        y\,\frac{du}{dy} + u = f(u)
                        \quad\Rightarrow\quad
                        \frac{du}{f(u)-u} = \frac{dy}{y}.
                    \]

              \item 若 $y' = f\!\left(\frac{ax+by+c}{a_1x+b_1y+c_1}\right)$,则:
                    \begin{enumerate}
                        \item 若 $c^2 + c_1^2 = 0$,可化为 $y' = g\!\left(\frac{y}{x}\right)$;
                        \item 若 $c^2 + c_1^2 \neq 0$ 且 $\frac{a}{a_1}=\frac{b}{b_1}$,
                              可化为 $y' = g(ax+by)$;
                        \item 若 $c^2 + c_1^2 \neq 0$ 且 $\frac{a}{a_1}\neq\frac{b}{b_1}$,
                              由
                              \[
                                  \begin{cases}
                                      ax+by+c=0, \\
                                      a_1x+b_1y+c_1=0
                                  \end{cases}
                                  \quad\Rightarrow\quad (x_0,y_0)
                              \]
                              令
                              \[
                                  \begin{cases}
                                      x = X + x_0, \\
                                      y = Y + y_0,
                                  \end{cases}
                              \]
                              则方程化为
                              \[
                                  y' = f\!\left(\frac{aX+bY}{a_1X+b_1Y}\right)
                                  \Rightarrow
                                  \frac{dY}{dX} = g\!\left(\frac{Y}{X}\right).
                              \]
                    \end{enumerate}
          \end{enumerate}

          % ========== 一阶线性型 ==========
    \item \textbf{一阶线性型}(或可换元化为此形式)

          若 $y' + p(x)y = q(x)$,
          则通解为:
          \[
              y = e^{-\int p(x)\,dx}
              \left[
                  \int e^{\int p(x)\,dx}\, q(x)\,dx + C
                  \right].
          \]

          % ========== 伯努利型 ==========
    \item \textbf{伯努利方程}

          若 $y' + p(x)y = q(x) y^n \ (n\neq0,1)$,
          变形、换元如下:
          \[
              y^{-n} y' + p(x) y^{1-n} = q(x),
              \qquad
              \text{令 } z = y^{1-n},
          \]
          则
          \[
              \frac{1}{1-n}\frac{dz}{dx} + p(x)z = q(x),
          \]
          为一阶线性方程,可按前述方法求解.
\end{enumerate}

\subsection{二阶可降阶微分方程的求解}
\DTwoThree

若二阶方程中缺少某个变量($x$ 或 $y$),可通过设 $y' = p$ 将其降为一阶方程.

\begin{enumerate}
    % ============================================================
    \item \textbf{缺 $y$ 的情形}:
          方程可写成
          \[
              y'' = f(x, y') \quad \text{或} \quad y'' = f(y').
          \]
          \begin{enumerate}[label=(\roman*)]
              \item 令 $y' = p$,则 $y'' = p' = \dfrac{dp}{dx}$,
                    方程化为一阶方程
                    \[
                        \frac{dp}{dx} = f(x, p)
                        \quad \text{或} \quad
                        \frac{dp}{dx} = f(p).
                    \]
              \item 若其通解为 $p = \varphi(x, C_1)$,即
                    \[
                        y' = \varphi(x, C_1),
                    \]
                    再对 $x$ 积分得:
                    \[
                        \boxed{
                            y = \int \varphi(x, C_1)\,dx + C_2.
                        }
                    \]
          \end{enumerate}

          % ============================================================
    \item \textbf{缺 $x$ 的情形}:
          方程可写成
          \[
              y'' = f(y, y').
          \]
          \begin{enumerate}[label=(\roman*)]
              \item 令 $y' = p$,则
                    \[
                        y'' = \frac{dp}{dx}
                        = \frac{dp}{dy} \cdot \frac{dy}{dx}
                        = p\,\frac{dp}{dy}.
                    \]
                    代入原方程得一阶方程:
                    \[
                        p\,\frac{dp}{dy} = f(y, p).
                    \]
              \item 若其通解为 $p = \varphi(y, C_1)$,
                    则由 $p = \dfrac{dy}{dx}$ 得
                    \[
                        \frac{dy}{dx} = \varphi(y, C_1),
                        \qquad
                        \text{即 } \frac{dy}{\varphi(y, C_1)} = dx.
                    \]
              \item 两边积分得:
                    \[
                        \boxed{
                            \int \frac{dy}{\varphi(y, C_1)} = x + C_2,
                        }
                    \]
                    即为原方程的通解.
          \end{enumerate}
\end{enumerate}
\subsection{高阶常系数线性微分方程的求解}
\DTwoThree

一般形式为
\[
    y'' + p y' + q y = f(x),
\]
其中 $p,q$ 为常数,$f(x)$ 为已知函数。

\subsubsection*{微分算子法}

约定:
\[
    D=\frac{d}{dx},\quad
    Dy=\frac{dy}{dx},\quad
    D^2y=\frac{d^2y}{dx^2}.
\]
于是微分方程可写为
\[
    (D^{2}+pD+q)y=f(x),
\]
记算子多项式为
\[
    F(D)=D^{2}+pD+q.
\]
则特解可形式写为:
\[
    y^{*}=\frac{1}{F(D)}\,f(x).
\]

并约定 $\frac{1}{D}$ 表示积分操作,例如
\[
    D\sin x=\cos x,\qquad
    \frac{1}{D}\sin x=-\cos x\ (C=0).
\]

\subsubsection*{1.\quad$\frac{1}{F(D)}e^{\alpha x}$ 型}

\[
    F(D)e^{\alpha x}=F(\alpha)e^{\alpha x}.
\]

\begin{enumerate}
    \item  若 $F(\alpha)\neq 0$,则
          \[
              y^{*}=\frac{1}{F(\alpha)}e^{\alpha x}.
          \]

    \item  若 $F(\alpha)=0,\ F'(\alpha)\neq 0$(一次重复根):
          \[
              y^{*}=x\frac{1}{F'(\alpha)}e^{\alpha x}.
          \]

    \item  若 $F(\alpha)=F'(\alpha)=0,\ F''(\alpha)\neq 0$(二次重复根):
          \[
              y^{*}=x^{2}\frac{1}{F''(\alpha)}e^{\alpha x}.
          \]
\end{enumerate}

\paragraph*{例题}
\begin{enumerate}
    \item $y'' + y' - 2y = 2$

          \[
              y^* = \frac{2e^{0x}}{F(0)}
              =\frac{2}{-2}=-1.
          \]

    \item $y'' + y' - 2y = e^x$

          \[
              F(1)=0,\qquad F'(1)=3.
          \]
          \[
              y^*=x\frac{1}{3}e^x=\frac13 xe^x.
          \]

    \item $y'' - 2y' + y = e^x$

          \[
              F(1)=0,\quad F'(1)=0,\quad F''(1)=2.
          \]
          \[
              y^{*}=\frac12 x^{2}e^{x}.
          \]
\end{enumerate}


\subsubsection*{2.\quad$\frac{1}{F(D)}\sin\beta x,\ \frac{1}{F(D)}\cos\beta x$ 型}

\paragraph{(1) 当 $F(D)=D^{2}+q$}

\[
    F(i\beta)=q-\beta^{2}.
\]

\begin{enumerate}
    \item 若 $q-\beta^{2}\neq 0$:
          \[
              y^*=\frac{1}{q-\beta^{2}}\sin\beta x,\qquad
              y^*=\frac{1}{q-\beta^{2}}\cos\beta x.
          \]

    \item  若 $q-\beta^{2}=0$(共振):
          \[
              y^{*}=x\frac{1}{F'(D)}\sin\beta x.
          \]
\end{enumerate}

\paragraph{(2) 当 $F(D)=D^{2}+pD+q$}

代入 $D^{2}=-\beta^{2}$ 得
\[
    F(D)\big|_{D^{2}=\beta i^{2}}=pD+q-\beta^{2}.
\]

\paragraph*{例题}
\begin{enumerate}
    \item $y''-y=\sin x$

          \[
              F(i)= -1 -1 = -2\neq 0.
          \]
          \[
              y^{*}=-\frac12\sin x.
          \]

    \item $y''+4y=\sin 2x$

          \[
              F(2i)= -4+4=0 \Rightarrow \text{共振}.
          \]
          \[
              y^{*}=x\frac{1}{2D}\sin 2x
              =-\frac14 x\cos 2x.
          \]

    \item $y''-3y'+2y=-\frac12\cos 2x$

          \[
              F(D)\to -4-3D+2=-(3D+2)
          \]
          \[
              y^{*}=\frac12\cdot\frac{1}{3D+2}\cos 2x
              =\frac{1}{40}(3\sin 2x+\cos 2x).
          \]
\end{enumerate}

\subsubsection*{3.\quad$\frac{1}{F(D)}P_k(x)$(多项式)}

将
\[
    \frac{1}{F(D)}
\]
展开成 $k$ 次泰勒多项式:
\[
    Q_k(D)=b_0+b_1D+\cdots+b_kD^k.
\]

则特解为:
\[
    y^{*}=Q_k(D)\,P_k(x).
\]

\paragraph*{例题}
$y''+y'=x^{2}+1$

\[
    y^{*}=\frac{1}{D(D+1)}(x^{2}+1)
    =\frac{1}{D}(1-D+D^{2})(x^{2}+1).
\]

计算得:
\[
    y^{*}=\frac13 x^{3}-x^{2}+3x-1.
\]


\subsubsection*{4.\quad$\frac{1}{F(D)}e^{ax}v(x)$ 型(平移算子)}

\[
    y^{*}=\frac{1}{F(D)}e^{ax}v(x)
    =e^{ax}\cdot\frac{1}{F(D+a)}v(x).
\]

\paragraph*{例题}
\begin{enumerate}
    \item $y''+4y'+5y=e^{-2x}\sin x$

          \[
              y^{*}=e^{-2x}\frac{1}{(D-2)^{2}+4(D-2)+5}\sin x
              =e^{-2x}\frac{1}{D^{2}+1}\sin x.
          \]
          \[
              y^{*}=-\frac12 x e^{-2x}\cos x.
          \]

    \item $y''-3y'+2y=2xe^x$

          \[
              y^{*}=2e^x\cdot\frac{1}{(D+1)^2-3(D+1)+2}x
              =2e^x\cdot\frac{1}{D(D-1)}x.
          \]

          \[
              y^{*}
              =2e^x\Bigl(-\frac{1}{D}-1\Bigr)x
              =-x(x+2)e^{x}.
          \]
\end{enumerate}

\begin{enumerate}
    % ============================================================
    \item \textbf{二阶非齐次方程:}
          \[
              y'' + p y' + q y = f(x).
          \]
          \begin{enumerate}[label=(\roman*)]
              \item 写出特征方程:
                    \[
                        \lambda^2 + p\lambda + q = 0,
                        \quad \Rightarrow \quad
                        \lambda_1,\,\lambda_2.
                    \]
              \item 得齐次方程通解:
                    \[
                        y_h =
                        \begin{cases}
                            C_1 e^{\lambda_1 x} + C_2 e^{\lambda_2 x}, & \lambda_1 \neq \lambda_2, \\[0.5em]
                            (C_1 + C_2 x)e^{\lambda x},                & \lambda_1 = \lambda_2.
                        \end{cases}
                    \]
              \item 设特解为 $\dot{y}$,代入原方程求待定系数;
              \item 写出通解:
                    \[
                        \boxed{y = y_h + \dot{y}.}
                    \]
          \end{enumerate}

          % ============================================================
    \item \textbf{右端项可分解时:}
          \[
              y'' + p y' + q y = f_1(x) + f_2(x).
          \]
          \begin{enumerate}[label=(\roman*)]
              \item 对每一项分别求特解:
                    \[
                        \begin{cases}
                            y'' + p y' + q y = f_1(x) \quad \Rightarrow \quad \text{特解 } \dot{y}_1, \\[0.3em]
                            y'' + p y' + q y = f_2(x) \quad \Rightarrow \quad \text{特解 } \dot{y}_2.
                        \end{cases}
                    \]
              \item 总特解为两者之和:
                    \[
                        \boxed{\dot{y} = \dot{y}_1 + \dot{y}_2.}
                    \]
              \item 写出总通解:
                    \[
                        y = y_h + \dot{y}.
                    \]
          \end{enumerate}

          % ============================================================
    \item \textbf{欧拉方程(Cauchy–Euler 型)}
          \[
              x^2 y'' + p x y' + q y = f(x).
          \]
          \begin{enumerate}[label=(\roman*)]
              \item 当 $x>0$ 时,令 $x = e^t$,则 $t = \ln x$,有
                    \[
                        \frac{dy}{dx} = \frac{1}{x}\frac{dy}{dt},
                        \quad
                        \frac{d^2y}{dx^2}
                        = \frac{1}{x^2}\left( \frac{d^2y}{dt^2} - \frac{dy}{dt} \right).
                    \]
                    代入得
                    \[
                        \frac{d^2y}{dt^2} + (p - 1)\frac{dy}{dt} + qy = f(e^t).
                    \]
                    解此方程后,用 $t = \ln x$ 回代.
              \item 当 $x<0$ 时,令 $x = -e^t$,同理可解.
          \end{enumerate}

          % ============================================================
    \item \textbf{$n$阶齐次线性方程:}
          \[
              y^{(n)} + a_1 y^{(n-1)} + \cdots + a_n y = 0.
          \]
          其特征方程为
          \[
              \lambda^n + a_1 \lambda^{n-1} + \cdots + a_n = 0.
          \]
          根据根的类型写通解:
          \begin{enumerate}[label=(\roman*)]
              \item 若 $\lambda$ 为\textbf{单实根}:
                    \[
                        y = C e^{\lambda x};
                    \]
              \item 若 $\lambda$ 为\textbf{$k$重实根}:
                    \[
                        y = (C_1 + C_2 x + \cdots + C_k x^{k-1}) e^{\lambda x};
                    \]
              \item 若 $\lambda = \alpha \pm \beta i$ 为\textbf{单复根}:
                    \[
                        y = e^{\alpha x}(C_1 \cos \beta x + C_2 \sin \beta x);
                    \]
              \item 若 $\lambda = \alpha \pm \beta i$ 为\textbf{二重复根}:
                    \[
                        y = e^{\alpha x}(C_1 \cos \beta x + C_2 \sin \beta x
                        + C_3 x \cos \beta x + C_4 x \sin \beta x).
                    \]
          \end{enumerate}
\end{enumerate}

\subsection{用换元法求解微分方程}

\DTwoThree
\begin{enumerate}
    \item \textbf{根据表达式形式直接换元}
          \begin{enumerate}
              \item 若出现 $f(x\pm y)$,令 $t = x \pm y$;
              \item 若出现 $f(xy)$,令 $t = xy$;
              \item 若出现 $f\!\left(\dfrac{y}{x}\right)$,令 $t = \dfrac{y}{x}$;
              \item 若出现 $f(x^{2}\pm y^{2})$,令 $t = x^{2} \pm y^{2}$.
          \end{enumerate}
          (说明:这类方程常见于可分离或齐次方程.换元后需写出 $dt$ 与 $dx,dy$ 的关系以进行分离.)

    \item \textbf{逆用求导公式进行换元}

          若方程中某部分可看作已知复合函数的导数形式,可通过“逆用求导”来换元简化.
          \[
              \begin{cases}
                  \text{见到 } f^{\prime}[g(x)] \cdot g^{\prime}(x) \ \Rightarrow \ \left(f[g(x)]\right)^{\prime},      & \text{令 } u = f[g(x)];   \\
                  \text{见到 } f^{\prime}(x) g(x) + f(x) g^{\prime}(x) \ \Rightarrow \ \left(f(x) g(x)\right)^{\prime}, & \text{令 } u = f(x) g(x).
              \end{cases}
          \]
          (说明:此法相当于“识别导数结构”,遇到链式或积的求导形式时可直接降阶或积分.)

    \item \textbf{交换自变量与因变量}

          当微分方程关于 $x$ 的形式复杂、而关于 $y$ 的形式更简单时,可以交换 $x,y$ 的角色,
          即令 $x$ 为 $y$ 的函数,用 $\dfrac{dx}{dy}$ 代替 $\dfrac{dy}{dx}$.

          (说明:尤其当方程“缺 $x$”时,令 $p = \dfrac{dy}{dx}$,转为 $p\dfrac{dp}{dy} = f(y,p)$ 类型;若“缺 $y$”,则令 $p = \dfrac{dy}{dx}$,转为 $p' = f(x,p)$.)
\end{enumerate}

\section{建立微分方程并求解}
寻找信息点$A$与信息点$B$,根据题设关系,建立方程

\DThree
\subsection{用极限、导数、积分表达式建方程}
\DTwoTwo+\DThree
\begin{enumerate}
    \item 信息点$A,B$为极限、导数、积分表达式且为等量关系时,令$A=B$,建立方程。
    \item 信息点为$f(x),g(x)$及$f^\prime(x),g^{\prime}(x)$的等量关系组,用求导、消元建方程.如:

          $$\begin{cases}f'(x)+xf'(-x)=x,\\f'(-x)-xf'(x)=-x\end{cases}\Rightarrow f'(x)=\frac{x+x^2}{1+x^2}\Rightarrow f(x)=x+\frac{1}{2}\ln(1+x^2)-\arctan x+C\:.$$
    \item 信息点为关于$x,y$的恒等式,如$f(xy)=yf(x)+xf(y)$,代入特殊点,并写$f^\prime(x)$定义.
\end{enumerate}
\subsection{用几何量表达式建方程}
\DThree

\begin{enumerate}
    \item 用曲线的切线斜率.
          $$k = f'(x_0) = \tan \alpha.$$
    \item 用两曲线 $f(x)$ 与 $g(x)$ 的公切线斜率.
          $$f'(x_0) = g'(x_0).$$
    \item 用截距.
          $$Y - y = y'(X - x) \begin{cases} \text{令 } Y = 0, 则 X = x - \frac{y}{y'} (\text{x轴上的截距}); \\ \text{令 } X = 0, 则 Y = y - xy' (\text{y轴上的截距}). \end{cases}$$
          如, 令 $X = Y$, 建等式(方程).
    \item 用面积.
          $$
              \int_a^b f(x) \, \mathrm{d}x.$$
    \item 用体积.

          $$V_{x}=\int_{a}^{b}\pi f^{2}(x)\mathrm{d}x\:,\:V_{y}=\int_{a}^{b}2\pi x\big|f(x)\big|\mathrm{d}x\:.$$

    \item 用平均值.


          $$\overline{f}=\frac{1}{b-a}\int_{a}^{b}f(x)\mathrm{d}x=f(\xi)\:.$$

    \item 用弧长.

          $$s=\int_{a}^{b}\sqrt{1+\left(y_{x}^{\prime}\right)^{2}}\mathrm{d}x\:.$$

    \item 用侧面积.


          $$S=\int_{a}^{b}2\pi\big|y(x)\big|\sqrt{1+(y_{x}^{\prime})^{2}}\mathrm{d}x\:.$$

    \item 用曲率.
          $$k=\frac{|y^{\prime\prime}|}{\left[1+(y^{\prime})^2\right]^{\frac{3}{2}}}.$$

    \item 用形心.

          $$\overline{x}=\frac{\iint_Dx\:\mathrm{d}\sigma}{\iint_D\mathrm{d}\sigma}\:,\:\overline{y}=\frac{\iint_Dy\:\mathrm{d}\sigma}{\iint_D\mathrm{d}\sigma}\:.$$
\end{enumerate}
\subsection{用变化率建方程}
\DThree
\begin{enumerate}
    \item  信息点A的变化率与B成比例,令$\frac{\mathrm{d}A}{\mathrm{d}t}=\pm kB$.

          物理背景:位移、速度、加速度,$F=ma$.
    \item 见到“$P$点的运动方向始终指向$Q$点”,立即寻找◯信息点$A:P$点处的切线斜率$.\textcircled{2}$信息点$B:PQ$

          连线与水平线夹角的正切值.令(1)=2(注意正负).
\end{enumerate}



\GAOchapter{无穷级数}

\section{级数 $\sum_{n=1}^{\infty} u_n$ 的敛散性判别}

\subsection{第一步:计算 $\displaystyle \lim_{n \to \infty} u_n$}

若 $\lim_{n \to \infty} u_n \neq 0$,则级数发散;若等于 0,则需进一步判别。

\begin{example}{}{}
    判别级数 $\displaystyle \sum_{n=2}^{\infty} \left(1-\frac{1}{n}\right)^n$ 的敛散性。
\end{example}
\begin{solution}
    $\displaystyle \lim_{n\to\infty}\left(1-\frac{1}{n}\right)^n=e^{-1}\neq0$,故该级数发散。
\end{solution}

\begin{example}{}{}
    判别级数 $\displaystyle \sum_{n=2}^{\infty}\left(2-\frac{2}{n}\right)^n\ln\left(\frac{1}{2^n}+1\right)$ 的敛散性。
\end{example}
\begin{solution}
    $\displaystyle \lim_{n\to\infty}\left(2-\frac{2}{n}\right)^n\ln\left(\frac{1}{2^n}+1\right)
        =\lim_{n\to\infty}\left(1-\frac{1}{n}\right)^n=e^{-1}\neq0$,
    故该级数发散。
\end{solution}


\subsection{第二步:研究 $u_n$ 的结构特征}

\begin{enumerate}
    \item \textbf{见到 $f(n)$ 型通项}

          \begin{enumerate}
              \item \textbf{恒等变形与放缩技巧}
                    \begin{itemize}
                        \item 若含 $a^n - b^n$,提取 $a^n$ 得 $a^n[1 - (b/a)^n]$。
                        \item 含 $\ln$ 时:
                              \[
                                  \ln b - \ln a = \ln\frac{b}{a}, \quad
                                  \ln(1+n) < n, \quad \ln n < n.
                              \]
                        \item 含 $e^{f(n)}$ 或复杂 $f(n)\pm g(n)$ 时,作 \textbf{泰勒展开},与 $\frac{1}{n^p}$ 比较阶。
                    \end{itemize}

              \item \textbf{含 $(-1)^n$ 时}
                    \begin{itemize}
                        \item 若 $(-1)^n$ 不影响正负性,可去掉。
                        \item 若影响,考虑交错级数判别(莱布尼茨判别法)。
                        \item 例如:
                              \[
                                  \sum_{n=2}^{\infty}\frac{(-1)^n}{\sqrt{n}+(-1)^n}
                                  =\sum_{n=2}^{\infty}\frac{(-1)^n\sqrt{n}}{n-1}-\sum_{n=2}^{\infty}\frac{1}{n-1},
                              \]
                              发散。
                    \end{itemize}
          \end{enumerate}

    \item \textbf{见到 $f(n)$ 与 $f'(n)$}
          \begin{itemize}
              \item 用 \textbf{拉格朗日中值定理}。
              \item 或用求和恒等式:
                    \[
                        \sum_{k=1}^{n} [f(k+1)-f(k)] = f(n+1)-f(1)。
                    \]
          \end{itemize}

    \item \textbf{见到 $f(n)-f(n-1)$}
          优先考虑:
          \begin{itemize}
              \item \textbf{有理化}(处理分子);
              \item \textbf{通分}(处理分母)。
          \end{itemize}

    \item \textbf{见到 $f(a_n)$}
          \begin{itemize}
              \item 若 $\{a_n\}$ 收敛于 $a>0$,则对充分大 $n$,有 $a_n>\tfrac{a}{2}>0$。
              \item 若 $\lim n^2 a_n = a>0$,则 $|a_n|\le \tfrac{M}{n^2}$。
              \item 若 $\lim n^2(a_n-b_n)=k<\infty$,则 $\sum (a_n-b_n)$ 收敛。
              \item 若 $\lim_{n\to\infty}a_n=p$,则
                    \[
                        \sum\frac{1}{n^{a_n}}
                        \begin{cases}
                            \text{发散,} & p<1, \\
                            \text{收敛,} & p>1, \\
                            \text{不定,} & p=1。
                        \end{cases}
                    \]
          \end{itemize}

    \item \textbf{见到 $f(a_n,a_{n+1})$}
          \begin{itemize}
              \item 若给出 $f(a_n,a_{n+1})$ 的关系式,可尝试写为 $S_n$ 或差分形式 $f(a_{n+1})-f(a_n)$。
          \end{itemize}

    \item \textbf{见到 $f(a_n,b_n)$}
          \begin{itemize}
              \item 建立关系:
                    \[
                        a_n b_n = n a_n \cdot \frac{b_n}{n}, \quad
                        a_n = (a_n-b_n)+b_n,
                    \]
                    或令 $\frac{b_n}{a_n}=c_n$,转化为单变量函数 $f(c_n)$。
          \end{itemize}

    \item \textbf{见到 $f(a_n, n^p)$}
          常用不等式:
          \[
              |ab|\le \frac{a^2+b^2}{2}。
          \]

    \item \textbf{见到 $f(a_n,S_n)$}
          \begin{itemize}
              \item 代入定义 $a_n=S_n-S_{n-1}$;
              \item 写出 $\sum a_n = S_n$,检查是否望项消去;
              \item $\sum a_n$ 收敛 $\Rightarrow S_n$ 有界;
              \item $\sum a_n$ 收敛 $\Leftarrow S_n$ 有界且 $a_n\to 0$。
          \end{itemize}

    \item \textbf{见到 $(-1)^n - 1$ 型通项}
          \begin{itemize}
              \item 可作 \textbf{泰勒展开} 分项讨论;
              \item 若可能,造出交错形式,用 \textbf{莱布尼茨判别法};
              \item 若失败,尝试 \textbf{绝对收敛判别法};
              \item 拆项法:若
                    \[
                        \sum\frac{f(n)\pm g(n)}{h(n)} = \sum\frac{f(n)}{h(n)} \pm \sum\frac{g(n)}{h(n)},
                    \]
                    分别讨论敛散性。
          \end{itemize}
\end{enumerate}


\subsection{常用判别法总结}
\begin{itemize}
    \item \textbf{比较判别法}($u_n>0$)
          \[
              \text{若 }0<u_n<v_n\text{ 且 }\sum v_n\text{ 收敛,则 }\sum u_n\text{ 收敛。}
          \]
    \item \textbf{比值判别法}:若 $\displaystyle \lim_{n\to\infty}\frac{u_{n+1}}{u_n}=q$,
          \[
              \begin{cases}
                  q<1 & \Rightarrow \text{收敛}, \\
                  q>1 & \Rightarrow \text{发散}, \\
                  q=1 & \Rightarrow \text{不定。}
              \end{cases}
          \]
    \item \textbf{根值判别法}:若 $\displaystyle \lim_{n\to\infty}\sqrt[n]{|u_n|}=q$,
          结论同上。
    \item \textbf{积分判别法}:若 $u_n=f(n)$ 且 $f(x)\ge 0$ 单调递减,
          \[
              \sum_{n=1}^{\infty} f(n)\text{ 收敛} \iff \int_1^{\infty} f(x)\,dx\text{ 收敛。}
          \]
    \item \textbf{莱布尼茨判别法}(交错级数):
          若 $a_n>0$ 且 $a_n\downarrow 0$,则 $\sum (-1)^{n-1}a_n$ 收敛。
\end{itemize}

\section{求幂级数的和函数}

\subsection{确定收敛域}

\begin{enumerate}
    \item \textbf{具体型问题}
          \begin{enumerate}
              \item 对于一般幂级数 $\displaystyle \sum_{n=0}^{\infty} a_n x^n$:

                    \textbf{收敛半径:}
                    \[
                        R =
                        \begin{cases}
                            \dfrac{1}{\rho}, & \rho = \lim\limits_{n\to\infty}\left|\dfrac{a_{n+1}}{a_n}\right|
                            \text{ 或 } \rho = \lim\limits_{n\to\infty}\sqrt[n]{|a_n|},                          \\[0.8em]
                            +\infty,         & \rho = 0,                                                        \\[0.3em]
                            0,               & \rho = +\infty.
                        \end{cases}
                    \]

                    \textbf{收敛区间与收敛域:}
                    在区间 $(-R, R)$ 内绝对收敛。
                    当 $x = \pm R$ 时需单独讨论敛散性,故收敛域可能为
                    $(-R, R)$、$[-R, R)$、$(-R, R]$ 或 $[-R, R]$。

              \item 对于缺项幂级数或一般函数项级数 $\sum u_n(x)$:

                    \begin{enumerate}
                        \item 取绝对值写作 $\sum |u_n(x)|$;
                        \item 用比值(或根值)判别法求出收敛区间 $(a,b)$;
                        \item 再讨论 $x=a,b$ 时的敛散性,从而确定收敛域。
                    \end{enumerate}
          \end{enumerate}

    \item \textbf{抽象型问题——阿贝尔定理的应用}
          \begin{enumerate}
              \item 若 $\sum a_n x^n$ 在 $x=x_1(\ne 0)$ 处收敛,则当 $|x|<|x_1|$ 时绝对收敛;
                    若在 $x=x_2(\ne 0)$ 处发散,则当 $|x|>|x_2|$ 时发散。

              \item 根据阿贝尔定理:
                    \[
                        \begin{cases}
                            \text{若在 } x_1 \text{ 处收敛}   & \Rightarrow R \ge |x_1 - x_0|, \\
                            \text{若在 } x_1 \text{ 处发散}   & \Rightarrow R \le |x_1 - x_0|, \\
                            \text{若在 } x_1 \text{ 处条件收敛} & \Rightarrow R = |x_1 - x_0|.
                        \end{cases}
                    \]

              \item 已知 $\sum a_n (x-x_1)^n$ 的敛散性,讨论 $\sum b_m (x-x_2)^m$:
                    \begin{enumerate}
                        \item 平移或提出 $(x-x_0)^k$,收敛半径不变;
                        \item \textbf{逐项求导:}收敛半径不变,收敛域可能缩小;
                        \item \textbf{逐项积分:}收敛半径不变,收敛域可能扩大。
                    \end{enumerate}
          \end{enumerate}
\end{enumerate}

\subsection{求和函数的方法}

\begin{enumerate}
    \item \textbf{先积后导、先导后积法}
          \begin{enumerate}
              \item $\sum (a n + b)x^n$:先积分再求导;
              \item $\sum \dfrac{x^n}{a n + b}$:先求导再积分;
              \item $\sum \dfrac{c n^2 + d n + e}{a n + b}x^n$:拆为若干简单级数相加。
          \end{enumerate}

    \item \textbf{常用幂级数公式}
          \begin{multicols}{2}
              \begin{enumerate}
                  \item $\displaystyle \sum_{n=0}^{\infty} x^n = \frac{1}{1-x}$, ($|x|<1$);
                  \item $\displaystyle \sum_{n=1}^{\infty} n x^{n-1} = \frac{1}{(1-x)^2}$;
                  \item $\displaystyle \sum_{n=2}^{\infty} n(n-1)x^{n-2} = \frac{2}{(1-x)^3}$;
                  \item $\displaystyle \sum_{n=1}^{\infty} (-1)^{n-1}\frac{x^{n}}{n} = \ln(1+x)$;
                  \item $\displaystyle \sum_{n=1}^{\infty} \frac{x^n}{n} = -\ln(1-x)$;
                  \item $\displaystyle \sum_{n=0}^{\infty} \frac{x^{2n+1}}{2n+1} = \frac{1}{2}\ln\frac{1+x}{1-x}$;
                  \item $\displaystyle \sum_{n=0}^{\infty} \frac{(-1)^n x^{2n+1}}{2n+1} = \arctan x$;
                  \item $\displaystyle \sum_{n=0}^{\infty} \frac{x^n}{n!} = e^x$;
                  \item $\displaystyle \sum_{n=0}^{\infty} \frac{x^{2n}}{(2n)!} = \cosh x = \frac{e^x+e^{-x}}{2}$;
                  \item $\displaystyle \sum_{n=0}^{\infty} \frac{(-1)^n x^{2n+1}}{(2n+1)!} = \sin x$;
                  \item $\displaystyle \sum_{n=0}^{\infty} \frac{(-1)^n x^{2n}}{(2n)!} = \cos x$.
              \end{enumerate}
          \end{multicols}
\end{enumerate}

\subsection{微分方程法求和函数}

\begin{enumerate}
    \item \textbf{已给微分方程型:}
          \begin{enumerate}
              \item 验证级数满足某微分方程;
              \item 求通解;
              \item 根据初值条件确定常数;
              \item 代入特定 $x$(如 $0,\frac{1}{2},1$)求具体和。
          \end{enumerate}

    \item \textbf{由通项关系建立微分方程型:}
          \begin{enumerate}
              \item 根据 $a_{n+1}$、$a_n$、$a_{n-1}$ 的关系式建立方程;
              \item 求解通解;
              \item 展开为 $\sum a_n x^n$ 并确定 $a_n$ 通项。
          \end{enumerate}
\end{enumerate}

\textbf{总结:}

\begin{itemize}
    \item 求收敛域首选比值法或根值法;
    \item 边界常考:$\sum \frac{x^n}{n}$、$\sum \frac{(-1)^n x^n}{n}$;
    \item 熟记导数与积分规律:求导 $\Rightarrow$ 乘 $n$;积分 $\Rightarrow$ 除 $n$;
    \item 收敛半径不变的三种操作:提因式、逐项求导、逐项积分;
    \item 求和函数常从 $\sum x^n$ 推导。
\end{itemize}

\section{函数展开成幂级数}

\begin{enumerate}
    \item \textbf{含对数函数的展开}

          \begin{enumerate}
              \item $\ln(a + bx)$ 型:
                    \[
                        \ln(a + bx) = \ln a + \ln\left(1 + \frac{b}{a}x\right), \quad a > 0.
                    \]
                    若 $|x| < \frac{a}{b}$,则
                    \[
                        \ln(1 + \tfrac{b}{a}x) = \sum_{n=1}^{\infty}(-1)^{n-1}\frac{1}{n}\left(\frac{b}{a}x\right)^n.
                    \]
              \item $\ln(1 + ax + bx^2)$ 型:
                    \[
                        \ln(1 + ax + bx^2)
                        = \ln(1 + cx) + \ln(1 + dx),
                    \]
                    其中 $a = c + d,\ b = cd.$
          \end{enumerate}

    \item \textbf{含分式的展开}

          \begin{enumerate}
              \item $\dfrac{1}{a + bx}$ 型:
                    \[
                        \frac{1}{a + bx}
                        = \frac{1}{a} \cdot \frac{1}{1 + \frac{b}{a}x}
                        = \frac{1}{a}\sum_{n=0}^{\infty}(-1)^n\left(\frac{b}{a}x\right)^n, \quad |x| < \frac{a}{b}.
                    \]
              \item $\dfrac{1}{(x+a)(x+b)}$ 型:
                    \[
                        \frac{1}{(x+a)(x+b)}
                        = \frac{1}{b - a}\left(\frac{1}{x+a} - \frac{1}{x+b}\right).
                    \]
          \end{enumerate}

    \item \textbf{含三角平方函数的化简}

          \[
              \sin^2x = \frac{1 - \cos 2x}{2}, \qquad
              \cos^2x = \frac{1 + \cos 2x}{2}.
          \]

          (常用于将 $\sin^2 x$、$\cos^2 x$ 转化为含单角的可积形式。)
\end{enumerate}

\section{傅里叶级数}

\subsection{周期为 $2l$ 的傅里叶级数定义}

设 $f(x)$ 是以 $2l$ 为周期的可积函数,则称
\[
    a_n = \frac{1}{l}\int_{-l}^{l} f(x)\cos\frac{n\pi x}{l}\,\mathrm{d}x, \quad
    b_n = \frac{1}{l}\int_{-l}^{l} f(x)\sin\frac{n\pi x}{l}\,\mathrm{d}x
\]
为 $f(x)$ 的傅里叶系数。

则其傅里叶级数为
\[
    f(x) \sim \frac{a_0}{2}
    + \sum_{n=1}^{\infty}\left(a_n\cos\frac{n\pi x}{l}
    + b_n\sin\frac{n\pi x}{l}\right).
\]

\subsection{狄利克雷收敛定理}

若周期为 $2l$ 的函数 $f(x)$ 在区间 $[-l,l]$ 上满足:

\begin{enumerate}
    \item 连续或仅有有限个第一类间断点;
    \item 至多具有有限个极值点,
\end{enumerate}

则其傅里叶级数在 $[-l,l]$ 上处处收敛,和函数为
\[
    S(x) =
    \begin{cases}
        f(x),                      & x\ \text{为连续点}, \\[0.4em]
        \dfrac{f(x-0)+f(x+0)}{2},  & x\ \text{为间断点}, \\[0.8em]
        \dfrac{f(-l+0)+f(l-0)}{2}, & x = \pm l.
    \end{cases}
\]
\subsection{正弦级数与余弦级数}

\begin{enumerate}
    \item 当 $f(x)$ 为\textbf{奇函数}时:
          \[
              f(x) \sim \sum_{n=1}^{\infty} b_n \sin\frac{n\pi x}{l},
              \quad b_n = \frac{2}{l}\int_0^l f(x)\sin\frac{n\pi x}{l}\,\mathrm{d}x.
          \]

    \item 当 $f(x)$ 为\textbf{偶函数}时:
          \[
              f(x) \sim \frac{a_0}{2}
              + \sum_{n=1}^{\infty} a_n \cos\frac{n\pi x}{l},
              \quad a_n = \frac{2}{l}\int_0^l f(x)\cos\frac{n\pi x}{l}\,\mathrm{d}x.
          \]
\end{enumerate}

\subsection{区间 $[0,l]$ 上函数的傅里叶展开}

若 $f(x)$ 仅定义在 $[0,l]$ 上,则先作\textbf{周期延拓}:

\begin{enumerate}
    \item \textbf{周期奇延拓(正弦级数)}

          \[
              F(x) =
              \begin{cases}
                  f(x),   & 0 < x \le l, \\
                  -f(-x), & -l < x < 0,  \\
                  0,      & x = 0,
              \end{cases}
          \]
          再令 $F(x)$ 为以 $2l$ 为周期的函数,则
          \[
              f(x) \sim \sum_{n=1}^{\infty} b_n \sin\frac{n\pi x}{l},
              \quad b_n = \frac{2}{l}\int_0^l f(x)\sin\frac{n\pi x}{l}\,\mathrm{d}x.
          \]

    \item \textbf{周期偶延拓(余弦级数)}

          \[
              F(x) =
              \begin{cases}
                  f(x),  & 0 \le x \le l, \\
                  f(-x), & -l < x < 0,
              \end{cases}
          \]
          再令 $F(x)$ 为以 $2l$ 为周期的函数,则
          \[
              f(x) \sim \frac{a_0}{2}
              + \sum_{n=1}^{\infty} a_n \cos\frac{n\pi x}{l},
              \quad a_n = \frac{2}{l}\int_0^l f(x)\cos\frac{n\pi x}{l}\,\mathrm{d}x.
          \]
\end{enumerate}

\subsection{总结}

\begin{itemize}
    \item 傅里叶级数是将函数在区间上展开为 $\sin$ 与 $\cos$ 线性组合;
    \item 奇函数 $\Rightarrow$ 正弦级数;偶函数 $\Rightarrow$ 余弦级数;
    \item 延拓思想是将 $[0,l]$ 上函数扩展为周期函数;
    \item 计算核心:求 $a_n$、$b_n$ 两类系数;
    \item 在间断点取左右极限的平均值;
    \item 傅里叶级数是高等数学与信号分析的桥梁。
\end{itemize}
\GAOchapter{多元函数积分学的预备知识}
\GAOchapter{多元函数积分学}
\section{计算三重积分}

\subsection{和式极限定义}

设区域
\[
    \Omega = \{(x, y, z) \mid a \le x \le b,\, c \le y \le d,\, e \le z \le f\},
\]
则三重积分定义为
\[
    \iiint_{\Omega} g(x, y, z) \, \mathrm{d}v
    = \lim_{n \to \infty}
    \sum_{i=1}^{n} \sum_{j=1}^{n} \sum_{k=1}^{n}
    g\!\left(a + \tfrac{b-a}{n}i,\, c + \tfrac{d-c}{n}j,\, e + \tfrac{f-e}{n}k\right)
    \frac{(b-a)(d-c)(f-e)}{n^3}.
\]

\subsection{积分次序的交换}

将所给积分次序还原为
\[
    \iiint_\Omega f(x,y,z)\, \mathrm{d}v,
\]
再根据区域特征或函数形式,选择新的积分次序以便计算(如先 $z$ 后 $x,y$ 或先 $x$ 后 $y,z$).

\subsection{积分的保号性}

\begin{enumerate}
    \item 若 $f(x,y,z)$ 在 $\Omega$ 上连续且 $f \ge 0$ 且不恒为零,则
          \[
              \iiint_{\Omega} f(x,y,z)\, \mathrm{d}v > 0.
          \]
    \item 若连续函数 $f(x,y,z)$ 满足对任意有界闭区域 $\Omega$,
          \[
              \iiint_{\Omega} f(x,y,z)\, \mathrm{d}v = 0,
          \]
          则必有 $f(x,y,z) \equiv 0$ 于该区域.
\end{enumerate}

\subsection{对称性的应用}

与二重积分完全类似.

\begin{enumerate}
    \item \textbf{普通对称性.}
          \begin{enumerate}
              \item 若 $\Omega$ 关于 $xOz$ 面对称,则
                    \[
                        \iiint_{\Omega} f(x, y, z)\, \mathrm{d}v =
                        \begin{cases}
                            2\iiint_{\Omega_1} f(x,y,z)\, \mathrm{d}v, & f(x,y,z) = f(x,-y,z),  \\[4pt]
                            0,                                         & f(x,y,z) = -f(x,-y,z),
                        \end{cases}
                    \]
                    其中 $\Omega_1$ 为 $\Omega$ 在 $xOz$ 面右侧部分.
              \item 若 $\Omega$ 关于三个坐标面对称,$\Omega_1$ 为第一卦限部分,则
                    \[
                        \iiint_{\Omega} f(x,y,z)\, \mathrm{d}v =
                        \begin{cases}
                            8\iiint_{\Omega_1} f(x,y,z)\, \mathrm{d}v, & f(x,y,z)=f(-x,-y,-z),  \\[4pt]
                            0,                                         & f(x,y,z)=-f(-x,-y,-z).
                        \end{cases}
                    \]
          \end{enumerate}

    \item \textbf{轮换对称性.}
          若 $\Omega$ 在交换 $x,y$ 后不变,则
          \[
              \iiint_{\Omega} f(x,y,z)\, \mathrm{d}v
              = \iiint_{\Omega} f(y,x,z)\, \mathrm{d}v.
          \]
          特别地,若 $\Omega=\{x^2+y^2+z^2\le R^2\}$,则
          \[
              I = \iiint_{\Omega} f(x)\, \mathrm{d}v
              = \iiint_{\Omega} f(y)\, \mathrm{d}v
              = \iiint_{\Omega} f(z)\, \mathrm{d}v
              = \tfrac{1}{3} \iiint_{\Omega} [f(x)+f(y)+f(z)]\, \mathrm{d}v.
          \]
\end{enumerate}

\subsection{直角坐标系下的积分方法}

\begin{enumerate}
    \item \textbf{先一后二法(投影穿线法)}
          \begin{enumerate}
              \item 适用:区域 $\Omega$ 由下曲面 $z=z_1(x,y)$ 与上曲面 $z=z_2(x,y)$ 所围.
              \item 计算公式:
                    \[
                        \iiint_{\Omega} f(x,y,z)\, \mathrm{d}v
                        = \iint_{D_{xy}} \mathrm{d}\sigma
                        \int_{z_1(x,y)}^{z_2(x,y)} f(x,y,z)\, \mathrm{d}z.
                    \]
          \end{enumerate}

    \item \textbf{先二后一法(定限截面法)}
          \begin{enumerate}
              \item 适用:$\Omega$ 是旋转体或按 $z$ 分层.
              \item 计算公式:
                    \[
                        \iiint_{\Omega} f(x,y,z)\, \mathrm{d}v
                        = \int_{a}^{b} \mathrm{d}z
                        \iint_{D_z} f(x,y,z)\, \mathrm{d}\sigma.
                    \]
          \end{enumerate}
\end{enumerate}

\subsection{柱面坐标系积分法}

若积分区域适于极坐标表示,令
\[
    \begin{cases}
        x = r\cos\theta, \\
        y = r\sin\theta,
    \end{cases}
\]
则
\[
    \iiint_{\Omega} f(x,y,z)\, \mathrm{d}x\mathrm{d}y\mathrm{d}z
    = \iiint_{\Omega} f(r\cos\theta,r\sin\theta,z)\, r\, \mathrm{d}r\,\mathrm{d}\theta\,\mathrm{d}z.
\]

\subsection{球面坐标系积分法}

\begin{enumerate}
    \item \textbf{适用场合:}
          \begin{enumerate}
              \item 被积函数含有 $f(x^2 + y^2 + z^2)$;
              \item 积分区域为球体或锥体的部分.
          \end{enumerate}
    \item \textbf{坐标变换:}
          \[
              \begin{cases}
                  x = r\sin\varphi\cos\theta, \\
                  y = r\sin\varphi\sin\theta, \\
                  z = r\cos\varphi,
              \end{cases}
              \quad
              \mathrm{d}v = r^2 \sin\varphi\, \mathrm{d}r\, \mathrm{d}\varphi\, \mathrm{d}\theta.
          \]
    \item \textbf{积分形式:}
          \[
              \iiint_{\Omega} f(x,y,z)\, \mathrm{d}v
              = \int_{\theta_1}^{\theta_2} \!\mathrm{d}\theta
              \int_{\varphi_1(\theta)}^{\varphi_2(\theta)} \!\mathrm{d}\varphi
              \int_{r_1(\varphi,\theta)}^{r_2(\varphi,\theta)}\!
              f(r\sin\varphi\cos\theta, r\sin\varphi\sin\theta, r\cos\varphi)
              r^2 \sin\varphi\, \mathrm{d}r.
          \]
\end{enumerate}
\subsection{重积分的应用}

\begin{enumerate}
    \item \textbf{体积:}
          \[
              V = \iiint_{\Omega} 1\, \mathrm{d}v.
          \]

    \item \textbf{重心(质心):}
          若体密度为 $\rho(x,y,z)$,则
          \[
              \overline{x} = \frac{\iiint_{\Omega} x\rho\, \mathrm{d}v}{\iiint_{\Omega} \rho\, \mathrm{d}v},\quad
              \overline{y} = \frac{\iiint_{\Omega} y\rho\, \mathrm{d}v}{\iiint_{\Omega} \rho\, \mathrm{d}v},\quad
              \overline{z} = \frac{\iiint_{\Omega} z\rho\, \mathrm{d}v}{\iiint_{\Omega} \rho\, \mathrm{d}v}.
          \]

    \item \textbf{引力:}
          对物体外一点 $M_0(x_0,y_0,z_0)$,质量为 $m$,体密度 $\rho(x,y,z)$,
          \[
              \begin{aligned}
                  F_x & = Gm \iiint_{\Omega} \frac{\rho(x,y,z)(x-x_0)}{[(x-x_0)^2+(y-y_0)^2+(z-z_0)^2]^{3/2}}\, \mathrm{d}v, \\
                  F_y & = Gm \iiint_{\Omega} \frac{\rho(x,y,z)(y-y_0)}{[(x-x_0)^2+(y-y_0)^2+(z-z_0)^2]^{3/2}}\, \mathrm{d}v, \\
                  F_z & = Gm \iiint_{\Omega} \frac{\rho(x,y,z)(z-z_0)}{[(x-x_0)^2+(y-y_0)^2+(z-z_0)^2]^{3/2}}\, \mathrm{d}v.
              \end{aligned}
          \]

    \item \textbf{转动惯量:}
          \[
              \begin{aligned}
                  I_x & = \iiint_{\Omega} \rho(y^2+z^2)\, \mathrm{d}v,     \\
                  I_y & = \iiint_{\Omega} \rho(x^2+z^2)\, \mathrm{d}v,     \\
                  I_z & = \iiint_{\Omega} \rho(x^2+y^2)\, \mathrm{d}v,     \\
                  I_O & = \iiint_{\Omega} \rho(x^2+y^2+z^2)\, \mathrm{d}v.
              \end{aligned}
          \]
\end{enumerate}
\section{计算第一型曲线积分}

\subsection{定义与物理意义}

设函数 $f(x,y)$ 定义在平面曲线 $L$ 上,或函数 $f(x,y,z)$ 定义在空间曲线 $\Gamma$ 上.
若 $f(x,y)\ge 0$(或 $f(x,y,z)\ge 0$)表示线密度,则第一型曲线积分
\[
    \int_L f(x,y)\,\mathrm{d}s,
    \qquad
    \int_\Gamma f(x,y,z)\,\mathrm{d}s
\]
表示沿曲线的物质曲杆的总质量.
此积分通过“分割—近似—求和—取极限”定义,与定积分、二重积分的思想一致.

\subsection{代入曲线方程}

若曲线 $L$ 可表示为 $y = y(x)$,则需将曲线方程代入被积函数以化简:
\[
    f(x,y) \;\Rightarrow\; f\bigl(x,\,y(x)\bigr),
\]
且弧长微元为
\[
    \mathrm{d}s = \sqrt{1 + (y')^2}\,\mathrm{d}x.
\]
空间曲线可令 $\Gamma : (x(t),y(t),z(t))$,并代入参数计算.

\subsection{几何意义}

若 $f(x,y)\equiv 1$(或 $f(x,y,z)\equiv 1$),则
\[
    \int_L \mathrm{d}s = \ell_L,
\]
即为曲线 $L$ 的弧长.

\subsection{利用形心公式}

曲线 $L$ 的形心 $\bar{x}$ 满足
\[
    \bar{x} = \frac{\int_L x\,\mathrm{d}s}{\int_L \mathrm{d}s},
\]
从而
\[
    \int_L x\,\mathrm{d}s = \bar{x} \cdot \ell_L.
\]
当曲线为规则图形(形心 $\bar{x}$ 已知且长度易求)时,此公式能简化计算.

\subsection{利用对称性质}

\begin{enumerate}
    \item \textbf{普通对称性}

          若曲线 $L$ 关于 $y$ 轴对称,则
          \[
              \int_L f(x,y)\,\mathrm{d}s
              =
              \begin{cases}
                  2\displaystyle\int_{L_1} f(x,y)\,\mathrm{d}s, & f(x,y)=f(-x,y),  \\[6pt]
                  0,                                            & f(x,y)=-f(-x,y),
              \end{cases}
          \]
          其中 $L_1$ 为 $L$ 的右半部分.
          关于 $x$ 轴对称或空间曲线关于坐标面对称的情况类似.

    \item \textbf{轮换对称性}

          若曲线在交换 $x,y$ 后保持不变,则
          \[
              \int_L f(x,y)\,\mathrm{d}s = \int_L f(y,x)\,\mathrm{d}s.
          \]
          因此可写成对称形式:
          \[
              \int_L f(x,y)\,\mathrm{d}s
              = \frac12 \int_L \bigl[f(x,y)+f(y,x)\bigr]\,\mathrm{d}s.
          \]
\end{enumerate}

\subsection{物理应用}

\begin{enumerate}
    \item \textbf{质心(形心)}

          若曲线线密度为 $\rho(x,y,z)$,则曲杆的形心为
          \[
              \bar{x}=\frac{\int_L x\rho\,\mathrm{d}s}{\int_L \rho\,\mathrm{d}s},\quad
              \bar{y}=\frac{\int_L y\rho\,\mathrm{d}s}{\int_L \rho\,\mathrm{d}s},\quad
              \bar{z}=\frac{\int_L z\rho\,\mathrm{d}s}{\int_L \rho\,\mathrm{d}s}.
          \]

    \item \textbf{转动惯量}

          曲线对各坐标轴的转动惯量为
          \[
              I_x=\int_L (y^2+z^2)\rho\,\mathrm{d}s,\quad
              I_y=\int_L (z^2+x^2)\rho\,\mathrm{d}s,
          \]
          \[
              I_z=\int_L (x^2+y^2)\rho\,\mathrm{d}s,\quad
              I_O=\int_L (x^2+y^2+z^2)\rho\,\mathrm{d}s.
          \]
\end{enumerate}

\section{计算第一型曲面积分}

\subsection{定义与物理意义}

设函数 $f(x,y,z)$ 定义在空间曲面 $\Sigma$ 上.
第一型曲面积分的物理意义是:若 $f(x,y,z) \ge 0$ 表示曲面上各点的面密度,则
\[
    \iint_{\Sigma} f(x,y,z)\,\mathrm{d}S
\]
表示该物质曲面的总质量.
其定义方式与二重积分、三重积分类似,都是“分割—近似—求和—取极限”的结果.


\subsection{代入曲面方程}

若曲面 $\Sigma$ 可表示为 $z = z(x,y)$,则需将其代入被积函数中化简:
\[
    f(x,y,z) \;\Rightarrow\; f\big(x,\,y,\,z(x,y)\big).
\]


\subsection{几何意义}

若 $f(x,y,z) \equiv 1$,则
\[
    \iint_{\Sigma} \mathrm{d}S = S_{\Sigma},
\]
即为曲面 $\Sigma$ 的面积.

\subsection{利用形心公式}

由形心定义:
\[
    \overline{x} = \frac{\iint_{\Sigma} x\,\mathrm{d}S}{\iint_{\Sigma}\mathrm{d}S},
\]
可得
\[
    \iint_{\Sigma} x\,\mathrm{d}S = \overline{x} \cdot S_{\Sigma}.
\]
当 $\Sigma$ 为规则图形(形心坐标 $\overline{x}$ 已知且面积易求)时,此公式尤为方便.


\subsection{利用对称性质}

\begin{enumerate}
    \item \textbf{普通对称性:}
          若 $\Sigma$ 关于 $xOz$ 面对称,则
          \[
              \iint_{\Sigma} f(x,y,z)\,\mathrm{d}S =
              \begin{cases}
                  2\displaystyle\iint_{\Sigma_1} f(x,y,z)\,\mathrm{d}S, & f(x,y,z)=f(x,-y,z),  \\[6pt]
                  0,                                                    & f(x,y,z)=-f(x,-y,z),
              \end{cases}
          \]
          其中 $\Sigma_1$ 为 $\Sigma$ 在 $xOz$ 面右侧部分.
          其余坐标面对称情况类似.

    \item \textbf{轮换对称性:}
          若曲面 $\Sigma: z=z(x,y)$ 在交换 $x,y$ 后保持不变,则
          \[
              \iint_{\Sigma} f(x,y,z)\,\mathrm{d}S = \iint_{\Sigma} f(y,x,z)\,\mathrm{d}S.
          \]
          若 $\Sigma$ 关于 $x,y,z$ 三坐标完全对称,则
          \[
              \iint_{\Sigma} f(x,y,z)\,\mathrm{d}S
              = \frac{1}{3}\iint_{\Sigma}\!\big[f(x,y,z)+f(y,z,x)+f(z,x,y)\big]\,\mathrm{d}S.
          \]
\end{enumerate}


\subsection{物理应用}

\begin{enumerate}
    \item \textbf{形心(重心)计算:}
          对面密度为 $\rho(x,y,z)$ 的光滑曲面 $\Sigma$,其形心 $(\overline{x},\overline{y},\overline{z})$ 为
          \[
              \overline{x} = \frac{\iint_{\Sigma} x\rho(x,y,z)\,\mathrm{d}S}{\iint_{\Sigma}\rho(x,y,z)\,\mathrm{d}S},\quad
              \overline{y} = \frac{\iint_{\Sigma} y\rho(x,y,z)\,\mathrm{d}S}{\iint_{\Sigma}\rho(x,y,z)\,\mathrm{d}S},\quad
              \overline{z} = \frac{\iint_{\Sigma} z\rho(x,y,z)\,\mathrm{d}S}{\iint_{\Sigma}\rho(x,y,z)\,\mathrm{d}S}.
          \]

    \item \textbf{转动惯量:}
          对同一曲面,其关于各坐标轴与原点的转动惯量分别为:
          \[
              I_x = \iint_{\Sigma}(y^2+z^2)\rho(x,y,z)\,\mathrm{d}S,\quad
              I_y = \iint_{\Sigma}(z^2+x^2)\rho(x,y,z)\,\mathrm{d}S,
          \]
          \[
              I_z = \iint_{\Sigma}(x^2+y^2)\rho(x,y,z)\,\mathrm{d}S,\quad
              I_O = \iint_{\Sigma}(x^2+y^2+z^2)\rho(x,y,z)\,\mathrm{d}S.
          \]
\end{enumerate}


\subsection*{小结}

\begin{itemize}
    \item 第一型曲面积分计算的是曲面“面积型”量(如面积、质量、形心、惯量等);
    \item 对称性与形心法常用于快速求积分;
    \item 若 $f(x,y,z)=1$,积分即为曲面面积.
\end{itemize}

\section{计算第二型线面积分}
\subsection{第二型曲线积分}

\begin{enumerate}
    \item \textbf{定义与物理意义(做功)}

          设向量场
          \[
              \mathbf{F}(x,y)=P(x,y)\mathbf{i}+Q(x,y)\mathbf{j}
              \quad\text{或}\quad
              \mathbf{F}(x,y,z)=P(x,y,z)\mathbf{i}+Q(x,y,z)\mathbf{j}+R(x,y,z)\mathbf{k},
          \]
          定义在有向曲线 $L$(或空间曲线 $\Gamma$)上,则第二型曲线积分表示变力 $\mathbf{F}$ 沿该曲线从起点到终点所做的功:
          \[
              \int_{L} P\,dx + Q\,dy
              \quad\text{或}\quad
              \int_{\Gamma} P\,dx + Q\,dy + R\,dz.
          \]

          与定积分、二重积分、三重积分、第一型曲线与曲面积分不同,
          第二型曲线积分是\textbf{向量场沿有向曲线的积分}(非几何量).

          \[
              \begin{cases}
                  \text{平面:} & \displaystyle\int_L (P,Q)\cdot(dx,dy) = \int_L P\,dx + Q\,dy,                        \\[6pt]
                  \text{空间:} & \displaystyle\int_\Gamma (P,Q,R)\cdot(dx,dy,dz) = \int_\Gamma P\,dx + Q\,dy + R\,dz.
              \end{cases}
          \]


    \item \textbf{计算方法}

          \begin{enumerate}
              \item \textbf{对称性法(类对称)}

                    若 $L^*$ 可分为关于某直线类对称的两部分 $L_1,L_2$,且对称点处 $P$ 绝对值相等,则
                    \[
                        \int_{L^*} P\,dx =
                        \begin{cases}
                            2\displaystyle\int_{L_1} P\,dx, & P(x,y)=P(-x,y),  \\[6pt]
                            0,                              & P(x,y)=-P(-x,y),
                        \end{cases}
                        \qquad
                        \int_{L^*} Q\,dy =
                        \begin{cases}
                            0,                              & Q(x,y)=Q(-x,y),  \\[6pt]
                            2\displaystyle\int_{L_1} Q\,dy, & Q(x,y)=-Q(-x,y).
                        \end{cases}
                    \]

              \item \textbf{参数化法——一投二代三计算(化为定积分)}

                    若 $L$ 的参数方程为
                    \[
                        \begin{cases}
                            x = x(t), \\
                            y = y(t),
                        \end{cases}\quad t\in[\alpha,\beta],
                    \]
                    则
                    \[
                        \int_{L} P\,dx + Q\,dy
                        = \int_{\alpha}^{\beta}\!\big[P(x(t),y(t))x'(t) + Q(x(t),y(t))y'(t)\big]\,dt.
                    \]
                    起点终点由参数 $\alpha,\beta$ 对应,顺序须与曲线方向一致.

              \item \textbf{格林公式(将曲线积分化为二重积分)}

                    设平面区域 $D$ 由分段光滑的正向闭曲线 $L$ 围成,且 $P,Q$ 在 $D$ 上具有连续一阶偏导数,则
                    \[
                        \oint_{L} P\,dx + Q\,dy = \iint_{D}\!\left(\frac{\partial Q}{\partial x}-\frac{\partial P}{\partial y}\right)\!d\sigma.
                    \]
                    \begin{enumerate}
                        \item 若 $L$ 为闭曲线且内部无奇点,可直接使用格林公式.
                        \item 若有奇点但除奇点外 $\dfrac{\partial Q}{\partial x}=\dfrac{\partial P}{\partial y}$,可换路径封闭.
                        \item 若非封闭曲线且 $\dfrac{\partial Q}{\partial x}=\dfrac{\partial P}{\partial y}$,可在区域内换一条起终点相同的简路径计算.
                        \item 若非封闭曲线且两偏导不等,可补线成闭合曲线 $L=L_{AB}+C_{BA}$,应用格林公式后再减去补线部分.
                    \end{enumerate}

              \item \textbf{积分与路径无关(保守场条件)}

                    若 $P,Q$ 在单连通区域 $D$ 内具有一阶连续偏导,则下列命题等价:
                    \begin{enumerate}[label=(\alph*)]
                        \item $\int_{L_{AB}} P\,dx+Q\,dy$ 与路径无关;
                        \item 对任意闭曲线 $\oint_L P\,dx+Q\,dy=0$;
                        \item 存在函数 $u(x,y)$ 使 $du = P\,dx + Q\,dy$;
                        \item $\mathbf{F}=(P,Q)$ 为某函数的梯度场;
                        \item $\dfrac{\partial P}{\partial y}=\dfrac{\partial Q}{\partial x}$.
                    \end{enumerate}

                    \textbf{求原函数法:}
                    \[
                        u(x,y) = \int_{(x_0,y_0)}^{(x,y)} P\,dx + Q\,dy,
                    \]
                    或沿折线路径计算:
                    \[
                        u(x,y) = \int_{x_0}^{x} P(x,y_0)\,dx + \int_{y_0}^{y} Q(x,y)\,dy.
                    \]
                    若 $\frac{\partial Q}{\partial x}=\frac{\partial P}{\partial y}$ 不成立,则 $u(x,y)$ 不存在.

                    \textbf{凑微分法:}
                    若能写出 $P\,dx+Q\,dy=d[u(x,y)]$,则
                    \[
                        \int_{L_{AB}} P\,dx + Q\,dy = u(B)-u(A).
                    \]

              \item \textbf{两类曲线积分关系式}
                    \[
                        \int_{\Gamma} P\,dx + Q\,dy + R\,dz
                        = \int_{\Gamma} (P\cos\alpha + Q\cos\beta + R\cos\gamma)\,ds,
                    \]
                    其中 $(\cos\alpha,\cos\beta,\cos\gamma)$ 为 $\Gamma$ 上点处的单位切向量.

              \item \textbf{空间曲线的两种计算法}
                    \begin{enumerate}
                        \item \textbf{参数法(一投二代三计算):}
                              \[
                                  \Gamma:\begin{cases}
                                      x=x(t), \\
                                      y=y(t), \\
                                      z=z(t),\quad t\in[a,b],
                                  \end{cases}
                              \]
                              \[
                                  \int_{\Gamma} P\,dx+Q\,dy+R\,dz
                                  = \int_a^b [P x'(t)+Q y'(t)+R z'(t)]\,dt.
                              \]

                        \item \textbf{斯托克斯公式:}
                              若 $\Gamma=\partial\Sigma$ 为曲面 $\Sigma$ 的正向边界,则
                              \[
                                  \oint_{\Gamma} P\,dx+Q\,dy+R\,dz
                                  = \iint_{\Sigma}
                                  \begin{vmatrix}
                                      dy\,dz                       & dz\,dx                       & dx\,dy                       \\
                                      \dfrac{\partial}{\partial x} & \dfrac{\partial}{\partial y} & \dfrac{\partial}{\partial z} \\
                                      P                            & Q                            & R
                                  \end{vmatrix}
                                  = \iint_{\Sigma}
                                  \begin{vmatrix}
                                      \cos\alpha                   & \cos\beta                    & \cos\gamma                   \\
                                      \dfrac{\partial}{\partial x} & \dfrac{\partial}{\partial y} & \dfrac{\partial}{\partial z} \\
                                      P                            & Q                            & R
                                  \end{vmatrix} dS.
                              \]
                              其中 $(\cos\alpha,\cos\beta,\cos\gamma)$ 为 $\Sigma$ 的单位外法线方向余弦.

                        \item 若 $\operatorname{rot}\mathbf{F}=0$(无旋场),则积分与路径无关,可换路径计算.
                    \end{enumerate}
          \end{enumerate}
\end{enumerate}

\subsection*{小结}

\begin{itemize}
    \item 第二型曲线积分体现\textbf{向量场沿曲线的累积作用}(常为功、环流等);
    \item 计算常用方法:参数化、一投二代三计算、对称性、格林公式;
    \item 若 $\nabla\times\mathbf{F}=0$,则积分与路径无关;
    \item 与第一型曲线积分关系:$\displaystyle\int_\Gamma (P,Q,R)\cdot(dx,dy,dz) = \int_\Gamma (P\cos\alpha+Q\cos\beta+R\cos\gamma)\,ds.$
\end{itemize}

\subsection{第二型曲面积分}

\subsubsection{定义与物理意义(通量)}

设向量场
\[
    \mathbf{F}(x, y, z) = P(x, y, z)\mathbf{i} + Q(x, y, z)\mathbf{j} + R(x, y, z)\mathbf{k},
\]
定义在光滑有向曲面 $\Sigma$ 上.
第二型曲面积分表示向量场 $\mathbf{F}$ 通过曲面 $\Sigma$ 的\textbf{通量}:
\[
    \iint_{\Sigma} P\,\mathrm{d}y\,\mathrm{d}z + Q\,\mathrm{d}z\,\mathrm{d}x + R\,\mathrm{d}x\,\mathrm{d}y
    = \iint_{\Sigma} \mathbf{F} \cdot (\mathrm{d}y\,\mathrm{d}z,\, \mathrm{d}z\,\mathrm{d}x,\, \mathrm{d}x\,\mathrm{d}y).
\]
它反映了向量场穿过曲面的量,与第一型曲面积分(面积量)不同,应注意区分.


\subsubsection{计算方法:一投二代三计算}

若曲面 $\Sigma$ 可写为 $z = z(x, y)$,则:
\[
    \iint_{\Sigma} R(x, y, z)\,\mathrm{d}x\,\mathrm{d}y
    = \pm \iint_{D_{xy}} R[x, y, z(x, y)]\,\mathrm{d}x\,\mathrm{d}y,
\]
其中 $D_{xy}$ 为 $\Sigma$ 在 $xOy$ 平面的投影区域.符号“$\pm$”根据法向量方向确定:
- 当上侧为正($\cos\gamma > 0$)取“+”;
- 当下侧为正($\cos\gamma < 0$)取“-”.

若曲面垂直于投影面,则该积分为零.
若曲面在投影中有重叠部分,应先剖分为互不重叠的曲面片再计算.

\subsubsection{转换投影法}

\paragraph{1. 法向量表达式}
若曲面 $\Sigma: z = z(x, y)$,则其单位法向量为
\[
    \mathbf{n} = \pm \frac{1}{\sqrt{1 + z_x^2 + z_y^2}} \, (-z_x, -z_y, 1),
\]
上侧为正取“+”,下侧为正取“-”.

\paragraph{2. 转换投影公式}
设 $P, Q, R$ 在 $\Sigma$ 上连续,且 $z = z(x, y)$ 有连续一阶偏导,则
\[
    \iint_{\Sigma} P\,\mathrm{d}y\,\mathrm{d}z + Q\,\mathrm{d}z\,\mathrm{d}x + R\,\mathrm{d}x\,\mathrm{d}y
    = \pm \iint_{D_{xy}} \big(-P\,z_x - Q\,z_y + R\big)\,\mathrm{d}x\,\mathrm{d}y.
\]


\subsubsection{“类”对称性}

若 $\Sigma^*$ 关于某平面对称,且对称点处 $R(x, y, z)$ 绝对值相等,则
\[
    \iint_{\Sigma^*} R(x, y, z)\,\mathrm{d}x\,\mathrm{d}y
    = \begin{cases}
        2 \displaystyle\iint_{\Sigma_1} R(x, y, z)\,\mathrm{d}x\,\mathrm{d}y, & R(x, y, z)\mathrm{d}x\,\mathrm{d}y \text{ 同号}, \\[6pt]
        0,                                                                    & R(x, y, z)\mathrm{d}x\,\mathrm{d}y \text{ 异号},
    \end{cases}
\]
其中 $\Sigma_1$ 为 $\Sigma^*$ 的一侧部分.
对 $P\,\mathrm{d}y\,\mathrm{d}z$、$Q\,\mathrm{d}z\,\mathrm{d}x$ 可类比处理.


\subsubsection{高斯公式(散度定理)}

设空间闭区域 $\Omega$ 由光滑闭曲面 $\Sigma$ 围成,且 $\Sigma$ 外侧为正.若 $P, Q, R$ 在 $\Omega$ 上具有连续一阶偏导数,则有
\[
    \iint_{\Sigma} P\,\mathrm{d}y\,\mathrm{d}z + Q\,\mathrm{d}z\,\mathrm{d}x + R\,\mathrm{d}x\,\mathrm{d}y
    = \iiint_{\Omega} \left(
    \frac{\partial P}{\partial x} + \frac{\partial Q}{\partial y} + \frac{\partial R}{\partial z}
    \right)\mathrm{d}v.
\]

\paragraph{常用技巧:}
\begin{enumerate}
    \item \textbf{封闭曲面、无奇点:} 直接用高斯公式;
    \item \textbf{封闭曲面、有奇点,且 $\mathrm{div}\,\mathbf{F}=0$:} 可换为包含奇点的其他封闭曲面;
    \item \textbf{非封闭曲面,且 $\mathrm{div}\,\mathbf{F}=0$:} 可换路径但边界需相同;
    \item \textbf{非封闭曲面,$\mathrm{div}\,\mathbf{F}\neq0$:} 可补面封闭(加面减面法);
    \item \textbf{若 $\mathrm{div}\,\mathbf{F}=0$ 对任意闭曲面成立:} 可建立方程求未知函数 $f(x)$.
\end{enumerate}

\subsubsection{两类曲面积分的关系}

设曲面 $\Sigma$ 有单位法向量 $\mathbf{n} = (\cos\alpha, \cos\beta, \cos\gamma)$,则
\[
    \iint_{\Sigma} P\,\mathrm{d}y\,\mathrm{d}z + Q\,\mathrm{d}z\,\mathrm{d}x + R\,\mathrm{d}x\,\mathrm{d}y
    = \iint_{\Sigma} (P\cos\alpha + Q\cos\beta + R\cos\gamma)\,\mathrm{d}S.
\]
右式即为第一型曲面积分的形式,表示通量的几何意义.


\subsubsection{总结}

\begin{itemize}
    \item \textbf{第一型曲面积分:} 几何量(面积、质量);
    \item \textbf{第二型曲面积分:} 向量场通量;
    \item \textbf{高斯公式:} 封闭曲面 $\leftrightarrow$ 体积分;
    \item \textbf{斯托克斯公式:} 曲线 $\leftrightarrow$ 曲面;
    \item 若 $\mathrm{div}\,\mathbf{F}=0$(无源场),则通量与选取曲面无关.
\end{itemize}

\XIANchapter{行列式}

\section{具体型行列式的计算:$a_{ij}$已给出}
\begin{enumerate}
    \item 化为基本型行列式
          \DTwoThree
          \begin{enumerate}
              \item 主对角线行列式
                    $$\begin{vmatrix}
                            a_{11} & a_{12} & \cdots & a_{1n} \\
                            0      & a_{22} & \cdots & a_{2n} \\
                            \vdots & \vdots &        & \vdots \\
                            0      & 0      & \cdots & a_{nn}
                        \end{vmatrix}=
                        \begin{vmatrix}
                            a_{11} & 0      & \cdots & 0      \\
                            a_{21} & a_{22} & \cdots & 0      \\
                            \vdots & \vdots &        & \vdots \\
                            a_{n1} & a_{n2} & \cdots & a_{nn}
                        \end{vmatrix}=
                        \begin{vmatrix}
                            a_{11} & 0      & \cdots & 0      \\
                            0      & a_{22} & \cdots & 0      \\
                            \vdots & \vdots &        & \vdots \\
                            0      & 0      & \cdots & a_{nn}
                        \end{vmatrix}=\prod_{i=1}^na_{ii}.$$
              \item 副对角线行列式
                    $$\begin{aligned}
                            \begin{vmatrix}
                                a_{11} & a_{12} & \cdots & a_{1,n-1} & a_{1n} \\
                                a_{21} & a_{22} & \cdots & a_{2,n-1} & 0      \\
                                \vdots & \vdots &        & \vdots    & \vdots \\
                                a_{n1} & 0      & \cdots & 0         & 0
                            \end{vmatrix} & =
                            \begin{vmatrix}
                                0      & \cdots & 0         & a_{1n} \\
                                0      & \cdots & a_{2,n-1} & a_{2n} \\
                                \vdots &        & \vdots    & \vdots \\
                                a_{n1} & \cdots & a_{n,n-1} & a_{nn}
                            \end{vmatrix}=
                            \begin{vmatrix}
                                0      & \cdots & 0         & a_{1n} \\
                                0      & \cdots & a_{2,n-1} & 0      \\
                                \vdots &        & \vdots    & \vdots \\
                                a_{n1} & \cdots & 0         & 0
                            \end{vmatrix}                                                                     \\
                                                                             & =(-1)^{\frac{n(n-1)}{2}}a_{1n}a_{2,n-1}\cdots a_{n1}.
                        \end{aligned}$$
              \item 拉普拉斯展开式

                    设$A$为$m$阶矩阵,$B$为$n$阶矩阵,则
                    $$\begin{gathered}
                            \begin{vmatrix}
                                A & O \\
                                O & B
                            \end{vmatrix}=
                            \begin{vmatrix}
                                A & C \\
                                O & B
                            \end{vmatrix}=
                            \begin{vmatrix}
                                A & O \\
                                C & B
                            \end{vmatrix}=|A||B|, \\
                            \begin{vmatrix}
                                O & A \\
                                B & O
                            \end{vmatrix}=
                            \begin{vmatrix}
                                C & A \\
                                B & O
                            \end{vmatrix}=
                            \begin{vmatrix}
                                O & A \\
                                B & C
                            \end{vmatrix}=(-1)^{mn}|A||B|.
                        \end{gathered}$$
              \item 范德蒙德行列式
                    $$\begin{vmatrix}
                            1         & 1         & \cdots & 1         \\
                            x_1       & x_2       & \cdots & x_n       \\
                            x_1^2     & x_2^2     & \cdots & x_n^2     \\
                            \vdots    & \vdots    &        & \vdots    \\
                            x_1^{n-1} & x_2^{n-1} & \cdots & x_n^{n-1}
                        \end{vmatrix}=\prod_{1\leq i<j\leq n}(x_j-x_i),n\geq2.$$
          \end{enumerate}
          \begin{note}{行列式计算思路}{行列式计算思路}
              \begin{enumerate}
                  \item 若所给行列式就是基本形或接近基本形,则直接套公式或经过简单处理化成基本形后套公式.
                  \item 简单处理的手段: \DTwoOne
                        \begin{enumerate}
                            \item 按零元素多的行或列展开;
                            \item 用行列式的性质对差别最小的“对应位置元素”进行处理,尽可能多地化出零元素,再按此行或列展开;
                            \item 对于行和或列和相等的情形,将所有列加到第1列或将所有行加到第1行,提出公因式,再用上述方法,等等.
                        \end{enumerate}
                  \item 具体型行列式的元素中若含$x$,则其为$x$的多项式.
              \end{enumerate}
          \end{note}
    \item 加边法
          \DTwoThree
          对于某些一开始不宜使用“互换”“倍乘”“倍加”性质的行列式,可以考虑使用加边法:$n$阶行列式中添加一行、一列升至$n+1$阶行列式.若添加在第 $1$ 列,且添加的是$[ 1, 0,...,0]^\mathrm{T}$,则第 $1$ 行其余元素可以任意添加,行列式的值不变,即
          $$D_n=
              \begin{vmatrix}
                  a_{11} & a_{12} & \cdots & a_{1n} \\
                  a_{21} & a_{22} & \cdots & a_{2n} \\
                  \vdots & \vdots &        & \vdots \\
                  a_{n1} & a_{n2} & \cdots & a_{nn}
              \end{vmatrix}=
              \begin{vmatrix}
                  1      & *      & *      & \cdots & *      \\
                  0      & a_{11} & a_{12} & \cdots & a_{1n} \\
                  0      & a_{21} & a_{22} & \cdots & a_{2n} \\
                  \vdots & \vdots & \vdots &        & \vdots \\
                  0      & a_{n1} & a_{n2} & \cdots & a_{nn}
              \end{vmatrix},$$
    \item 递推法
          \DOne+\DTwoOne

          \begin{enumerate}
              \item 建立递推公式,即建立$D_n$ 与$D_{n-1}$的关系,有些复杂的题甚至要建立$D_n$,$D_{n-1}$与$D_{n-2}$的关系.
              \item$D_{n-1}$与$D_n$要有完全相同的元素分布规律,只是$D_n-1$ 比 $D_n$ 低了一阶.
          \end{enumerate}
    \item 数学归纳法
          \PThree

          涉及$n$阶行列式的证明型计算问题,即告知行列式计算结果,让考生证明之,可考虑数学归纳法.
          \begin{enumerate}
              \item 第一数学归纳法 (适用于 $F(D_{n},D_{n-1})=0$):
                    \begin{enumerate}
                        \item 验证当 $n=1$ 时,命题成立;
                        \item 假设当 $n=k$ (≥2)时,命题成立;
                        \item 证明当 $n=k+1$ 时,命题成立.
                    \end{enumerate}
                    则命题对任意正整数 $n$ 成立.
              \item 第二数学归纳法 (适用于 $F(D_{n},D_{n-1},D_{n-2})=0$):
                    \begin{enumerate}
                        \item 验证当 $n=1$ 和 $n=2$ 时,命题成立;
                        \item 假设当 $n<k$ 时,命题成立;
                        \item 证明当 $n=k$ ($≥3$)时,命题成立.
                    \end{enumerate}
                    则命题对任意正整数 $n$ 成立.

          \end{enumerate}
\end{enumerate}
\section{抽象型行列式的计算:$a_{ij}$未给出}

\begin{enumerate}
    \item 用行列式的性质

          用行列式的性质将所求行列式进一步化成已知行列式.
    \item 用矩阵知识
          \begin{enumerate}
              \item 设$C=AB,A,B$为同阶方阵,则$|C|=|AB|=|A||B|.$
              \item 设$C=A+B,A,B$ 为同阶方阵,则 $|C|=|A+B|$,但由于$|A+B|$ 不一定等于$|A|+|B|$,故需对$|A+B|$作恒等变形,转化为矩阵乘积的行列式.这里的恒等变形一般是:
                    \begin{enumerate}
                        \item 由题设条件,如 $E=AA^\mathrm{T}$;
                        \item 用$E=AA^{-1}$等.
                    \end{enumerate}
          \end{enumerate}
\end{enumerate}
\XIANchapter{余子式和代数余子式的计算}
\section{用矩阵}

当$|A|\neq0$时,
$$A^*=|A|A^{-1}.$$
由于$A^{*}$由$A_{ij}$组成,用上式求出$A^{*}$,即得到所有的$A_{ij}$.但要注意,此方法要求$|A|\neq0$,这既是前提,也是一种限制.

\section{用特征值}
设 $A$ 为 3 阶方阵,当 $A$ 为可逆矩阵时,记其特征值为 $\lambda_{1}$,$\lambda_{2}$,$\lambda_{3}$,则 $A^{-1}$ 的特征值为 $\lambda_{1}^{-1}$,$\lambda_{2}^{-1}$,$\lambda_{3}^{-1}$,且由 $A^{*} = |A|A^{-1} = \lambda_{1}\lambda_{2}\lambda_{3}A^{-1}$,可知 $A^{*}$ 的特征值为
$$\lambda_{1}^{*} = \lambda_{1}\lambda_{2}\lambda_{3} \cdot \lambda_{1}^{-1} = \lambda_{2}\lambda_{3}, \quad \lambda_{2}^{*} = \lambda_{1}\lambda_{2}\lambda_{3} \cdot \lambda_{2}^{-1} = \lambda_{1}\lambda_{3}, \quad \lambda_{3}^{*} = \lambda_{1}\lambda_{2}\lambda_{3} \cdot \lambda_{3}^{-1} = \lambda_{1}\lambda_{2},$$
故由
$$
    A^{*} = \begin{bmatrix}
        A_{11} & A_{21} & A_{31} \\
        A_{12} & A_{22} & A_{32} \\
        A_{13} & A_{23} & A_{33}
    \end{bmatrix},
$$
知 $A_{11} + A_{22} + A_{33} = \operatorname{tr}(A^{*}) = \lambda_{1}^{*} + \lambda_{2}^{*} + \lambda_{3}^{*} = \lambda_{2}\lambda_{3} + \lambda_{1}\lambda_{3} + \lambda_{1}\lambda_{2}$.
\XIANchapter{矩阵运算}

\XIANchapter{矩阵的秩}
\section{定义}
设 $A$ 是 $m\times n$ 矩阵,$A$ 中最大的不为零的子式的阶数称为矩阵 $A$ 的秩,记为 $r(A)$.也可以这样定义:若存在 $k$ 阶子式不为零,而任意 $k+1$ 阶子式全为零(如果有的话),则 $r(A)=k$,且若 $A$ 为 $n\times n$ 矩阵,则
$$r(A_{n\times n})=n\Leftrightarrow|A|\neq0\Leftrightarrow A \text{可逆.}$$
\section{公式}
\begin{enumerate}
    \item 设$A$ 是 m×n 矩阵,则 $0\leqslant r(A)\leqslant\min\{m,n\}.$
    \item 设$A$ 是$m\times n$ 矩阵,则$r(kA)=r(A)(k\neq0).$
    \item 设$A$ 是$m\times n$ 矩阵,$P,Q$分别是$m$阶、$n$阶可逆矩阵,则

          $$r(A)=r(PA)=r(AQ)=r(PAQ).$$


\end{enumerate}

\chapter{线性方程组}
\section{线性方程组理论总结}
\DOne
\begin{enumerate}
    \item 齐次线性方程组$Ax=0$ \DOne
    \item 非齐次线性方程组$Ax=b$ \DOne
\end{enumerate}


\section{线性方程组问题}
\begin{enumerate}
    \item 一般求解问题
    \item 公共解问题
    \item 同解问题
          \DOne+\DTwoTwo
          \begin{detail}{齐次线性方程组\DTwoTwo}{}
          \end{detail}

          \begin{detail}{非齐次线性方程组\DTwoTwo}{}
              设(\RomanSymbols Ⅰ)$A_{m\times n}x=\beta$与(Ⅱ)$B_{s\times n}x=\gamma$均有解,则

              ①(\RomanSymbols Ⅰ)与(Ⅱ)同解

              $\Leftrightarrow$②$A_{m\times n}x=0$与$B_{s\times n}x=0$同解且(\RomanSymbols Ⅰ)与(Ⅱ)有公共解

              $\Leftrightarrow$③$r\left(\begin{bmatrix}A & \beta \\ B & \gamma\end{bmatrix}\right)=r\left(\begin{bmatrix}A \\ B\end{bmatrix}\right)=r(A)=r(B)$

              $\Leftrightarrow$④$[A,\beta]$与$[B,\gamma]$的行向量组等价.
          \end{detail}
\end{enumerate}


\section{线性方程组的几何意义}



\XIANchapter{向量组}

\section{研究具体型向量关系}

\subsection{定义法}
\subsection{求极大线性无关组}
\section{研究抽象型向量关系}
\subsection{定义法}
\subsection{综合问题}
\DOne+\DTwoThree

$$x = \eta^{*} + k_{1} \xi_{1} + \ldots + k_{n-r} \xi_{n-r}.$$
$$
    A \xi_{i} = 0.
$$
$$A \eta^{*} = \beta.$$
\section{研究向量组等价}
\section{向量空间}
\subsection{概念}
\subsection{过渡矩阵}
\subsection{坐标变换}
\XIANchapter{特征向量与特征值}

\term{求解利用} $A$ 的特征值与特征向量

% ======================================================
\section{利用特征值命题}
\DOne + \DTwoTwo
\begin{enumerate}

    % ------------------------------------------------------
    \item \textbf{特征值判定}
          \[
              \lambda_0 \text{ 是 }A\text{ 的特征值}
              \;\Longleftrightarrow\;
              |\lambda_0E - A| = 0,
          \]
          \[
              \lambda_0 \text{ 不是特征值}
              \;\Longleftrightarrow\;
              |\lambda_0E - A| \ne 0
              \quad(\text{矩阵可逆、满秩}).
          \]

          % ------------------------------------------------------
    \item \textbf{特征值与行列式、迹的关系}

          若 $A$ 的特征值为 $\lambda_1,\ldots,\lambda_n$,则
          \[
              |A|=\lambda_1\lambda_2\cdots\lambda_n,
              \qquad
              \mathrm{tr}(A)=\lambda_1+\lambda_2+\cdots+\lambda_n.
          \]

          % ------------------------------------------------------
    \item \textbf{矩阵函数与特征值对应表}

          \begin{table}[h]
              \centering
              \begin{tabular}{|c|c|c|c|c|c|c|}
                  \hline
                  矩阵   & $A$       & $f(A)$       & $A^{-1}$            & $A^{*}$ 的 $f$         & $P^{-1}AP=B$ & $P^{-1}f(A)P=B$ \\
                  \hline
                  特征值  & $\lambda$ & $f(\lambda)$ & $\frac{1}{\lambda}$ & $\frac{|A|}{\lambda}$ & $\lambda$    & $f(\lambda)$    \\
                  \hline
                  特征向量 & $\xi$     & $\xi$        & $\xi$               & $\xi$                 & $P^{-1}\xi$  & $P^{-1}\xi$     \\
                  \hline
              \end{tabular}
          \end{table}

          (注:分母出现 $\lambda$ 时均要求 $\lambda\neq 0$)

          \begin{note}{}{}
              若 $\lambda\neq 0$,则
              \[
                  af(A)\pm bA^{-1}\pm cA^{*}
              \]
              的特征值为
              \[
                  a f(\lambda)\pm \frac{b}{\lambda}\pm c\frac{|A|}{\lambda},
              \]
              对应特征向量仍为 $\xi$。但 $f(A),A^{-1},A^{*}$ 与 $A^{T}$ 的线性组合特征向量一般不同,需重新计算。
          \end{note}

          % ------------------------------------------------------
    \item \textbf{关于转置矩阵}

          虽然 $A^{T}$ 与 $A$ 特征值相同,但其特征向量一般不同。

          \begin{note}{}{}
              $A^{T}$ 与 $A$ 属于不同特征值的特征向量相互正交。
          \end{note}

          % ------------------------------------------------------
    \item \textbf{归零原则(常用)}

          \begin{enumerate}
              \item \textbf{归零准则一(对任意矩阵)}
                    若 $f(A)=O$,则 $A$ 的任意特征值 $\lambda$ 满足
                    \[
                        f(\lambda)=0.
                    \]

              \item \textbf{归零准则二(凯莱–哈密顿定理)}
                    若 $A$ 的特征多项式为
                    \[
                        f(\lambda)=|\lambda E-A|
                        =\lambda^n+a_{n-1}\lambda^{n-1}+\cdots+a_0,
                    \]
                    则
                    \[
                        f(A)=A^n+a_{n-1}A^{n-1}+\cdots +a_0E=O.
                    \]
          \end{enumerate}

\end{enumerate}

% ======================================================
\section{利用特征向量命题}
\DOne + \DTwoTwo
\begin{enumerate}

    % ------------------------------------------------------
    \item \textbf{特征向量判定}
          \[
              \xi\neq 0 \text{ 是 }A\text{ 的属于 }\lambda_0\text{ 的特征向量}
              \;\Longleftrightarrow\;
              (\lambda_0E-A)\xi=0.
          \]

          % ------------------------------------------------------
    \item \textbf{重要结论}

          \begin{enumerate}
              \item 单根特征值恰有 $1$ 个线性无关特征向量。
              \item $k$ 重特征值 $\lambda$ 至多有 $k$ 个线性无关特征向量。
              \item 不同特征值对应的特征向量必线性无关。
              \item 同一特征值下,若 $\xi_1,\xi_2$ 为特征向量,则 $k_1\xi_1+k_2\xi_2$($k_1k_2\neq 0$)仍是特征向量。
              \item 不同特征值下 $\xi_1,\xi_2$,线性组合 $k_1\xi_1+k_2\xi_2$ 不是任何特征向量($k_1k_2\neq 0$)。
              \item 若 $\xi$ 属于 $\lambda_1$,则不可能属于其它 $\lambda_2\neq\lambda_1$。
              \item 若 $A$ 只有 1 个线性无关特征向量,则唯一特征值必须是 $n$ 重特征值。
              \item 若 $AB=BA$ 且 $A$ 有 $n$ 个互异特征值,则 $A$ 的全部特征向量都是 $B$ 的特征向量。
              \item 若 $r(A)+r(B)<n$,则 $Ax=0,Bx=0$ 至少有一个公共非零解。
          \end{enumerate}

\end{enumerate}

% ======================================================
\section{利用矩阵方程命题}
\DOne+\DTwoTwo+\DTwoThree
\begin{enumerate}

    % ------------------------------------------------------
    \item $AB=O$
          \[
              AB=O\Rightarrow
              A[\beta_1,\ldots,\beta_n]=[0,\ldots,0],
          \]
          若每个 $\beta_i\neq 0$,则 $\beta_i$ 为 $A$ 属于特征值 $0$ 的特征向量。

          % ------------------------------------------------------
    \item 若任意 $\xi\ne 0$ 都是 $(\lambda E-A)x=0$ 的解
          取标准基 $e_1,\ldots,e_n$,得
          \[
              (\lambda E-A)B=O,\quad B=[e_1,\ldots,e_n] \text{ 可逆},
          \]
          故
          \[
              \lambda E-A=O \Rightarrow A=\lambda E.
          \]

          % ------------------------------------------------------
    \item 若 $AB=C$
          令 $B=[\beta_1,\ldots,\beta_n], C=[\gamma_1,\ldots,\gamma_n]$,若
          \[
              \gamma_i=\lambda_i\beta_i,
          \]
          则
          \[
              A\beta_i=\lambda_i\beta_i,
          \]
          所以 $\beta_i$ 是 $A$ 的属于 $\lambda_i$ 的特征向量。

          % ------------------------------------------------------
    \item 若 $AP=PB$,$P$ 可逆
          \[
              P^{-1}AP=B \Rightarrow A\sim B \Rightarrow \lambda_A=\lambda_B.
          \]

          % ------------------------------------------------------
    \item 若 $A$ 每行元素之和均为 $k$
          \[
              A \begin{bmatrix}1\\1\\ \vdots\\1\end{bmatrix}
              =
              k\begin{bmatrix}1\\1\\ \vdots\\1\end{bmatrix},
          \]
          因此上述向量为特征向量,特征值为 $k$。

          % ------------------------------------------------------
    \item 若 $A$ 可逆且每行元素之和为 $k$
          则 $A^{-1}$ 的每行元素之和为
          \[
              \frac{1}{k}.
          \]

          % ------------------------------------------------------
    \item 若 $A$ 每行元素之和为 $k$
          则 $A^n$ 每行元素之和为
          \[
              k^n.
          \]

\end{enumerate}
\XIANchapter{相似理论}
\section{化归相似对角化的基本局面}
\DOne+\DTwoThree

若 $n$ 阶矩阵 $A$ 有 $n$ 个线性无关的特征向量,则 $A$ 可相似对角化,且有
$$[\xi_1, \xi_2, \cdots, \xi_n]^{-1} A [\xi_1, \xi_2, \cdots, \xi_n] = \begin{bmatrix} \lambda_1 & & \\ & \lambda_2 & \\ & & \ddots \\ & & & \lambda_n \end{bmatrix},$$
牢记这个形式.

\section{用各种条件判$A$能否相似对角化}
\DOne+\DTwoTwo

$\star \star \star$

\begin{enumerate}
    \item 充要条件
          \begin{enumerate}
              \item $A$有$n$个线性无关的特征向量 $\Leftrightarrow A \sim\Lambda$
              \item $n_i = n - r(\lambda_i E - A) \Leftrightarrow A \sim \Lambda$.
          \end{enumerate}
    \item 充分条件
          \begin{enumerate}
              \item $A$是实对称矩阵$\Leftrightarrow A \sim \Lambda$.
              \item $A$有$n$个互异特征值$\Leftrightarrow A \sim \Lambda$.
              \item $A^k=E$($k$为正整数) $\Leftrightarrow A \sim \Lambda$.
              \item $A^2 - (k_1 + k_2)A + k_1 k_2 E = O$ 且 $k_1 \neq k_2 \Rightarrow A \sim \Lambda$.
              \item $r(A) = 1$ 且 $\text{tr}(A) \neq 0 \Rightarrow A \sim \Lambda$.
          \end{enumerate}
    \item 必要条件

          $A \sim \Lambda \Rightarrow r(A) =$ 非零特征值的个数 (重根按重数算).
    \item 否定条件
          \begin{enumerate}
              \item $A \neq O$, $A^k = O$ ($k$ 为大于 1 的整数) $\Rightarrow A$ 不可相似对角化.
              \item $A$ 的特征值全为 $k$, 但 $A \neq kE \Rightarrow A$ 不可相似对角化.
          \end{enumerate}
\end{enumerate}

\section{非对称矩阵$A$与实对称矩阵$A$相似对角化的异同}
\DOne+\DTwoOne

\begin{enumerate}
    \item 非对称矩阵$A$不存在正交矩阵$Q$,使其相似对角化
    \item 实对称矩阵$A$存在正交矩阵$Q$,使其相似对角化
\end{enumerate}
\section{$A$与$B$相似}
\DOne+\DTwoOne+\PFour

$\star \star \star$

\begin{enumerate}
    \item 若$A$相似于$B$,则
          \begin{enumerate}
              \item $|A| = |B|$;
              \item $r(A)=r(B)$;
              \item $tr(A)=tr(B)$;
              \item $\lambda_{A} = \lambda_{B}$ (或 |$\lambda E - A$| = |$\lambda E - B$|);
              \item 属于 $\lambda_{A}$ 的线性无关的特征向量的个数等于属于 $\lambda_{B}$ 的线性无关的特征向量的个数;
              \item $A, B$  的各阶主子式之和分别相等.
          \end{enumerate}

    \item 若$A$相似于$\Lambda$,$B$相似于$\Lambda$,则$A$相似于$B$.
    \item 若$A$相似于$B$,$B$相似于$\Lambda$,则$A$相似于$\Lambda$.
    \item $A$与$B$的相似手段的“三同一不同”.

          若 $P^{-1}AP = B$, 则 $P^{-1}f(A)P = f(B)$, $P^{-1}A^{-1}P = B^{-1}$, $P^{-1}A^{*}P = B^{*}$, 即 $f(A)$ 与 $f(B)$, $A^{-1}$ 与 $B^{-1}$, $A^{*}$ 与 $B^{*}$ 相似的手段相同, 也即 $P^{-1}[af(A) + bA^{-1} + cA^{*}]P = af(B) + bB^{-1} + cB^{*}$. 但 $A^{T}$ 与 $B^{T}$ 相似的手段与上面不同.
\end{enumerate}
\section{相似对角化的应用}
\DOne+\DTwoTwo

\begin{example}{}{}
    已知数列 $\{x_n\}$, $\{y_n\}$, $\{z_n\}$ 满足 $x_0 = -1$, $y_0 = 0$, $z_0 = 2$, 且
    $$\begin{cases}
            x_n = -2x_{n-1} + 2z_{n-1}, \\
            y_n = -2y_{n-1} - 2z_{n-1}, \\
            z_n = -6x_{n-1} - 3y_{n-1} + 3z_{n-1},
        \end{cases}$$
    记 $\alpha_n = \begin{bmatrix} x_n \\ y_n \\ z_n \end{bmatrix}$, 写出满足 $\alpha_n = A\alpha_{n-1}$ 的矩阵 $A$, 并求 $A^n$ 及 $x_n$, $y_n$, $z_n (n=1,2,\cdots)$.
\end{example}
\begin{solution}
    由题设得$\begin{bmatrix}x_n\\y_n\\z_n\end{bmatrix}=\begin{bmatrix}-2&0&2\\0&-2&-2\\-6&-3&3\end{bmatrix}\begin{bmatrix}x_{n-1}\\y_{n-1}\\z_{n-1}\end{bmatrix}$,得矩阵$A=\begin{bmatrix}-2&0&2\\0&-2&-2\\-6&-3&3\end{bmatrix}$满足$\alpha_n=A\alpha_{n-1}$.

    因为

    $|\lambda E-A|=\begin{vmatrix}\lambda+2&0&-2\\0&\lambda+2&2\\6&3&\lambda-3\end{vmatrix}=\lambda(\lambda-1)(\lambda+2)$,

    所以矩阵$A$的特征值为$\lambda_1=0$,$\lambda_2=1$,$\lambda_3=-2$.

    当$\lambda_1=0$时,解方程组$(0E-A)x=0$,得特征向量$\xi_1=\begin{bmatrix}1&-1&1\end{bmatrix}^{T}$;

    当$\lambda_2=1$时,解方程组$(E-A)x=0$,得特征向量$\xi_2=\begin{bmatrix}2&-2&3\end{bmatrix}^{T}$;

    当$\lambda_3=-2$时,解方程组$(-2E-A)x=0$,得特征向量$\xi_3=\begin{bmatrix}-1&2&0\end{bmatrix}^T$.

    令$P=[\xi_1,\xi_2,\xi_3]=\begin{bmatrix}1&2&-1\\-1&-2&2\\1&3&0\end{bmatrix}$,则$P^{-1}AP=\begin{bmatrix}0&0&0\\0&1&0\\0&0&-2\end{bmatrix}$,即$A=P\begin{bmatrix}0&0&0\\0&1&0\\0&0&-2\end{bmatrix}P^{-1}$,从而得$A^n=P\begin{bmatrix}0&0&0\\0&1&0\\0&0&-2\end{bmatrix}^nP^{-1}=\begin{bmatrix}1&2&-1\\-1&-2&2\\1&3&0\end{bmatrix}\begin{bmatrix}0&0&0\\0&1&0\\0&0&(-2)^n\end{bmatrix}\begin{bmatrix}6&3&-2\\-2&-1&1\\1&1&0\end{bmatrix}$

    $$=\begin{bmatrix}-4-(-2)^n&-2-(-2)^n&2\\4-(-2)^{n+1}&2-(-2)^{n+1}&-2\\-6&-3&3\end{bmatrix}$$.

    由递推式$\alpha_n=A\alpha_{n-1}$知$\alpha_n=A^n\alpha_0$,其中$\alpha_0=\begin{bmatrix}-1&0&2\end{bmatrix}^T$,所以

    $\alpha_n=A^n\alpha_0=\begin{bmatrix}-4-(-2)^n&-2-(-2)^n&2\\4-(-2)^{n+1}&2-(-2)^{n+1}&-2\\-6&-3&3\end{bmatrix}\begin{bmatrix}-1\\0\\2\end{bmatrix}=\begin{bmatrix}8+(-2)^n\\-8+(-2)^{n+1}\\12\end{bmatrix}$,

    故$x_n=8+(-2)^n$,$y_n=-8+(-2)^{n+1}$,$z_n=12(n=1,2,\cdots)$.
\end{solution}

\section{正交矩阵及其使用}
\DOne + \DTwoOne

$\star \star \star$

\begin{enumerate}
    \item 若$A$为正交矩阵,则
          $$A^\top A = E \Leftrightarrow A^{-1} = A^\top$$
          $$\Leftrightarrow A \text{ 由规范正交基组成 }$$
          $$\Leftrightarrow A^{\mathrm{T}}\text{是正交矩阵}$$
          $$\Leftrightarrow A^{-1}\text{是正交矩阵}$$
          $$\Leftrightarrow A^{*}\text{是正交矩阵}$$
          $$\Leftrightarrow-A\text{是正交矩阵.}$$
    \item 若$A,B$为同阶正交矩阵,则$AB$为正交矩阵,但$A+B$不一定为正交矩阵.
    \item ${\text{若 }A\text{ 为正交矩阵,则其实特征值的取值范围为}\{-1,1\}}.$
    \item 设$A$为$n$阶非零矩阵,$\begin{cases}\text{若}a_{ij}=A_{ij},\text{则}A^{\mathrm{T}}=A^{*},AA^{\mathrm{T}}=E,\text{且}\mid A\mid=1;\\\text{若}a_{ij}=-A_{ij},\text{则}A^{\mathrm{T}}=-A^*,AA^{\mathrm{T}}=E,\text{且}\mid A\mid=-1.\end{cases}$
\end{enumerate}
\XIANchapter{二次型}
\section{$f=x^TAx$ 中$A$的表示}
\DOne+\DTwoThree

\begin{enumerate}
    \item 给出非对称矩阵 $B$,令$A=\frac{B+B^T}{2}$,则$A=A^\mathrm{T}.$
    \item 通过题设或基本变形显化出 $A.$
\end{enumerate}

\section{配方法与正交变换法的异同}

\begin{enumerate}
    \item 命题语言
          \DTwoTwo
          \begin{enumerate}
              \item 配方法

                    二次型语言:将 $f = x^T A x$ 通过配方法化为标准形,并求出可逆变换矩阵 $C$.

                    矩阵语言:求可逆矩阵 $C$,使得 $C^T A C = \Lambda$.
              \item 正交变换法

                    二次型语言:将 $f = x^T A x$ 通过正交变换法化为标准形,并求出正交矩阵 $Q$.

                    矩阵语言:求正交矩阵 $Q$,使得 $Q^T A Q = \Lambda$.
          \end{enumerate}
    \item 过程与结果的异同
          \DTwoThree

          设$f(x)=x^TAx$.
          \begin{enumerate}
              \item 配方法(可逆线性变换)

                    $x=Cy$,$C$可逆.使得$f\xlongequal{x=\boldsymbol{C}y}y^T\Lambda y$,其中$C^TAC=\Lambda$(使$A$合同于对角矩阵).
              \item 正交变换法(可逆线性变换):

                    $x=Qy$(这里的$Q$不仅可逆,还满足$Q^{-1}=Q^{T}$),使得$f\xlongequal{x=\boldsymbol{Q}y}y^T\Lambda y$,其中$Q^{T}AQ=Q^{-1}AQ=A$.
          \end{enumerate}
          二者区别:在配方法中,$c$只满足可逆,所以$c^{-1}$不一定等于$c^T$,但是在正交变换法中,变换手段$Q$满足$Q^{- 1}= Q^T$ .

          二者相同点:它们的正、负惯性指数是对应相等的.
    \item 惯性指数
          \begin{example}{}{}
              $f(x_{1},x_{2},x_{3})=-2x_{1}x_{2}-2x_{1}x_{3}+6x_{2}x_{3}$的正惯性指数为(  ).
          \end{example}
          \begin{solution}
              令$\begin{cases}x_{1}=y_{1}+y_{2},\\x_{2}=y_{1}-y_{2},\end{cases}$则

              $$f=-2y_{1}^{2}+2y_{2}^{2}+4y_{1}y_{3}-8y_{2}y_{3}
                  =-2(y_{1}-y_{3})^{2}+2(y_{2}-2y_{3})^{2}-6y_{3}^{2},$$

              再令$\begin{cases}z_{1}=y_{1}-y_{3},\\z_{2}=y_{2}-2y_{3},\end{cases}$则

              $$f=-2z_{1}^{2}+2z_{2}^{2}-6z_{3}^{2},$$

              故$f$的正惯性指数为1.
          \end{solution}
\end{enumerate}

\section{伪配方法}
\DTwoThree

“平方和式$A^2+B^2+C^2$”未必就是(拉格朗日)配方法得来的结果,故若非拉格朗日配方法,则称伪配方法.要注意伪配方法的变换矩阵是否有可逆性.
\begin{enumerate}
    \item 如果变换没有可逆性,则有可能改变表达式的几何性质,如封闭性,此时不能得出平方和式正定;
    \item 如果变换是可逆的,则平方和式正定.
\end{enumerate}


\begin{note}{}{}
    对于$f(x_{1},x_{2},x_{3})=(a_{1}x_{1}+a_{2}x_{2}+a_{3}x_{3})^{2}+(b_{1}x_{1}+b_{2}x_{2}+b_{3}x_{3})^{2}+(c_{1}x_{1}+c_{2}x_{2}+c_{3}x_{3})^{2}$的情形,可总结如下做题方法:

    令$f=0$,即$\begin{cases}a_{1}x_{1}+a_{2}x_{2}+a_{3}x_{3}=0,\\b_{1}x_{1}+b_{2}x_{2}+b_{3}x_{3}=0,\\c_{1}x_{1}+c_{2}x_{2}+c_{3}x_{3}=0,\end{cases}$计算$|\boldsymbol{A}|=\begin{vmatrix}a_{1}&a_{2}&a_{3}\\b_{1}&b_{2}&b_{3}\\c_{1}&c_{2}&c_{3}\end{vmatrix}$,若$|\boldsymbol{A}|\neq0$,则$f$正定;若$|\boldsymbol{A}|=0$,则$f$不正定.
\end{note}
\section{正交变换法的传递性}
\DOne+\DTwoThree

若$A$相似于$B$,则$B$相似于$C$,则$A$相似于$C$.这里$B$常为$\Lambda$.

\section{合同的判定与手段}
\DOne+\DTwoThree

\begin{enumerate}
    \item 同阶实对称矩阵$A,B$合同的判定

          用正、负惯性指数:$A,B$合同$\Leftrightarrow p_A=p_B,q_A=q_B$(相同的正、负惯性指数).
    \item 已知$A$,$\Lambda$($\Lambda$是对角矩阵),求可逆矩阵$C$,使得$C^TAC=\Lambda$
    \item 已知$A$,$B$($B$不是对角矩阵),求可逆矩阵$C$,使得$C^TAC=B$
\end{enumerate}
\begin{idea}{求可逆矩阵$C$,使得$C^TAC=\Lambda$}{}
    \begin{enumerate}
        \item 配方 盯着$\Lambda$的对角线元素,提出对应系数
        \item 换元
        \item 求逆
    \end{enumerate}
\end{idea}
\begin{idea}{求可逆矩阵$C$,使得$C^TAC=B$}{}
    \begin{enumerate}
        \item 对$f$配方、换元,写$D_1$
        \item 对$g$配方、换元,写$D_2$
        \item 令$D_1x=D_2y$,求$D_2^{-1}D_1$
    \end{enumerate}
\end{idea}
\section{合同与相似的异同}
\DOne+\DTwoThree

对于实对称矩阵$A$与$B$,相似必合同,反之不成立.
\begin{example}{合同与相似的异同}{合同与相似的异同}
    已知二次型
    $$f(x_{1}, x_{2}, x_{3}) = x_{1}^{2} + 2x_{2}^{2} + 2x_{3}^{2} + 2x_{1}x_{2} - 2x_{1}x_{3}$$
    $$
        g(y_{1}, y_{2}, y_{3}) = y_{1}^{2} + y_{2}^{2} + y_{3}^{2} + 2y_{2}y_{3}$$
    \begin{enumerate}
        \item 求可逆变换 $x = Py$,将 $f(x_{1}, x_{2}, x_{3})$ 化成 $g(y_{1}, y_{2}, y_{3})$.
        \item 是否存在正交变换 $x = Qy$,将 $f(x_{1}, x_{2}, x_{3})$ 化成 $g(y_{1}, y_{2}, y_{3})$?
    \end{enumerate}
\end{example}
\begin{idea}{解题思路 \ref{ex:合同与相似的异同}}{}
    \begin{enumerate}
        \item 求可逆变换用配方法
        \item 判断是否存在正交变换,如果存在必相似,使用相似的充分条件和充要条件
    \end{enumerate}
\end{idea}
\section{正定的判定与应用}
\DOne+\DTwoThree

$\star\star\star$

\begin{enumerate}
    \item 前提
          $A=A^T$($A$是实对称矩阵)
    \item 二次型$f=x^TAx$正定的充要条件 \DTwo

          $n$元二次型$f=x^{T}Ax$正定

          $\Leftrightarrow$对任意的$x\neq 0$,有$x^{T}Ax>0$(定义)

          $\Leftrightarrow A$的特征值$\lambda_{i}>0(i=1,2,\cdots,n)$

          $\Leftrightarrow f$的正惯性指数$p=n$

          $\Leftrightarrow$存在可逆矩阵$D$,使得$A=D^{T}D$

          $\Leftrightarrow A$与$E$合同

          $\Leftrightarrow A$的各阶顺序主子式均大于0.
    \item 二次型$f=x^TAx$正定的必要条件
          \begin{enumerate}
              \item $a_{ii}>0\left(i=1,2,\cdots,n\right).$
              \item $| A| > 0$.
          \end{enumerate}
    \item 重要结论
          \begin{enumerate}
              \item 若$A$正定,则$A^-1,A^{*},A^{m}(m$为正整数$),kA(k>0),C^{\mathrm{T}}AC(C$可逆 )均正定 .

              \item 若$A,B$正定,则$A+B$正定,$\begin{bmatrix}A&O\\O&B\end{bmatrix}$正定.

              \item ${\text{若}A,B}$正定,则$AB$正定的充要条件是$AB= BA$ .
          \end{enumerate}

\end{enumerate}

\section{二次型的最值}
\DOne+\DTwoTwo+\DTwoThree

设 $A$ 为实对称矩阵,令
\[
    f(x)=x^T A x,\qquad x\in\mathbb R^n.
\]
若 $A=Q\Lambda Q^T$($Q$ 正交, $\Lambda=\mathrm{diag}(\lambda_i)$),令 $x=Qy$ 得
\[
    f(x)=\sum_{i=1}^n\lambda_i y_i^2.
\]
记 $\lambda_{\min}=\min_i\lambda_i,\ \lambda_{\max}=\max_i\lambda_i$,则
\[
    \boxed{\;\lambda_{\min}\|x\|^2 \le x^TAx \le \lambda_{\max}\|x\|^2\;},
\]
两端同时除以 $x^Tx$ 得 Rayleigh 商界:
\[
    \lambda_{\min}\le \frac{x^TAx}{x^Tx}\le \lambda_{\max}.
\]
等号成立当且仅当 $x$ 为对应的特征向量.

给定对称 $A$ 与对称且正定的 $B$,考题常要求求
\[
    \max_{x\ne0}\frac{x^TAx}{x^TBx}\quad\text{或}\quad\min_{x\ne0}\frac{x^TAx}{x^TBx}.
\]

若 $B$ 为对称正定矩阵,则可作分解 $B=P^TP$.记 $y=Px$,则有
$\frac{x^TAx}{x^TBx}\xlongequal[x=P^{-1}y]{y=Px} \frac{(P^{-1}y)^TAP^{-1}y}{y^Ty}=\frac{y^T(PAP^{-1})y}{y^Ty}=\frac{y^TCy}{y^Ty}$,
其中 $C=PAP^{-1}$ 为对称矩阵.

于是,$\dfrac{x^TAx}{x^TBx}$ 的最大值、最小值,分别等于 $C$ 的最大、最小特征值;相应的 $x$ 可由 $y=Px$ 反变换得到.
\chapter{随机事件和概率}
\GAIchapter{一维随机变量及其分布}

\section{判分布}
\DOne + \DTwoTwo

\begin{enumerate}
    \item \textbf{分布函数判定}
          \begin{enumerate}
              \item \textbf{充要条件:}
                    \[
                        F(x)\text{ 是分布函数 } \Leftrightarrow
                        \begin{cases}
                            F(x) \text{ 单调不减且右连续}, \\
                            F(-\infty)=0,\quad F(+\infty)=1.
                        \end{cases}
                    \]

              \item \textbf{典型形式:}
                    \begin{enumerate}
                        \item 若 $F_i(x)$ 为分布函数,$\lambda_i>0,\ \sum\lambda_i=1$,则 $\sum\lambda_iF_i(x)$ 仍为分布函数;
                        \item 若 $F(x)$ 为分布函数,则其平移均值
                              \[
                                  \frac{1}{T}\int_x^{x+T}F(t)\,dt
                              \]
                              仍为分布函数;
                        \item 几何平均 $\sqrt{F_1(x)F_2(x)}$ 为分布函数;
                        \item $[F(x)]^n$ 与 $1-[1-F(x)]^n$ 亦为分布函数.
                    \end{enumerate}
          \end{enumerate}

    \item \textbf{分布律判定}
          \[
              \{p_i\}\text{ 为概率分布 } \Leftrightarrow p_i\ge0,\ \sum_i p_i=1.
          \]

    \item \textbf{概率密度判定}
          \begin{enumerate}
              \item \textbf{充要条件:}
                    \[
                        f(x)\text{ 为概率密度 } \Leftrightarrow
                        f(x)\ge0,\quad
                        \int_{-\infty}^{+\infty}f(x)\,dx=1.
                    \]
              \item \textbf{常见构造形式:}
                    \begin{enumerate}
                        \item 若 $f_i(x)$ 为概率密度,$\sum\lambda_i=1$,则 $\sum\lambda_i f_i(x)$ 仍为密度;
                        \item 平均形式 $\dfrac{1}{T}\!\int_x^{x+T}\!f(t)\,dt$;
                        \item 混合型:$\dfrac{2}{n}\sum F_i(x)f_i(x)$;
                        \item 累积分布组合:
                              \[
                                  f_1F_2\cdots F_n + F_1f_2\cdots F_n + \cdots + F_1F_2\cdots f_n;
                              \]
                        \item 若 $F,f$ 对应,则 $n[F(x)]^{n-1}f(x)$、$n[1-F(x)]^{n-1}f(x)$ 皆为密度.
                    \end{enumerate}
          \end{enumerate}

    \item \textbf{反问题:建立方程求参数}
          \[
              \begin{cases}
                  F(-\infty)=0, \\
                  F(+\infty)=1, \\
                  \sum p_i = 1, \\
                  \int_{-\infty}^{+\infty} f(x)\,dx = 1.
              \end{cases}
          \]
\end{enumerate}


\section{用分布}
\DOne

\subsection{离散型分布}
\begin{enumerate}
    \item \textbf{0-1分布(伯努利试验)}
          \[
              X\sim B(1,p),\quad P(X=1)=p,\ P(X=0)=1-p.
          \]

    \item \textbf{二项分布}
          \[
              X\sim B(n,p),\quad P(X=k)=C_n^k p^k(1-p)^{n-k},
          \]
          \[
              E X = np,\quad D X = np(1-p).
          \]
          性质:
          \begin{itemize}
              \item $Y=n-X\sim B(n,1-p)$;
              \item 概率$P(X=k)$先增后减,最大值在$k=\lfloor (n+1)p\rfloor$附近.
          \end{itemize}

    \item \textbf{几何分布(首次成功试验次数)}
          \[
              P(X=k)=p(1-p)^{k-1},\quad k=1,2,\dots;\quad E X=\tfrac{1}{p},\ D X=\tfrac{1-p}{p^2}.
          \]

    \item \textbf{超几何分布}
          \[
              P(X=k)=\frac{C_M^kC_{N-M}^{n-k}}{C_N^n},\quad E X=\frac{nM}{N}.
          \]

    \item \textbf{泊松分布}
          \[
              P(X=k)=\frac{\lambda^k e^{-\lambda}}{k!},\quad E X = D X = \lambda.
          \]
          用于描述稀有事件次数.
\end{enumerate}


\subsection{连续型分布}

\begin{enumerate}
    \item \textbf{均匀分布} $U(a,b)$
          \[
              f(x)=
              \begin{cases}
                  \dfrac{1}{b-a}, & a<x<b,     \\
                  0,              & \text{其他},
              \end{cases}\quad
              F(x)=
              \begin{cases}
                  0,                & x<a,      \\
                  \dfrac{x-a}{b-a}, & a\le x<b, \\
                  1,                & x\ge b.
              \end{cases}
          \]
          $E X=\tfrac{a+b}{2},\quad D X=\tfrac{(b-a)^2}{12}.$

    \item \textbf{指数分布(连续型首次冲击)}
          \[
              f(x)=
              \begin{cases}
                  \lambda e^{-\lambda x}, & x\ge0, \\
                  0,                      & x<0,
              \end{cases}\quad
              F(x)=1-e^{-\lambda x}.
          \]
          \[
              E X=\tfrac{1}{\lambda},\quad D X=\tfrac{1}{\lambda^2}.
          \]
          无记忆性:$P(X>t+s\mid X>t)=P(X>s)$.

          \begin{note}{}{}
              特殊关系:
              \[
                  E\!\left(\tfrac12\right)\equiv\chi^2(2),\quad
                  2\lambda X\sim\chi^2(2).
              \]
          \end{note}

    \item \textbf{自由度为1的$t$分布(标准柯西分布)}
          \[
              f(x)=\frac{1}{\pi(1+x^2)},\quad X\sim t(1).
          \]

    \item \textbf{正态分布}
          \[
              f(x)=\frac{1}{\sqrt{2\pi}\sigma}e^{-\frac{(x-\mu)^2}{2\sigma^2}},\quad
              X\sim N(\mu,\sigma^2).
          \]
          \begin{note}{}{}
              \begin{enumerate}
                  \item 标准正态:$N(0,1)$,
                        \[
                            \varphi(x)=\frac{1}{\sqrt{2\pi}}e^{-x^2/2},\quad
                            \Phi(x)=\int_{-\infty}^x\varphi(t)\,dt.
                        \]
                  \item 性质:
                        \[
                            \Phi(-x)=1-\Phi(x),\quad
                            P(|X|\le a)=2\Phi(a)-1.
                        \]
                  \item 标准化:
                        \[
                            Z=\frac{X-\mu}{\sigma}\sim N(0,1),
                            \quad F(x)=\Phi\!\left(\frac{x-\mu}{\sigma}\right).
                        \]
                  \item 含参数密度$f(x)=ke^{-(ax^2+bx+c)}$,
                        其规范化常数$k=\sqrt{\frac{a}{\pi}}e^{\frac{4ac-b^2}{4a}}$.
              \end{enumerate}
          \end{note}
\end{enumerate}


\section{利用分布求概率}
\DOne

\begin{enumerate}
    \item 若 $X\sim F(x)$,则
          \[
              P(X\le a)=F(a),\quad
              P(a<X<b)=F(b-0)-F(a),\quad
              P(X=a)=F(a)-F(a-0).
          \]
    \item 若 $X\sim p_i$,则
          \[
              P(X\in I)=\sum_{x_i\in I}p_i.
          \]
    \item 若 $X\sim f(x)$,则
          \[
              P(X\in I)=\int_I f(x)\,dx.
          \]
    \item \textbf{反问题:}已知概率表达式,建立方程求参数.
\end{enumerate}


\section{求分布}
\DOne

根据题意建立
\[
    F(x)=P\{X\le x\},
\]
利用定义及已知条件计算并验证其合法性.
\GAIchapter{一维随机变量函数的分布}

\section{离散型$\rightarrow$离散型}
设离散型随机变量 $X$ 的分布为 $P\{X = x_i\} = p_i \ (i=1,2,\cdots)$,若 $Y = g(X)$,则 $Y$ 仍为离散型随机变量,其分布为
$$
    Y \sim
    \begin{pmatrix}
        g(x_1) & g(x_2) & \cdots \\
        p_1    & p_2    & \cdots
    \end{pmatrix}.
$$
若若干个 $g(x_k)$ 取相同值,则合并为一项,并将对应概率相加.

\section{连续型$\rightarrow$连续型(或混合型)}
设连续型随机变量 $X$ 的分布函数与密度分别为 $F_X(x)$、$f_X(x)$,若 $Y = g(X)$,则可用以下两种方法求其分布:

\subsection*{(1) 分布函数法}
由定义直接求:
$$
    F_Y(y) = P\{Y \le y\} = P\{g(X) \le y\} = \int_{g(x) \le y} f_X(x) \, dx.
$$
若 $F_Y(y)$ 连续且可导,则 $f_Y(y) = F_Y'(y)$.

\subsection*{(2) 公式法(单调可导变换)}
若 $y = g(x)$ 在 $(a,b)$ 上严格单调且可导,则存在反函数 $x = h(y)$,其概率密度为
$$
    f_Y(y) = f_X[h(y)] \cdot |h'(y)|, \quad \alpha < y < \beta,
$$
其中
$$
    \alpha = \min\{\lim_{x\to a^+} g(x), \lim_{x\to b^-} g(x)\}, \quad
    \beta = \max\{\lim_{x\to a^+} g(x), \lim_{x\to b^-} g(x)\}.
$$

\section{连续型$\rightarrow$离散型}
若 $X \sim f_X(x)$ 且 $Y = g(X)$ 为离散型变量,先求出 $Y$ 的可能取值 $a_i$,再由
$$
    P\{Y = a_i\} = \int_{g(x)=a_i} f_X(x) \, dx
$$
得出其分布.
\section{两种重要的随机变量变换}

\subsection*{(1) 变换于 $U(0,1)$}
\begin{example}{}{}
    设随机变量 $X$ 的分布函数 $F_X(x)$ 严格单调递增,反函数 $F_X^{-1}(y)$ 存在,令 $Y = F_X(X)$,则 $Y \sim U(0,1)$.
\end{example}
\begin{proof}
    由定义:
    $$
        F_Y(y) = P\{Y \le y\} = P\{F_X(X) \le y\} = P\{X \le F_X^{-1}(y)\} = F_X[F_X^{-1}(y)] = y,
    $$
    对 $0 \le y < 1$ 成立;此外,$F_Y(y) = 0 (y < 0)$,$F_Y(y) = 1 (y \ge 1)$,即
    $$
        F_Y(y) =
        \begin{cases}
            0, & y < 0,       \\
            y, & 0 \le y < 1, \\
            1, & y \ge 1,
        \end{cases}
    $$
    故 $Y \sim U(0,1)$.
\end{proof}


\subsection*{(2) 变换于 $E(1)$}
\begin{example}{}{}
    设 $X$ 的分布函数 $F_X(x)$ 连续且在其密度区间上严格单调,令
    $$
        Y = -\ln[1 - F_X(X)],
    $$
    则 $Y \sim E(1)$.
\end{example}
\begin{proof}
    由定义:
    $$
        P\{Y \le y\} = P\{-\ln[1 - F_X(X)] \le y\} = P\{F_X(X) \le 1 - e^{-y}\}.
    $$
    由于 $F_X(X) \sim U(0,1)$,故
    $$
        F_Y(y) = 1 - e^{-y}, \quad y > 0,
    $$
    即 $Y \sim E(1)$.
\end{proof}
\GAIchapter{多维随机变量及其分布}

\section{离散型问题}
一般不作重点考查。

\section{连续型问题}
\begin{enumerate}
      \item \textbf{二维均匀分布}

            若 $(X,Y)$ 的概率密度为
            $$
                  f(x,y)=
                  \begin{cases}
                        \dfrac{1}{S_D}, & (x,y)\in D, \\
                        0,              & \text{其他,}
                  \end{cases}
            $$
            其中 $S_D$ 为区域 $D$ 的面积,则称 $(X,Y)$ 在平面有界区域 $D$ 上服从\textbf{二维均匀分布}。

      \item \textbf{二维正态分布}

            若 $(X,Y)$ 的概率密度为
            $$
                  f(x,y)=\frac{1}{2\pi\sigma_1\sigma_2\sqrt{1-\rho^2}}
                  \exp\!\left\{
                  -\frac{1}{2(1-\rho^2)}
                  \left[
                        \left(\frac{x-\mu_1}{\sigma_1}\right)^2
                        -2\rho\!\left(\frac{x-\mu_1}{\sigma_1}\right)\!\left(\frac{y-\mu_2}{\sigma_2}\right)
                        +\left(\frac{y-\mu_2}{\sigma_2}\right)^2
                        \right]
                  \right\},
            $$
            其中 $\mu_1,\mu_2\in\mathbb{R}$,$\sigma_1,\sigma_2>0$,$-1<\rho<1$,则称
            $$
                  (X,Y)\sim N(\mu_1,\mu_2;\sigma_1^2,\sigma_2^2;\rho)
            $$
            为\textbf{二维正态分布}。
\end{enumerate}

\section{边缘分布、条件分布与独立性}

\subsection*{(1) 边缘分布}
\begin{enumerate}
      \item 分布函数:
            $$
                  F_X(x)=F(x,+\infty), \quad F_Y(y)=F(+\infty,y).
            $$
      \item 离散情形:
            $$
                  p_{i\cdot}=\sum_j p_{ij}, \quad p_{\cdot j}=\sum_i p_{ij}.
            $$
      \item 连续情形:
            $$
                  f_X(x)=\int_{-\infty}^{+\infty} f(x,y)\,dy, \quad
                  f_Y(y)=\int_{-\infty}^{+\infty} f(x,y)\,dx.
            $$
\end{enumerate}

\subsection*{(2) 条件分布}
\begin{enumerate}
      \item 离散型:
            $$
                  P\{Y=y_j|X=x_i\}=\frac{p_{ij}}{p_{i\cdot}},\quad
                  P\{X=x_i|Y=y_j\}=\frac{p_{ij}}{p_{\cdot j}}.
            $$
      \item 连续型:
            $$
                  f_{Y|X}(y|x)=\frac{f(x,y)}{f_X(x)}, \quad
                  f_{X|Y}(x|y)=\frac{f(x,y)}{f_Y(y)}.
            $$
            对应分布函数为:
            $$
                  F(x|y)=\int_{-\infty}^x f_{X|Y}(u|y)\,du, \quad
                  F(y|x)=\int_{-\infty}^y f_{Y|X}(v|x)\,dv.
            $$
\end{enumerate}

\subsection*{(3) 独立性判定}
\begin{enumerate}
      \item $X$ 与 $Y$ 独立 $\Leftrightarrow$
            $$
                  F(x,y)=F_X(x)F_Y(y).
            $$
      \item 离散型:
            $$
                  X,Y\text{独立} \Leftrightarrow p_{ij}=p_{i\cdot}p_{\cdot j}.
            $$
      \item 连续型:
            $$
                  X,Y\text{独立} \Leftrightarrow f(x,y)=f_X(x)f_Y(y).
            $$
      \item 若要证明“不独立”,可取 $x_0,y_0$ 使得
            $$
                  F(x_0,y_0)\ne F_X(x_0)F_Y(y_0),
            $$
            或 $\{X\le x_0\}\subseteq\{Y\le y_0\}$ 等。
\end{enumerate}


\section{用分布求概率及反问题}
\begin{enumerate}
      \item 若 $(X,Y)\sim p_{ij}$,则
            $$
                  P\{(X,Y)\in D\}=\sum_{(x_i,y_j)\in D} p_{ij}.
            $$
      \item 若 $(X,Y)\sim f(x,y)$,则
            $$
                  P\{(X,Y)\in D\}=\iint_D f(x,y)\,dx\,dy.
            $$
      \item 若为混合型,用全概率公式分情况计算。
      \item \textbf{反问题:}已知概率关系,建立方程求参数。
\end{enumerate}


\section*{小结}
\begin{itemize}
      \item 边缘分布:对另一变量积分(或求和);
      \item 条件分布:联合分布除以边缘分布;
      \item 独立性:$f(x,y)=f_X(x)f_Y(y)$;
      \item 常见二维分布:均匀、正态;
      \item 二维正态分布由参数 $(\mu_1,\mu_2,\sigma_1,\sigma_2,\rho)$ 唯一确定。
\end{itemize}
\GAIchapter{多维随机变量函数的分布}

\section{多维$\rightarrow$一维}

\subsection*{(1) 离散型$\rightarrow$离散型}
\begin{enumerate}
    \item 若 $(X,Y)\sim p_{ij}$,$Z=g(X,Y)$,则
          $$
              Z\sim q_k, \quad q_k = \sum_{g(x_i,y_j)=z_k} p_{ij}.
          $$
    \item 若 $X\sim p_i$, $Y\sim q_j$ 且相互独立,可考
          $$Z=X+Y,\; XY,\; \max\{X,Y\},\; \min\{X,Y\},$$
          等离散型函数.
\end{enumerate}


\subsection*{(2) 连续型$\rightarrow$连续型}

\begin{enumerate}
    \item \textbf{分布函数法}
          $$
              (X,Y)\sim f(x,y), \quad Z=g(X,Y),
          $$
          则
          $$
              F_Z(z)=P\{g(X,Y)\le z\}=\iint_{g(x,y)\le z} f(x,y)\,dx\,dy,
              \qquad f_Z(z)=F_Z'(z).
          $$

    \item \textbf{换元法(雅可比法)}
          \begin{itemize}
              \item 设 $\begin{cases} u=u(x,y), \\ v=v(x,y) \end{cases}$ 连续可微,存在反函数 $\begin{cases} x=x(u,v), \\ y=y(u,v) \end{cases}$,
                    且
                    $$
                        J=\frac{\partial(x,y)}{\partial(u,v)}
                        =\begin{vmatrix}
                            \dfrac{\partial x}{\partial u} & \dfrac{\partial x}{\partial v} \\[4pt]
                            \dfrac{\partial y}{\partial u} & \dfrac{\partial y}{\partial v}
                        \end{vmatrix}\neq 0.
                    $$
              \item 则 $(U,V)$ 的联合密度为
                    $$
                        f_{U,V}(u,v)=f[x(u,v),y(u,v)]\cdot|J|.
                    $$
              \item 边缘密度:
                    $$
                        f_U(u)=\int_{-\infty}^{+\infty}f_{U,V}(u,v)\,dv, \quad
                        f_V(v)=\int_{-\infty}^{+\infty}f_{U,V}(u,v)\,du.
                    $$
          \end{itemize}

    \item \textbf{最值函数的分布}
          \begin{enumerate}
              \item 最大值:
                    $$
                        Z=\max\{X,Y\},\quad F_{\max}(z)=P\{\max(X,Y)\le z\}=P\{X\le z, Y\le z\}=F(z,z).
                    $$
                    若 $X,Y$ 独立:
                    $$
                        F_{\max}(z)=F_X(z)F_Y(z).
                    $$

              \item 最小值:
                    $$
                        Z=\min\{X,Y\},\quad F_{\min}(z)=P\{\min(X,Y)\le z\}
                        =F_X(z)+F_Y(z)-F(z,z).
                    $$
                    若 $X,Y$ 独立:
                    $$
                        F_{\min}(z)=1-[1-F_X(z)][1-F_Y(z)].
                    $$
          \end{enumerate}

          \textbf{推广到 $n$ 个独立随机变量 $X_1,\dots,X_n$:}
          $$
              F_{\max}(z)=\prod_{i=1}^n F_{X_i}(z),\quad
              F_{\min}(z)=1-\prod_{i=1}^n[1-F_{X_i}(z)].
          $$
          若 $X_i$ 同分布($F,f$):
          $$
              F_{\max}(z)=[F(z)]^n,\quad f_{\max}(z)=n[F(z)]^{n-1}f(z),
          $$
          $$
              F_{\min}(z)=1-[1-F(z)]^n,\quad f_{\min}(z)=n[1-F(z)]^{n-1}f(z).
          $$
\end{enumerate}
---

\subsection*{(3) 离散型、连续型混合}
设 $X\sim p_i$, $Y\sim f_Y(y)$, $Z=g(X,Y)$.

\begin{enumerate}
    \item 若 $X,Y$ 独立,用全概率公式:
          $$
              F_Z(z)=\sum_i p_i P\{g(x_i,Y)\le z\}=\sum_i p_i \int_{g(x_i,y)\le z} f_Y(y)\,dy.
          $$
    \item 若 $X,Y$ 不独立,直接使用分布函数法.
\end{enumerate}


\section{一维$\rightarrow$多维}

\begin{enumerate}
    \item 离散型:
          $$
              X\sim p_i,\;
              \begin{cases}U=g(X),\\V=h(X)\end{cases}
              \Rightarrow (U,V)\sim q_{ij}.
          $$
    \item 连续型:
          $$
              X\sim f(x),\;
              \begin{cases}U=g(X),\\V=h(X)\end{cases}
              \Rightarrow (U,V)\sim f_{U,V}(u,v)\text{ 或 }p_{ij}.
          $$
\end{enumerate}


\section{多维$\rightarrow$多维}

\begin{enumerate}
    \item 离散型:
          $$
              (X,Y)\sim p_{ij},\;
              \begin{cases}U=g(X,Y),\\V=h(X,Y)\end{cases}
              \Rightarrow (U,V)\sim q_{ij}.
          $$
    \item 连续型:
          $$
              (X,Y)\sim f(x,y),\;
              \begin{cases}U=g(X,Y),\\V=h(X,Y)\end{cases}
              \Rightarrow (U,V)\sim f_{U,V}(u,v)\text{ 或 }p_{ij}.
          $$
    \item 混合型:
          $$
              X\sim p_i,\;Y\sim f_Y(y),\;
              \begin{cases}U=g(X,Y),\\V=h(X,Y)\end{cases}
              \Rightarrow (U,V)\sim q_{ij}.
          $$
\end{enumerate}


\section*{小结}
\begin{itemize}
    \item 多维 $\to$ 一维:常用分布函数法、换元法、最值法.
    \item 一维 $\to$ 多维:常由函数对映 $(U,V)=(g(X),h(X))$ 构造.
    \item 换元法关键在雅可比行列式 $J$.
    \item 最值函数分布为命题型常考,记忆:
          $$F_{\max}(z)=[F(z)]^n,\quad F_{\min}(z)=1-[1-F(z)]^n.$$
\end{itemize}
\chapter{数字特征}
\GAIchapter{大数定律与中心极限定理}

\section{判别或证明依概率收敛}
设随机变量 $X$ 与随机变量序列 $\{X_n\}$ $(n=1,2,3,\cdots)$,如果对任意的 $\varepsilon > 0$,有
$$\lim_{n\to\infty}P\{|X_n-X|\geqslant\varepsilon\}=0 \text{ 或 } \lim_{n\to\infty}P\{|X_n-X|<\varepsilon\}=1,$$
则称随机变量序列 $\{X_n\}$ 依概率收敛于随机变量 $X$,记为 $\lim_{n\to\infty}X_n=X(P)$ 或 $X_n\xrightarrow{P}X(n\to\infty)$.

\section{利用大数定律计算收敛值}
\begin{enumerate}
      \item 切比雪夫大数定律

            假设 $\{X_{n}\}(n=1,2,\cdots)$ 是相互独立的随机变量序列,如果方差$DX_i(i\geqslant1)$存在且一致有上界,即存在常数 $C$, 使 $DX_i\leq C$ 对 一 切  $i\geqslant 1$ 均 成 立 , 则  $\{ X_n\}$服从大数定律:$\frac1n\sum_{i=1}^{n}X_i\xrightarrow{p}\frac1n\sum_{i=1}^{n}EX_i$
      \item 伯努利大数定律

            假设$\mu_n$是$n$重伯努利试验中事件$A$发生的次数,在每次试验中事件$A$ 发 生 的 概 率 为 $p( 0< p< 1)$,则$\frac{\mu_n}n\xrightarrow{P}p$ ,即对任意 $\varepsilon>0$ ,有$\lim_{n\to\infty} P\left\{\left|\frac{\mu_n}{n}-p\right|<\varepsilon\right\}=1$
      \item 辛钦大数定律

            假设 $\{X_{n}\} (n=1,2,\cdots)$ 是独立同分布的随机变量序列,如果数学期望 $EX_{i}=\mu (i=1,2,\cdots)$ 存在,则$\frac{1}{n}\sum_{i=1}^{n}X_{i} \xrightarrow{p} \mu $, 即对任意$ \varepsilon > 0, $有$ \lim_{n \to \infty} P\left\{\left|\frac{1}{n}\sum_{i=1}^{n}X_{i} - \mu\right| < \varepsilon\right\} = 1 $.
      \item 考结论

            在满足一定条件时,大数定律都在讲同一个结论,即
            $$\frac{1}{n}\sum_{i=1}^{n}X_{i}\xrightarrow{P}E\biggl(\frac{1}{n}\sum_{i=1}^{n}X_{i}\biggr)\:.$$
\end{enumerate}


\section{用中心极限定理求概率}
\begin{enumerate}
      \item 列维-林德伯格定理

            假设$\{X_n\}(n=1,2,\cdotp\cdotp\cdotp)$是独立同分布的随机变量序列,如果$EX_i=\mu,DX_i=\sigma^2>0(i=1,2,\cdotp\cdotp\cdotp)$存在,则
            对任意的实数$x$,有
            $$\lim_{n\to\infty}P\left\{\frac{\sum_{i=1}^{n}X_{i}-n\mu}{\sqrt{n}\sigma}\leqslant x\right\}=\frac{1}{\sqrt{2\pi}}\int_{-\infty}^{x}\mathrm{e}^{-\frac{1}{2}t^{2}}\mathrm{d}t=\Phi(x)\:.$$
      \item 棣莫弗-拉普拉斯定理

            假设随机变量$Y_n\sim B(n,p)(0<p<1,n\geqslant1)$,则对任意实数 $x$,有
            $$\lim_{n\to\infty}P\left\{\frac{Y_{n}-np}{\sqrt{np(1-p)}}\leqslant x\right\}=\frac{1}{\sqrt{2\pi}}\int_{-\infty}^{x}\mathrm{e}^{-\frac{t^{2}}{2}}\mathrm{d}t=\Phi(x).$$
      \item 考结论

            设 $X_i$ 独立同分布于某一分布,期望、方差均存在,则当 $n \to \infty$ 时,$\sum_{i=1}^{n} X_i$ 服从正态分布,即对任意的 $X_i \sim F(\mu, \sigma^2)$,$\mu = EX_i$,$\sigma^2 = DX_i$,都有在 $n \to \infty$ 时,$\sum_{i=1}^{n} X_i \sim N(n\mu, n\sigma^2)$,$\frac{\sum_{i=1}^{n} X_i - n\mu}{\sqrt{n}\sigma} \sim N(0, 1)$,即
            $$\lim_{n \to \infty} P\left\{ \frac{\sum_{i=1}^{n} X_i - n\mu}{\sqrt{n}\sigma} \leqslant x \right\} = \Phi(x) \, .$$
\end{enumerate}


\chapter{统计量及其分布}
\section{统计量及其数字特征}
设 $X_{1}, X_{2}, \cdots, X_{n}$ 是来自总体 $X$ 的简单随机样本,则
\begin{enumerate}
    \item 样本均值 $\bar{X}=\frac{1}{n} \sum_{i=1}^{n} X_{i}$.
    \item 样本方差 $S^{2}=\frac{1}{n-1} \sum_{i=1}^{n}\left(X_{i}-\bar{X}\right)^{2}=\frac{1}{n-1}\left(\sum_{i=1}^{n} X_{i}^{2}-n \bar{X}^{2}\right)$.

          样本标准差 $S=\sqrt{\frac{1}{n-1} \sum_{i=1}^{n}\left(X_{i}-\bar{X}\right)^{2}}$.
    \item 样本 $k$ 阶原点矩 $A_{k}=\frac{1}{n} \sum_{i=1}^{n} X_{i}^{k}(k=1,2, \cdots)$.
    \item 样本 $k$ 阶中心矩 $B_{k}=\frac{1}{n} \sum_{i=1}^{n}\left(X_{i}-\bar{X}\right)^{k}(k=2,3, \cdots)$.
    \item 顺序统计量

          将样本 $X_{1}, X_{2}, \cdots, X_{n}$ 的 $n$ 个观测量按其取值从小到大的顺序排列,得
          $$X_{(1)} \leqslant X_{(2)} \leqslant \cdots \leqslant X_{(n)}.$$
          随机变量 $X_{(k)}(k=1,2, \cdots, n)$ 称作第 $k$ 顺序统计量,其中 $X_{(1)}$ 是最小观测量, $X_{(n)}$ 是最大观测量,即
          $$X_{(1)}=\min \left\{X_{1}, X_{2}, \cdots, X_{n}\right\}, \quad X_{(n)}=\max \left\{X_{1}, X_{2}, \cdots, X_{n}\right\}.$$
\end{enumerate}

\section{判别统计量的分布}
定义:统计量的分布称为抽样分布

\begin{enumerate}
    \item 正态分布
          \begin{enumerate}
              \item 概念

                    如果 $X$ 的概率密度为
                    $$f(x) = \frac{1}{\sqrt{2 \pi} \sigma} \mathrm{e}^{-\frac{1}{2} \left( \frac{x - \mu}{\sigma} \right)^2} \quad (-\infty < x < +\infty),$$
                    其中 $-\infty < \mu < +\infty$, $\sigma > 0$, 则称 $X$ 服从参数为 $(\mu, \sigma^2)$ 的正态分布或称 $X$ 为正态变量, 记为 $X \sim N(\mu, \sigma^2)$.
              \item 上$\alpha$分位数

                    若 $X \sim N(0, 1)$, $P\{X > \mu_\alpha\} = \alpha$ ( $0 < \alpha < 1$ ), 则称 $\mu_\alpha$ 为标准正态分布的上 $\alpha$ 分位数
              \item 性质

                    $f(x)$ 的图形关于直线 $x=\mu$对称,即$f(\mu-x)=f(\mu+x)$,并在$x=\mu$处有唯一最大值
                    $$f(\mu)=\frac{1}{\sqrt{2\pi}\sigma}.$$
                    通常称$\mu=0$ , $\sigma=1$时的正态分布$N(0,1)$为标准正态分布,记标准正态分布的概率密度为
                    $\varphi(x)=\frac{1}{\sqrt{2\pi}}\mathrm{e}^{-\frac{1}{2}x^{2}}$,分布函数为$Q(x)=\frac1{\sqrt{2\pi}}\int_{-\infty}^{x}\mathrm{e}^{-\frac{t^{2}}{2}}dt$ .显然$\varphi(x)$为偶函数,且有
                    $$\Phi(0)=\frac{1}{2},\Phi(-x)=1-\Phi(x).$$
          \end{enumerate}
    \item $\chi^2$分布
          \begin{enumerate}
              \item 概念

                    若随机变量 $X_{1}, X_{2}, \cdots, X_{n}$ 相互独立,且都服从标准正态分布,则随机变量 $X = \sum_{i=1}^{n} X_{i}^{2}$ 服从自由度为 $n$ 的 $\chi^{2}$ 分布,记为 $X \sim \chi^{2}(n)$.
              \item 上$\alpha$分位数

                    对给定的 $\alpha (0 < \alpha < 1)$,称满足
                    $$P\{\chi^{2} > \chi_{\alpha}^{2}(n)\} = \int_{\chi_{\alpha}^{2}(n)}^{+\infty} f(x) \, \mathrm{d}x = \alpha$$
                    的 $\chi_{\alpha}^{2}(n)$ 为 $\chi^{2}(n)$ 分布的上 $\alpha$ 分位数(见图). 对于不同的 $\alpha, n$,$\chi^{2}(n)$ 分布上 $\alpha$ 分位数可通过查表求得.
              \item 性质
                    \begin{enumerate}
                        \item  若 $X_{1} \sim \chi^{2}(n_{1})$,$X_{2} \sim \chi^{2}(n_{2})$,$X_{1}$ 与 $X_{2}$ 相互独立,则
                              $$
                                  X_{1} + X_{2} \sim \chi^{2}(n_{1} + n_{2}).$$
                              此结论可推广至有限多个随机变量的和.
                        \item $若X\sim\chi^{2}(n)$,则$EX=n,DX=2n.$
                    \end{enumerate}
          \end{enumerate}
    \item $t$ 分布
          \begin{enumerate}
              \item 概念
                    设随机变量 $X \sim N(0,1)$, $Y \sim \chi^2(n)$, $X$ 与 $Y$ 相互独立, 则随机变量 $t = \frac{X}{\sqrt{Y/n}}$ 服从自由度为 $n$ 的 $t$ 分布, 记为 $t \sim t(n)$.
              \item 上$\alpha$分位数

                    对给定的 $\alpha(0 < \alpha < 1)$, 称满足
                    $$P\{t > t_{\alpha}(n)\} = \alpha$$
                    的 $t_{\alpha}(n)$ 为 $t(n)$ 分布的上 $\alpha$ 分位数.
              \item 性质
                    \begin{enumerate}
                        \item  $t$ 分布概率密度 $f(x)$ 的图形关于 $x = 0$ 对称, 因此
                              $$Et = 0 \quad (n \geqslant 2).$$
                        \item 由 $t$ 分布概率密度 $f(x)$ 图形的对称性, 知 $P\{t > -t_{\alpha}(n)\} = P\{t > t_{1-\alpha}(n)\}$, 故 $t_{1-\alpha}(n) = -t_{\alpha}(n)$. 当 $\alpha$ 值在表中没有时, 可用此式求得上 $\alpha$ 分位数.
                    \end{enumerate}
          \end{enumerate}
    \item $F$ 分布
          \begin{enumerate}
              \item 概念

                    设随机变量 $X \sim \chi^2(n_1)$,$Y \sim \chi^2(n_2)$,且 $X$ 与 $Y$ 相互独立,则 $F = \frac{X / n_1}{Y / n_2}$ 服从自由度为 $(n_1, n_2)$ 的 $F$ 分布,记为 $F \sim F(n_1, n_2)$,其中 $n_1$ 称为第一自由度,$n_2$ 称为第二自由度.$F$ 分布的概率密度 $f(x)$ 的图形.
              \item 上$\alpha$分位数

                    对给定的 $\alpha (0 < \alpha < 1)$,称满足
                    $$P\{F > F_\alpha(n_1, n_2)\} = \alpha$$
                    的 $F_\alpha(n_1, n_2)$ 为 $F(n_1, n_2)$ 分布的上 $\alpha$ 分位数.
              \item 性质
                    \begin{enumerate}
                        \item 若 $F \sim F(n_1, n_2)$,则 $\frac{1}{F} \sim F(n_2, n_1)$.
                        \item $F_{1-\alpha}(n_1, n_2) = \frac{1}{F_\alpha(n_2, n_1)}$.常用来求 $F$ 分布表中未列出的上 $\alpha$ 分位数,显然,有些特殊值可直接得出,如 $1-\alpha = \alpha$,$n_1 = n_2 = n$ 时,有 $F_{0.5}(n, n) = \frac{1}{F_{0.5}(n, n)}$,且 $F_{0.5}(n, n) > 0$,故 $F_{0.5}(n, n) = 1$.
                        \item 若 $t \sim t(n)$,则 $t^2 \sim F(1, n)$.
                    \end{enumerate}
          \end{enumerate}
\end{enumerate}
\section{用正态总体下的常用结论判别分布、计算概率}
设 $X_{1}, X_{2}, \cdots, X_{n}$ 是取自正态总体 $N(\mu, \sigma^{2})$ 的一个样本, $\bar{X}$ , $S^{2}$ 分别是样本均值和样本方差,则

\begin{enumerate}
    \item $\bar{X} \sim N\left(\mu, \frac{\sigma^{2}}{n}\right)$ ,即 $\frac{\bar{X}-\mu}{\frac{\sigma}{\sqrt{n}}} = \frac{\sqrt{n}(\bar{X}-\mu)}{\sigma} \sim N(0,1)$
    \item $\frac{1}{\sigma^{2}} \sum_{i=1}^{n}(X_{i}-\mu)^{2} \sim \chi^{2}(n)$ ;
    \item $\frac{(n-1)S^{2}}{\sigma^{2}} = \sum_{i=1}^{n}\left(\frac{X_{i}-\bar{X}}{\sigma}\right)^{2} \sim \chi^{2}(n-1)$ ( $\mu$ 未知,在 “2.” 中用 $\bar{X}$ 替代 $\mu$ );
    \item $\bar{X}$ 与 $S^{2}$ 相互独立, $\frac{\sqrt{n}(\bar{X}-\mu)}{S} \sim t(n-1)$ ( $\sigma$ 未知,在 “1.” 中用 $S$ 替代 $\sigma$ ). 进一步有
          $$\frac{n(\bar{X}-\mu)^{2}}{S^{2}} \sim F(1, n-1).$$
\end{enumerate}


\chapter{参数估计与假设检验}
\section{求点估计、作评价}
\begin{enumerate}
      \item 矩估计
            \begin{enumerate}
                  \item 对于一个参数$\begin{cases}\text{a.用一阶矩建立方程:令}\bar{X}=EX;\\\\\text{b.若“a”不能用,用二阶矩建立方程:令}\frac1n\sum_{i=1}^nX_i^2=E(X^2)\:.\end{cases}$

                        一个方程解出一个参数即可作为矩估计.
                  \item 对于两个参数,用一阶矩与二阶矩建立两个方程,即$\bar{X}=EX$ 与$\frac1n\sum_{i=1}^nX_i^2=E(X^2)$,两个方程解出两
                        个参数即可作为矩估计.
            \end{enumerate}
      \item 最大似然估计

            对未知参数$\theta$进行估计时,在该参数可能的取值范围$I$内选取,使“样本获此观测值$x_1,x_2,\cdotp\cdotp\cdotp,x_n$”的概率最大的参数值$\hat{\theta}$作为$\theta$的估计,这样选定的$\hat{\theta}$最有利于$x_1,x_2,\cdotp\cdotp\cdotp,x_n$的出现,即“参数$\theta$为多少时,观测值出现的概率最大”.
            \begin{enumerate}
                  \item 写似然函数$L(x_{1},x_{2},\cdots,x_{n};\theta)=\left\{\begin{array}{l}{\prod_{i=1}^{n}p(x_{i};\theta)\ (\text{离散型;})}\\{\prod_{i=1}^{n}f(x_{i};\theta)\ (\text{连续型.})}\end{array}\right.$
                  \item 求参数$\left\{\begin{array}{l}\text{若似然函数有驻点,则令}\frac{dL}{d\theta}=0\text{或}\frac{d(\ln L)}{d\theta}=0\text{,解出}\hat{\theta};\\\text{若似然函数无驻点(单调),则用定义求}\hat{\theta};\\\text{若似然函数为常数,则用定义求}\hat{\theta},\text{此时}\hat{\theta}\text{不唯一}.\end{array}\right.$
                  \item 最大似然估计量的不变性原则.

                        设$\hat{\theta}$是总体分布中未知参数$\theta$的最大似然估计,函数$u=u(\theta)$具有单值的反函数$\theta=\theta(u)$,则$\hat{u}=u(\hat{\theta})$是$u(\theta)$的最大似然估计.
                  \item 双总体的最大似然估计.
            \end{enumerate}
      \item 估计量的评价
            \begin{enumerate}
                  \item 无偏性.

                        对于估计量 $\hat{\theta}$,若 $E\hat{\theta}=\theta$,则称 $\hat{\theta}$ 为 $\theta$ 的无偏估计量.
                  \item 有效性.

                        若 $E\hat{\theta}_{1}=\theta$,$E\hat{\theta}_{2}=\theta$,即 $\hat{\theta}_{1}$,$\hat{\theta}_{2}$ 均是 $\theta$ 的无偏估计量,当 $D\hat{\theta}_{1}<D\hat{\theta}_{2}$ 时,称 $\hat{\theta}_{1}$ 比 $\hat{\theta}_{2}$ 有效.
                  \item 一致性(相合性).(只针对大样本$n\rightarrow\infty$)

                        若 $\hat{\theta}$ 为 $\theta$ 的估计量,则对任意 $\varepsilon>0$,有
                        $$\lim_{n\rightarrow\infty}P\{\left|\hat{\theta}-\theta\right|\geqslant\varepsilon\}=0,$$
                        或
                        $$
                              \lim_{n\rightarrow\infty}P\{\left|\hat{\theta}-\theta\right|<\varepsilon\}=1,$$
                        即当 $\hat{\theta}\stackrel{P}{\longrightarrow}\theta$ 时,称 $\hat{\theta}$ 为 $\theta$ 的一致(或相合)估计.
            \end{enumerate}
\end{enumerate}

\section{作区间估计、假设检验、求两类错误}
\begin{enumerate}
      \item 区间估计
            \begin{enumerate}
                  \item 概念.

                        设$\theta$是总体$X$的分布函数的一个未知参数,对于给定$\alpha(0<\alpha<1)$ ,如果由样本$X_1,X_2,\cdotp\cdotp\cdotp,X_n$确定的
                        两个统计量$\hat{\theta } _1= \hat{\theta } _1( X_1, X_2, \cdotp \cdotp \cdotp , X_n)$ , $\hat{\theta } _2= \hat{\theta } _2( X_1, X_2, \cdotp \cdotp \cdotp , X_n)$,使
                        $$P\{\hat{\theta}_1(X_1,X_2,\cdotp\cdotp\cdotp,X_n)<\theta<\hat{\theta}_2(X_1,X_2,\cdotp\cdotp\cdotp,X_n)\}=1-\alpha\:,$$
                        则称随机区间($\hat{\theta}_{\mathrm{l}},\hat{\theta}_{\mathrm{2}})$是$\theta$的置信度为 $1-\alpha$的置信区间,$\hat{\theta}_{\mathrm{l}}$ 和$\hat{\theta}_{\mathrm{2}}$分别称为$\theta$的置信度为 $1-\alpha$的双侧置信区间的置信下限和置信上限,$1-\alpha$称为置信度或置信水平,$\alpha$称为显著性水平.如果$P\{\theta<\hat{\theta}_1\}=P\{\theta>\hat{\theta}_2\}=\frac\alpha2,则称$
                        这种置信区间为等尾置信区间.
                  \item 单个正态总体均值和方差的置信区间.

                        设 $X \sim N(\mu, \sigma^2)$,从总体 $X$ 中抽取样本 $X_1, X_2, \cdots, X_n$,样本均值为 $\bar{X}$,样本方差为 $S^2$.
                        \begin{enumerate}
                              \item  $\sigma^2$ 已知,$\mu$ 的置信水平是 $1-\alpha$ 的置信区间为
                                    $$\left( \bar{X} - \frac{\sigma}{\sqrt{n}} z_{\frac{\alpha}{2}}, \bar{X} + \frac{\sigma}{\sqrt{n}} z_{\frac{\alpha}{2}} \right).$$
                              \item $\sigma^2$ 未知,$\mu$ 的置信水平是 $1-\alpha$ 的置信区间为
                                    $$\left( \bar{X} - \frac{S}{\sqrt{n}} t_{\frac{\alpha}{2}}(n-1), \bar{X} + \frac{S}{\sqrt{n}} t_{\frac{\alpha}{2}}(n-1) \right).$$
                              \item $\mu$ 已知,$\sigma^2$ 的置信水平是 $1-\alpha$ 的置信区间为
                                    $$\left( \frac{\sum_{i=1}^{n} (X_i - \mu)^2}{\chi^2_{\frac{\alpha}{2}}(n)}, \frac{\sum_{i=1}^{n} (X_i - \mu)^2}{\chi^2_{1-\frac{\alpha}{2}}(n)} \right).$$
                              \item $\mu$ 未知,$\sigma^2$ 的置信水平是 $1-\alpha$ 的置信区间为
                                    $$\left( \frac{(n-1)S^2}{\chi^2_{\frac{\alpha}{2}}(n-1)}, \frac{(n-1)S^2}{\chi^2_{1-\frac{\alpha}{2}}(n-1)} \right).$$
                        \end{enumerate}

            \end{enumerate}
      \item 假设检验
            \begin{enumerate}
                  \item 概念.

                        关于总体(分布中的未知参数,分布的类型、特征、相关性、独立性$\cdots\cdots)$ 的每一种论断(“看法”) 称为统计假设,然后根据样本观察数据或试验结果所提供的信息去推断(检验)这个“看法”(即假设) 是否成立,这类统计推断问题称为假设检验.
                  \item 原假设与备择假设.

                        常常把没有充分理由不能轻易否定的假设取为原假设(基本假设或零假设),记为$H_{0}$,将其否定的
                        陈述(假设)称为对立假设或备择假设,记为$H_{\mathrm{l}}.$
                  \item 小概率原理与显著性水平.
                        \begin{enumerate}
                              \item 小概率原理

                                    对假设进行检验的基本思想是采用某种带有概率性质的反证法.这种方法的依据是小概率原理—— 概率很接近于0 的事件在一次试验或观察中认为备择假设不会发生.若小概率事件发生了,则拒绝原假设.
                              \item 显著性水平$\alpha$

                                    小概率事件中“小概率”的值没有统一规定,通常是根据实际问题的要求,规定一个界限$\alpha(0<\alpha<1)$,当一个事件的概率不大于 $\alpha$时,即认为它是小概率事件.在假设检验问题中,$\alpha$称为显著性水平,通常取$\alpha=0.1,0.05,0.01$等.
                        \end{enumerate}
                  \item 正态总体下的六大检验及拒绝域.
                        \begin{enumerate}
                              \item $\sigma^2$已知,$\mu$未知.$H_0:\mu=\mu_0,H_1:\mu\neq\mu_0$,则拒绝域为$\left(-\infty,\mu_{0}-\frac{\sigma}{\sqrt{n}}z_{\frac{\alpha}{2}}\right]\cup\left[\mu_{0}+\frac{\sigma}{\sqrt{n}}z_{\frac{\alpha}{2}},+\infty\right)$.
                              \item $\sigma^2$未知,$\mu$未知.$H_0:\mu=\mu_0,H_1:\mu\neq\mu_0$,则拒绝域为
                                    $$\left(-\infty,\mu_0-\frac{S}{\sqrt{n}}t_{\frac{\alpha}{2}}(n-1)\right]\cup\left[\mu_0+\frac{S}{\sqrt{n}}t_{\frac{\alpha}{2}}(n-1),+\infty\right).$$
                              \item $\sigma^{2}$已知,$\mu$未知.$H_{0}$:$\mu\leqslant\mu_{0}$(或写$\mu=\mu_{0}$),$H_{1}$:$\mu>\mu_{0}$,则拒绝域为$\left[\mu_{0}+\frac{\sigma}{\sqrt{n}}z_{\alpha},+\infty\right)$.
                              \item $\sigma^{2}$已知,$\mu$未知.$H_{0}$:$\mu\geqslant\mu_{0}$(或写$\mu=\mu_{0}$),$H_{1}$:$\mu<\mu_{0}$,则拒绝域为$\left[-\infty,\mu_{0}-\frac{\sigma}{\sqrt{n}}z_{\alpha}\right)$.
                              \item $\sigma^{2}$未知,$\mu$未知.$H_{0}$:$\mu\leqslant\mu_{0}$(或写$\mu=\mu_{0}$),$H_{1}$:$\mu>\mu_{0}$, 则拒绝域为$\left[\mu_{0}+\frac{S}{\sqrt{n}}t_{\alpha}(n-1),+\infty\right)$.
                              \item $\sigma^{2}$未知,$\mu$未知.$H_{0}$:$\mu\geqslant\mu_{0}$(或写$\mu=\mu_{0}$),$H_{1}$:$\mu<\mu_{0}$,则拒绝域为$\left(-\infty,\mu_{0}-\frac{S}{\sqrt{n}}t_{\alpha}(n-1)\right]$.
                        \end{enumerate}
            \end{enumerate}
      \item 两类错误

            第一类错误(“弃真”):若$H_0$为真,按检验法则,否定$H_{0}$,此时犯了“弃真”的错误,这种错误称为第一类错误,犯第一类错误的概率为 $\alpha=P\{$拒绝$H_0|H_0$为真 $\}.$

            第二类错误(“取伪”):若$H_0$不真,按检验法则,接受$H_0$,此时犯了“取伪”的错误,这种错误称为第二类错误,犯第二类错误的概率为 $\beta=P\{$ 接受$H_0|H_0$为假 $\}.$
\end{enumerate}




\GAIgroupsancheck
\XIANgroupsancheck
\GAOgroupsancheck

\makeatletter
\let\chapter\@std@chapter
\let\@std@chapter\relax
\makeatother

\backmatter
{\small
  \printindex
  \printindex[sym]
}

\end{document}
