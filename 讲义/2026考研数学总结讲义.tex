\makeatletter
   \def\input@path{{..}} % 搜索上层目录的 LALUbook
\makeatother

\documentclass[
  UTF8
    % colors = false,
    geometry = a4,
]{LALUbook}

\usepackage{mathdots}
\usepackage{booktabs} % Excel 导出的大表格
\usepackage{rotating}
\usepackage{extarrows}
\usepackage{blkarray}
\usepackage{cases}

\usepackage{float}
\usepackage{diagbox}
\usepackage{caption}

\usepackage{pgfplots}
\usetikzlibrary{cd, arrows, arrows.meta, calc, intersections, decorations.pathreplacing, patterns, decorations.markings,angles,quotes, graphs, positioning, shapes.geometric}
\pgfplotsset{compat=newest}

\usepackage[xindy, splitindex]{imakeidx}
\makeindex[
    columns=1,
    program=truexindy,
    intoc=true,
    options=-M texindy -I xelatex -C utf8,
    title={名词索引}
] % 名词索引
\makeindex[
    columns=3,
    program=truexindy,
    intoc=true,
    options=-M numeric-sort -M latex -M latex-loc-fmts -M makeindex -I xelatex -C utf8,
    name=sym,
    title={符号索引}
] % 符号索引


% OPD 系列
\newcommand{\OOne}{\textcolor{blue}{\textbf{$O$ (盯住目标)}}}
\newcommand{\DOne}{\textcolor{blue}{\textbf{$D_1$ (常规操作)}}}
\newcommand{\DTwo}{\textcolor{blue}{\textbf{$D_2$ (脱胎换骨)}}}
\newcommand{\DThree}{\textcolor{blue}{\textbf{$D_3$ (移花接木)}}}
\newcommand{\DFour}{\textcolor{blue}{\textbf{$D_4$ (可圈可点)}}}
\newcommand{\DFourThree}{\textcolor{blue}{\textbf{$D_{43}$ (数形结合)}}}
\newcommand{\DTwoOne}{\textcolor{blue}{\textbf{$D_{21}$ (观察研究对象)}}}
\newcommand{\DTwoTwo}{\textcolor{blue}{\textbf{$D_{22}$ (转换等价表述)}}}
\newcommand{\DTwoThree}{\textcolor{blue}{\textbf{$D_{23}$ (化归经典形式)}}}
\newcommand{\POne}{\textcolor{blue}{\textbf{$P_{1}$ (常规思路)}}}
\newcommand{\POneOne}{\textcolor{blue}{\textbf{$P_{11}$ (正向思路)}}}
\newcommand{\POneTwo}{\textcolor{blue}{\textbf{$P_{12}$ (反向思路)}}}
\newcommand{\POneThree}{\textcolor{blue}{\textbf{$P_{13}$ (双向思路)}}}
\newcommand{\PTwo}{\textcolor{blue}{\textbf{$P_{2}$ (反证思路)}}}
\newcommand{\PThree}{\textcolor{blue}{\textbf{$P_{3}$ (数学归纳)}}}
\newcommand{\PFour}{\textcolor{blue}{\textbf{$P_{4}$ (逆否思路)}}}
\newfontfamily\RomanSymbols{Arial Unicode MS}

\newtcbtheorem[number within=section]{detail}{细节}{laluthmstyle={red}}{det}
\newtcbtheorem[number within=section]{idea}{思路}{laluthmstyle={teal}}{ide}

% 内容总结
\newenvironment{summary}{%
    \hypersetup{bookmarksnumbered=false}%
    \titleformat{\subsection}[block]{\centering\heiti\Large}{}{1em}{}%
    \phantomsection%
    \subsection*{内容总结}%
}{}


% 标题格式
% \chapter   正常专题
% \Wchapter 高数强化专题
% \Xchapter 线代强化专题
% \Gchapter 概率强化专题
\newcounter{GAOchapter}
\newcounter{XIANchapter}
\newcounter{GAIchapter}

\makeatletter
\newcommand*{\GAOgroupsancheck}{%
  \expandafter\@ifundefined{@exist@GAOchapter@\arabic{chapter}}%
    {}%
    {\endgroup}%
}
\newcommand*{\XIANgroupsancheck}{%
  \expandafter\@ifundefined{@exist@XIANchapter@\arabic{chapter}}%
    {}%
    {\endgroup}%
}
\newcommand*{\GAIgroupsancheck}{%
  \expandafter\@ifundefined{@exist@GAIchapter@\arabic{chapter}}%
    {}%
    {\endgroup}%
}

\let\@std@chapter\chapter
\renewcommand*{\chapter}{%
  \GAOgroupsancheck%
  \XIANgroupsancheck%
  \GAIgroupsancheck%
  \@std@chapter%
}

\newcommand{\GAOchapter}[1]{%
%   \LUgroupsancheck
  \begingroup
  \addtocounter{chapter}{-1}
  \refstepcounter{GAOchapter}
  % 去掉希腊字母,只保留强化专题 + 中文编号
  \renewcommand*{\thechapter}{\arabic{GAOchapter}}
%   \renewcommand*{\theHchapter}{LU.\arabic{LUchapter}}
  \ctexset{
      chapter={
        format={\centering\Huge\bfseries},
        name={高数专题,},      % “强化专题”前缀
        number={\zhnumber{\arabic{GAOchapter}}} % 中文数字
      },
  }
  \csname @std@chapter\endcsname{#1}
  \expandafter\xdef\csname @exist@GAOchapter@\arabic{chapter}\endcsname{\null}
}


\newcommand{\XIANchapter}[1]{%
%   \LUgroupsancheck
  \begingroup
  \addtocounter{chapter}{-1}
  \refstepcounter{XIANchapter}
  % 去掉希腊字母,只保留强化专题 + 中文编号
  \renewcommand*{\thechapter}{\arabic{XIANchapter}}
%   \renewcommand*{\theHchapter}{LU.\arabic{LUchapter}}
  \ctexset{
      chapter={
        format={\centering\Huge\bfseries},
        name={线代专题,},      % “强化专题”前缀
        number={\zhnumber{\arabic{XIANchapter}}} % 中文数字
      },
  }
  \csname @std@chapter\endcsname{#1}
  \expandafter\xdef\csname @exist@XIANchapter@\arabic{chapter}\endcsname{\null}
}


\newcommand{\GAIchapter}[1]{%
%   \LUgroupsancheck
  \begingroup
  \addtocounter{chapter}{-1}
  \refstepcounter{GAIchapter}
  % 去掉希腊字母,只保留强化专题 + 中文编号
  \renewcommand*{\thechapter}{\arabic{GAIchapter}}
%   \renewcommand*{\theHchapter}{LU.\arabic{LUchapter}}
  \ctexset{
      chapter={
        format={\centering\Huge\bfseries},
        name={概率专题,},      % “强化专题”前缀
        number={\zhnumber{\arabic{GAIchapter}}} % 中文数字
      },
  }
  \csname @std@chapter\endcsname{#1}
  \expandafter\xdef\csname @exist@GAOchapter@\arabic{chapter}\endcsname{\null}
}
\makeatother

\ctexset{
    chapter={format={\centering\Huge\bfseries},name={强化专题,},number={\zhnumber{\arabic{chapter}}}},
    % chapter={format={\centering\Huge\bfseries},name={第,讲},number=\arabic{chapter}},
    section={format={\raggedright\Large\bfseries},name={,},number={\thechapter.\arabic{section}}},
    subsection={format={\raggedright\large\bfseries},name={,},number={\thesection.\arabic{subsection}}},
    subsubsection={format={\raggedright\normalsize\bfseries},name={,},number={\thesubsection.\arabic{subsubsection}}},
}

\title{\heiti 临沂大学 2025--2026 学年 \\ 2026考研数学总结讲义}

\AtEndPreamble{\hypersetup{
    hypertexnames=true,
    pdfauthor={林子立},
    pdftitle={2026考研数学总结讲义},
}}

\begin{document}

\title{2026考研数学总结讲义}
\author{林子立}
\date{\today}
\maketitle

% \songti
\pagenumbering{Roman}
\clearpage

\pdfbookmark[0]{目录}{contents}
\tableofcontents

\addtolength{\parskip}{.5em}

\mainmatter
\GAOchapter{函数极限与连续}

\section{判定类型,做好计算}

\begin{enumerate}
    %-------------------------------------------------------
    \item \textbf{未定式整体判定}

          一般地,见到以下形式的极限:
          \[
              \frac{?}{0}, \quad \frac{0}{?}, \quad \frac{\infty}{?}, \quad
              \frac{?}{\infty}, \quad ?\cdot\infty, \quad
              \infty\cdot?, \quad \infty-?, \quad \infty^{?}, \quad ?^{\infty},
          \]
          可直接判断它们分别对应的七种未定式类型:
          \[
              \frac{0}{0}, \quad \frac{\infty}{\infty}, \quad 0\cdot\infty, \quad
              \infty-\infty, \quad \infty^{0}, \quad 0^{0}, \quad 1^{\infty}.
          \]
          若题设形式不属于这七类,则通常不是未定式问题。

          %-------------------------------------------------------
    \item \textbf{未定式局部判定}

          例如:
          \[
              \lim_{n\to\infty}\frac{1+x}{1+nx^{2n}}=
              \begin{cases}
                  0,   & x=\pm1, \\
                  1+x, & |x|<1,  \\
                  0,   & |x|>1.
              \end{cases}
          \]
          关键在于判定局部项 $nx^{2n}$ 的极限性质。常见局部极限总结如下:

          \[
              \lim_{n\to\infty}|x|^n=
              \begin{cases}
                  \infty, & |x|>1, \\
                  1,      & |x|=1, \\
                  0,      & |x|<1;
              \end{cases}
              \quad
              \lim_{x\to0^+}x^a=
              \begin{cases}
                  0,       & a>0, \\
                  1,       & a=0, \\
                  +\infty, & a<0;
              \end{cases}
          \]
          \[
              \lim_{n\to\infty}nx^{2n}=
              \begin{cases}
                  +\infty, & |x|\ge1, \\
                  0,       & |x|<1;
              \end{cases}
              \quad
              \lim_{n\to\infty}e^{nx}=
              \begin{cases}
                  +\infty, & x>0, \\
                  1,       & x=0, \\
                  0,       & x<0;
              \end{cases}
          \]
          \[
              \lim_{n\to\infty}n^x=
              \begin{cases}
                  +\infty, & x>0, \\
                  1,       & x=0, \\
                  0,       & x<0.
              \end{cases}
          \]

          %-------------------------------------------------------
    \item \textbf{常用的无穷小量阶的比较}

          \begin{enumerate}
              \item \textbf{普通函数型:}

                    当 $x\to0$ 时,
                    \[
                        \sin x \sim x, \quad \tan x \sim x, \quad
                        \arcsin x \sim x, \quad \arctan x \sim x,
                    \]
                    \[
                        e^x - 1 \sim x, \quad \ln(1+x) \sim x, \quad
                        \ln(x + \sqrt{1+x^2}) \sim x,
                    \]
                    \[
                        a^x - 1 = e^{x\ln a}-1 \sim x\ln a \ (a>0,a\ne1),
                        \quad 1 - \cos x \sim \tfrac{1}{2}x^2,
                    \]
                    \[
                        1 - \cos^{\alpha}x \sim \tfrac{\alpha}{2}x^2 \ (\alpha\ne0),
                        \quad (1+x)^{\alpha}-1 \sim \alpha x,
                        \quad (1+x)^x - 1 \sim x^2.
                    \]

              \item \textbf{差函数型:}

                    当 $x\to0$ 时,
                    \[
                        x - \sin x \sim \tfrac{1}{6}x^3, \quad
                        x - \arcsin x \sim -\tfrac{1}{6}x^3, \quad
                        x - \tan x \sim -\tfrac{1}{3}x^3, \quad
                        x - \arctan x \sim \tfrac{1}{3}x^3,
                    \]
                    \[
                        x - \ln(1+x) \sim \tfrac{1}{2}x^2,
                        \quad e^x - 1 - x \sim \tfrac{1}{2}x^2.
                    \]

                    可通过恒等变形“创造”差函数,如:
                    \[
                        \begin{cases}
                            x - \ln(1+\tan x) = (x - \tan x) + (\tan x - \ln(1+\tan x)); \\[4pt]
                            \sin x + \ln(1 - \sin x) = -[-\sin x - \ln(1 - \sin x)];     \\[4pt]
                            f(x) - \tan x = (f(x) - x) + (x - \tan x).
                        \end{cases}
                    \]

              \item \textbf{复合函数型:}

                    若 $f(x)\sim ax^m,\ g(x)\sim bx^n$ 且 $ab\ne0$,则
                    \[
                        f[g(x)] \sim ab^m x^{mn}.
                    \]

              \item \textbf{变上限积分型:}
                    \begin{enumerate}
                        \item 若 $f(x)\sim ax^m$,则
                              \[
                                  \int_0^x f(t)\,\mathrm{d}t \sim \int_0^x a t^m\,\mathrm{d}t.
                              \]
                        \item 若 $\lim_{x\to0}f(x)=A\ne0$, $h(x)\to0$,则
                              \[
                                  \int_0^{h(x)} f(t)\,\mathrm{d}t \sim A\,h(x).
                              \]
                    \end{enumerate}

              \item \textbf{复合+变上限积分型:}
                    \[
                        f(x)\sim ax^m,\ g(x)\sim bx^n
                        \Rightarrow
                        \int_0^{g(x)}f(t)\,\mathrm{d}t \sim \int_0^{bx^n}a t^m\,\mathrm{d}t.
                    \]

              \item \textbf{带头大哥型:}

                    若 $\alpha=o(\beta)$,则
                    \[
                        \textcircled{1}\ \alpha+\beta\sim\beta, \qquad
                        \textcircled{2}\ \alpha+\beta\text{ 与 }\beta\text{同号}, \qquad
                        \textcircled{3}\ \alpha\beta=o(\beta^2).
                    \]
          \end{enumerate}

          %-------------------------------------------------------
    \item \textbf{常用无穷大量阶的比较}

          \[
              \begin{cases}
                  \text{当 }x\to+\infty: & \ln^p x \ll x^q \ll a^x \ll x^x,        \\[4pt]
                  \text{当 }n\to\infty:  & \ln^p n \ll n^q \ll a^n \ll n! \ll n^n,
              \end{cases}\quad (p,q>0,a>1)
          \]
          因此:
          \[
              \lim_{n\to\infty}\frac{\ln^p n}{n^q}=0,\quad
              \lim_{n\to\infty}\frac{n^q}{a^n}=0,\quad
              \lim_{n\to\infty}\frac{a^n}{n!}=0,\quad
              \lim_{n\to\infty}\frac{n!}{n^n}=0.
          \]

          %-------------------------------------------------------
    \item \textbf{涉及 $\infty$ 的计算问题}

          关于 $\infty-\infty$ 型,注意以下四点:

          \begin{enumerate}
              \item 若 $f(x)$ 在 $|x|$ 足够大时有定义,则
                    \[
                        \lim_{x\to\infty}f(x) \text{ 存在}
                        \iff
                        \lim_{x\to+\infty}f(x) = \lim_{x\to-\infty}f(x).
                    \]

              \item 若出现差式 $f(x)-f(x)$(如三角、对数、反三角函数差),
                    可用\textbf{拉格朗日中值定理}变形再求极限。

              \item 若出现幂次差:
                    \[
                        [f_1(x)]^{g(x)}-[f_2(x)]^{g(x)} = [f_2(x)]^{g(x)}
                        \left(\left[\frac{f_1(x)}{f_2(x)}\right]^{g(x)}-1\right),
                    \]
                    \[
                        [f(x)]^{g_1(x)}-[f(x)]^{g_2(x)} = [f(x)]^{g_2(x)}
                        \left([f(x)]^{g_1(x)-g_2(x)}-1\right).
                    \]

              \item 若见 $\lim_{x\to\infty}[f(x)-ax]$,
                    常通过\textbf{恒等变形}将其改写为乘除形式处理。
          \end{enumerate}
\end{enumerate}

\section{判定连续与间断}

\begin{enumerate}
    %---------------------------------------------------
    \item \textbf{常见备选点判定}

          常见函数的无定义点或分段点如下表所示:

          \begin{enumerate}
              \item $\displaystyle \mathrm{e}^{\frac{1}{x}} \Rightarrow x=0$ 为无定义点;
              \item $\displaystyle \frac{1}{\int_{1}^{x}|\sin t|\,\mathrm{d}t} \Rightarrow x=\pm1$ 为无定义点;
              \item $\displaystyle \frac{1}{\sin x} \Rightarrow x=k\pi \ (k=0,\pm1,\pm2,\cdots)$ 为无定义点;
              \item $\displaystyle \frac{1}{\arctan x} \Rightarrow x=0$ 为无定义点;
              \item $\displaystyle \frac{1}{\tan\!\left(x-\frac{\pi}{4}\right)},\ 0<x<2\pi
                        \Rightarrow x=\frac{\pi}{4},\frac{3\pi}{4},\frac{5\pi}{4},\frac{7\pi}{4}$ 为无定义点;
              \item $\displaystyle \frac{1}{|x|(x^2-1)} \Rightarrow x=0,\pm1$ 为无定义点;
              \item $[x] \Rightarrow x=n\ (n=0,\pm1,\pm2,\cdots)$ 为分段点;
              \item $\displaystyle |x|^{\frac{1}{(1-x)(x-2)}} \Rightarrow x=0,1,2$ 为无定义点。
          \end{enumerate}

          \vspace{0.5em}
          \textit{技巧提示:}
          若题目中出现分母、根号、对数或分段定义函数,应先检查其定义域边界与分段点,这些往往是可能的间断点。

          %---------------------------------------------------
    \item \textbf{计算三个关键值}

          判定连续性需依次计算以下三个值(或判断其是否存在):
          \[
              \lim_{x\to x_0^-}f(x), \quad
              \lim_{x\to x_0^+}f(x), \quad
              f(x_0).
          \]
          然后进行比较。

          %---------------------------------------------------
    \item \textbf{根据定义作出结论}

          若以上三者不全相等或有不存在的情况,按下列规则分类:

          \begin{enumerate}
              \item \textbf{跳跃间断点(第一类间断)}:
                    \[
                        \lim_{x\to x_0^-}f(x) \neq \lim_{x\to x_0^+}f(x),
                    \]
                    且左右极限都存在。

              \item \textbf{可去间断点}:
                    \[
                        \lim_{x\to x_0}f(x) \text{ 存在}, \quad
                        \text{但 } f(x_0)\text{ 未定义或 }\lim f(x)\neq f(x_0).
                    \]
                    若通过重新定义 $f(x_0)=\lim f(x)$ 可使函数连续。

              \item \textbf{无穷间断点}:
                    \[
                        \lim_{x\to x_0^\pm}f(x)=\pm\infty.
                    \]
                    函数趋于无穷大,图像在此处“竖直渐近”。

              \item \textbf{振荡间断点(第二类间断)}:
                    \[
                        \lim_{x\to x_0}f(x) \text{ 不存在且无穷振荡}.
                    \]
                    例如 $\sin\frac{1}{x}$ 在 $x=0$ 处。
          \end{enumerate}
\end{enumerate}
\section{研究 $x \to \cdot$ 时 $f(x)$ 的微观性态}

\begin{enumerate}
    %--------------------------
    \item \textbf{定义法}

          极限的 $\varepsilon$–$\delta$ 定义:
          \[
              \lim_{x \to x_0} f(x) = A
              \;\Leftrightarrow\;
              \forall \varepsilon > 0,\ \exists \delta > 0,\
              \text{当 } 0 < |x - x_0| < \delta \text{ 时},\ |f(x) - A| < \varepsilon.
          \]

          ✅ \textit{说明:}
          表明当 $x$ 无限接近 $x_0$ 时,$f(x)$ 可以被控制在 $A$ 的任意邻域内。

          %--------------------------
    \item \textbf{局部保号性}

          \begin{enumerate}
              \item 若 $f(x) \to A \ (x \to x_0)$ 且 $A > 0$(或 $A < 0$),
                    则存在 $\delta > 0$,使得当 $0 < |x - x_0| < \delta$ 时,
                    $f(x) > 0$(或 $f(x) < 0$)。

              \item 若在 $x_0$ 的某去心邻域内 $f(x) \ge 0$(或 $\le 0$),
                    且 $\displaystyle \lim_{x \to x_0} f(x) = A$,
                    则 $A \ge 0$(或 $A \le 0$)。
          \end{enumerate}

          ✅ \textit{应用技巧:}
          在求极限符号问题(如 $\lim f(x)/g(x)$ 是否为正)时,常结合保号性与等价无穷小判断符号。

          %--------------------------
    \item \textbf{夹逼准则(两边夹法)}

          若存在函数 $g(x)$、$h(x)$ 使:
          \begin{enumerate}
              \item $h(x) \le f(x) \le g(x)$;
              \item $\displaystyle \lim_{x \to x_0} h(x) = \lim_{x \to x_0} g(x) = A$;
          \end{enumerate}
          则 $\displaystyle \lim_{x \to x_0} f(x) = A$。

          ✅ \textit{常见应用:}
          \[
              \lim_{x \to 0} x^2 \sin\frac{1}{x} = 0, \quad
              \lim_{x \to 0} x \sin\frac{1}{x^2} = 0.
          \]

          %--------------------------
    \item \textbf{单调有界准则}

          若存在 $\delta > 0$,使得:

          \begin{enumerate}
              \item $f(x)$ 在区间 $(x_0, x_0 + \delta)$ 内单调且有界,
                    则右极限 $\displaystyle \lim_{x \to x_0^+} f(x)$ 存在;
              \item $f(x)$ 在区间 $(x_0 - \delta, x_0)$ 内单调且有界,
                    则左极限 $\displaystyle \lim_{x \to x_0^-} f(x)$ 存在。
          \end{enumerate}

          同理,若 $f(x)$ 在 $(a, +\infty)$ 或 $(-\infty, b)$ 上单调有界,
          则 $\displaystyle \lim_{x \to +\infty} f(x)$ 或 $\lim_{x \to -\infty} f(x)$ 存在。

          ✅ \textit{典型函数:}
          \[
              f(x) = 1 - \frac{1}{x},\quad f(x)=\arctan x,\quad f(x)=1 - \frac{1}{2^x}.
          \]
\end{enumerate}
\GAOchapter{数列极限}

\section{含 $f(x_n, x_{n+1})$ 的等式关系}
\DTwoThree

\begin{enumerate}
    \item 若 $x_{n+1} = f(x_n)$,且 $f(x)$ 可导,且存在常数 $k<1$ 使得
          \[
              |f'(x)| \le k,
          \]
          则由 \textbf{压缩映射原理},数列 $\{x_n\}$ 收敛。

    \item 若 $x_{n+1} = f(x_n)$,但 $f(x)$ 不易求导或 $|f'(x)| \nleq k < 1$,可从以下角度分析:
          \begin{enumerate}
              \item 比较 $x_{n+1}$ 与 $x_n$ 的大小,判断单调性;
              \item 作差:$x_{n+1} - x_n$,根据符号判断单调性;
              \item 作商:$\dfrac{x_{n+1}}{x_n}$(当 $x_{n+1}$ 与 $x_n$ 同号时),根据与 1 的大小比较单调性;
              \item 结合题设提示(往往在第(1)问给出),判断有界性或单调性。
          \end{enumerate}
\end{enumerate}

\section{含 $f(x_n, x_{n+1})$ 的不等式关系}

若题设给出不等式关系,可通过比较 $x_n$ 与 $x_{n+1}$ 的大小,
确定 $\{x_n\}$ 的上、下界,从而判断其是否单调有界。

\section{初值 $x_1$ 对递推式收敛性的影响}

若递推式为 $x_{n+1}=f(x_n)$,且初值 $x_1$ 仅给出取值范围,
需根据 $x_1$ 所在区间分情况讨论收敛或发散情形。

\section{双通项数列问题($a_n, b_n$ 型)}
\DTwoOne+\DTwoThree

\begin{enumerate}
    \item 将 $a_n,b_n$ 满足的式子联立,消去其中一个,化为单通项递推问题。
          常用技巧:
          \begin{enumerate*}[itemjoin=\quad]
              \item 恒等变形;
              \item 无穷小比阶;
              \item 放缩比较;
              \item 函数单调性。
          \end{enumerate*}

    \item 令 $c_n = \dfrac{a_n}{b_n}$,转化为求 $\lim\limits_{n\to\infty} c_n$。
          常用技巧:
          \begin{enumerate*}[itemjoin=\quad]
              \item 单调有界准则;
              \item 夹逼准则;
              \item 极限保号性。
          \end{enumerate*}
\end{enumerate}

\section{复合函数的极限}
\DTwoOne

\begin{enumerate}
    \item \textbf{因变量极限定理}

          设 $y=f[g(x)]$,令 $u=g(x)$,若
          \[
              \begin{cases}
                  \displaystyle \lim_{x\to x_0} g(x) = u_0, \\[3pt]
                  \displaystyle \lim_{u\to u_0} f(u) = a,   \\[3pt]
                  g(x)\ne u_0 \text{ 当 } x\ne x_0,
              \end{cases}
          \]
          则有
          \[
              \lim_{x\to x_0} f[g(x)] = a.
          \]

    \item \textbf{中间变量极限定理}

          若 $\{u_n\}$ 取自有限区间 $I$,且 $f(x)$ 在 $I$ 上严格单调,若
          \[
              \lim_{n\to\infty} f(u_n) \text{ 存在},
          \]
          则 $\lim_{n\to\infty} u_n$ 也存在。
\end{enumerate}
\GAOchapter{一元函数微分学的概念}
\GAOchapter{一元函数微分学的计算}

\section{泰勒展开法}
\DTwoThree

若 $f(x)$ 是 $e^x$、$\ln(1+x)$、$\sin x$、$\cos x$、$\frac{1}{1+x}$ 等函数或其“亲戚”,
可通过适当的恒等变形,将其化为已知展开式形式,再利用\textbf{泰勒展开的唯一性}求得 $f^{(n)}(x_0)$。

\begin{enumerate}
    \item \textbf{通分的逆运算(瓦解敌人,各个击破)}
          \[
              \frac{1}{x(x+1)} = \frac{1}{x} - \frac{1}{x+1}.
          \]

    \item \textbf{对数运算性质}
          \[
              \ln(2+x) = \ln\!\left[2\!\left(1+\frac{x}{2}\right)\right]
              = \ln 2 + \ln\!\left(1+\frac{x}{2}\right).
          \]

    \item \textbf{三角恒等式}
          \[
              \sin^2 x = \frac{1-\cos 2x}{2}.
          \]

    \item \textbf{广义化:} 若 $x \to 0$,可作类似极限形式的推广分析。

    \item \textbf{“偏导数化”:} 若函数中含多元关系,可视作 $x$ 的偏导形式进行处理。
\end{enumerate}

\section{莱布尼茨公式法}
\DTwoThree

若函数含有可有限次求导的多项式部分,如
\[
    f(x)\cdot (a x^2 + b x + c),
\]
例如 $e^x(1+x^2)$,则用\textbf{莱布尼茨乘积求导公式}更为高效。
因为 $(a x^2 + b x + c)^{(3)} = 0$,使用该公式后仅保留前三项。

常用 $n$ 阶导数公式($n$ 为正整数)如下:
\[
    \begin{aligned}
        (a^x)^{(n)}                      & = a^x (\ln a)^n,                       \\
        \left(\frac{1}{1+x}\right)^{(n)} & = \frac{(-1)^n n!}{(1+x)^{n+1}},       \\
        [\ln(1+x)]^{(n)}                 & = \frac{(-1)^{n-1}(n-1)!}{(1+x)^n},    \\
        (\sin x)^{(n)}                   & = \sin\!\left(x+n\frac{\pi}{2}\right), \\
        (\cos x)^{(n)}                   & = \cos\!\left(x+n\frac{\pi}{2}\right).
    \end{aligned}
\]

\section{求导转化法}
\DTwoThree
\begin{enumerate}
    \item 若函数既非“亲戚”,又不便于恒等变形,可先求一阶、二阶导,再转化为熟悉形式。
          例如:
          \[
              y = \arctan x, \quad y' = \frac{1}{1+x^2} \Rightarrow y'(1+x^2)=1,
          \]
          从而转化为“莱布尼茨公式法”情形。

    \item 必须熟记前述莱布尼茨法中的常用五个公式,并掌握其\textbf{递推规律}。
\end{enumerate}

\section{特殊点的高阶导数}
\begin{enumerate}
    \item 分段函数的分段点;
    \item 含绝对值的函数。
\end{enumerate}


\section{奇偶与周期函数的高阶导数}

\begin{enumerate}
    \item 若 $f(x)$ 为奇函数:
          \[
              \begin{cases}
                  f^{(2n)}(x)\text{ 为奇函数}, \\
                  f^{(2n+1)}(x)\text{ 为偶函数}.
              \end{cases}
          \]
    \item 若 $f(x)$ 为偶函数:
          \[
              \begin{cases}
                  f^{(2n)}(x)\text{ 为偶函数}, \\
                  f^{(2n+1)}(x)\text{ 为奇函数}.
              \end{cases}
          \]
    \item 若 $f(x)$ 为周期函数,则 $f^{(n)}(x)$ 亦为周期函数。
\end{enumerate}


\section{隐函数的二阶导}

若 $F(x, y) = 0$,且 $y = y(x)$,则对 $x$ 求导:
\[
    F_x + F_y y' = 0 \Rightarrow y' = -\frac{F_x}{F_y}.
\]
再对该式关于 $x$ 求导,得:
\[
    y'' = -\frac{F_{xx} + 2F_{xy}y' + F_{yy}(y')^2}{F_y}.
\]


\section{参数方程的二阶导}

若
\[
    \begin{cases}
        x = x(t), \\
        y = y(t),
    \end{cases}
\]
则
\[
    \frac{\mathrm{d}y}{\mathrm{d}x}
    = \frac{\mathrm{d}y/\mathrm{d}t}{\mathrm{d}x/\mathrm{d}t}
    = \frac{y'(t)}{x'(t)} = \varphi(t),
\]
进而
\[
    \frac{\mathrm{d}^2y}{\mathrm{d}x^2}
    = \frac{\mathrm{d}\varphi/\mathrm{d}t}{\mathrm{d}x/\mathrm{d}t}
    = \frac{\varphi'(t)}{x'(t)}.
\]


\section{反函数的二阶导}

若 $y=f(x)$ 单调且二阶可导,且 $f'(x)\ne0$,则存在反函数 $x=\varphi(y)$。

设
\[
    f'(x)=y'_x,\quad \varphi'(y)=x'_y,
\]
则
\[
    y'_x=\frac{1}{x'_y}, \qquad
    y''_{xx}=-\frac{x''_{yy}}{(x'_y)^3}.
\]
反过来,有
\[
    x'_y = \frac{1}{y'_x}, \qquad
    x''_{yy} = -\frac{y''_{xx}}{(y'_x)^3}.
\]
\GAOchapter{一元函数微分学的应用(一)——几何应用}

\section{切线、法线与截距}
\DTwoTwo

设 $y = f(x)$ 在 $x_0$ 处可导,切点为 $(x_0, y_0)$。

\begin{enumerate}
    \item \textbf{切线方程:}
          \[
              y - y_0 = f'(x_0)(x - x_0).
          \]
    \item \textbf{法线方程:}
          \[
              y - y_0 = -\frac{1}{f'(x_0)}(x - x_0).
          \]
    \item \textbf{各截距公式:}
          \[
              \begin{aligned}
                  \text{x轴切线截距:} & \quad x_0 - \frac{y_0}{f'(x_0)}, \\
                  \text{y轴切线截距:} & \quad y_0 - x_0 f'(x_0),         \\
                  \text{x轴法线截距:} & \quad x_0 + y_0 f'(x_0),         \\
                  \text{y轴法线截距:} & \quad y_0 + \frac{x_0}{f'(x_0)}.
              \end{aligned}
          \]
\end{enumerate}


\section{单调性、极值、凹凸性与拐点}
\DTwoTwo
\subsection{单调性判别}

设 $f(x)$ 在 $[a,b]$ 上连续、在 $(a,b)$ 内可导:
\begin{enumerate}
    \item 若 $f'(x) \ge 0$ 且仅在有限点取等号,则 $f(x)$ 在 $[a,b]$ 上\textbf{严格递增};
    \item 若 $f'(x) \le 0$ 且仅在有限点取等号,则 $f(x)$ 在 $[a,b]$ 上\textbf{严格递减}。
\end{enumerate}

\subsection{极值的定义与判别}

若存在 $x_0$ 的某邻域,使得
\[
    f(x) \le f(x_0) \ (\text{或 } f(x) \ge f(x_0)),
\]
则称 $x_0$ 为 $f(x)$ 的\textbf{极大值点}(或极小值点)。

\textbf{常用判别:}
\begin{enumerate}
    \item 一阶导数法:$f'(x_0) = 0$;
    \item 二阶导数法:若 $f''(x_0) > 0$,则为极小值;若 $f''(x_0) < 0$,则为极大值。
\end{enumerate}

\subsection{凹凸性的定义与判别}

\begin{enumerate}
    \item 定义法:
          \[
              f\!\left(\frac{x_1+x_2}{2}\right)
              \begin{cases}
                  < \frac{f(x_1)+f(x_2)}{2}, & \text{凹函数;} \\[4pt]
                  > \frac{f(x_1)+f(x_2)}{2}, & \text{凸函数.}
              \end{cases}
          \]
    \item 导数法:
          若 $f''(x) > 0$,则函数在该区间上\textbf{凹向上(凸)};
          若 $f''(x) < 0$,则函数在该区间上\textbf{凹向下(凹)}。
\end{enumerate}

\subsection{拐点}

连续曲线的凹弧与凸弧的分界点称为\textbf{拐点}。

\textbf{判别条件:}
\[
    f''(x_0) = 0 \quad \text{且 } f''(x) \text{ 在 } x_0 \text{ 附近变号}.
\]

\subsection{重要结论总结}

\[
    \boxed{
        \begin{aligned}
            \text{有极值点}        & \Leftrightarrow f'(x) \text{ 有零点},     \\[3pt]
            \text{有拐点}         & \Leftrightarrow f''(x) \text{ 有零点且变号}, \\[3pt]
            f'(x) \text{ 无零点}  & \Rightarrow f(x) \text{ 单调性不变},        \\[3pt]
            f''(x) \text{ 无零点} & \Rightarrow f'(x) \text{ 单调性不变.}
        \end{aligned}
    }
\]

\section{渐近线}

\textbf{定义:} 曲线 $y=f(x)$ 的渐近线是指曲线无限接近的一条直线。

\begin{enumerate}
    \item \textbf{竖直渐近线:}
          若 $\displaystyle \lim_{x\to a^\pm}f(x)=\infty$,则 $x=a$ 为竖直渐近线。

    \item \textbf{水平渐近线:}
          若 $\displaystyle \lim_{x\to\infty}f(x)=A$ 或 $\lim_{x\to-\infty}f(x)=B$,
          则 $y=A$ 或 $y=B$ 为水平渐近线。

    \item \textbf{斜渐近线:}
          若 $\displaystyle \lim_{x\to\infty}\frac{f(x)}{x}=k$ 且 $\lim_{x\to\infty}[f(x)-kx]=b$,
          则 $y=kx+b$ 为斜渐近线。
\end{enumerate}

\section{最值与值域求法}
\DTwoThree
\begin{enumerate}
    \item 若 $f(x)$ 在 $[a,b]$ 上连续,则\textbf{最值只可能出现在:}
          \[
              \text{驻点、导数不存在点、区间端点}.
          \]
    \item 若 $f(x)$ 在 $(a,b)$ 内连续,且仅有一个极值点 $x_0$,
          则 $x_0$ 即为全区间的最值点。
    \item 若难以直接判断最值,可使用:
          \begin{enumerate}
              \item 平方和放缩法;
              \item 三角代换法;
              \item 单调性区间法;
              \item 取值范围不等式(如 $a^2+b^2\ge2ab$).
          \end{enumerate}
\end{enumerate}
\GAOchapter{一元函数微分学的应用(二)——中值定理、微分等式与微分不等式}

\section{寻找元函数}
\POneTwo+\POneThree+\DTwoThree

\subsection{一阶乘积求导公式的逆用}
\[
    (uv)' = u'v + uv'
\]
常见形式:
\begin{enumerate}
    \item $[f(x) x^n]' = x^{n-1}[x f'(x) + n f(x)]$;
    \item $[f(x) e^{nx}]' = e^{nx}[f'(x) + n f(x)]$;
    \item $[f(x) e^{x^n}]' = e^{x^n}[f'(x) + n x^{n-1} f(x)]$;
    \item $[f(x) e^{\varphi(x)}]' = e^{\varphi(x)}[f'(x) + f(x)\varphi'(x)]$;
    \item $\displaystyle \left\{ f(x)e^{\int_0^x [f(t)]^{n-1}dt} \right\}' = e^{\int_0^x [f(t)]^{n-1}dt}\{f'(x) + [f(x)]^n\}$;
    \item $[f(x)f'(x)]' = [f'(x)]^2 + f(x)f''(x)$;
    \item $[f(x)g(x)]' = f'(x)g(x) + f(x)g'(x)$;
    \item $[f(x)\arctan x]' = f'(x)\arctan x + \dfrac{f(x)}{1+x^2}$;
    \item $[f(x)\sin x]' = f'(x)\sin x + f(x)\cos x = [f'(x)\tan x + f(x)]\cos x$。
\end{enumerate}

\subsection{二阶乘积求导公式的逆用}
\[
    (uv)'' = u''v + 2u'v' + uv'', \quad [f(x)e^x]'' = e^x[f''(x) + 2f'(x) + f(x)]。
\]

\subsection{一阶商求导公式的逆用}
\[
    \left(\frac{u}{v}\right)' = \frac{u'v - uv'}{v^2}
\]
\begin{enumerate}
    \item $\displaystyle \left[\frac{f(x)}{x}\right]' = \frac{f'(x)x - f(x)}{x^2}$
          见到 $f'(x)x - f(x)$,$x \ne 0$,令 $F(x)=\frac{f(x)}{x}$;
    \item $\displaystyle \left[\frac{f'(x)}{f(x)}\right]' = \frac{f''(x)f(x) - [f'(x)]^2}{f^2(x)}$
          见到 $f''(x)f(x) - [f'(x)]^2$,令 $F(x)=\frac{f'(x)}{f(x)}$;
    \item $[\ln f(x)]'' = \left[\frac{f'(x)}{f(x)}\right]' = \frac{f''(x)f(x) - [f'(x)]^2}{f^2(x)}$
          见到 $f''(x)f(x) - [f'(x)]^2$,$f(x)>0$ 时可令 $F(x)=\ln f(x)$。
\end{enumerate}

\subsection{祖孙三代传承法}
若欲证结论“差辈分”,如
\[
    f''(\xi)=f(\xi)\quad\text{或}\quad f'(\xi)=\int_0^\xi f(t)\,dt,
\]
则补齐辈分作恒等变形:

\paragraph{例1:}
\[
    f''(\xi)-f'(\xi)+f'(\xi)-f(\xi)=0,
\]
令 $F'(x)=[f''(x)-f'(x)]e^x + [f'(x)-f(x)]e^x$,则
$F(x)=[f'(x)-f(x)]e^x$。

\paragraph{例2:}
\[
    f'(\xi)-f(\xi)+f(\xi)-\int_0^\xi f(t)\,dt = 0,
\]
令
\[
    F'(x)=\big([f'(x)-f(x)]+[f(x)-\int_0^x f(t)\,dt]\big)e^x,
\]
则 $F(x)=\left[f(x)-\int_0^x f(t)\,dt\right]e^x$。

\subsection{小伎俩(障眼法)}
\begin{enumerate}[label=(\arabic*)]
    \item \textbf{简单化:}
          $2\xi-1\rightarrow x^2-x,\ f'(ξ)f(ξ)\rightarrow\frac{1}{2}f^2(x),\
              \dfrac{f'(ξ)}{f(ξ)}\rightarrow\ln f(x),\ \dfrac{1}{ξ}\rightarrow\ln x$。

    \item \textbf{升辈降辈:}
          $f(x)\to f'(x)$ 或 $f(x)\to \int_0^x f(t)dt$。

    \item \textbf{平移:}
          \[
              [f(x)(x+1)^2]'=(x+1)\big[f'(x)(x+1)+2f(x)\big],
          \]
          \[
              [f(x)e^{(x-1)^2}]'=e^{(x-1)^2}[f'(x)+2(x-1)f(x)]。
          \]

    \item \textbf{恒等变形:}
          \begin{enumerate}
              \item 移项:令右端为 0;
              \item 乘除:如 $\big[f'(x)g(x)-f(x)g'(x)\big]_{x=\xi}=0 \Rightarrow \dfrac{f'(ξ)}{g'(ξ)}=\dfrac{f(ξ)}{g(ξ)}$。
          \end{enumerate}
\end{enumerate}

\subsection{积分型题设}
当题设或结论出现 $\displaystyle \int_a^b f(x)\,dx = c$、$\int_a^\xi f(x)\,dx$ 等形式时:
\begin{enumerate}
    \item 令被积函数为辅助函数;
    \item 令 $F(x)=\int_a^x f(t)\,dt$ 或 $F(x)=\int_x^b f(t)\,dt$;
    \item 用推广积分中值定理:
          $\displaystyle \int_a^b f(x)\,dx=f(\xi)(b-a)$;
    \item 用分部积分:$\displaystyle \int_a^b u\,d\nu=u\nu\Big|_a^b-\int_a^b\nu\,du$;
    \item 或作泰勒展开再积分。
\end{enumerate}


\section{证明 $f'(\xi)=0$}
\DTwoTwo+\DTwoThree

常用三种情形:
\begin{enumerate}
    \item 区间内部最值点:费马定理;
    \item 区间端点值相等:罗尔定理;
    \item 区间端点导数异号:导数介值定理。
\end{enumerate}


\section{证明含高阶导数的等式或不等式}
\DTwoTwo+\DTwoThree

\subsection{利用中值定理}
\begin{enumerate}
    \item 拉格朗日中值定理:
          若 $f(x)$ 在 $[a,b]$ 上连续且在 $(a,b)$ 内可导,则
          $\exists \xi \in (a,b)$,使得
          \[
              f'(\xi)=\frac{f(b)-f(a)}{b-a}。
          \]
    \item 推论:$f'(x)=0\Rightarrow f(x)$ 在区间上为常数。
    \item 写作 $f(x)-f(a)=f'(\xi)(x-a)$,常用于出现“$f-f$”或“$f-f'$”关系;
    \item 写作 $\dfrac{f(b)-f(a)}{b-a}=f'(\xi)$,常用于“切线斜率”;
    \item 写作 $f(x)-f(a)=f'(\theta x)x$,$0<\theta<1$,用于极限估值;
    \item 写作 $\int_a^x f(t)\,dt=f(\xi)(x-a)$,用于积分中值定理。
\end{enumerate}

\subsection{柯西中值定理}
当式子可化为 $\dfrac{f(b)-f(a)}{g(b)-g(a)}$ 或 $\dfrac{f'(\xi)}{g'(\xi)}$ 时使用。
\begin{enumerate}
    \item 若一函数取具体形式,可利用 $\displaystyle f(\xi)=\left[\int_a^x f(t)\,dt\right]'_{x=\xi}$;
    \item 若函数值含 $0$ 或 $1$,可利用 $f(a)=0$、$e^0=1$、$\int_a^a f=0$ 等;
    \item 若 $f(a)=0$,可作
          \[
              f(\xi)=f'(\eta)(\xi-a),\ \eta\in(a,\xi),
          \]
          再代回主式,得:
          \[
              f'(\eta)(b^2-a^2)=\frac{2\xi}{\xi-a}\int_a^b f(x)\,dx。
          \]
\end{enumerate}


\section{用泰勒公式}
\DTwoThree+\DThree

\subsection{常用泰勒展开式或形式展开式大观}

\begin{enumerate}
    \item \textbf{第一组:基本代数型展开式}
          \begin{enumerate}
              \item $\sqrt{1\pm x} = 1 \pm \dfrac{1}{2}x - \dfrac{1}{8}x^2 + \cdots, \quad |x| < 1$;
              \item $\dfrac{1}{\sqrt{1\pm x}} = 1 \mp \dfrac{1}{2}x + \dfrac{3}{8}x^2 + \cdots, \quad |x| < 1$;
              \item $\dfrac{1}{1+x} = 1 - x + x^2 - x^3 + \cdots = \displaystyle\sum_{n=0}^{\infty}(-1)^n x^n, \quad |x| < 1$;
              \item $\dfrac{1}{1-x} = 1 + x + x^2 + x^3 + \cdots = \displaystyle\sum_{n=0}^{\infty} x^n, \quad |x| < 1$;
              \item $\dfrac{1}{(1-x)^2} = 1 + 2x + 3x^2 + \cdots = \displaystyle\sum_{n=0}^{\infty}(n+1)x^n, \quad |x| < 1$;
              \item $\dfrac{1}{(1+x)^2} = 1 - 2x + 3x^2 - 4x^3 + \cdots = \displaystyle\sum_{n=0}^{\infty}(-1)^n (n+1)x^n, \quad |x| < 1$。
          \end{enumerate}

    \item \textbf{第二组:指数型展开式}
          \begin{enumerate}
              \item $e^x = \displaystyle\sum_{n=0}^{\infty} \dfrac{x^n}{n!}$;
              \item $a^x = e^{x\ln a} = \displaystyle\sum_{n=0}^{\infty} \dfrac{(\ln a)^n x^n}{n!}$;
              \item $\sinh x = \dfrac{e^x - e^{-x}}{2} = \displaystyle\sum_{n=0}^{\infty} \dfrac{x^{2n+1}}{(2n+1)!}$;
              \item $\cosh x = \dfrac{e^x + e^{-x}}{2} = \displaystyle\sum_{n=0}^{\infty} \dfrac{x^{2n}}{(2n)!}$。
          \end{enumerate}

    \item \textbf{第三组:三角与反三角展开式}
          \begin{enumerate}
              \item $\sin x = x - \dfrac{x^3}{3!} + \dfrac{x^5}{5!} - \cdots = \displaystyle\sum_{n=0}^{\infty} (-1)^n \dfrac{x^{2n+1}}{(2n+1)!}$;
              \item $\cos x = 1 - \dfrac{x^2}{2!} + \dfrac{x^4}{4!} - \cdots = \displaystyle\sum_{n=0}^{\infty} (-1)^n \dfrac{x^{2n}}{(2n)!}$;
              \item $\tan x = x + \dfrac{1}{3}x^3 + \dfrac{2}{15}x^5 + \cdots, \quad |x| < \dfrac{\pi}{2}$;
              \item $\arcsin x = x + \dfrac{1}{6}x^3 + \dfrac{3}{40}x^5 + \cdots, \quad |x| < 1$;
              \item $\arctan x = x - \dfrac{1}{3}x^3 + \dfrac{1}{5}x^5 - \cdots = \displaystyle\sum_{n=0}^{\infty} (-1)^n \dfrac{x^{2n+1}}{2n+1}, \quad |x| \le 1$。
          \end{enumerate}

    \item \textbf{第四组:对数型展开式}
          \begin{enumerate}
              \item $\ln(1+x) = x - \dfrac{1}{2}x^2 + \dfrac{1}{3}x^3 - \dfrac{1}{4}x^4 + \cdots = \displaystyle\sum_{n=1}^{\infty}(-1)^{n+1}\dfrac{x^n}{n}, \quad -1 < x \le 1$;
              \item $\ln(1-x) = -\displaystyle\sum_{n=1}^{\infty} \dfrac{x^n}{n}, \quad -1 \le x < 1$;
              \item $\ln x = \ln(1+(x-1)) = (x-1) - \dfrac{1}{2}(x-1)^2 + \dfrac{1}{3}(x-1)^3 - \cdots = \displaystyle\sum_{n=1}^{\infty}(-1)^{n+1}\dfrac{(x-1)^n}{n}, \quad 0 < x \le 2$;
              \item $\ln(a+x) = \ln a + \ln\!\left(1+\dfrac{x}{a}\right) = \ln a + \dfrac{x}{a} - \dfrac{1}{2a^2}x^2 + \cdots, \quad a>0,\ -a<x\le a$;
              \item $\ln\dfrac{1+x}{1-x} = 2\left(x + \dfrac{x^3}{3} + \dfrac{x^5}{5} + \cdots\right) = 2\displaystyle\sum_{n=0}^{\infty} \dfrac{x^{2n+1}}{2n+1}, \quad |x| < 1$;
              \item $\ln\dfrac{x+1}{x-1} = 2\displaystyle\sum_{n=0}^{\infty} \dfrac{1}{(2n+1)x^{2n+1}}, \quad |x| > 1$;
              \item $\ln\!\big(x+\sqrt{x^2+1}\big) = x - \dfrac{1}{6}x^3 + \dfrac{3}{40}x^5 + \cdots, \quad |x| < 1$。
          \end{enumerate}
\end{enumerate}

\subsection{本质与用途}
泰勒公式是函数的 $n$ 阶近似:
\[
    f(x)=\sum_{k=0}^n \frac{f^{(k)}(x_0)}{k!}(x-x_0)^k + R_n(x),
\]
其中余项可取:
\[
    R_n(x)=o\big((x-x_0)^n\big)\quad\text{或}\quad R_n(x)=\frac{f^{(n+1)}(\xi)}{(n+1)!}(x-x_0)^{n+1}。
\]

\textbf{常用场景:}
\begin{enumerate}
    \item 求极限;
    \item 判断无穷小阶;
    \item 求 $f^{(n)}(x_0)$;
    \item 证明估值问题。
\end{enumerate}

\textbf{$x_0$ 与 $x$ 的选取:}
\begin{enumerate}
    \item $x_0$ 取使导数简单的点,如 0、1、端点;
    \item $x$ 取任意点或关于 $x_0$ 的对称点。
\end{enumerate}

\section{讨论 $f(x)=0$ 的根的个数}

\begin{enumerate}
    \item \textbf{零点定理及其推广:}
          若 $f(x)$ 在 $[a,b]$ 上连续,且 $f(a)f(b)<0$,则方程 $f(x)=0$ 在 $(a,b)$ 内至少有一个根。
    \item \textbf{反证思想:}
          若要证 $f(x)$ 存在零点,可设 $f(x)$ 无零点,按条件推矛盾;反之亦然。
    \item \textbf{导数工具:}
          用 $f'(x)$ 研究函数单调性、极值性态,以判断零点个数。
    \item \textbf{罗尔定理推论:}
          若 $f^{(n)}(x)=0$ 至多有 $k$ 个根,则 $f(x)=0$ 至多有 $k+n$ 个根。
    \item \textbf{实系数奇次方程:}
          方程 $x^{2n+1}+a_1x^{2n}+\cdots+a_{2n}x+a_{2n+1}=0$ 至少有一个实根。
    \item \textbf{渐近性态:}
          若 $\lim_{x\to+\infty}f(x)$ 与 $\lim_{x\to-\infty}f(x)$ 异号,则方程至少有一个实根。
\end{enumerate}

\section{证明不等式}

\begin{enumerate}
    \item \textbf{用单调性:}
          \begin{enumerate}
              \item 若 $\displaystyle\lim_{x\to a^+}F(x)\ge0$ 且 $F'(x)\ge0$,则 $(a,b)$ 内 $F(x)\ge0$;
              \item 若 $\displaystyle\lim_{x\to b^-}F(x)\ge0$ 且 $F'(x)\le0$,则 $(a,b)$ 内 $F(x)\ge0$。
          \end{enumerate}

    \item \textbf{用最值:}
          若 $F(x)$ 在 $(a,b)$ 有最小值 $m$,则 $F(x)\ge m$;若有最大值 $M$,则 $F(x)\le M$。

    \item \textbf{用凹凸性:}
          若 $F''(x)>0$,则:
          \begin{enumerate}
              \item $\displaystyle \frac{F(x_1)+F(x_2)}{2}\ge F\!\left(\frac{x_1+x_2}{2}\right)$;
              \item 对任意 $\lambda_1,\lambda_2>0$ 且 $\lambda_1+\lambda_2=1$,有
                    $\lambda_1F(x_1)+\lambda_2F(x_2)\ge F(\lambda_1x_1+\lambda_2x_2)$;
              \item 对任意 $x\ne x_0$,有 $F(x)\ge F(x_0)+F'(x_0)(x-x_0)$。
          \end{enumerate}
          若 $F''(x)<0$,则上述不等式方向反向。

    \item \textbf{用拉格朗日中值定理:}
          若 $F'(x)\ge A$(或 $\le A$),则
          $F(b)-F(a)\ge A(b-a)$(或 $\le A(b-a)$)。

    \item \textbf{用柯西中值定理:}
          若 $\dfrac{F'(x)}{G'(x)}\ge A$(或 $\le A$),则
          $\dfrac{F(b)-F(a)}{G(b)-G(a)}\ge A$(或 $\le A$)。

    \item \textbf{用带余项的泰勒公式:}
          若 $F''(x)$ 存在且恒 $>0$(或 $<0$),则在 $x_0$ 处展开:
          \[
              F(x)=F(x_0)+F'(x_0)(x-x_0)+\frac{1}{2}F''(\xi)(x-x_0)^2,
          \]
          其中 $\xi$ 介于 $x$ 与 $x_0$,从而有
          $F(x)\ge F(x_0)+F'(x_0)(x-x_0)$(或反向不等式)。
\end{enumerate}

\section{求解含参等式或不等式问题}

\begin{enumerate}
    \item \textbf{导数中不含参数:}
          辅助函数 $f'(x)$ 不含参数。研究函数性态时不考虑参数,最后再根据参数取值讨论与 $x$ 轴位置关系。
    \item \textbf{导数中含参数:}
          辅助函数 $f'(x)$ 含参数。研究性态时需分类讨论参数取值。常见形式:
          \begin{enumerate}
              \item 函数型:$f(x)=k$ 或 $f(x,k)$;
              \item 参数方程型:
                    \[
                        \begin{cases}
                            x = x(t,k), \\
                            y = y(t,k)。
                        \end{cases}
                    \]
          \end{enumerate}
\end{enumerate}
\GAOchapter{一元函数微分学的应用(三)——物理应用}

\section{寻找、建立相关变化率等式并求解}

\begin{enumerate}
    \item \textbf{运动学关系:}
          若质点的位移关于时间的函数为 $x=x(t)$,则:
          \[
              v = \frac{dx}{dt}, \quad
              a = \frac{dv}{dt} = \frac{dv}{dx}\frac{dx}{dt} = v\frac{dv}{dx}.
          \]
          (其中 $v$ 为速度,$a$ 为加速度。)

    \item \textbf{参数形式下的变化率关系:}
          若函数 $y=f(x)$ 由参数方程
          \[
              \begin{cases}
                  x = x(t), \\
                  y = y(t)
              \end{cases}
          \]
          所确定,且 $y$ 对 $t$ 的变化率与 $x$ 对 $t$ 的变化率成比例,
          即
          \[
              \frac{dy}{dt} = k a \frac{dx}{dt}, \quad (k \neq 0),
          \]
          则有
          \[
              \frac{dy}{dx} = k a.
          \]
          (常用于求相关变化率或速度比。)
\end{enumerate}

\section{根据题设写出物理量微元(微段常量化),建立等式并求解}

若题中给出“A 对时间 $t$ 的变化率与 $B$ 成正比”,则有:
\[
    \frac{dA}{dt} = kB, \quad (k \neq 0),
\]
由此建立微分方程并求解即可。

\subsection*{常见模型举例}
\begin{enumerate}
    \item \textbf{匀加速直线运动:}
          \[
              \frac{dv}{dt} = a_0 \Rightarrow v = a_0t + v_0, \quad x = \frac{1}{2}a_0t^2 + v_0t + x_0.
          \]
    \item \textbf{牛顿冷却定律:}
          物体温度 $T$ 的变化率与其与环境温度 $T_0$ 的温差成正比:
          \[
              \frac{dT}{dt} = -k(T - T_0),
          \]
          解得
          \[
              T = T_0 + Ce^{-kt}.
          \]
    \item \textbf{放射性衰变 / 电容放电:}
          数量 $N$ 的减少率与其本身成正比:
          \[
              \frac{dN}{dt} = -kN,
          \]
          解得
          \[
              N = N_0 e^{-kt}.
          \]
    \item \textbf{人口增长模型(逻辑斯蒂型):}
          若增长率受限于总容量 $M$,
          \[
              \frac{dN}{dt} = kN\left(1-\frac{N}{M}\right),
          \]
          解得
          \[
              N = \frac{M}{1+Ce^{-kt}}.
          \]
\end{enumerate}

\subsection*{小结:解题步骤}
\begin{enumerate}
    \item 根据题意确定变量间的依赖关系;
    \item 用比例、速率等物理语言写出微分方程;
    \item 若能分离变量,则两边积分;
    \item 结合初值(如 $t=0$ 时条件)求出常数;
    \item 给出解析式或函数关系。
\end{enumerate}
\GAOchapter{一元函数积分学的概念与性质}
\GAOchapter{一元函数积分学的计算}
\GAOchapter{一元函数积分学的应用(一)——几何应用}
\GAOchapter{一元函数积分学的应用(二)——积分等式与积分不等式}

\section{定积分等式问题}

\begin{enumerate}
    \item \textbf{与字母无关性}
          $$
              \int_{a}^{b} f(x)\,\mathrm{d}x = \int_{a}^{b} f(y)\,\mathrm{d}y.
          $$

    \item \textbf{线性性}
          $$
              \int_{a}^{b}\!\!\big[k_1f_1(x)+k_2f_2(x)\big]\mathrm{d}x
              =k_1\!\int_{a}^{b}\!f_1(x)\mathrm{d}x+k_2\!\int_{a}^{b}\!f_2(x)\mathrm{d}x.
          $$

    \item \textbf{方向性}
          $$
              \int_{a}^{b}f(x)\,\mathrm{d}x=-\int_{b}^{a}f(x)\,\mathrm{d}x.
          $$

    \item \textbf{可加(可拆)性}
          $$
              \int_{a}^{b}f(x)\,\mathrm{d}x=\int_{a}^{c}f(x)\,\mathrm{d}x+\int_{c}^{b}f(x)\,\mathrm{d}x.
          $$

    \item \textbf{“祖孙三代”函数的奇偶性规律}
          \begin{enumerate}
              \item 若 $f(x)$ 为可导的奇函数,则 $f'(x)$ 为偶函数;
              \item 若 $f(x)$ 为可导的偶函数,则 $f'(x)$ 为奇函数;
              \item 若 $f(x)$ 为可积的奇函数,则 $\displaystyle \int_{0}^{x}f(t)\mathrm{d}t$ 为偶函数;
              \item 若 $f(x)$ 为可积的偶函数,则 $\displaystyle \int_{0}^{x}f(t)\mathrm{d}t$ 为奇函数;
              \item 对称积分公式:
                    $$
                        \int_{-a}^{a}f(x)\,\mathrm{d}x=
                        \begin{cases}
                            2\int_{0}^{a}f(x)\,\mathrm{d}x, & f(x)\text{为偶函数}, \\[4pt]
                            0,                              & f(x)\text{为奇函数}.
                        \end{cases}
                    $$
              \item 一般形式:
                    $$
                        \int_{-a}^{a}f(x)\,\mathrm{d}x
                        =\tfrac{1}{2}\!\int_{-a}^{a}[f(x)+f(-x)]\,\mathrm{d}x
                        =\!\int_{0}^{a}[f(x)+f(-x)]\,\mathrm{d}x.
                    $$
                    (常见例:$f(x)=\tfrac{1}{1+e^{x}},\ \tfrac{1}{1+e^{1/x}},\ \tfrac{e^x+1}{e^x-1}$ 等)
          \end{enumerate}

    \item \textbf{“祖孙三代”函数的周期性规律}
          \begin{enumerate}
              \item 若 $f(x)$ 可导且以 $T$ 为周期,则 $f'(x)$ 亦以 $T$ 为周期;
              \item 若 $f(x)$ 可积且以 $T$ 为周期,则
                    $\displaystyle \int_{0}^{x}f(t)\mathrm{d}t$
                    以 $T$ 为周期 $\Leftrightarrow \int_{0}^{T}f(x)\mathrm{d}x=0$;
              \item 对任意常数 $a$,有
                    $$
                        \int_{a}^{a+T}f(x)\,\mathrm{d}x=\int_{0}^{T}f(x)\,\mathrm{d}x,
                        \quad
                        \int_{a}^{a+nT}f(x)\,\mathrm{d}x=n\int_{0}^{T}f(x)\,\mathrm{d}x.
                    $$
          \end{enumerate}

    \item \textbf{区间再现公式}
          \begin{enumerate}
              \item $\displaystyle \int_{a}^{b}f(x)\mathrm{d}x=\int_{a}^{b}f(a+b-x)\mathrm{d}x$;
              \item $\displaystyle \int_{a}^{b}f(x)\mathrm{d}x=\tfrac{1}{2}\int_{a}^{b}[f(x)+f(a+b-x)]\mathrm{d}x$;
              \item $\displaystyle \int_{a}^{b}f(x)\mathrm{d}x=\int_{a}^{\frac{a+b}{2}}[f(x)+f(a+b-x)]\mathrm{d}x$;
              \item $\displaystyle \int_{0}^{\pi}xf(\sin x)\mathrm{d}x=\tfrac{\pi}{2}\int_{0}^{\pi}f(\sin x)\mathrm{d}x$;
              \item $\displaystyle \int_{0}^{\pi}xf(\sin x)\mathrm{d}x=\pi\int_{0}^{\frac{\pi}{2}}f(\sin x)\mathrm{d}x$;
              \item $\displaystyle \int_{0}^{\frac{\pi}{2}}f(\sin x)\mathrm{d}x=\int_{0}^{\frac{\pi}{2}}f(\cos x)\mathrm{d}x$;
              \item $\displaystyle \int_{0}^{\frac{\pi}{2}}f(\sin x,\cos x)\mathrm{d}x=\int_{0}^{\frac{\pi}{2}}f(\cos x,\sin x)\mathrm{d}x$。
          \end{enumerate}

    \item \textbf{华里士公式(Wallis Formula)}
          \begin{enumerate}
              \item $\displaystyle \int_{0}^{\frac{\pi}{2}}\!\sin^n x\,\mathrm{d}x
                        =\!\int_{0}^{\frac{\pi}{2}}\!\cos^n x\,\mathrm{d}x
                        =\begin{cases}
                            \frac{n-1}{n}\frac{n-3}{n-2}\cdots\frac{2}{3}\cdot1,             & n\text{奇}, \\[4pt]
                            \frac{n-1}{n}\frac{n-3}{n-2}\cdots\frac{1}{2}\cdot\frac{\pi}{2}, & n\text{偶}.
                        \end{cases}$

              \item $\displaystyle \int_{0}^{\pi}\!\sin^n x\,\mathrm{d}x=
                        \begin{cases}
                            2\!\times\!\frac{n-1}{n}\!\cdots\!\frac{2}{3}\cdot1,             & n\text{奇}, \\[4pt]
                            2\!\times\!\frac{n-1}{n}\!\cdots\!\frac{1}{2}\cdot\frac{\pi}{2}, & n\text{偶}.
                        \end{cases}$

              \item $\displaystyle \int_{0}^{\pi}\!\cos^n x\,\mathrm{d}x=
                        \begin{cases}
                            0,                                                               & n\text{奇}, \\[4pt]
                            2\!\times\!\frac{n-1}{n}\!\cdots\!\frac{1}{2}\cdot\frac{\pi}{2}, & n\text{偶}.
                        \end{cases}$

              \item $\displaystyle \int_{0}^{2\pi}\!\cos^n x\,\mathrm{d}x
                        =\int_{0}^{2\pi}\!\sin^n x\,\mathrm{d}x
                        =\begin{cases}
                            0,                                                               & n\text{奇}, \\[4pt]
                            4\!\times\!\frac{n-1}{n}\!\cdots\!\frac{1}{2}\cdot\frac{\pi}{2}, & n\text{偶}.
                        \end{cases}$
          \end{enumerate}

    \item \textbf{积分中值定理}
          若 $f(x)$ 在 $[a,b]$ 上连续,则存在 $\xi\in[a,b]$,使得
          $$
              \int_{a}^{b}f(x)\,\mathrm{d}x=f(\xi)(b-a).
          $$

    \item \textbf{定积分的换元法}
          设 $f(x)$ 在 $[a,b]$ 上连续,若 $x=\varphi(t)$ 连续可导,且 $\varphi(\alpha)=a,\ \varphi(\beta)=b$,则
          $$
              \int_{a}^{b}f(x)\,\mathrm{d}x=\int_{\alpha}^{\beta}f[\varphi(t)]\varphi'(t)\,\mathrm{d}t.
          $$

    \item \textbf{定积分的分部积分法}
          $$
              \int_{a}^{b}u(x)v'(x)\mathrm{d}x
              =u(x)v(x)\Big|_{a}^{b}-\int_{a}^{b}v(x)u'(x)\mathrm{d}x,
          $$
          其中 $u'(x),v'(x)$ 在 $[a,b]$ 上连续。

          \begin{enumerate}
              \item \textbf{对称性判断:}
                    \begin{example}{}{}
                        计算:
                        \begin{enumerate}
                            \item $\displaystyle \int_{0}^{2}(x-1)\mathrm{d}x$;
                            \item $\displaystyle \int_{0}^{2}x(x-1)(x-2)\mathrm{d}x$。
                        \end{enumerate}
                    \end{example}
                    \begin{solution}
                        令 $t=x-1$:
                        \begin{align*}
                            \int_{0}^{2}(x-1)\mathrm{d}x       & = \int_{-1}^{1}t\,\mathrm{d}t = 0,                                       \\
                            \int_{0}^{2}x(x-1)(x-2)\mathrm{d}x & = \int_{-1}^{1}(t+1)t(t-1)\mathrm{d}t=\int_{-1}^{1}(t^3-t)\mathrm{d}t=0.
                        \end{align*}
                    \end{solution}

              \item \textbf{判断积分正负:}
                    \begin{example}{}{}
                        设 $I=\displaystyle\int_{0}^{\frac{3\pi}{2}}\frac{\cos x}{2x-3\pi}\mathrm{d}x$,则 $I$(  )。
                        \begin{center}
                            (A) 正 \quad (B) 负 \quad (C) 零 \quad (D) 发散
                        \end{center}
                    \end{example}
                    \begin{solution}
                        \vspace{-0.5em}
                        \begin{align*}
                            I & = \tfrac{1}{2}\int_{0}^{\frac{3\pi}{2}}\frac{\sin t}{t}\,\mathrm{d}t
                            = \tfrac{1}{2}\int_{0}^{\frac{\pi}{2}}\!\!\!\frac{\sin t}{t}\,\mathrm{d}t
                            + \tfrac{1}{2}\int_{\frac{\pi}{2}}^{\pi}\!\!\!\frac{\sin t}{t}\,\mathrm{d}t
                            - \tfrac{1}{2}\int_{0}^{\frac{\pi}{2}}\!\!\!\frac{\sin u}{\pi+u}\,\mathrm{d}u > 0.
                        \end{align*}
                        故选 (A)。
                    \end{solution}

              \item \textbf{升降阶应用:}
                    \begin{example}{}{}
                        设 $f(x)=xg'(2x)$,$g(x)$ 的一个原函数为 $\ln(x+1)$,则 $\displaystyle \int_{0}^{1}f(x)\mathrm{d}x=$\underline{\hspace{2cm}}。
                    \end{example}
                    \begin{solution}
                        由分部积分:
                        \[
                            \int_{0}^{1}xg'(2x)\mathrm{d}x
                            =\tfrac{1}{4}\int_{0}^{2}x\,\mathrm{d}[g(x)]
                            =\tfrac{1}{4}\Big[xg(x)-\!\int_{0}^{2}g(x)\mathrm{d}x\Big]
                            =\tfrac{1}{4}\Big(\tfrac{2}{3}-\ln3\Big)
                            =\tfrac{1}{6}-\tfrac{1}{4}\ln3.
                        \]
                    \end{solution}
          \end{enumerate}

    \item \textbf{牛顿–莱布尼茨公式}
          若 $f(x)$ 在 $[a,b]$ 上连续,$F(x)$ 是其原函数,则
          $$
              \int_{a}^{b}f(x)\,\mathrm{d}x=F(b)-F(a).
          $$
\end{enumerate}
\section{定积分不等式问题}

\begin{enumerate}
    \item \textbf{比较定理(保号性)}
          \begin{itemize}
              \item 若 $f(x),g(x)$ 在 $[a,b]$ 上连续,且对任意 $x\in[a,b]$,有 $f(x)\le g(x)$,则
                    $$
                        \int_{a}^{b}f(x)\,\mathrm{d}x\le\int_{a}^{b}g(x)\,\mathrm{d}x,\quad (b>a)。
                    $$
              \item 若 $f(x)\ge0$,则 $\displaystyle \int_{a}^{b}f(x)\,\mathrm{d}x\ge0$。
          \end{itemize}
          \textit{几何意义:曲线 $f$ 位于 $g$ 之下时,其“面积”亦小于后者。}

    \item \textbf{估值定理(夹估不等式)}
          \begin{itemize}
              \item 若 $f(x)$ 在 $[a,b]$ 上连续,且
                    $$
                        m\le f(x)\le M,\quad \forall x\in[a,b],
                    $$
                    则有
                    $$
                        m(b-a)\le\int_{a}^{b}f(x)\,\mathrm{d}x\le M(b-a),\quad (b>a)。
                    $$
              \item 当 $f(x)$ 在区间上恒为常值时,上下界取等号。
          \end{itemize}
          \textit{直观理解:积分相当于“平均值 × 区间长度”,而平均值必位于函数值的上下界之间。}

    \item \textbf{绝对值不等式}
          若 $f(x)$ 在 $[a,b]$ 上连续,则
          $$
              \left|\int_{a}^{b}f(x)\,\mathrm{d}x\right|
              \le\int_{a}^{b}|f(x)|\,\mathrm{d}x,\quad (b>a)。
          $$
          \textit{解释:符号抵消只会减小积分的绝对值。等号成立当且仅当 $f(x)$ 在 $[a,b]$ 上符号恒定。}

    \item \textbf{黎曼思想(分区思想)}
          将“大区间问题”拆分为若干“小区间”分别求解,是微积分分析中最核心的思想之一:
          $$
              \int_{a}^{b}f(x)\,\mathrm{d}x
              =\int_{a}^{c}f(x)\,\mathrm{d}x+\int_{c}^{b}f(x)\,\mathrm{d}x.
          $$
          \textit{启示:当函数在整体上难处理时,分段、局部分析后再综合,是定积分思想的本质体现。}
\end{enumerate}
\GAOchapter{一元函数积分学的应用(三)——物理应用}
\GAOchapter{多元函数微分学}
\GAOchapter{二重积分}
\GAOchapter{微分方程}
\section{求解微分方程并研究解的性质}
\subsection{一阶微分方程的求解}
\DTwoThree

若题目中出现 $y'$ 或 $dy=\cdots dx$,则通常属于以下类型:

\begin{enumerate}
    % ========== 可分离变量型 ==========
    \item \textbf{可分离变量型}(或可换元化为此形式)
          \begin{enumerate}
              \item 若可写为 $y' = f(x)g(y)$,
                    则分离变量得:
                    \[
                        \frac{dy}{g(y)} = f(x)\,dx,
                        \qquad
                        \int \frac{dy}{g(y)} = \int f(x)\,dx.
                    \]

              \item 若可写为 $y' = f(ax+by+c)$,
                    令 $u=ax+by+c$,则 $u' = a + b f(u)$,
                    于是:
                    \[
                        \frac{du}{a + b f(u)} = dx,
                        \qquad
                        \int \frac{du}{a + b f(u)} = \int dx.
                    \]
          \end{enumerate}

          % ========== 齐次型 ==========
    \item \textbf{齐次型}(或可换元化为此形式)
          \begin{enumerate}
              \item 若 $y' = f\!\left(\frac{y}{x}\right)$,
                    令 $\displaystyle \frac{y}{x} = u$,即 $y=ux,\ y' = u + x u'$,
                    代入得:
                    \[
                        x\,\frac{du}{dx} + u = f(u)
                        \quad\Rightarrow\quad
                        \frac{du}{f(u)-u} = \frac{dx}{x}.
                    \]

              \item 若 $\displaystyle \frac{1}{y'} = f\!\left(\frac{x}{y}\right)$,
                    令 $\displaystyle \frac{x}{y} = u$,即 $x=uy,\ x' = u + y u'$,
                    代入得:
                    \[
                        y\,\frac{du}{dy} + u = f(u)
                        \quad\Rightarrow\quad
                        \frac{du}{f(u)-u} = \frac{dy}{y}.
                    \]

              \item 若 $y' = f\!\left(\frac{ax+by+c}{a_1x+b_1y+c_1}\right)$,则:
                    \begin{enumerate}
                        \item 若 $c^2 + c_1^2 = 0$,可化为 $y' = g\!\left(\frac{y}{x}\right)$;
                        \item 若 $c^2 + c_1^2 \neq 0$ 且 $\frac{a}{a_1}=\frac{b}{b_1}$,
                              可化为 $y' = g(ax+by)$;
                        \item 若 $c^2 + c_1^2 \neq 0$ 且 $\frac{a}{a_1}\neq\frac{b}{b_1}$,
                              由
                              \[
                                  \begin{cases}
                                      ax+by+c=0, \\
                                      a_1x+b_1y+c_1=0
                                  \end{cases}
                                  \quad\Rightarrow\quad (x_0,y_0)
                              \]
                              令
                              \[
                                  \begin{cases}
                                      x = X + x_0, \\
                                      y = Y + y_0,
                                  \end{cases}
                              \]
                              则方程化为
                              \[
                                  y' = f\!\left(\frac{aX+bY}{a_1X+b_1Y}\right)
                                  \Rightarrow
                                  \frac{dY}{dX} = g\!\left(\frac{Y}{X}\right).
                              \]
                    \end{enumerate}
          \end{enumerate}

          % ========== 一阶线性型 ==========
    \item \textbf{一阶线性型}(或可换元化为此形式)

          若 $y' + p(x)y = q(x)$,
          则通解为:
          \[
              y = e^{-\int p(x)\,dx}
              \left[
                  \int e^{\int p(x)\,dx}\, q(x)\,dx + C
                  \right].
          \]

          % ========== 伯努利型 ==========
    \item \textbf{伯努利方程}

          若 $y' + p(x)y = q(x) y^n \ (n\neq0,1)$,
          变形、换元如下:
          \[
              y^{-n} y' + p(x) y^{1-n} = q(x),
              \qquad
              \text{令 } z = y^{1-n},
          \]
          则
          \[
              \frac{1}{1-n}\frac{dz}{dx} + p(x)z = q(x),
          \]
          为一阶线性方程,可按前述方法求解.
\end{enumerate}

\subsection{二阶可降阶微分方程的求解}
\DTwoThree

若二阶方程中缺少某个变量($x$ 或 $y$),可通过设 $y' = p$ 将其降为一阶方程.

\begin{enumerate}
    % ============================================================
    \item \textbf{缺 $y$ 的情形}:
          方程可写成
          \[
              y'' = f(x, y') \quad \text{或} \quad y'' = f(y').
          \]
          \begin{enumerate}[label=(\roman*)]
              \item 令 $y' = p$,则 $y'' = p' = \dfrac{dp}{dx}$,
                    方程化为一阶方程
                    \[
                        \frac{dp}{dx} = f(x, p)
                        \quad \text{或} \quad
                        \frac{dp}{dx} = f(p).
                    \]
              \item 若其通解为 $p = \varphi(x, C_1)$,即
                    \[
                        y' = \varphi(x, C_1),
                    \]
                    再对 $x$ 积分得:
                    \[
                        \boxed{
                            y = \int \varphi(x, C_1)\,dx + C_2.
                        }
                    \]
          \end{enumerate}

          % ============================================================
    \item \textbf{缺 $x$ 的情形}:
          方程可写成
          \[
              y'' = f(y, y').
          \]
          \begin{enumerate}[label=(\roman*)]
              \item 令 $y' = p$,则
                    \[
                        y'' = \frac{dp}{dx}
                        = \frac{dp}{dy} \cdot \frac{dy}{dx}
                        = p\,\frac{dp}{dy}.
                    \]
                    代入原方程得一阶方程:
                    \[
                        p\,\frac{dp}{dy} = f(y, p).
                    \]
              \item 若其通解为 $p = \varphi(y, C_1)$,
                    则由 $p = \dfrac{dy}{dx}$ 得
                    \[
                        \frac{dy}{dx} = \varphi(y, C_1),
                        \qquad
                        \text{即 } \frac{dy}{\varphi(y, C_1)} = dx.
                    \]
              \item 两边积分得:
                    \[
                        \boxed{
                            \int \frac{dy}{\varphi(y, C_1)} = x + C_2,
                        }
                    \]
                    即为原方程的通解.
          \end{enumerate}
\end{enumerate}

\subsection{高阶常系数线性微分方程的求解}
\DTwoThree

一般形式为
\[
    y'' + p y' + q y = f(x),
\]
其中 $p,q$ 为常数,$f(x)$ 为已知函数.

\begin{enumerate}
    % ============================================================
    \item \textbf{二阶非齐次方程:}
          \[
              y'' + p y' + q y = f(x).
          \]
          \begin{enumerate}[label=(\roman*)]
              \item 写出特征方程:
                    \[
                        \lambda^2 + p\lambda + q = 0,
                        \quad \Rightarrow \quad
                        \lambda_1,\,\lambda_2.
                    \]
              \item 得齐次方程通解:
                    \[
                        y_h =
                        \begin{cases}
                            C_1 e^{\lambda_1 x} + C_2 e^{\lambda_2 x}, & \lambda_1 \neq \lambda_2, \\[0.5em]
                            (C_1 + C_2 x)e^{\lambda x},                & \lambda_1 = \lambda_2.
                        \end{cases}
                    \]
              \item 设特解为 $\dot{y}$,代入原方程求待定系数;
              \item 写出通解:
                    \[
                        \boxed{y = y_h + \dot{y}.}
                    \]
          \end{enumerate}

          % ============================================================
    \item \textbf{右端项可分解时:}
          \[
              y'' + p y' + q y = f_1(x) + f_2(x).
          \]
          \begin{enumerate}[label=(\roman*)]
              \item 对每一项分别求特解:
                    \[
                        \begin{cases}
                            y'' + p y' + q y = f_1(x) \quad \Rightarrow \quad \text{特解 } \dot{y}_1, \\[0.3em]
                            y'' + p y' + q y = f_2(x) \quad \Rightarrow \quad \text{特解 } \dot{y}_2.
                        \end{cases}
                    \]
              \item 总特解为两者之和:
                    \[
                        \boxed{\dot{y} = \dot{y}_1 + \dot{y}_2.}
                    \]
              \item 写出总通解:
                    \[
                        y = y_h + \dot{y}.
                    \]
          \end{enumerate}

          % ============================================================
    \item \textbf{欧拉方程(Cauchy–Euler 型)}
          \[
              x^2 y'' + p x y' + q y = f(x).
          \]
          \begin{enumerate}[label=(\roman*)]
              \item 当 $x>0$ 时,令 $x = e^t$,则 $t = \ln x$,有
                    \[
                        \frac{dy}{dx} = \frac{1}{x}\frac{dy}{dt},
                        \quad
                        \frac{d^2y}{dx^2}
                        = \frac{1}{x^2}\left( \frac{d^2y}{dt^2} - \frac{dy}{dt} \right).
                    \]
                    代入得
                    \[
                        \frac{d^2y}{dt^2} + (p - 1)\frac{dy}{dt} + qy = f(e^t).
                    \]
                    解此方程后,用 $t = \ln x$ 回代.
              \item 当 $x<0$ 时,令 $x = -e^t$,同理可解.
          \end{enumerate}

          % ============================================================
    \item \textbf{$n$阶齐次线性方程:}
          \[
              y^{(n)} + a_1 y^{(n-1)} + \cdots + a_n y = 0.
          \]
          其特征方程为
          \[
              \lambda^n + a_1 \lambda^{n-1} + \cdots + a_n = 0.
          \]
          根据根的类型写通解:
          \begin{enumerate}[label=(\roman*)]
              \item 若 $\lambda$ 为\textbf{单实根}:
                    \[
                        y = C e^{\lambda x};
                    \]
              \item 若 $\lambda$ 为\textbf{$k$重实根}:
                    \[
                        y = (C_1 + C_2 x + \cdots + C_k x^{k-1}) e^{\lambda x};
                    \]
              \item 若 $\lambda = \alpha \pm \beta i$ 为\textbf{单复根}:
                    \[
                        y = e^{\alpha x}(C_1 \cos \beta x + C_2 \sin \beta x);
                    \]
              \item 若 $\lambda = \alpha \pm \beta i$ 为\textbf{二重复根}:
                    \[
                        y = e^{\alpha x}(C_1 \cos \beta x + C_2 \sin \beta x
                        + C_3 x \cos \beta x + C_4 x \sin \beta x).
                    \]
          \end{enumerate}
\end{enumerate}

\subsection{用换元法求解微分方程}

\DTwoThree
\begin{enumerate}
    \item \textbf{根据表达式形式直接换元}
          \begin{enumerate}
              \item 若出现 $f(x\pm y)$,令 $t = x \pm y$;
              \item 若出现 $f(xy)$,令 $t = xy$;
              \item 若出现 $f\!\left(\dfrac{y}{x}\right)$,令 $t = \dfrac{y}{x}$;
              \item 若出现 $f(x^{2}\pm y^{2})$,令 $t = x^{2} \pm y^{2}$.
          \end{enumerate}
          (说明:这类方程常见于可分离或齐次方程.换元后需写出 $dt$ 与 $dx,dy$ 的关系以进行分离.)

    \item \textbf{逆用求导公式进行换元}

          若方程中某部分可看作已知复合函数的导数形式,可通过“逆用求导”来换元简化.
          \[
              \begin{cases}
                  \text{见到 } f^{\prime}[g(x)] \cdot g^{\prime}(x) \ \Rightarrow \ \left(f[g(x)]\right)^{\prime},      & \text{令 } u = f[g(x)];   \\
                  \text{见到 } f^{\prime}(x) g(x) + f(x) g^{\prime}(x) \ \Rightarrow \ \left(f(x) g(x)\right)^{\prime}, & \text{令 } u = f(x) g(x).
              \end{cases}
          \]
          (说明:此法相当于“识别导数结构”,遇到链式或积的求导形式时可直接降阶或积分.)

    \item \textbf{交换自变量与因变量}

          当微分方程关于 $x$ 的形式复杂、而关于 $y$ 的形式更简单时,可以交换 $x,y$ 的角色,
          即令 $x$ 为 $y$ 的函数,用 $\dfrac{dx}{dy}$ 代替 $\dfrac{dy}{dx}$.

          (说明:尤其当方程“缺 $x$”时,令 $p = \dfrac{dy}{dx}$,转为 $p\dfrac{dp}{dy} = f(y,p)$ 类型;若“缺 $y$”,则令 $p = \dfrac{dy}{dx}$,转为 $p' = f(x,p)$.)
\end{enumerate}

\section{建立微分方程并求解}
寻找信息点$A$与信息点$B$,根据题设关系,建立方程

\DThree
\subsection{用极限、导数、积分表达式建方程}
\DTwoTwo+\DThree
\subsection{用几何量表达式建方程}
\DThree
\subsection{用变化率建方程}
\DThree



\GAOchapter{无穷级数}

\section{级数 $\sum_{n=1}^{\infty} u_n$ 的敛散性判别}

\subsection{第一步:计算 $\displaystyle \lim_{n \to \infty} u_n$}

若 $\lim_{n \to \infty} u_n \neq 0$,则级数发散;若等于 0,则需进一步判别。

\begin{example}{}{}
    判别级数 $\displaystyle \sum_{n=2}^{\infty} \left(1-\frac{1}{n}\right)^n$ 的敛散性。
\end{example}
\begin{solution}
    $\displaystyle \lim_{n\to\infty}\left(1-\frac{1}{n}\right)^n=e^{-1}\neq0$,故该级数发散。
\end{solution}

\begin{example}{}{}
    判别级数 $\displaystyle \sum_{n=2}^{\infty}\left(2-\frac{2}{n}\right)^n\ln\left(\frac{1}{2^n}+1\right)$ 的敛散性。
\end{example}
\begin{solution}
    $\displaystyle \lim_{n\to\infty}\left(2-\frac{2}{n}\right)^n\ln\left(\frac{1}{2^n}+1\right)
        =\lim_{n\to\infty}\left(1-\frac{1}{n}\right)^n=e^{-1}\neq0$,
    故该级数发散。
\end{solution}


\subsection{第二步:研究 $u_n$ 的结构特征}

\begin{enumerate}
    \item \textbf{见到 $f(n)$ 型通项}

          \begin{enumerate}
              \item \textbf{恒等变形与放缩技巧}
                    \begin{itemize}
                        \item 若含 $a^n - b^n$,提取 $a^n$ 得 $a^n[1 - (b/a)^n]$。
                        \item 含 $\ln$ 时:
                              \[
                                  \ln b - \ln a = \ln\frac{b}{a}, \quad
                                  \ln(1+n) < n, \quad \ln n < n.
                              \]
                        \item 含 $e^{f(n)}$ 或复杂 $f(n)\pm g(n)$ 时,作 \textbf{泰勒展开},与 $\frac{1}{n^p}$ 比较阶。
                    \end{itemize}

              \item \textbf{含 $(-1)^n$ 时}
                    \begin{itemize}
                        \item 若 $(-1)^n$ 不影响正负性,可去掉。
                        \item 若影响,考虑交错级数判别(莱布尼茨判别法)。
                        \item 例如:
                              \[
                                  \sum_{n=2}^{\infty}\frac{(-1)^n}{\sqrt{n}+(-1)^n}
                                  =\sum_{n=2}^{\infty}\frac{(-1)^n\sqrt{n}}{n-1}-\sum_{n=2}^{\infty}\frac{1}{n-1},
                              \]
                              发散。
                    \end{itemize}
          \end{enumerate}

    \item \textbf{见到 $f(n)$ 与 $f'(n)$}
          \begin{itemize}
              \item 用 \textbf{拉格朗日中值定理}。
              \item 或用求和恒等式:
                    \[
                        \sum_{k=1}^{n} [f(k+1)-f(k)] = f(n+1)-f(1)。
                    \]
          \end{itemize}

    \item \textbf{见到 $f(n)-f(n-1)$}
          优先考虑:
          \begin{itemize}
              \item \textbf{有理化}(处理分子);
              \item \textbf{通分}(处理分母)。
          \end{itemize}

    \item \textbf{见到 $f(a_n)$}
          \begin{itemize}
              \item 若 $\{a_n\}$ 收敛于 $a>0$,则对充分大 $n$,有 $a_n>\tfrac{a}{2}>0$。
              \item 若 $\lim n^2 a_n = a>0$,则 $|a_n|\le \tfrac{M}{n^2}$。
              \item 若 $\lim n^2(a_n-b_n)=k<\infty$,则 $\sum (a_n-b_n)$ 收敛。
              \item 若 $\lim_{n\to\infty}a_n=p$,则
                    \[
                        \sum\frac{1}{n^{a_n}}
                        \begin{cases}
                            \text{发散,} & p<1, \\
                            \text{收敛,} & p>1, \\
                            \text{不定,} & p=1。
                        \end{cases}
                    \]
          \end{itemize}

    \item \textbf{见到 $f(a_n,a_{n+1})$}
          \begin{itemize}
              \item 若给出 $f(a_n,a_{n+1})$ 的关系式,可尝试写为 $S_n$ 或差分形式 $f(a_{n+1})-f(a_n)$。
          \end{itemize}

    \item \textbf{见到 $f(a_n,b_n)$}
          \begin{itemize}
              \item 建立关系:
                    \[
                        a_n b_n = n a_n \cdot \frac{b_n}{n}, \quad
                        a_n = (a_n-b_n)+b_n,
                    \]
                    或令 $\frac{b_n}{a_n}=c_n$,转化为单变量函数 $f(c_n)$。
          \end{itemize}

    \item \textbf{见到 $f(a_n, n^p)$}
          常用不等式:
          \[
              |ab|\le \frac{a^2+b^2}{2}。
          \]

    \item \textbf{见到 $f(a_n,S_n)$}
          \begin{itemize}
              \item 代入定义 $a_n=S_n-S_{n-1}$;
              \item 写出 $\sum a_n = S_n$,检查是否望项消去;
              \item $\sum a_n$ 收敛 $\Rightarrow S_n$ 有界;
              \item $\sum a_n$ 收敛 $\Leftarrow S_n$ 有界且 $a_n\to 0$。
          \end{itemize}

    \item \textbf{见到 $(-1)^n - 1$ 型通项}
          \begin{itemize}
              \item 可作 \textbf{泰勒展开} 分项讨论;
              \item 若可能,造出交错形式,用 \textbf{莱布尼茨判别法};
              \item 若失败,尝试 \textbf{绝对收敛判别法};
              \item 拆项法:若
                    \[
                        \sum\frac{f(n)\pm g(n)}{h(n)} = \sum\frac{f(n)}{h(n)} \pm \sum\frac{g(n)}{h(n)},
                    \]
                    分别讨论敛散性。
          \end{itemize}
\end{enumerate}


\subsection{常用判别法总结}
\begin{itemize}
    \item \textbf{比较判别法}($u_n>0$)
          \[
              \text{若 }0<u_n<v_n\text{ 且 }\sum v_n\text{ 收敛,则 }\sum u_n\text{ 收敛。}
          \]
    \item \textbf{比值判别法}:若 $\displaystyle \lim_{n\to\infty}\frac{u_{n+1}}{u_n}=q$,
          \[
              \begin{cases}
                  q<1 & \Rightarrow \text{收敛}, \\
                  q>1 & \Rightarrow \text{发散}, \\
                  q=1 & \Rightarrow \text{不定。}
              \end{cases}
          \]
    \item \textbf{根值判别法}:若 $\displaystyle \lim_{n\to\infty}\sqrt[n]{|u_n|}=q$,
          结论同上。
    \item \textbf{积分判别法}:若 $u_n=f(n)$ 且 $f(x)\ge 0$ 单调递减,
          \[
              \sum_{n=1}^{\infty} f(n)\text{ 收敛} \iff \int_1^{\infty} f(x)\,dx\text{ 收敛。}
          \]
    \item \textbf{莱布尼茨判别法}(交错级数):
          若 $a_n>0$ 且 $a_n\downarrow 0$,则 $\sum (-1)^{n-1}a_n$ 收敛。
\end{itemize}

\section{求幂级数的和函数}

\subsection{确定收敛域}

\begin{enumerate}
    \item \textbf{具体型问题}
          \begin{enumerate}
              \item 对于一般幂级数 $\displaystyle \sum_{n=0}^{\infty} a_n x^n$:

                    \textbf{收敛半径:}
                    \[
                        R =
                        \begin{cases}
                            \dfrac{1}{\rho}, & \rho = \lim\limits_{n\to\infty}\left|\dfrac{a_{n+1}}{a_n}\right|
                            \text{ 或 } \rho = \lim\limits_{n\to\infty}\sqrt[n]{|a_n|},                          \\[0.8em]
                            +\infty,         & \rho = 0,                                                        \\[0.3em]
                            0,               & \rho = +\infty.
                        \end{cases}
                    \]

                    \textbf{收敛区间与收敛域:}
                    在区间 $(-R, R)$ 内绝对收敛。
                    当 $x = \pm R$ 时需单独讨论敛散性,故收敛域可能为
                    $(-R, R)$、$[-R, R)$、$(-R, R]$ 或 $[-R, R]$。

              \item 对于缺项幂级数或一般函数项级数 $\sum u_n(x)$:

                    \begin{enumerate}
                        \item 取绝对值写作 $\sum |u_n(x)|$;
                        \item 用比值(或根值)判别法求出收敛区间 $(a,b)$;
                        \item 再讨论 $x=a,b$ 时的敛散性,从而确定收敛域。
                    \end{enumerate}
          \end{enumerate}

    \item \textbf{抽象型问题——阿贝尔定理的应用}
          \begin{enumerate}
              \item 若 $\sum a_n x^n$ 在 $x=x_1(\ne 0)$ 处收敛,则当 $|x|<|x_1|$ 时绝对收敛;
                    若在 $x=x_2(\ne 0)$ 处发散,则当 $|x|>|x_2|$ 时发散。

              \item 根据阿贝尔定理:
                    \[
                        \begin{cases}
                            \text{若在 } x_1 \text{ 处收敛}   & \Rightarrow R \ge |x_1 - x_0|, \\
                            \text{若在 } x_1 \text{ 处发散}   & \Rightarrow R \le |x_1 - x_0|, \\
                            \text{若在 } x_1 \text{ 处条件收敛} & \Rightarrow R = |x_1 - x_0|.
                        \end{cases}
                    \]

              \item 已知 $\sum a_n (x-x_1)^n$ 的敛散性,讨论 $\sum b_m (x-x_2)^m$:
                    \begin{enumerate}
                        \item 平移或提出 $(x-x_0)^k$,收敛半径不变;
                        \item \textbf{逐项求导:}收敛半径不变,收敛域可能缩小;
                        \item \textbf{逐项积分:}收敛半径不变,收敛域可能扩大。
                    \end{enumerate}
          \end{enumerate}
\end{enumerate}

\subsection{求和函数的方法}

\begin{enumerate}
    \item \textbf{先积后导、先导后积法}
          \begin{enumerate}
              \item $\sum (a n + b)x^n$:先积分再求导;
              \item $\sum \dfrac{x^n}{a n + b}$:先求导再积分;
              \item $\sum \dfrac{c n^2 + d n + e}{a n + b}x^n$:拆为若干简单级数相加。
          \end{enumerate}

    \item \textbf{常用幂级数公式}
          \begin{multicols}{2}
              \begin{enumerate}
                  \item $\displaystyle \sum_{n=0}^{\infty} x^n = \frac{1}{1-x}$, ($|x|<1$);
                  \item $\displaystyle \sum_{n=1}^{\infty} n x^{n-1} = \frac{1}{(1-x)^2}$;
                  \item $\displaystyle \sum_{n=2}^{\infty} n(n-1)x^{n-2} = \frac{2}{(1-x)^3}$;
                  \item $\displaystyle \sum_{n=1}^{\infty} \frac{x^n}{n} = -\ln(1-x)$;
                  \item $\displaystyle \sum_{n=0}^{\infty} \frac{x^{2n+1}}{2n+1} = \frac{1}{2}\ln\frac{1+x}{1-x}$;
                  \item $\displaystyle \sum_{n=0}^{\infty} \frac{(-1)^n x^{2n+1}}{2n+1} = \arctan x$;
                  \item $\displaystyle \sum_{n=0}^{\infty} \frac{x^n}{n!} = e^x$;
                  \item $\displaystyle \sum_{n=0}^{\infty} \frac{x^{2n}}{(2n)!} = \cosh x = \frac{e^x+e^{-x}}{2}$;
                  \item $\displaystyle \sum_{n=0}^{\infty} \frac{(-1)^n x^{2n+1}}{(2n+1)!} = \sin x$;
                  \item $\displaystyle \sum_{n=0}^{\infty} \frac{(-1)^n x^{2n}}{(2n)!} = \cos x$.
              \end{enumerate}
          \end{multicols}
\end{enumerate}

\subsection{微分方程法求和函数}

\begin{enumerate}
    \item \textbf{已给微分方程型:}
          \begin{enumerate}
              \item 验证级数满足某微分方程;
              \item 求通解;
              \item 根据初值条件确定常数;
              \item 代入特定 $x$(如 $0,\frac{1}{2},1$)求具体和。
          \end{enumerate}

    \item \textbf{由通项关系建立微分方程型:}
          \begin{enumerate}
              \item 根据 $a_{n+1}$、$a_n$、$a_{n-1}$ 的关系式建立方程;
              \item 求解通解;
              \item 展开为 $\sum a_n x^n$ 并确定 $a_n$ 通项。
          \end{enumerate}
\end{enumerate}

\textbf{总结:}

\begin{itemize}
    \item 求收敛域首选比值法或根值法;
    \item 边界常考:$\sum \frac{x^n}{n}$、$\sum \frac{(-1)^n x^n}{n}$;
    \item 熟记导数与积分规律:求导 $\Rightarrow$ 乘 $n$;积分 $\Rightarrow$ 除 $n$;
    \item 收敛半径不变的三种操作:提因式、逐项求导、逐项积分;
    \item 求和函数常从 $\sum x^n$ 推导。
\end{itemize}

\section{函数展开成幂级数}

\begin{enumerate}
    \item \textbf{含对数函数的展开}

          \begin{enumerate}
              \item $\ln(a + bx)$ 型:
                    \[
                        \ln(a + bx) = \ln a + \ln\left(1 + \frac{b}{a}x\right), \quad a > 0.
                    \]
                    若 $|x| < \frac{a}{b}$,则
                    \[
                        \ln(1 + \tfrac{b}{a}x) = \sum_{n=1}^{\infty}(-1)^{n-1}\frac{1}{n}\left(\frac{b}{a}x\right)^n.
                    \]
              \item $\ln(1 + ax + bx^2)$ 型:
                    \[
                        \ln(1 + ax + bx^2)
                        = \ln(1 + cx) + \ln(1 + dx),
                    \]
                    其中 $a = c + d,\ b = cd.$
          \end{enumerate}

    \item \textbf{含分式的展开}

          \begin{enumerate}
              \item $\dfrac{1}{a + bx}$ 型:
                    \[
                        \frac{1}{a + bx}
                        = \frac{1}{a} \cdot \frac{1}{1 + \frac{b}{a}x}
                        = \frac{1}{a}\sum_{n=0}^{\infty}(-1)^n\left(\frac{b}{a}x\right)^n, \quad |x| < \frac{a}{b}.
                    \]
              \item $\dfrac{1}{(x+a)(x+b)}$ 型:
                    \[
                        \frac{1}{(x+a)(x+b)}
                        = \frac{1}{b - a}\left(\frac{1}{x+a} - \frac{1}{x+b}\right).
                    \]
          \end{enumerate}

    \item \textbf{含三角平方函数的化简}

          \[
              \sin^2x = \frac{1 - \cos 2x}{2}, \qquad
              \cos^2x = \frac{1 + \cos 2x}{2}.
          \]

          (常用于将 $\sin^2 x$、$\cos^2 x$ 转化为含单角的可积形式。)
\end{enumerate}

\section{傅里叶级数}

\subsection{周期为 $2l$ 的傅里叶级数定义}

设 $f(x)$ 是以 $2l$ 为周期的可积函数,则称
\[
    a_n = \frac{1}{l}\int_{-l}^{l} f(x)\cos\frac{n\pi x}{l}\,\mathrm{d}x, \quad
    b_n = \frac{1}{l}\int_{-l}^{l} f(x)\sin\frac{n\pi x}{l}\,\mathrm{d}x
\]
为 $f(x)$ 的傅里叶系数。

则其傅里叶级数为
\[
    f(x) \sim \frac{a_0}{2}
    + \sum_{n=1}^{\infty}\left(a_n\cos\frac{n\pi x}{l}
    + b_n\sin\frac{n\pi x}{l}\right).
\]

\subsection{狄利克雷收敛定理}

若周期为 $2l$ 的函数 $f(x)$ 在区间 $[-l,l]$ 上满足:

\begin{enumerate}
    \item 连续或仅有有限个第一类间断点;
    \item 至多具有有限个极值点,
\end{enumerate}

则其傅里叶级数在 $[-l,l]$ 上处处收敛,和函数为
\[
    S(x) =
    \begin{cases}
        f(x),                      & x\ \text{为连续点}, \\[0.4em]
        \dfrac{f(x-0)+f(x+0)}{2},  & x\ \text{为间断点}, \\[0.8em]
        \dfrac{f(-l+0)+f(l-0)}{2}, & x = \pm l.
    \end{cases}
\]
\subsection{正弦级数与余弦级数}

\begin{enumerate}
    \item 当 $f(x)$ 为\textbf{奇函数}时:
          \[
              f(x) \sim \sum_{n=1}^{\infty} b_n \sin\frac{n\pi x}{l},
              \quad b_n = \frac{2}{l}\int_0^l f(x)\sin\frac{n\pi x}{l}\,\mathrm{d}x.
          \]

    \item 当 $f(x)$ 为\textbf{偶函数}时:
          \[
              f(x) \sim \frac{a_0}{2}
              + \sum_{n=1}^{\infty} a_n \cos\frac{n\pi x}{l},
              \quad a_n = \frac{2}{l}\int_0^l f(x)\cos\frac{n\pi x}{l}\,\mathrm{d}x.
          \]
\end{enumerate}

\subsection{区间 $[0,l]$ 上函数的傅里叶展开}

若 $f(x)$ 仅定义在 $[0,l]$ 上,则先作\textbf{周期延拓}:

\begin{enumerate}
    \item \textbf{周期奇延拓(正弦级数)}

          \[
              F(x) =
              \begin{cases}
                  f(x),   & 0 < x \le l, \\
                  -f(-x), & -l < x < 0,  \\
                  0,      & x = 0,
              \end{cases}
          \]
          再令 $F(x)$ 为以 $2l$ 为周期的函数,则
          \[
              f(x) \sim \sum_{n=1}^{\infty} b_n \sin\frac{n\pi x}{l},
              \quad b_n = \frac{2}{l}\int_0^l f(x)\sin\frac{n\pi x}{l}\,\mathrm{d}x.
          \]

    \item \textbf{周期偶延拓(余弦级数)}

          \[
              F(x) =
              \begin{cases}
                  f(x),  & 0 \le x \le l, \\
                  f(-x), & -l < x < 0,
              \end{cases}
          \]
          再令 $F(x)$ 为以 $2l$ 为周期的函数,则
          \[
              f(x) \sim \frac{a_0}{2}
              + \sum_{n=1}^{\infty} a_n \cos\frac{n\pi x}{l},
              \quad a_n = \frac{2}{l}\int_0^l f(x)\cos\frac{n\pi x}{l}\,\mathrm{d}x.
          \]
\end{enumerate}

\subsection{总结}

\begin{itemize}
    \item 傅里叶级数是将函数在区间上展开为 $\sin$ 与 $\cos$ 线性组合;
    \item 奇函数 $\Rightarrow$ 正弦级数;偶函数 $\Rightarrow$ 余弦级数;
    \item 延拓思想是将 $[0,l]$ 上函数扩展为周期函数;
    \item 计算核心:求 $a_n$、$b_n$ 两类系数;
    \item 在间断点取左右极限的平均值;
    \item 傅里叶级数是高等数学与信号分析的桥梁。
\end{itemize}
\GAOchapter{多元函数积分学的预备知识}
\GAOchapter{多元函数积分学}
\section{计算三重积分}

\subsection{和式极限定义}

设区域
\[
      \Omega = \{(x, y, z) \mid a \le x \le b,\, c \le y \le d,\, e \le z \le f\},
\]
则三重积分定义为
\[
      \iiint_{\Omega} g(x, y, z) \, \mathrm{d}v
      = \lim_{n \to \infty}
      \sum_{i=1}^{n} \sum_{j=1}^{n} \sum_{k=1}^{n}
      g\!\left(a + \tfrac{b-a}{n}i,\, c + \tfrac{d-c}{n}j,\, e + \tfrac{f-e}{n}k\right)
      \frac{(b-a)(d-c)(f-e)}{n^3}.
\]

\subsection{积分次序的交换}

将所给积分次序还原为
\[
      \iiint_\Omega f(x,y,z)\, \mathrm{d}v,
\]
再根据区域特征或函数形式,选择新的积分次序以便计算(如先 $z$ 后 $x,y$ 或先 $x$ 后 $y,z$)。

\subsection{积分的保号性}

\begin{enumerate}
      \item 若 $f(x,y,z)$ 在 $\Omega$ 上连续且 $f \ge 0$ 且不恒为零,则
            \[
                  \iiint_{\Omega} f(x,y,z)\, \mathrm{d}v > 0.
            \]
      \item 若连续函数 $f(x,y,z)$ 满足对任意有界闭区域 $\Omega$,
            \[
                  \iiint_{\Omega} f(x,y,z)\, \mathrm{d}v = 0,
            \]
            则必有 $f(x,y,z) \equiv 0$ 于该区域。
\end{enumerate}

\subsection{对称性的应用}

与二重积分完全类似。

\begin{enumerate}
      \item \textbf{普通对称性.}
            \begin{enumerate}
                  \item 若 $\Omega$ 关于 $xOz$ 面对称,则
                        \[
                              \iiint_{\Omega} f(x, y, z)\, \mathrm{d}v =
                              \begin{cases}
                                    2\iiint_{\Omega_1} f(x,y,z)\, \mathrm{d}v, & f(x,y,z) = f(x,-y,z),  \\[4pt]
                                    0,                                         & f(x,y,z) = -f(x,-y,z),
                              \end{cases}
                        \]
                        其中 $\Omega_1$ 为 $\Omega$ 在 $xOz$ 面右侧部分。
                  \item 若 $\Omega$ 关于三个坐标面对称,$\Omega_1$ 为第一卦限部分,则
                        \[
                              \iiint_{\Omega} f(x,y,z)\, \mathrm{d}v =
                              \begin{cases}
                                    8\iiint_{\Omega_1} f(x,y,z)\, \mathrm{d}v, & f(x,y,z)=f(-x,-y,-z),  \\[4pt]
                                    0,                                         & f(x,y,z)=-f(-x,-y,-z).
                              \end{cases}
                        \]
            \end{enumerate}

      \item \textbf{轮换对称性.}
            若 $\Omega$ 在交换 $x,y$ 后不变,则
            \[
                  \iiint_{\Omega} f(x,y,z)\, \mathrm{d}v
                  = \iiint_{\Omega} f(y,x,z)\, \mathrm{d}v.
            \]
            特别地,若 $\Omega=\{x^2+y^2+z^2\le R^2\}$,则
            \[
                  I = \iiint_{\Omega} f(x)\, \mathrm{d}v
                  = \iiint_{\Omega} f(y)\, \mathrm{d}v
                  = \iiint_{\Omega} f(z)\, \mathrm{d}v
                  = \tfrac{1}{3} \iiint_{\Omega} [f(x)+f(y)+f(z)]\, \mathrm{d}v.
            \]
\end{enumerate}

\subsection{直角坐标系下的积分方法}

\begin{enumerate}
      \item \textbf{先一后二法(投影穿线法)}
            \begin{enumerate}
                  \item 适用:区域 $\Omega$ 由下曲面 $z=z_1(x,y)$ 与上曲面 $z=z_2(x,y)$ 所围。
                  \item 计算公式:
                        \[
                              \iiint_{\Omega} f(x,y,z)\, \mathrm{d}v
                              = \iint_{D_{xy}} \mathrm{d}\sigma
                              \int_{z_1(x,y)}^{z_2(x,y)} f(x,y,z)\, \mathrm{d}z.
                        \]
            \end{enumerate}

      \item \textbf{先二后一法(定限截面法)}
            \begin{enumerate}
                  \item 适用:$\Omega$ 是旋转体或按 $z$ 分层。
                  \item 计算公式:
                        \[
                              \iiint_{\Omega} f(x,y,z)\, \mathrm{d}v
                              = \int_{a}^{b} \mathrm{d}z
                              \iint_{D_z} f(x,y,z)\, \mathrm{d}\sigma.
                        \]
            \end{enumerate}
\end{enumerate}

\subsection{柱面坐标系积分法}

若积分区域适于极坐标表示,令
\[
      \begin{cases}
            x = r\cos\theta, \\
            y = r\sin\theta,
      \end{cases}
\]
则
\[
      \iiint_{\Omega} f(x,y,z)\, \mathrm{d}x\mathrm{d}y\mathrm{d}z
      = \iiint_{\Omega} f(r\cos\theta,r\sin\theta,z)\, r\, \mathrm{d}r\,\mathrm{d}\theta\,\mathrm{d}z.
\]

\subsection{球面坐标系积分法}

\begin{enumerate}
      \item \textbf{适用场合:}
            \begin{enumerate}
                  \item 被积函数含有 $f(x^2 + y^2 + z^2)$;
                  \item 积分区域为球体或锥体的部分。
            \end{enumerate}
      \item \textbf{坐标变换:}
            \[
                  \begin{cases}
                        x = r\sin\varphi\cos\theta, \\
                        y = r\sin\varphi\sin\theta, \\
                        z = r\cos\varphi,
                  \end{cases}
                  \quad
                  \mathrm{d}v = r^2 \sin\varphi\, \mathrm{d}r\, \mathrm{d}\varphi\, \mathrm{d}\theta.
            \]
      \item \textbf{积分形式:}
            \[
                  \iiint_{\Omega} f(x,y,z)\, \mathrm{d}v
                  = \int_{\theta_1}^{\theta_2} \!\mathrm{d}\theta
                  \int_{\varphi_1(\theta)}^{\varphi_2(\theta)} \!\mathrm{d}\varphi
                  \int_{r_1(\varphi,\theta)}^{r_2(\varphi,\theta)}\!
                  f(r\sin\varphi\cos\theta, r\sin\varphi\sin\theta, r\cos\varphi)
                  r^2 \sin\varphi\, \mathrm{d}r.
            \]
\end{enumerate}
\subsection{重积分的应用}

\begin{enumerate}
      \item \textbf{体积:}
            \[
                  V = \iiint_{\Omega} 1\, \mathrm{d}v.
            \]

      \item \textbf{重心(质心):}
            若体密度为 $\rho(x,y,z)$,则
            \[
                  \overline{x} = \frac{\iiint_{\Omega} x\rho\, \mathrm{d}v}{\iiint_{\Omega} \rho\, \mathrm{d}v},\quad
                  \overline{y} = \frac{\iiint_{\Omega} y\rho\, \mathrm{d}v}{\iiint_{\Omega} \rho\, \mathrm{d}v},\quad
                  \overline{z} = \frac{\iiint_{\Omega} z\rho\, \mathrm{d}v}{\iiint_{\Omega} \rho\, \mathrm{d}v}.
            \]

      \item \textbf{引力:}
            对物体外一点 $M_0(x_0,y_0,z_0)$,质量为 $m$,体密度 $\rho(x,y,z)$,
            \[
                  \begin{aligned}
                        F_x & = Gm \iiint_{\Omega} \frac{\rho(x,y,z)(x-x_0)}{[(x-x_0)^2+(y-y_0)^2+(z-z_0)^2]^{3/2}}\, \mathrm{d}v, \\
                        F_y & = Gm \iiint_{\Omega} \frac{\rho(x,y,z)(y-y_0)}{[(x-x_0)^2+(y-y_0)^2+(z-z_0)^2]^{3/2}}\, \mathrm{d}v, \\
                        F_z & = Gm \iiint_{\Omega} \frac{\rho(x,y,z)(z-z_0)}{[(x-x_0)^2+(y-y_0)^2+(z-z_0)^2]^{3/2}}\, \mathrm{d}v.
                  \end{aligned}
            \]

      \item \textbf{转动惯量:}
            \[
                  \begin{aligned}
                        I_x & = \iiint_{\Omega} \rho(y^2+z^2)\, \mathrm{d}v,     \\
                        I_y & = \iiint_{\Omega} \rho(x^2+z^2)\, \mathrm{d}v,     \\
                        I_z & = \iiint_{\Omega} \rho(x^2+y^2)\, \mathrm{d}v,     \\
                        I_O & = \iiint_{\Omega} \rho(x^2+y^2+z^2)\, \mathrm{d}v.
                  \end{aligned}
            \]
\end{enumerate}
\section{计算第一型曲面积分}

\subsection{定义}

设 $f(x,y,z)$ 定义在光滑曲面 $\Sigma$ 上,若将 $\Sigma$ 分割为若干小面元 $\Delta S_i$,取 $\Sigma_i$ 上一点 $M_i(x_i,y_i,z_i)$,当面元最大直径趋于零时,若极限
\[
      \lim_{\lambda \to 0} \sum_i f(x_i,y_i,z_i)\Delta S_i
\]
存在,则称其为函数 $f(x,y,z)$ 在曲面 $\Sigma$ 上的\textbf{第一型曲面积分},记作
\[
      \iint_{\Sigma} f(x,y,z)\, \mathrm{d}S.
\]

其物理意义:当 $f(x,y,z)\ge0$ 表示面密度时,该积分即为曲面薄片的质量。


\subsection{代入曲面方程}

若曲面方程为 $\Sigma: z = z(x,y)$,则需将其代入被积函数:
\[
      f(x,y,z) \;\Rightarrow\; f(x,y,z(x,y)),
\]
并结合面积微元
\[
      \mathrm{d}S = \sqrt{1 + z_x'^2 + z_y'^2}\, \mathrm{d}x\, \mathrm{d}y,
\]
从而化为二重积分形式:
\[
      \iint_{\Sigma} f(x,y,z)\, \mathrm{d}S
      = \iint_{D_{xy}} f(x,y,z(x,y)) \sqrt{1 + z_x'^2 + z_y'^2}\, \mathrm{d}x\, \mathrm{d}y.
\]


\subsection{几何意义}

当 $f(x,y,z)\equiv1$ 时,
\[
      \iint_{\Sigma} \mathrm{d}S
\]
即为曲面 $\Sigma$ 的面积。


\subsection{形心公式的应用}

若曲面 $\Sigma$ 的形心(或质心)坐标为 $(\overline{x},\overline{y},\overline{z})$,则
\[
      \overline{x} = \frac{\iint_{\Sigma} x\, \mathrm{d}S}{\iint_{\Sigma} \mathrm{d}S}, \quad
      \overline{y} = \frac{\iint_{\Sigma} y\, \mathrm{d}S}{\iint_{\Sigma} \mathrm{d}S}, \quad
      \overline{z} = \frac{\iint_{\Sigma} z\, \mathrm{d}S}{\iint_{\Sigma} \mathrm{d}S}.
\]
若 $\Sigma$ 为规则对称图形(形心坐标已知,面积易求),可得
\[
      \iint_{\Sigma} x\, \mathrm{d}S = \overline{x} \, S_{\Sigma}, \quad
      \iint_{\Sigma} y\, \mathrm{d}S = \overline{y} \, S_{\Sigma}, \quad
      \iint_{\Sigma} z\, \mathrm{d}S = \overline{z} \, S_{\Sigma}.
\]


\subsection{对称性}

\begin{enumerate}
      \item \textbf{普通对称性.}
            若 $\Sigma$ 关于 $xOz$ 面对称,则
            \[
                  \iint_{\Sigma} f(x,y,z)\, \mathrm{d}S =
                  \begin{cases}
                        2 \iint_{\Sigma_1} f(x,y,z)\, \mathrm{d}S, & f(x,y,z) = f(x,-y,z),  \\[4pt]
                        0,                                         & f(x,y,z) = -f(x,-y,z),
                  \end{cases}
            \]
            其中 $\Sigma_1$ 为 $\Sigma$ 在 $xOz$ 面右侧部分。
            其他关于 $yOz$ 或 $xOy$ 的情形类似。

      \item \textbf{轮换对称性.}
            若曲面 $\Sigma: z=z(x,y)$ 在交换 $x,y$ 后不变,则
            \[
                  \iint_{\Sigma} f(x,y,z)\, \mathrm{d}S
                  = \iint_{\Sigma} f(y,x,z)\, \mathrm{d}S.
            \]
            若 $\Sigma$ 关于 $x,y,z$ 完全对称,则
            \[
                  \iint_{\Sigma} f(x,y,z)\, \mathrm{d}S
                  = \frac{1}{3} \iint_{\Sigma} [f(x,y,z) + f(y,z,x) + f(z,x,y)]\, \mathrm{d}S.
            \]
\end{enumerate}


\subsection{物理应用}

\begin{enumerate}
      \item \textbf{重心(质心)或形心.}
            对光滑曲面薄片 $\Sigma$,面密度为 $\rho(x,y,z)$,则
            \[
                  \overline{x} = \frac{\iint_{\Sigma} x\rho(x,y,z)\, \mathrm{d}S}{\iint_{\Sigma} \rho(x,y,z)\, \mathrm{d}S}, \quad
                  \overline{y} = \frac{\iint_{\Sigma} y\rho(x,y,z)\, \mathrm{d}S}{\iint_{\Sigma} \rho(x,y,z)\, \mathrm{d}S}, \quad
                  \overline{z} = \frac{\iint_{\Sigma} z\rho(x,y,z)\, \mathrm{d}S}{\iint_{\Sigma} \rho(x,y,z)\, \mathrm{d}S}.
            \]

      \item \textbf{转动惯量.}
            面密度为 $\rho(x,y,z)$ 的曲面薄片 $\Sigma$,其关于 $x$、$y$、$z$ 轴及原点的转动惯量分别为:
            \[
                  \begin{aligned}
                        I_x & = \iint_{\Sigma} (y^2 + z^2)\rho(x,y,z)\, \mathrm{d}S,       \\
                        I_y & = \iint_{\Sigma} (z^2 + x^2)\rho(x,y,z)\, \mathrm{d}S,       \\
                        I_z & = \iint_{\Sigma} (x^2 + y^2)\rho(x,y,z)\, \mathrm{d}S,       \\
                        I_O & = \iint_{\Sigma} (x^2 + y^2 + z^2)\rho(x,y,z)\, \mathrm{d}S.
                  \end{aligned}
            \]
\end{enumerate}

\section{计算第一型曲面积分}

\subsection{定义与物理意义}

设函数 $f(x,y,z)$ 定义在空间曲面 $\Sigma$ 上。
第一型曲面积分的物理意义是:若 $f(x,y,z) \ge 0$ 表示曲面上各点的面密度,则
\[
      \iint_{\Sigma} f(x,y,z)\,\mathrm{d}S
\]
表示该物质曲面的总质量。
其定义方式与二重积分、三重积分类似,都是“分割—近似—求和—取极限”的结果。


\subsection{代入曲面方程}

若曲面 $\Sigma$ 可表示为 $z = z(x,y)$,则需将其代入被积函数中化简:
\[
      f(x,y,z) \;\Rightarrow\; f\big(x,\,y,\,z(x,y)\big).
\]


\subsection{几何意义}

若 $f(x,y,z) \equiv 1$,则
\[
      \iint_{\Sigma} \mathrm{d}S = S_{\Sigma},
\]
即为曲面 $\Sigma$ 的面积。

\subsection{利用形心公式}

由形心定义:
\[
      \overline{x} = \frac{\iint_{\Sigma} x\,\mathrm{d}S}{\iint_{\Sigma}\mathrm{d}S},
\]
可得
\[
      \iint_{\Sigma} x\,\mathrm{d}S = \overline{x} \cdot S_{\Sigma}.
\]
当 $\Sigma$ 为规则图形(形心坐标 $\overline{x}$ 已知且面积易求)时,此公式尤为方便。


\subsection{利用对称性质}

\begin{enumerate}
      \item \textbf{普通对称性:}
            若 $\Sigma$ 关于 $xOz$ 面对称,则
            \[
                  \iint_{\Sigma} f(x,y,z)\,\mathrm{d}S =
                  \begin{cases}
                        2\displaystyle\iint_{\Sigma_1} f(x,y,z)\,\mathrm{d}S, & f(x,y,z)=f(x,-y,z),  \\[6pt]
                        0,                                                    & f(x,y,z)=-f(x,-y,z),
                  \end{cases}
            \]
            其中 $\Sigma_1$ 为 $\Sigma$ 在 $xOz$ 面右侧部分。
            其余坐标面对称情况类似。

      \item \textbf{轮换对称性:}
            若曲面 $\Sigma: z=z(x,y)$ 在交换 $x,y$ 后保持不变,则
            \[
                  \iint_{\Sigma} f(x,y,z)\,\mathrm{d}S = \iint_{\Sigma} f(y,x,z)\,\mathrm{d}S.
            \]
            若 $\Sigma$ 关于 $x,y,z$ 三坐标完全对称,则
            \[
                  \iint_{\Sigma} f(x,y,z)\,\mathrm{d}S
                  = \frac{1}{3}\iint_{\Sigma}\!\big[f(x,y,z)+f(y,z,x)+f(z,x,y)\big]\,\mathrm{d}S.
            \]
\end{enumerate}


\subsection{物理应用}

\begin{enumerate}
      \item \textbf{形心(重心)计算:}
            对面密度为 $\rho(x,y,z)$ 的光滑曲面 $\Sigma$,其形心 $(\overline{x},\overline{y},\overline{z})$ 为
            \[
                  \overline{x} = \frac{\iint_{\Sigma} x\rho(x,y,z)\,\mathrm{d}S}{\iint_{\Sigma}\rho(x,y,z)\,\mathrm{d}S},\quad
                  \overline{y} = \frac{\iint_{\Sigma} y\rho(x,y,z)\,\mathrm{d}S}{\iint_{\Sigma}\rho(x,y,z)\,\mathrm{d}S},\quad
                  \overline{z} = \frac{\iint_{\Sigma} z\rho(x,y,z)\,\mathrm{d}S}{\iint_{\Sigma}\rho(x,y,z)\,\mathrm{d}S}.
            \]

      \item \textbf{转动惯量:}
            对同一曲面,其关于各坐标轴与原点的转动惯量分别为:
            \[
                  I_x = \iint_{\Sigma}(y^2+z^2)\rho(x,y,z)\,\mathrm{d}S,\quad
                  I_y = \iint_{\Sigma}(z^2+x^2)\rho(x,y,z)\,\mathrm{d}S,
            \]
            \[
                  I_z = \iint_{\Sigma}(x^2+y^2)\rho(x,y,z)\,\mathrm{d}S,\quad
                  I_O = \iint_{\Sigma}(x^2+y^2+z^2)\rho(x,y,z)\,\mathrm{d}S.
            \]
\end{enumerate}


\subsection*{小结}

\begin{itemize}
      \item 第一型曲面积分计算的是曲面“面积型”量(如面积、质量、形心、惯量等);
      \item 对称性与形心法常用于快速求积分;
      \item 若 $f(x,y,z)=1$,积分即为曲面面积。
\end{itemize}

\section{计算第二型线面积分}
\subsection{第二型曲线积分}

\begin{enumerate}
      \item \textbf{定义与物理意义(做功)}

            设向量场
            \[
                  \mathbf{F}(x,y)=P(x,y)\mathbf{i}+Q(x,y)\mathbf{j}
                  \quad\text{或}\quad
                  \mathbf{F}(x,y,z)=P(x,y,z)\mathbf{i}+Q(x,y,z)\mathbf{j}+R(x,y,z)\mathbf{k},
            \]
            定义在有向曲线 $L$(或空间曲线 $\Gamma$)上,则第二型曲线积分表示变力 $\mathbf{F}$ 沿该曲线从起点到终点所做的功:
            \[
                  \int_{L} P\,dx + Q\,dy
                  \quad\text{或}\quad
                  \int_{\Gamma} P\,dx + Q\,dy + R\,dz.
            \]

            与定积分、二重积分、三重积分、第一型曲线与曲面积分不同,
            第二型曲线积分是\textbf{向量场沿有向曲线的积分}(非几何量)。

            \[
                  \begin{cases}
                        \text{平面:} & \displaystyle\int_L (P,Q)\cdot(dx,dy) = \int_L P\,dx + Q\,dy,                        \\[6pt]
                        \text{空间:} & \displaystyle\int_\Gamma (P,Q,R)\cdot(dx,dy,dz) = \int_\Gamma P\,dx + Q\,dy + R\,dz.
                  \end{cases}
            \]


      \item \textbf{计算方法}

            \begin{enumerate}
                  \item \textbf{对称性法(类对称)}

                        若 $L^*$ 可分为关于某直线类对称的两部分 $L_1,L_2$,且对称点处 $P$ 绝对值相等,则
                        \[
                              \int_{L^*} P\,dx =
                              \begin{cases}
                                    2\displaystyle\int_{L_1} P\,dx, & P(x,y)=P(-x,y),  \\[6pt]
                                    0,                              & P(x,y)=-P(-x,y),
                              \end{cases}
                              \qquad
                              \int_{L^*} Q\,dy =
                              \begin{cases}
                                    0,                              & Q(x,y)=Q(-x,y),  \\[6pt]
                                    2\displaystyle\int_{L_1} Q\,dy, & Q(x,y)=-Q(-x,y).
                              \end{cases}
                        \]

                  \item \textbf{参数化法——一投二代三计算(化为定积分)}

                        若 $L$ 的参数方程为
                        \[
                              \begin{cases}
                                    x = x(t), \\
                                    y = y(t),
                              \end{cases}\quad t\in[\alpha,\beta],
                        \]
                        则
                        \[
                              \int_{L} P\,dx + Q\,dy
                              = \int_{\alpha}^{\beta}\!\big[P(x(t),y(t))x'(t) + Q(x(t),y(t))y'(t)\big]\,dt.
                        \]
                        起点终点由参数 $\alpha,\beta$ 对应,顺序须与曲线方向一致。

                  \item \textbf{格林公式(将曲线积分化为二重积分)}

                        设平面区域 $D$ 由分段光滑的正向闭曲线 $L$ 围成,且 $P,Q$ 在 $D$ 上具有连续一阶偏导数,则
                        \[
                              \oint_{L} P\,dx + Q\,dy = \iint_{D}\!\left(\frac{\partial Q}{\partial x}-\frac{\partial P}{\partial y}\right)\!d\sigma.
                        \]
                        \begin{enumerate}
                              \item 若 $L$ 为闭曲线且内部无奇点,可直接使用格林公式。
                              \item 若有奇点但除奇点外 $\dfrac{\partial Q}{\partial x}=\dfrac{\partial P}{\partial y}$,可换路径封闭。
                              \item 若非封闭曲线且 $\dfrac{\partial Q}{\partial x}=\dfrac{\partial P}{\partial y}$,可在区域内换一条起终点相同的简路径计算。
                              \item 若非封闭曲线且两偏导不等,可补线成闭合曲线 $L=L_{AB}+C_{BA}$,应用格林公式后再减去补线部分。
                        \end{enumerate}

                  \item \textbf{积分与路径无关(保守场条件)}

                        若 $P,Q$ 在单连通区域 $D$ 内具有一阶连续偏导,则下列命题等价:
                        \begin{enumerate}[label=(\alph*)]
                              \item $\int_{L_{AB}} P\,dx+Q\,dy$ 与路径无关;
                              \item 对任意闭曲线 $\oint_L P\,dx+Q\,dy=0$;
                              \item 存在函数 $u(x,y)$ 使 $du = P\,dx + Q\,dy$;
                              \item $\mathbf{F}=(P,Q)$ 为某函数的梯度场;
                              \item $\dfrac{\partial P}{\partial y}=\dfrac{\partial Q}{\partial x}$。
                        \end{enumerate}

                        \textbf{求原函数法:}
                        \[
                              u(x,y) = \int_{(x_0,y_0)}^{(x,y)} P\,dx + Q\,dy,
                        \]
                        或沿折线路径计算:
                        \[
                              u(x,y) = \int_{x_0}^{x} P(x,y_0)\,dx + \int_{y_0}^{y} Q(x,y)\,dy.
                        \]
                        若 $\frac{\partial Q}{\partial x}=\frac{\partial P}{\partial y}$ 不成立,则 $u(x,y)$ 不存在。

                        \textbf{凑微分法:}
                        若能写出 $P\,dx+Q\,dy=d[u(x,y)]$,则
                        \[
                              \int_{L_{AB}} P\,dx + Q\,dy = u(B)-u(A).
                        \]

                  \item \textbf{两类曲线积分关系式}
                        \[
                              \int_{\Gamma} P\,dx + Q\,dy + R\,dz
                              = \int_{\Gamma} (P\cos\alpha + Q\cos\beta + R\cos\gamma)\,ds,
                        \]
                        其中 $(\cos\alpha,\cos\beta,\cos\gamma)$ 为 $\Gamma$ 上点处的单位切向量。

                  \item \textbf{空间曲线的两种计算法}
                        \begin{enumerate}
                              \item \textbf{参数法(一投二代三计算):}
                                    \[
                                          \Gamma:\begin{cases}
                                                x=x(t), \\
                                                y=y(t), \\
                                                z=z(t),\quad t\in[a,b],
                                          \end{cases}
                                    \]
                                    \[
                                          \int_{\Gamma} P\,dx+Q\,dy+R\,dz
                                          = \int_a^b [P x'(t)+Q y'(t)+R z'(t)]\,dt.
                                    \]

                              \item \textbf{斯托克斯公式:}
                                    若 $\Gamma=\partial\Sigma$ 为曲面 $\Sigma$ 的正向边界,则
                                    \[
                                          \oint_{\Gamma} P\,dx+Q\,dy+R\,dz
                                          = \iint_{\Sigma}
                                          \begin{vmatrix}
                                                dy\,dz                       & dz\,dx                       & dx\,dy                       \\
                                                \dfrac{\partial}{\partial x} & \dfrac{\partial}{\partial y} & \dfrac{\partial}{\partial z} \\
                                                P                            & Q                            & R
                                          \end{vmatrix}
                                          = \iint_{\Sigma}
                                          \begin{vmatrix}
                                                \cos\alpha                   & \cos\beta                    & \cos\gamma                   \\
                                                \dfrac{\partial}{\partial x} & \dfrac{\partial}{\partial y} & \dfrac{\partial}{\partial z} \\
                                                P                            & Q                            & R
                                          \end{vmatrix} dS.
                                    \]
                                    其中 $(\cos\alpha,\cos\beta,\cos\gamma)$ 为 $\Sigma$ 的单位外法线方向余弦。

                              \item 若 $\operatorname{rot}\mathbf{F}=0$(无旋场),则积分与路径无关,可换路径计算。
                        \end{enumerate}
            \end{enumerate}
\end{enumerate}

\subsection*{小结}

\begin{itemize}
      \item 第二型曲线积分体现\textbf{向量场沿曲线的累积作用}(常为功、环流等);
      \item 计算常用方法:参数化、一投二代三计算、对称性、格林公式;
      \item 若 $\nabla\times\mathbf{F}=0$,则积分与路径无关;
      \item 与第一型曲线积分关系:$\displaystyle\int_\Gamma (P,Q,R)\cdot(dx,dy,dz) = \int_\Gamma (P\cos\alpha+Q\cos\beta+R\cos\gamma)\,ds.$
\end{itemize}

\subsection{第二型曲面积分}

\subsubsection{定义与物理意义(通量)}

设向量场
\[
      \mathbf{F}(x, y, z) = P(x, y, z)\mathbf{i} + Q(x, y, z)\mathbf{j} + R(x, y, z)\mathbf{k},
\]
定义在光滑有向曲面 $\Sigma$ 上。
第二型曲面积分表示向量场 $\mathbf{F}$ 通过曲面 $\Sigma$ 的\textbf{通量}:
\[
      \iint_{\Sigma} P\,\mathrm{d}y\,\mathrm{d}z + Q\,\mathrm{d}z\,\mathrm{d}x + R\,\mathrm{d}x\,\mathrm{d}y
      = \iint_{\Sigma} \mathbf{F} \cdot (\mathrm{d}y\,\mathrm{d}z,\, \mathrm{d}z\,\mathrm{d}x,\, \mathrm{d}x\,\mathrm{d}y).
\]
它反映了向量场穿过曲面的量,与第一型曲面积分(面积量)不同,应注意区分。


\subsubsection{计算方法:一投二代三计算}

若曲面 $\Sigma$ 可写为 $z = z(x, y)$,则:
\[
      \iint_{\Sigma} R(x, y, z)\,\mathrm{d}x\,\mathrm{d}y
      = \pm \iint_{D_{xy}} R[x, y, z(x, y)]\,\mathrm{d}x\,\mathrm{d}y,
\]
其中 $D_{xy}$ 为 $\Sigma$ 在 $xOy$ 平面的投影区域。符号“$\pm$”根据法向量方向确定:
- 当上侧为正($\cos\gamma > 0$)取“+”;
- 当下侧为正($\cos\gamma < 0$)取“-”。

若曲面垂直于投影面,则该积分为零。
若曲面在投影中有重叠部分,应先剖分为互不重叠的曲面片再计算。

\subsubsection{转换投影法}

\paragraph{1. 法向量表达式}
若曲面 $\Sigma: z = z(x, y)$,则其单位法向量为
\[
      \mathbf{n} = \pm \frac{1}{\sqrt{1 + z_x^2 + z_y^2}} \, (-z_x, -z_y, 1),
\]
上侧为正取“+”,下侧为正取“-”。

\paragraph{2. 转换投影公式}
设 $P, Q, R$ 在 $\Sigma$ 上连续,且 $z = z(x, y)$ 有连续一阶偏导,则
\[
      \iint_{\Sigma} P\,\mathrm{d}y\,\mathrm{d}z + Q\,\mathrm{d}z\,\mathrm{d}x + R\,\mathrm{d}x\,\mathrm{d}y
      = \pm \iint_{D_{xy}} \big(-P\,z_x - Q\,z_y + R\big)\,\mathrm{d}x\,\mathrm{d}y.
\]


\subsubsection{“类”对称性}

若 $\Sigma^*$ 关于某平面对称,且对称点处 $R(x, y, z)$ 绝对值相等,则
\[
      \iint_{\Sigma^*} R(x, y, z)\,\mathrm{d}x\,\mathrm{d}y
      = \begin{cases}
            2 \displaystyle\iint_{\Sigma_1} R(x, y, z)\,\mathrm{d}x\,\mathrm{d}y, & R(x, y, z)\mathrm{d}x\,\mathrm{d}y \text{ 同号}, \\[6pt]
            0,                                                                    & R(x, y, z)\mathrm{d}x\,\mathrm{d}y \text{ 异号},
      \end{cases}
\]
其中 $\Sigma_1$ 为 $\Sigma^*$ 的一侧部分。
对 $P\,\mathrm{d}y\,\mathrm{d}z$、$Q\,\mathrm{d}z\,\mathrm{d}x$ 可类比处理。


\subsubsection{高斯公式(散度定理)}

设空间闭区域 $\Omega$ 由光滑闭曲面 $\Sigma$ 围成,且 $\Sigma$ 外侧为正。若 $P, Q, R$ 在 $\Omega$ 上具有连续一阶偏导数,则有
\[
      \iint_{\Sigma} P\,\mathrm{d}y\,\mathrm{d}z + Q\,\mathrm{d}z\,\mathrm{d}x + R\,\mathrm{d}x\,\mathrm{d}y
      = \iiint_{\Omega} \left(
      \frac{\partial P}{\partial x} + \frac{\partial Q}{\partial y} + \frac{\partial R}{\partial z}
      \right)\mathrm{d}v.
\]

\paragraph{常用技巧:}
\begin{enumerate}
      \item \textbf{封闭曲面、无奇点:} 直接用高斯公式;
      \item \textbf{封闭曲面、有奇点,且 $\mathrm{div}\,\mathbf{F}=0$:} 可换为包含奇点的其他封闭曲面;
      \item \textbf{非封闭曲面,且 $\mathrm{div}\,\mathbf{F}=0$:} 可换路径但边界需相同;
      \item \textbf{非封闭曲面,$\mathrm{div}\,\mathbf{F}\neq0$:} 可补面封闭(加面减面法);
      \item \textbf{若 $\mathrm{div}\,\mathbf{F}=0$ 对任意闭曲面成立:} 可建立方程求未知函数 $f(x)$。
\end{enumerate}

\subsubsection{两类曲面积分的关系}

设曲面 $\Sigma$ 有单位法向量 $\mathbf{n} = (\cos\alpha, \cos\beta, \cos\gamma)$,则
\[
      \iint_{\Sigma} P\,\mathrm{d}y\,\mathrm{d}z + Q\,\mathrm{d}z\,\mathrm{d}x + R\,\mathrm{d}x\,\mathrm{d}y
      = \iint_{\Sigma} (P\cos\alpha + Q\cos\beta + R\cos\gamma)\,\mathrm{d}S.
\]
右式即为第一型曲面积分的形式,表示通量的几何意义。


\subsubsection{总结}

\begin{itemize}
      \item \textbf{第一型曲面积分:} 几何量(面积、质量);
      \item \textbf{第二型曲面积分:} 向量场通量;
      \item \textbf{高斯公式:} 封闭曲面 $\leftrightarrow$ 体积分;
      \item \textbf{斯托克斯公式:} 曲线 $\leftrightarrow$ 曲面;
      \item 若 $\mathrm{div}\,\mathbf{F}=0$(无源场),则通量与选取曲面无关。
\end{itemize}

\chapter{行列式}
\XIANchapter{余子式和代数余子式的计算}

\section{计算余子式、代数余子式的线性组合}

% \chapter{习题3}
\clearpage

\begin{question}{}{}
    对于任意给定的 $\varepsilon\in(0,1)$,总存在整数 $N$,当 $n>N$ 时,恒有 $|x_n-a|\le 2\varepsilon$ ,是数列 $\{x_n\}$ 收敛于 $a$ 的( )

    \begin{multicols}{2}  % 并排两列
        \begin{itemize}[label={}]
            \item A 充分条件但非必要条件
            \item B 必要条件但非充分条件
            \item C 充分必要条件
            \item D 既非充分条件又非必要条件
        \end{itemize}
    \end{multicols}
\end{question}
\begin{solution}
    该条件表述为:对任意 $\varepsilon\in(0,1)$,存在 $N$,当 $n>N$ 时 $|x_n-a|\le2\varepsilon$。要判别此条件与收敛 $x_n\to a$ 的等价性:

    若序列收敛于 $a$,则对任意 $\varepsilon>0$ 存在 $N$ 使得当 $n>N$ 有 $|x_n-a|<\varepsilon$,从而对任意 $\varepsilon\in(0,1)$ 同样成立(取同样的 $N$)。反过来,若题中条件成立,则对任意给定的正 $\varepsilon$:若 $\varepsilon\ge1$,不难直接满足;若 $0<\varepsilon<1$,条件已给出 $|x_n-a|\le2\varepsilon$,令 $\varepsilon'=\varepsilon/2\in(0,1)$,由题意存在 $N$ 使得当 $n>N$ 有 $|x_n-a|\le2\varepsilon'=\varepsilon$,故对任意 $\varepsilon>0$ 都能得到相应 $N$,即 $x_n\to a$。因此该条件与收敛等价。

    答案:C(充分必要条件)。
\end{solution}

\begin{question}{}{}
    设 $f^\prime(1)=0,\lim_{x\to1}\frac{f^{\prime}(x)}{\left(x-1\right)^3}=-2$, 则$x=1$为()
    \begin{multicols}{4}  % 并排两列
        \begin{itemize}[label={}]
            \item A 非极值点
            \item B 极大值点
            \item C 极小值点
            \item D 间断点
        \end{itemize}
    \end{multicols}
\end{question}
\begin{solution}
    由极限条件有,当 $x\to1$ 时
    \[
        f'(x)\sim -2(x-1)^3.
    \]
    因此对于 $x>1$(且靠近 1)有 $(x-1)^3>0$,所以 $f'(x)\approx -2(x-1)^3<0$;对于 $x<1$ 有 $(x-1)^3<0$,所以 $f'(x)\approx -2(x-1)^3>0$。即导数在 $x=1$ 处由正变负,函数先增后减,所以在 $x=1$ 处取得局部极大值。

    答案:B(极大值点)。
\end{solution}

\begin{question}{}{}

    设 $\lim _{x\to 0}\frac {\sin 6x+ xf( x) }{x^3}= 0$,则$\lim_{x\to0}\frac{6+f(x)}{x^2}=$
    \begin{multicols}{4}  % 并排两列
        \begin{itemize}[label={}]
            \item A 0
            \item B 6
            \item C 36
            \item D $\infty$
        \end{itemize}
    \end{multicols}
\end{question}
\begin{solution}
    使用 $\sin6x$ 的泰勒展开:
    \[
        \sin6x=6x-\frac{(6x)^3}{6}+o(x^3)=6x-36x^3+o(x^3).
    \]
    将其代入:
    \[
        \frac{\sin6x+xf(x)}{x^3}=\frac{6x-36x^3+xf(x)+o(x^3)}{x^3}
        =\frac{6}{x^2}-36+\frac{f(x)}{x^2}+o(1).
    \]
    极限为 0,故
    \[
        \frac{6+f(x)}{x^2}\to 36.
    \]
    答案:C(36)。
\end{solution}

\begin{question}{}{}

    求极限 $\lim_{x \to +\infty} \frac{\sqrt{5x}-1}{\sqrt{x+2}}$.
\end{question}
\begin{solution}
    按主项比较:
    \[
        \frac{\sqrt{5x}-1}{\sqrt{x+2}}
        = \frac{\sqrt{x}\left(\sqrt5 - \tfrac{1}{\sqrt{x}}\right)}{\sqrt{x}\sqrt{1+2/x}}
        \to \frac{\sqrt5}{1}=\sqrt5.
    \]
    答案:$\sqrt5$。
\end{solution}

\begin{question}{}{}

    求极限 $\lim_{x \to 2} \frac{|x-2|}{x^2-4}$.
\end{question}
\begin{solution}
    化简 ($x\neq2$):
    \[
        \frac{|x-2|}{x^2-4}=\frac{|x-2|}{(x-2)(x+2)}=\frac{1}{|x+2|}.
    \]
    当 $x\to2$,得 $1/4$。

    答案:$\displaystyle \frac14$。
\end{solution}

\begin{question}{}{}

    求极限 $\lim_{x \to 0} \frac{\left(\frac{2+\cos x}{3}\right)^x-1}{x^3}$.
\end{question}
\begin{solution}
    令 $a(x)=\dfrac{2+\cos x}{3}$. 当 $x\to0$,
    \[
        a(x)=1-\frac{x^2}{6}+O(x^4).
    \]
    因此
    \[
        \ln a(x)=-\frac{x^2}{6}+o(x^2).
    \]
    有
    \[
        \left(a(x)\right)^x=e^{x\ln a(x)}=1+x\ln a(x)+o(x\ln a(x))=1-\frac{x^3}{6}+o(x^3).
    \]
    故所求极限为 $-1/6$。
\end{solution}

\begin{question}{}{}

    求极限 $\lim_{x \to 0} \frac{e^{\tan x}-e^{\sin x}}{x^2 \ln(1+x)}$.
\end{question}
\begin{solution}
    先用展开:
    \[
        \tan x = x+\frac{x^3}{3}+o(x^3),\qquad \sin x = x-\frac{x^3}{6}+o(x^3),
    \]
    所以
    \[
        \tan x-\sin x=\frac{x^3}{2}+o(x^3).
    \]
    写
    \[
        e^{\tan x}-e^{\sin x}=e^{\sin x}\big(e^{\tan x-\sin x}-1\big)
        \sim e^{x}\big(\tan x-\sin x\big)\sim 1\cdot\frac{x^3}{2}.
    \]
    分母:
    \[
        x^2\ln(1+x)=x^2\big(x+o(x)\big)=x^3+o(x^3).
    \]
    因此极限为 $(1/2)/1=1/2$。
\end{solution}

\begin{question}{}{}

    求极限 $\lim_{x \to 0} \frac{e^{x^2}-e^{2-2\cos x}}{x^4}$.
\end{question}
\begin{solution}
    先展开:
    \[
        e^{x^2}=1+x^2+\tfrac{x^4}{2}+o(x^4),
    \]
    而
    \[
        2-2\cos x=2-2\big(1-\tfrac{x^2}{2}+\tfrac{x^4}{24}+o(x^4)\big)=x^2-\tfrac{x^4}{12}+o(x^4).
    \]
    于是
    \[
        e^{2-2\cos x}=1+(x^2-\tfrac{x^4}{12})+\tfrac{1}{2}x^4+o(x^4)
        =1+x^2+x^4\Big(-\tfrac{1}{12}+\tfrac{1}{2}\Big)+o(x^4)
        =1+x^2+\tfrac{5}{12}x^4+o(x^4).
    \]
    两者相减得
    \[
        e^{x^2}-e^{2-2\cos x}=\Big(\tfrac{1}{2}-\tfrac{5}{12}\Big)x^4+o(x^4)=\tfrac{1}{12}x^4+o(x^4).
    \]
    所以极限为 $1/12$。
\end{solution}

\begin{question}{}{}

    求极限 $\lim_{x \to +\infty} \left[x - x^2 \ln \left(1 + \frac{1}{x}\right)\right]$.
\end{question}
\begin{solution}
    使用 $\ln(1+u)=u-\tfrac{u^2}{2}+o(u^2)$,取 $u=1/x$:
    \[
        x^2\ln\Big(1+\frac{1}{x}\Big)=x^2\Big(\frac{1}{x}-\frac{1}{2x^2}+o(1/x^2)\Big)=x-\frac{1}{2}+o(1).
    \]
    因此差为约 $\tfrac12$,极限为 $1/2$。
\end{solution}

\begin{question}{}{}

    求极限 $\lim_{x \to 0} \frac{e^{x^2} - \cos x}{x^2}$.
\end{question}
\begin{solution}
    展开:
    \[
        e^{x^2}=1+x^2+\tfrac{x^4}{2}+o(x^4),\qquad \cos x=1-\tfrac{x^2}{2}+\tfrac{x^4}{24}+o(x^4).
    \]
    相减得主项为 $x^2+\tfrac{x^2}{2}=\tfrac{3}{2}x^2$,除以 $x^2$ 得 $3/2$。
\end{solution}

\begin{question}{}{}

    求极限 $\lim_{x \to 0} \frac{[\sin x - \sin (\sin x)] \sin x}{x^4}$.
\end{question}
\begin{solution}
    先展开:
    \[
        \sin x = x-\tfrac{x^3}{6}+o(x^3).
    \]
    令 $s=\sin x=x-\tfrac{x^3}{6}+o(x^3)$,则
    \[
        \sin(\sin x)=s-\tfrac{s^3}{6}+o(s^3)=\Big(x-\tfrac{x^3}{6}\Big)-\tfrac{1}{6}x^3+o(x^3)=x-\tfrac{x^3}{3}+o(x^3).
    \]
    因此
    \[
        \sin x-\sin(\sin x)=\tfrac{x^3}{6}+o(x^3).
    \]
    乘以 $\sin x\sim x$,分子 $\sim \tfrac{x^4}{6}$,故极限为 $1/6$。
\end{solution}

\begin{question}{}{}

    求极限 $\lim_{x \to 0} \left(\frac{2 + e^{\frac{1}{x}}}{1 + e^{\frac{4}{x}}} + \frac{\sin x}{|x|}\right)$.
\end{question}
\begin{solution}
    分左右极限考察。

    当 $x\to0^+$ 时,$1/x\to+\infty$,故 $e^{1/x}\to\infty,\ e^{4/x}\to\infty$,且
    \[
        \frac{2+e^{1/x}}{1+e^{4/x}}=\frac{e^{1/x}\big(1+2e^{-1/x}\big)}{e^{4/x}\big(1+e^{-4/x}\big)}=e^{-3/x}(1+o(1))\to0.
    \]
    同时 $\dfrac{\sin x}{|x|}=\dfrac{\sin x}{x}\to1$。和为 $1$。

    当 $x\to0^-$ 时,$1/x\to-\infty$,所以 $e^{1/x}\to0,\ e^{4/x}\to0$,首项 $\to 2$;而 $\dfrac{\sin x}{|x|}=\dfrac{\sin x}{-x}=-\dfrac{\sin x}{x}\to -1$。和为 $2-1=1$。

    左右极限相等,极限存在且等于 $1$。
\end{solution}

\begin{question}{}{}

    求极限$\lim_{n\to\infty}\left(\frac{1}{\sqrt{n^2+1^2}}+\frac{1}{\sqrt{n^2+2^2}}+\cdots+\frac{1}{\sqrt{n^2+n^2}}\right)$.
\end{question}
\begin{solution}
    写为求和的 Riemann 近似:
    \[
        \sum_{k=1}^n \frac{1}{\sqrt{n^2+k^2}}=\sum_{k=1}^n \frac{1}{n}\cdot\frac{1}{\sqrt{1+(k/n)^2}}.
    \]
    当 $n\to\infty$,该和趋于
    \[
        \int_0^1 \frac{dt}{\sqrt{1+t^2}}.
    \]
    计算不定积分,$\int \frac{dt}{\sqrt{1+t^2}}=\operatorname{arsinh}t=\ln\big(t+\sqrt{1+t^2}\big)$。因此值为
    \[
        \ln\big(1+\sqrt2\big).
    \]
\end{solution}

\begin{question}{}{}

    设$f(x)\in C[a,b]$,证明$\exists\xi\in[a,b]$,使得$\int_{a}^{b}f(x)dx=f(\xi)(b-a)$.
\end{question}
\begin{solution}
    这是积分中值定理(连续函数的平均值定理)。因为 $f$ 在 $[a,b]$ 上连续,令
    \[
        m=\min_{[a,b]} f,\qquad M=\max_{[a,b]} f.
    \]
    则 $m(b-a)\le\int_a^b f(x)\,dx\le M(b-a)$. 由于 $f$ 在 $[a,b]$ 连续,值域为 $[m,M]$,而平均值 $\dfrac{1}{b-a}\int_a^b f(x)\,dx$ 属于 $[m,M]$,由连续性和介值定理存在 $\xi\in[a,b]$ 使得
    \[
        f(\xi)=\frac{1}{b-a}\int_a^b f(x)\,dx,
    \]
    即 $\int_a^b f(x)\,dx=f(\xi)(b-a)$。证毕。
\end{solution}

\begin{question}{}{}

    设$x_{1}=\sqrt{2}$,$x_{n}=\sqrt{2+x_{n-1}}$,$n=2,3,\cdots$. 证明$\{x_{n}\}$极限存在,并求极限.
\end{question}
\begin{solution}
    首先证明单调性与有界性:

    (1) 计算数列初项:$x_1=\sqrt2\approx1.414$,$x_2=\sqrt{2+\sqrt2}\approx1.847\,$。猜测单调递增且有上界 2。

    (2) 若 $x_{n-1}<2$,则 $x_n=\sqrt{2+x_{n-1}}<\sqrt{4}=2$,所以若某一项 $<2$,后项都 $<2$,故有上界 2。且
    \[
        x_{n+1}-x_n=\sqrt{2+x_n}-\sqrt{2+x_{n-1}}=\frac{x_n-x_{n-1}}{\sqrt{2+x_n}+\sqrt{2+x_{n-1}}}.
    \]
    若已知 $x_n>x_{n-1}$,则右端 $>0$,由此可递推证明 $\{x_n\}$ 单调递增(因 $x_2>x_1$ 起)。

    因此序列单调有界,故收敛。设极限为 $L$,取极限于递推式得
    \[
        L=\sqrt{2+L}\quad\Rightarrow\quad L^2=2+L\quad\Rightarrow\quad L^2-L-2=0.
    \]
    解得 $L=\dfrac{1\pm\sqrt{1+8}}{2}=\dfrac{1\pm3}{2}$,因为项均为正且小于 2,取正根 $L=2$。

    结论:极限存在且为 $2$。
\end{solution}
\chapter{矩阵运算}

\chapter{矩阵的秩}
\chapter{线性方程组}
\section{线性方程组理论总结}
\DOne
\begin{enumerate}
    \item 齐次线性方程组$Ax=0$ \DOne
    \item 非齐次线性方程组$Ax=b$ \DOne
\end{enumerate}


\section{线性方程组问题}
\begin{enumerate}
    \item 一般求解问题
    \item 公共解问题
    \item 同解问题
          \DOne+\DTwoTwo
          \begin{detail}{齐次线性方程组\DTwoTwo}{}
          \end{detail}

          \begin{detail}{非齐次线性方程组\DTwoTwo}{}
              设(\RomanSymbols Ⅰ)$A_{m\times n}x=\beta$与(Ⅱ)$B_{s\times n}x=\gamma$均有解,则

              ①(\RomanSymbols Ⅰ)与(Ⅱ)同解

              $\Leftrightarrow$②$A_{m\times n}x=0$与$B_{s\times n}x=0$同解且(\RomanSymbols Ⅰ)与(Ⅱ)有公共解

              $\Leftrightarrow$③$r\left(\begin{bmatrix}A & \beta \\ B & \gamma\end{bmatrix}\right)=r\left(\begin{bmatrix}A \\ B\end{bmatrix}\right)=r(A)=r(B)$

              $\Leftrightarrow$④$[A,\beta]$与$[B,\gamma]$的行向量组等价.
          \end{detail}
\end{enumerate}


\section{线性方程组的几何意义}



\chapter{向量组}

\section{研究具体型向量关系}

\subsection{定义法}
\subsection{求极大线性无关组}
\section{研究抽象型向量关系}
\subsection{定义法}
\subsection{综合问题}
\DOne+\DTwoThree

$$x = \eta^{*} + k_{1} \xi_{1} + \ldots + k_{n-r} \xi_{n-r}.$$
$$
A \xi_{i} = 0.
$$
$$A \eta^{*} = \beta.$$
\section{研究向量组等价}
\section{向量空间}
\subsection{概念}
\subsection{过渡矩阵}
\subsection{坐标变换}
\chapter{特征向量与特征值}
\term{求解利用} $A$的特征值与特征向量

\section{利用特征值命题}
\DOne +\DTwoTwo
\begin{enumerate}
    \item $\lambda_0$是$A$ 的特征值$\Leftrightarrow|\lambda_0E-A|=0$(建立方程求参数或证明行列式 $|\lambda_0E-A|=0$ );
          $\lambda_0$不是$A$ 的特征值$\Leftrightarrow|\lambda_0E-A|\neq0$(矩阵可逆,满秩).
    \item 若$\lambda_1,\lambda_2,\cdots,\lambda_n$是$A$的 $n$个特征值,则

          $$\begin{cases}|A|=\lambda_1\lambda_2\cdots\lambda_n\:,\\\mathrm{tr}\left(A\right)=\lambda_1+\lambda_2+\cdots+\lambda_n\:.\end{cases}$$
    \item \begin{enumerate}
              \item 记住下表
                    \begin{table}[h]
                        \centering
                        \begin{tabular}{|c|c|c|c|c|c|c|}
                            \hline
                            矩阵      & $A$       & $f(A)$       & $A^{-1}$            & $f(A^*)$               & $P^{-1}AP=B$ & $P^{-1}f(A)P=B$ \\
                            \hline
                            特征值     & $\lambda$ & $f(\lambda)$ & $\frac{1}{\lambda}$ & $\frac{|A|}{\lambda} $ & $\lambda$    & $f(\lambda)$    \\
                            \hline
                            对应的特征向量 & $\xi$     & $\xi$        & $\xi$               & $\xi$                  & $P^{-1}\xi$  & $P^{-1}\xi$     \\
                            \hline
                        \end{tabular}
                    \end{table}

                    表中$\lambda$在分母上的,设 $\lambda != 0$
                    \begin{note}{}{}
                        当$\lambda\neq0$ 时,$af(A)\pm bA^{-1}\pm cA^{.}$的特征值为$af\left(\lambda\right)\pm b\frac1\lambda\pm c\frac{|A|}\lambda$,特征向量仍为$\xi$.但
                        $f(A),A^{-1},A^*$ 与$A^T$,$B$ 的线性组合因特征向量不同,无上述规律.
                    \end{note}
              \item $\text{虽然 }A^\mathrm{T}\text{ 的特征值与 }A\text{ 相同,但特征向量不再是 }\xi\text{,要单独计算才能得出 }.$
                    \begin{note}{}{}
                        $ A^\mathrm{T}\text{ 和 }A\text{ 属于不同特征值的特征向量正交 }.$
                    \end{note}
              \item 归零原则.
                    \begin{enumerate}
                        \item 归零准则一: 设$f(x)$为多项式, 若矩阵$A$满足$f(A)=O$, $\lambda$是$A$的任一特征值, 则$\lambda$满足$f(\lambda)=0$.
                        \item 归零准则二:设$n$ 阶方阵 $A$ 的特征多项式为$f(\lambda)=|\lambda E-A|=\lambda^n+a_{n-1}\lambda^{n-1}+\cdots+a_1\lambda+a_0$,则$A$ 的
                              多项式$f(A)$为零矩阵,即$f( A) = A^{n }+ a_{n- 1}A^{n- 1}+ \cdots + a_{1}A+ a_{0}E= O$ .
                    \end{enumerate}
          \end{enumerate}
\end{enumerate}

\section{利用特征向量命题}
\DOne + \DTwoTwo
\begin{enumerate}
    \item $\xi(\neq0)$是$A$的属于$\lambda_{0}$的特征向量$\Leftrightarrow\xi$是$(\lambda_{0}E-A)x=0$的非零解.\DTwoTwo
    \item 重要结论.
          \begin{enumerate}
              \item 单根恰有$1$个线性无关的特征向量.
              \item $k$重特征值$\lambda$至多只有$k$个线性无关的特征向量($k≥2$).
              \item 若$\xi_1,\xi_2$是$A$的属于不同特征值$\lambda_1,\lambda_2$的特征向量,则$\xi_1,\xi_2$线性无关.
              \item 若$\xi_1,\xi_2$是$A$ 的属于同一特征值 $\lambda$ 的特征向量,则当 $k_1k_2\neq0$ 时,非零向量 $k_1\xi_1+k_2\xi_2$仍是 $A$ 的属于特征值$\lambda$的特征向量(常考其中一个系数(如$k_2$)等于0的情形).
              \item 若$\xi_1,\xi_2$是$A$的属于不同特征值$\lambda_1,\lambda_2$的特征向量,则当$k_1\neq0,k_2\neq0$时,$k_1\xi_1+k_2\xi_2$不是$A$的任何特征值的特征向量(常考$k_1=k_2=1$的情形 ).
              \item 若 $\xi$ 是 $A$ 的属于特征值 $\lambda_1$ 的特征向量,$\lambda_1 \neq \lambda_2$,则 $\xi$ 不是 $\lambda_2$ 的特征向量.
              \item 若$A$只有$1$个线性无关的特征向量,即$\sum_{i=1}^{m}[n-r(\lambda_{i}E-A)]=1$,$\lambda_{i}(i=1,2,\cdots,m)$是A的m个不同特征值,则只能有一个$\lambda_{k}(1\leqslant k\leqslant m)$,使$r(\lambda_{k}E-A)=n-1$,而其余$r(\lambda_{i}E-A)=n$,这与$r(\lambda_{i}E-A)<n$矛盾。故A只能有一个$\lambda_{k}$,且此$\lambda_{k}$为n重特征值.
              \item 设$n$阶矩阵$A,B$满足$AB=BA$,且$A$有$n$个互不相同的特征值,则$A$的特征向量都是$B$的特征向量.
              \item 若 $r(A) + r(B) < n$,则 $Ax = 0$,$Bx = 0$ 至少有一个公共非零解 $\xi$.
          \end{enumerate}
\end{enumerate}
\section{利用矩阵方程命题}
\DOne + \DTwoTwo + \DTwoThree

\begin{enumerate}
    \item $AB=O \Rightarrow A[\beta_1, \beta_2, \cdots, \beta_n] = [0, 0, \cdots, 0]$, 即 $A\beta_i = 0\beta_i (i=1, 2, \cdots, n)$, 若 $\beta_i$ 均为非零列向量, 则 $\beta_i$ 为 $A$ 的属于特征值 $\lambda=0$ 的特征向量.
    \item 若任意 $n$ 维列向量 $\xi (\neq 0)$ 均为 $(\lambda E - A)x = 0$ 的解, 则令 $e_1 = \begin{bmatrix} 1 \\ 0 \\ \vdots \\ 0 \end{bmatrix}$, $e_2 = \begin{bmatrix} 0 \\ 1 \\ \vdots \\ 0 \end{bmatrix}$, $\cdots$, $e_n = \begin{bmatrix} 0 \\ \vdots \\ 0 \\ 1 \end{bmatrix}$,且$\boldsymbol{B}=[\boldsymbol{e}_{1},\boldsymbol{e}_{2},\cdots,\boldsymbol{e}_{n}]$,于是$(\lambda\boldsymbol{E}-\boldsymbol{A})\boldsymbol{B}=\boldsymbol{O}$,由于$\boldsymbol{B}$可逆,因此有$\lambda\boldsymbol{E}-\boldsymbol{A}=\boldsymbol{O}$,即$\boldsymbol{A}=\lambda\boldsymbol{E}$.
    \item $AB = C \Rightarrow A[\beta_1, \beta_2, \cdots, \beta_n] = [\gamma_1, \gamma_2, \cdots, \gamma_n] = [\lambda_1\beta_1, \lambda_2\beta_2, \cdots, \lambda_n\beta_n]$, 即 $A\beta_i = \lambda_i\beta_i (i=1, 2, \cdots, n)$, 其中 $\gamma_i = \lambda_i\beta_i$, $\beta_i$ 为非零列向量, 则 $\beta_i$ 为 $A$ 的属于特征值 $\lambda_i$ 的特征向量.
    \item  $AP = PB$, $P$ 可逆 $\Rightarrow P^{-1}AP = B \Rightarrow A \sim B \Rightarrow \lambda_A = \lambda_B$.
    \item $A$ 的每行元素之和均为 $k \Rightarrow A\begin{bmatrix} 1 \\ 1 \\ \vdots \\ 1 \end{bmatrix} = k\begin{bmatrix} 1 \\ 1 \\ \vdots \\ 1 \end{bmatrix}\Rightarrow k$ 是特征值, $\begin{bmatrix} 1 \\ 1 \\ \vdots \\ 1 \end{bmatrix}$ 是 $A$ 的属于特征值 $k$ 的特征向量.
    \item 若 $A$ 可逆, $A$ 的每行元素之和均为 $k$, 则 $A^{-1}$ 的每行元素之和均为 $\frac{1}{k}$.
    \item 若 $A$ 的每行元素之和均为 $k$, 则 $A^n$ 的每行元素之和均为 $k^n$.
\end{enumerate}
\chapter{相似理论}
\section{化归相似对角化的基本局面}
\DOne+\DTwoThree

若 $n$ 阶矩阵 $A$ 有 $n$ 个线性无关的特征向量,则 $A$ 可相似对角化,且有
$$[\xi_1, \xi_2, \cdots, \xi_n]^{-1} A [\xi_1, \xi_2, \cdots, \xi_n] = \begin{bmatrix} \lambda_1 & & \\ & \lambda_2 & \\ & & \ddots \\ & & & \lambda_n \end{bmatrix},$$
牢记这个形式.

\section{用各种条件判$A$能否相似对角化}
\DOne+\DTwoTwo

$\star \star \star$

\begin{enumerate}
    \item 充要条件
          \begin{enumerate}
              \item $A$有$n$个线性无关的特征向量 $\Leftrightarrow A \sim\Lambda$
              \item $n_i = n - r(\lambda_i E - A) \Leftrightarrow A \sim \Lambda$.
          \end{enumerate}
    \item 充分条件
          \begin{enumerate}
              \item $A$是实对称矩阵$\Leftrightarrow A \sim \Lambda$.
              \item $A$有$n$个互异特征值$\Leftrightarrow A \sim \Lambda$.
              \item $A^k=E$($k$为正整数) $\Leftrightarrow A \sim \Lambda$.
              \item $A^2 - (k_1 + k_2)A + k_1 k_2 E = O$ 且 $k_1 \neq k_2 \Rightarrow A \sim \Lambda$.
              \item $r(A) = 1$ 且 $\text{tr}(A) \neq 0 \Rightarrow A \sim \Lambda$.
          \end{enumerate}
    \item 必要条件

          $A \sim \Lambda \Rightarrow r(A) =$ 非零特征值的个数 (重根按重数算).
    \item 否定条件
          \begin{enumerate}
              \item $A \neq O$, $A^k = O$ ($k$ 为大于 1 的整数) $\Rightarrow A$ 不可相似对角化.
              \item $A$ 的特征值全为 $k$, 但 $A \neq kE \Rightarrow A$ 不可相似对角化.
          \end{enumerate}
\end{enumerate}

\section{非对称矩阵$A$与实对称矩阵$A$相似对角化的异同}
\DOne+\DTwoOne

\begin{enumerate}
    \item 非对称矩阵$A$不存在正交矩阵$Q$,使其相似对角化
    \item 实对称矩阵$A$存在正交矩阵$Q$,使其相似对角化
\end{enumerate}
\section{$A$与$B$相似}
\DOne+\DTwoOne+\PFour

$\star \star \star$

\begin{enumerate}
    \item 若$A$相似于$B$,则
          \begin{enumerate}
              \item $|A| = |B|$;
              \item $r(A)=r(B)$;
              \item $tr(A)=tr(B)$;
              \item $\lambda_{A} = \lambda_{B}$ (或 |$\lambda E - A$| = |$\lambda E - B$|);
              \item 属于 $\lambda_{A}$ 的线性无关的特征向量的个数等于属于 $\lambda_{B}$ 的线性无关的特征向量的个数;
              \item $A, B$  的各阶主子式之和分别相等.
          \end{enumerate}

    \item 若$A$相似于$\Lambda$,$B$相似于$\Lambda$,则$A$相似于$B$.
    \item 若$A$相似于$B$,$B$相似于$\Lambda$,则$A$相似于$\Lambda$.
    \item $A$与$B$的相似手段的“三同一不同”.

          若 $P^{-1}AP = B$, 则 $P^{-1}f(A)P = f(B)$, $P^{-1}A^{-1}P = B^{-1}$, $P^{-1}A^{*}P = B^{*}$, 即 $f(A)$ 与 $f(B)$, $A^{-1}$ 与 $B^{-1}$, $A^{*}$ 与 $B^{*}$ 相似的手段相同, 也即 $P^{-1}[af(A) + bA^{-1} + cA^{*}]P = af(B) + bB^{-1} + cB^{*}$. 但 $A^{T}$ 与 $B^{T}$ 相似的手段与上面不同.
\end{enumerate}
\section{相似对角化的应用}
\DOne+\DTwoTwo

\begin{example}{}{}
    已知数列 $\{x_n\}$, $\{y_n\}$, $\{z_n\}$ 满足 $x_0 = -1$, $y_0 = 0$, $z_0 = 2$, 且
    $$\begin{cases}
            x_n = -2x_{n-1} + 2z_{n-1}, \\
            y_n = -2y_{n-1} - 2z_{n-1}, \\
            z_n = -6x_{n-1} - 3y_{n-1} + 3z_{n-1},
        \end{cases}$$
    记 $\alpha_n = \begin{bmatrix} x_n \\ y_n \\ z_n \end{bmatrix}$, 写出满足 $\alpha_n = A\alpha_{n-1}$ 的矩阵 $A$, 并求 $A^n$ 及 $x_n$, $y_n$, $z_n (n=1,2,\cdots)$.
\end{example}
\begin{solution}
    由题设得$\begin{bmatrix}x_n\\y_n\\z_n\end{bmatrix}=\begin{bmatrix}-2&0&2\\0&-2&-2\\-6&-3&3\end{bmatrix}\begin{bmatrix}x_{n-1}\\y_{n-1}\\z_{n-1}\end{bmatrix}$,得矩阵$A=\begin{bmatrix}-2&0&2\\0&-2&-2\\-6&-3&3\end{bmatrix}$满足$\alpha_n=A\alpha_{n-1}$.

    因为

    $|\lambda E-A|=\begin{vmatrix}\lambda+2&0&-2\\0&\lambda+2&2\\6&3&\lambda-3\end{vmatrix}=\lambda(\lambda-1)(\lambda+2)$,

    所以矩阵$A$的特征值为$\lambda_1=0$,$\lambda_2=1$,$\lambda_3=-2$.

    当$\lambda_1=0$时,解方程组$(0E-A)x=0$,得特征向量$\xi_1=\begin{bmatrix}1&-1&1\end{bmatrix}^{T}$;

    当$\lambda_2=1$时,解方程组$(E-A)x=0$,得特征向量$\xi_2=\begin{bmatrix}2&-2&3\end{bmatrix}^{T}$;

    当$\lambda_3=-2$时,解方程组$(-2E-A)x=0$,得特征向量$\xi_3=\begin{bmatrix}-1&2&0\end{bmatrix}^T$.

    令$P=[\xi_1,\xi_2,\xi_3]=\begin{bmatrix}1&2&-1\\-1&-2&2\\1&3&0\end{bmatrix}$,则$P^{-1}AP=\begin{bmatrix}0&0&0\\0&1&0\\0&0&-2\end{bmatrix}$,即$A=P\begin{bmatrix}0&0&0\\0&1&0\\0&0&-2\end{bmatrix}P^{-1}$,从而得$A^n=P\begin{bmatrix}0&0&0\\0&1&0\\0&0&-2\end{bmatrix}^nP^{-1}=\begin{bmatrix}1&2&-1\\-1&-2&2\\1&3&0\end{bmatrix}\begin{bmatrix}0&0&0\\0&1&0\\0&0&(-2)^n\end{bmatrix}\begin{bmatrix}6&3&-2\\-2&-1&1\\1&1&0\end{bmatrix}$

    $$=\begin{bmatrix}-4-(-2)^n&-2-(-2)^n&2\\4-(-2)^{n+1}&2-(-2)^{n+1}&-2\\-6&-3&3\end{bmatrix}$$.

    由递推式$\alpha_n=A\alpha_{n-1}$知$\alpha_n=A^n\alpha_0$,其中$\alpha_0=\begin{bmatrix}-1&0&2\end{bmatrix}^T$,所以

    $\alpha_n=A^n\alpha_0=\begin{bmatrix}-4-(-2)^n&-2-(-2)^n&2\\4-(-2)^{n+1}&2-(-2)^{n+1}&-2\\-6&-3&3\end{bmatrix}\begin{bmatrix}-1\\0\\2\end{bmatrix}=\begin{bmatrix}8+(-2)^n\\-8+(-2)^{n+1}\\12\end{bmatrix}$,

    故$x_n=8+(-2)^n$,$y_n=-8+(-2)^{n+1}$,$z_n=12(n=1,2,\cdots)$.
\end{solution}

\section{正交矩阵及其使用}
\DOne + \DTwoOne

$\star \star \star$

\begin{enumerate}
    \item 若$A$为正交矩阵,则
          $$A^\top A = E \Leftrightarrow A^{-1} = A^\top$$
          $$\Leftrightarrow A \text{ 由规范正交基组成 }$$
          $$\Leftrightarrow A^{\mathrm{T}}\text{是正交矩阵}$$
          $$\Leftrightarrow A^{-1}\text{是正交矩阵}$$
          $$\Leftrightarrow A^{*}\text{是正交矩阵}$$
          $$\Leftrightarrow-A\text{是正交矩阵.}$$
    \item 若$A,B$为同阶正交矩阵,则$AB$为正交矩阵,但$A+B$不一定为正交矩阵.
    \item ${\text{若 }A\text{ 为正交矩阵,则其实特征值的取值范围为}\{-1,1\}}.$
    \item 设$A$为$n$阶非零矩阵,$\begin{cases}\text{若}a_{ij}=A_{ij},\text{则}A^{\mathrm{T}}=A^{*},AA^{\mathrm{T}}=E,\text{且}\mid A\mid=1;\\\text{若}a_{ij}=-A_{ij},\text{则}A^{\mathrm{T}}=-A^*,AA^{\mathrm{T}}=E,\text{且}\mid A\mid=-1.\end{cases}$
\end{enumerate}
\chapter{二次型}
\section{$f=x^TAx$ 中$A$的表示}
\DOne+\DTwoThree

\begin{enumerate}
    \item 给出非对称矩阵 $B$,令$A=\frac{B+B^T}{2}$,则$A=A^\mathrm{T}.$
    \item 通过题设或基本变形显化出 $A.$
\end{enumerate}

\section{配方法与正交变换法的异同}

\begin{enumerate}
    \item 命题语言
          \DTwoTwo
          \begin{enumerate}
              \item 配方法

                    二次型语言:将 $f = x^T A x$ 通过配方法化为标准形,并求出可逆变换矩阵 $C$.

                    矩阵语言:求可逆矩阵 $C$,使得 $C^T A C = \Lambda$.
              \item 正交变换法

                    二次型语言:将 $f = x^T A x$ 通过正交变换法化为标准形,并求出正交矩阵 $Q$.

                    矩阵语言:求正交矩阵 $Q$,使得 $Q^T A Q = \Lambda$.
          \end{enumerate}
    \item 过程与结果的异同
          \DTwoThree

          设$f(x)=x^TAx$.
          \begin{enumerate}
              \item 配方法(可逆线性变换)

                    $x=Cy$,$C$可逆.使得$f\xlongequal{x=\boldsymbol{C}y}y^T\Lambda y$,其中$C^TAC=\Lambda$(使$A$合同于对角矩阵).
              \item 正交变换法(可逆线性变换):

                    $x=Qy$(这里的$Q$不仅可逆,还满足$Q^{-1}=Q^{T}$),使得$f\xlongequal{x=\boldsymbol{Q}y}y^T\Lambda y$,其中$Q^{T}AQ=Q^{-1}AQ=A$.
          \end{enumerate}
          二者区别:在配方法中,$c$只满足可逆,所以$c^{-1}$不一定等于$c^T$,但是在正交变换法中,变换手段$Q$满足$Q^{- 1}= Q^T$ .

          二者相同点:它们的正、负惯性指数是对应相等的.
    \item 惯性指数
          \begin{example}{}{}
              $f(x_{1},x_{2},x_{3})=-2x_{1}x_{2}-2x_{1}x_{3}+6x_{2}x_{3}$的正惯性指数为(  ).
          \end{example}
          \begin{solution}
              令$\begin{cases}x_{1}=y_{1}+y_{2},\\x_{2}=y_{1}-y_{2},\end{cases}$则

              $$f=-2y_{1}^{2}+2y_{2}^{2}+4y_{1}y_{3}-8y_{2}y_{3}
                  =-2(y_{1}-y_{3})^{2}+2(y_{2}-2y_{3})^{2}-6y_{3}^{2},$$

              再令$\begin{cases}z_{1}=y_{1}-y_{3},\\z_{2}=y_{2}-2y_{3},\end{cases}$则

              $$f=-2z_{1}^{2}+2z_{2}^{2}-6z_{3}^{2},$$

              故$f$的正惯性指数为1.
          \end{solution}
\end{enumerate}

\section{伪配方法}
\DTwoThree

“平方和式$A^2+B^2+C^2$”未必就是(拉格朗日)配方法得来的结果,故若非拉格朗日配方法,则称伪配方法.要注意伪配方法的变换矩阵是否有可逆性.
\begin{enumerate}
    \item 如果变换没有可逆性,则有可能改变表达式的几何性质,如封闭性,此时不能得出平方和式正定;
    \item 如果变换是可逆的,则平方和式正定.
\end{enumerate}


\begin{note}{}{}
    对于$f(x_{1},x_{2},x_{3})=(a_{1}x_{1}+a_{2}x_{2}+a_{3}x_{3})^{2}+(b_{1}x_{1}+b_{2}x_{2}+b_{3}x_{3})^{2}+(c_{1}x_{1}+c_{2}x_{2}+c_{3}x_{3})^{2}$的情形,可总结如下做题方法:

    令$f=0$,即$\begin{cases}a_{1}x_{1}+a_{2}x_{2}+a_{3}x_{3}=0,\\b_{1}x_{1}+b_{2}x_{2}+b_{3}x_{3}=0,\\c_{1}x_{1}+c_{2}x_{2}+c_{3}x_{3}=0,\end{cases}$计算$|\boldsymbol{A}|=\begin{vmatrix}a_{1}&a_{2}&a_{3}\\b_{1}&b_{2}&b_{3}\\c_{1}&c_{2}&c_{3}\end{vmatrix}$,若$|\boldsymbol{A}|\neq0$,则$f$正定;若$|\boldsymbol{A}|=0$,则$f$不正定.
\end{note}
\section{正交变换法的传递性}
\DOne+\DTwoThree

若$A$相似于$B$,则$B$相似于$C$,则$A$相似于$C$.这里$B$常为$\Lambda$.

\section{合同的判定与手段}
\DOne+\DTwoThree

\begin{enumerate}
    \item 同阶实对称矩阵$A,B$合同的判定

          用正、负惯性指数:$A,B$合同$\Leftrightarrow p_A=p_B,q_A=q_B$(相同的正、负惯性指数).
    \item 已知$A$,$\Lambda$($\Lambda$是对角矩阵),求可逆矩阵$C$,使得$C^TAC=\Lambda$
    \item 已知$A$,$B$($B$不是对角矩阵),求可逆矩阵$C$,使得$C^TAC=B$
\end{enumerate}
\begin{idea}{求可逆矩阵$C$,使得$C^TAC=\Lambda$}{}
    \begin{enumerate}
        \item 配方 盯着$\Lambda$的对角线元素,提出对应系数
        \item 换元
        \item 求逆
    \end{enumerate}
\end{idea}
\begin{idea}{求可逆矩阵$C$,使得$C^TAC=B$}{}
    \begin{enumerate}
        \item 对$f$配方、换元,写$D_1$
        \item 对$g$配方、换元,写$D_2$
        \item 令$D_1x=D_2y$,求$D_2^{-1}D_1$
    \end{enumerate}
\end{idea}
\section{合同与相似的异同}
\DOne+\DTwoThree

对于实对称矩阵$A$与$B$,相似必合同,反之不成立.
\begin{example}{合同与相似的异同}{合同与相似的异同}
    已知二次型
    $$f(x_{1}, x_{2}, x_{3}) = x_{1}^{2} + 2x_{2}^{2} + 2x_{3}^{2} + 2x_{1}x_{2} - 2x_{1}x_{3}$$
    $$
        g(y_{1}, y_{2}, y_{3}) = y_{1}^{2} + y_{2}^{2} + y_{3}^{2} + 2y_{2}y_{3}$$
    \begin{enumerate}
        \item 求可逆变换 $x = Py$,将 $f(x_{1}, x_{2}, x_{3})$ 化成 $g(y_{1}, y_{2}, y_{3})$.
        \item 是否存在正交变换 $x = Qy$,将 $f(x_{1}, x_{2}, x_{3})$ 化成 $g(y_{1}, y_{2}, y_{3})$?
    \end{enumerate}
\end{example}
\begin{idea}{解题思路 \ref{ex:合同与相似的异同}}{}
    \begin{enumerate}
        \item 求可逆变换用配方法
        \item 判断是否存在正交变换,如果存在必相似,使用相似的充分条件和充要条件
    \end{enumerate}
\end{idea}
\section{正定的判定与应用}
\DOne+\DTwoThree

$\star\star\star$

\begin{enumerate}
    \item 前提
          $A=A^T$($A$是实对称矩阵)
    \item 二次型$f=x^TAx$正定的充要条件 \DTwo

          $n$元二次型$f=x^{T}Ax$正定

          $\Leftrightarrow$对任意的$x\neq 0$,有$x^{T}Ax>0$(定义)

          $\Leftrightarrow A$的特征值$\lambda_{i}>0(i=1,2,\cdots,n)$

          $\Leftrightarrow f$的正惯性指数$p=n$

          $\Leftrightarrow$存在可逆矩阵$D$,使得$A=D^{T}D$

          $\Leftrightarrow A$与$E$合同

          $\Leftrightarrow A$的各阶顺序主子式均大于0.
    \item 二次型$f=x^TAx$正定的必要条件
          \begin{enumerate}
              \item $a_{ii}>0\left(i=1,2,\cdots,n\right).$
              \item $| A| > 0$.
          \end{enumerate}
    \item 重要结论
          \begin{enumerate}
              \item 若$A$正定,则$A^-1,A^{*},A^{m}(m$为正整数$),kA(k>0),C^{\mathrm{T}}AC(C$可逆 )均正定 .

              \item 若$A,B$正定,则$A+B$正定,$\begin{bmatrix}A&O\\O&B\end{bmatrix}$正定.

              \item ${\text{若}A,B}$正定,则$AB$正定的充要条件是$AB= BA$ .
          \end{enumerate}

\end{enumerate}

\section{二次型的最值}
\DOne+\DTwoTwo+\DTwoThree



\GAIchapter{随机事件和概率}
\GAIchapter{一维随机变量及其分布}

\section{判分布}
\DOne + \DTwoTwo
\begin{enumerate}
      \item 判分布函数
            \begin{enumerate}
                  \item 充要条件

                        $F(x)$ 是分布函数 $\Leftrightarrow F(x)$ 是 $x$ 的单调不减且右连续的函数,且 $F(-\infty)=0$,$F(+\infty)=1$.
                  \item 分布函数形式大观.
                        \begin{enumerate}
                              \item 设 $F_i(x)$ 是分布函数,$\lambda_i>0$,$\sum_{i=1}^{n}\lambda_i=1$,则 $\sum_{i=1}^{n}\lambda_iF_i(x)$ 是分布函数.特别地,算术平均值 $\frac{F_1(x)+F_2(x)}{2}$ 是分布函数.
                              \item 设 $F(x)$ 是分布函数,则 $F(x)$ 在 $[x,x+T]$ $(T>0)$ 上的均值 $\frac{1}{T}\int_{x}^{x+T}F(t)dt$ 是分布函数.可见,线性组合 $\sum_{i=1}^{n}\lambda_iF_i(x)$ 及其连续形式均仍是分布函数.
                              \item 几何平均值 $\sqrt{F_1(x)F_2(x)}$ 是分布函数.
                              \item $[F(x)]^n$,$1-[1-F(x)]^n$ 是分布函数.
                        \end{enumerate}
            \end{enumerate}
      \item 判分布律的充要条件

            $\{p_i\}$ 是概率分布 $\Leftrightarrow p_i\geqslant 0$,且 $\sum_{i}p_i=1$.
      \item 判概率密度
            \begin{enumerate}
                  \item 充要条件

                        $f(x)$ 是概率密度 $\Leftrightarrow f(x) \geqslant 0$,且 $\int_{-\infty}^{+\infty} f(x) \mathrm{d}x = 1$.
                  \item 概率密度形式大观
                        \begin{enumerate}
                              \item 设 $f(x)$ 为概率密度,$\lambda_i > 0$,$\sum_{i=1}^{n} \lambda_i = 1$,则 $\sum_{i=1}^{n} \lambda_i f_i(x)$ 是概率密度.特别地,$\frac{1}{2}[f_1(x) + f_2(x)]$ 是概率密度.
                              \item 设 $f(x)$ 为概率密度,则 $f(x)$ 在 $[x, x+T]$($T > 0$)上的均值 $\frac{1}{T} \int_{x}^{x+T} f(t) \mathrm{d}t$ 是概率密度.
                              \item 设 $X_i$ 的分布函数为 $F_i(x)$,概率密度为 $f_i(x)$,则 $\frac{2}{n}\sum_{i=1}^{n}F_i(x)f_i(x)$ 是概率密度.
                              \item 设 $X_i$ 的分布函数为 $F_i(x)$,概率密度为 $f_i(x)$,则 $f_1(x)F_2(x)\cdots F_n(x) + F_1(x)f_2(x)\cdots F_n(x) + \cdots + F_1(x)F_2(x)\cdots f_n(x)$ 是概率密度.
                              \item 设 $F(x)$ 是分布函数,$f(x)$ 是对应的概率密度,则 $n[F(x)]^{n-1}f(x)$,$n[1-F(x)]^{n-1}f(x)$ 是概率密度.
                        \end{enumerate}
            \end{enumerate}
      \item 反问题
            $$\begin{cases}F(-\infty)=0,\\F(+\infty)=1,\\\sum_{i}p_{i}=1,\\\int_{-\infty}^{+\infty}f(x)\mathrm{d}x=1 \end{cases}$$
            建方程,求参数
\end{enumerate}
\section{用分布}
\DOne

\begin{enumerate}
      \item 离散型分布
            \begin{enumerate}
                  \item 0-1分布.

                        $X \sim B(1,p)$, $X$ (伯努利计数变量) $\sim \begin{pmatrix} 1 & 0 \\ p & 1-p \end{pmatrix}$.
                  \item 二项分布.

                        $X \sim B(n,p) \begin{cases} n\text{次试验相互独立;} \\ P(A) = p; \\ \text{只有}A, \overline{A}\text{两种结果}. \end{cases}$
                        记 $X$ 为 $A$ 发生的次数, 则$P\{X = k\} = C_n^k p^k (1-p)^{n-k}$, $k = 0, 1, 2, \cdots, n$,$EX = np$, $DX = np(1-p)$.

                        二项分布 $X \sim B(n,p)$ 还具有如下性质:
                        \begin{enumerate}
                              \item $Y = n - X$ 服从二项分布 $B(n,q)$, 其中 $q = 1 - p$.
                              \item 对固定的 $n$ 和 $p$,随着 $k$ 的增大,$P\{X=k\}$ 先上升到最大值而后下降
                                    \begin{enumerate}
                                          \item 当 $(n+1)p$ 为整数时,$P\{X=(n+1)p\} = P\{X=(n+1)p-1\}$ 最大.
                                          \item 当 $(n+1)p$ 不为整数时,$P\{X=\left[(n+1)p\right]\}$ 最大,其中 $\left[(n+1)p\right]$ 表示 $(n+1)p$ 的整数部分.
                                    \end{enumerate}
                        \end{enumerate}
                  \item 离散型首次冲击分布(几何分布).

                        在伯努利试验序列中 $P(A)=p,P(\bar{A})=1-p$, 首次出现 $A$ 即停止(即首次冲击即失效).令$X$为试
                        验次数,则$P\{X=k\}=p(1-p)^{k-1},k=1,2,\cdots$,其中$P\left\{X=1\right\}$最大,且$EX=\frac1p,DX=\frac{1-p}{p^{2}}.$
                  \item 超几何分布
                        $N$件产品中有$M$件正品, 从中无放回地随机抽取$n$件, 则取到$k$个正品的概率为

                        $P\{X=k\}=\frac{\mathrm{C}_{M}^{k} \mathrm{C}_{N-M}^{n-k}}{\mathrm{C}_{N}^{n}}$, $k$为整数, $\max \{0, n-N+M\} \leqslant k \leqslant \min \{n, M\}$, 且 $E X=\frac{n M}{N}$.
                  \item 泊松分布

                        泊松分布是指某单位时间段, 某场合下, 源源不断的随机质点流的个数, 也常用于描述稀有事件的概率.

                        $$P\{X=k\}=\frac{\lambda^{k}}{k !} \mathrm{e}^{-\lambda}(k=0,1, \cdots ; \lambda>0),$$

                        $\lambda$ 表示强度 $(E X=\lambda)$, 且 $P\{X=[\lambda]\}$ 最大, 其中 $[\lambda]$ 表示对 $\lambda$ 取整.
            \end{enumerate}

      \item 连续型分布
            \begin{enumerate}
                  \item 均匀分布$U(a,b)$.
                        如果随机变量X的概率密度和分布函数分别为
                        $$f(x)=\begin{cases}
                                    \frac{1}{b-a}, & a<x<b,     \\
                                    0,             & \text{其他},
                              \end{cases}$$
                        $$F(x)=\begin{cases}
                                    0,               & x<a,            \\
                                    \frac{x-a}{b-a}, & a\leqslant x<b, \\
                                    1,               & x\geqslant b,
                              \end{cases}$$
                        则称X在区间(a,b)上服从均匀分布,记为$X\sim U(a,b)$.
                  \item 连续型首次冲击分布(指数分布).

                        设随机质点流的计数过程为 $\{N_{t}\}(t\geqslant 0)$,$N_{t}$ 服从参数为 $\lambda t$ 的泊松分布.令 $T_{1}$ 表示第 1 个质点到来的时刻,则当 $t>0$ 时,令 $A=\{T_{1}>t\}$ 表示第 1 个质点在时刻 $t$ 之后到来,$B=\{N_{t}=0\}$ 表示在 $[0,t]$ 时间上有 0 个质点到来,即 $A$ 与 $B$ 是相等事件,故 $P(A)=P(B)$,即
                        $$P\{T_{1}>t\}=P\{N_{t}=0\}=\frac{(\lambda t)^{0}}{0!}e^{-\lambda t}=e^{-\lambda t},$$
                        于是
                        $$
                              F_{T_{1}}(t)=P\{T_{1}\leqslant t\}=1-e^{-\lambda t}, t>0,$$
                        即 $T_{1}$ 服从参数为 $\lambda$ 的指数分布。
                        如果 $X$ 的概率密度和分布函数分别为
                        $$f(x)=\begin{cases}
                                    \lambda e^{-\lambda x}, & x\geqslant 0, (\lambda>0), \\
                                    0,                      & \text{其他}
                              \end{cases}, F(x)=\begin{cases}
                                    1-e^{-\lambda x}, & x\geqslant 0, (\lambda>0), \\
                                    0,                & x<0
                              \end{cases},$$
                        则称 $X$ 服从参数为 $\lambda$ 的指数分布,记为 $X\sim E(\lambda)$.
                        \begin{note}{}{}
                              \begin{enumerate}
                                    \item 当$t,s > 0$时, $P\{X\geq t+s|X\geq t\}=P\{X\geq s\}$称为指数分布的无记忆性.
                                    \item $EX=\frac{1}{\lambda}$称为平均寿命, 也称为平均等待时间, $\lambda$称为失效频率, 它是一个常数, 只有失效频率不变, 元件无损耗, 才有无记忆性.
                                    \item 特别地, 当λ=$\frac{1}{2}$, 即X~f(x)=$\begin{cases}\frac{1}{2}e^{-\frac{x}{2}}, & x≥0\\0, & x<0\end{cases}$时, 也称$X$服从自由度为$2$的$\chi^2$ 分布, 故$E(\frac{1}{2})$与$\chi^2(2)$ 是同一分布.
                                    \item 若$X\sim E(1)$, 则$2X~E(\frac{1}{2})$, $2X\sim \chi^2(2)$ .
                                    \item 若$X\sim E(\lambda)$, 则$2\lambda X~E(\frac{1}{2})$, $2\lambda X \sim\chi^2(2)$ .
                              \end{enumerate}
                        \end{note}
                  \item 自由度为$1$的$t$分布(标准柯西分布).

                        若
                        $$X\sim f(x)=\frac{1}{\pi(1+x^{2})}\:,\:-\infty<x<+\infty\:,$$
                        则称$X$服从自由度为1的$t$分布(标准柯西分布),即$X\sim t(1)$这是用于描述受迫共振的一种分布.
                  \item 正态分布

                        若$X\sim f(x)=\frac{1}{\sqrt{2\pi}\sigma}\mathrm{e}^{-\frac{(x-\mu)^2}{2\sigma^2}}$,$-\infty<x<+\infty$,其中$-\infty<\mu<+\infty$,$\sigma>0$,则称$X$服从参数为$(\mu,\sigma^2)$的正态分布,记为$X\sim N(\mu,\sigma^2)$.
                        \begin{note}{}{}
                              \begin{enumerate}
                                    \item  $\mu=0$,$\sigma=1$时的正态分布为标准正态分布,记为$X\sim N(0,1)$.
                                          $$X\sim\varphi(x)=\frac{1}{\sqrt{2\pi}}\mathrm{e}^{-\frac{x^2}{2}}, \quad \Phi(x)=\int_{-\infty}^x\frac{1}{\sqrt{2\pi}}\mathrm{e}^{-\frac{t^2}{2}}\mathrm{d}t,$$
                                          且
                                          $$Y=\mid X\mid\sim f_{\gamma}(y)=\begin{cases}\dfrac{2}{\sqrt{2\pi}}\mathrm{e}^{-\frac{y^{2}}{2}},&y>0,\\0,&y\leqslant0\end{cases}=\begin{cases}2\varphi(y),&y>0,\\0,&y\leqslant0.\end{cases}$$
                                    \item 计算公式与重要数据.

                                          若$X\sim N(0,1)$,则有
                                          $$
                                                \Phi(-x)=1-\Phi(x); \Phi(0)=\frac{1}{2};$$
                                          $$P\{|X|\leqslant a\}=2\Phi(a)-1(a>0).$$

                                    \item 标准化.

                                          若$X\sim N(\mu,\sigma^2)$,则
                                          $$
                                                \frac{X-\mu}{\sigma}\sim N(0,1),$$
                                          且有
                                          $$F(x)=P\{X\leqslant x\}=\Phi\left(\frac{x-\mu}{\sigma}\right),$$
                                          $$
                                                P\{a\leqslant X\leqslant b\}=\Phi\left(\frac{b-\mu}{\sigma}\right)-\Phi\left(\frac{a-\mu}{\sigma}\right),$$
                                          $$P\{\mu-k\sigma\leqslant X\leqslant\mu+k\sigma\}=2\Phi(k)-1(k>0).$$
                                    \item 含参数的概率密度的结构.

                                          设函数$f(x)=k\mathrm{e}^{-(ax^2+bx+c)}$, $x\in(-\infty,+\infty)(a>0)$,则
                                          $$
                                                ax^2+bx+c=a\left[\left(x+\frac{b}{2a}\right)^2+\frac{4ac-b^2}{4a^2}\right],$$
                                          且$k=\sqrt{\frac{a}{\pi}}\mathrm{e}^{\frac{4ac-b^2}{4a}}$,如$f(x)=k\mathrm{e}^{-\left(\frac{x^2}{4}+\frac{x}{2}+\frac{1}{4}\right)}$,则
                                          $$\frac{x^2}{4}+\frac{x}{2}+\frac{1}{4}=\frac{1}{4}\left[\left(x+\frac{1}{2}\right)^2+\frac{4\cdot\frac{1}{4}\cdot\frac{1}{4}-\left(\frac{1}{2}\right)^2}{4\cdot\left(\frac{1}{4}\right)^2}\right]$$
                                          $$
                                                =\frac{1}{4}(x+1)^2,$$
                                          且$k=\sqrt{\frac{1}{4\pi}}\mathrm{e}^0=\frac{1}{2\sqrt{\pi}}$.
                              \end{enumerate}
                        \end{note}
            \end{enumerate}
      \item 利用分布求概率及反问题
            \begin{enumerate}
                  \item  $X \sim F(x)$,则
                        \begin{enumerate}
                              \item  $P\{X \leqslant a\} = F(a)$;
                              \item  $P\{X < a\} = F(a-0)$;
                              \item  $P\{X = a\} = P\{X \leqslant a\} - P\{X < a\} = F(a) - F(a-0)$;
                              \item  $P\{a < X < b\} = P\{X < b\} - P\{X \leqslant a\} = F(b-0) - F(a)$;
                              \item  $P\{a \leqslant X \leqslant b\} = P\{X \leqslant b\} - P\{X < a\} = F(b) - F(a-0)$.
                        \end{enumerate}
                  \item $X \sim p_{i}$,则
                        $$P\{X \in I\} = \sum_{x_{i} \in I} P\{X = x_{i}\}$$
                  \item $X \sim f(x)$,则
                        $$P\{X \in I\} = \int_{I} f(x) \, dx$$
                  \item 反问题:已知概率反求参数.
            \end{enumerate}
\end{enumerate}

\section{求分布}
\DOne

根据题设条件,建立$F(x) = P\{X\leq x\}$并计算此概率.
\GAIchapter{一维随机变量函数的分布}

\section{离散型$\rightarrow$离散型}
设离散型随机变量 $X$ 的分布为 $P\{X = x_i\} = p_i \ (i=1,2,\cdots)$,若 $Y = g(X)$,则 $Y$ 仍为离散型随机变量,其分布为
$$
      Y \sim
      \begin{pmatrix}
            g(x_1) & g(x_2) & \cdots \\
            p_1    & p_2    & \cdots
      \end{pmatrix}.
$$
若若干个 $g(x_k)$ 取相同值,则合并为一项,并将对应概率相加。

\section{连续型$\rightarrow$连续型(或混合型)}
设连续型随机变量 $X$ 的分布函数与密度分别为 $F_X(x)$、$f_X(x)$,若 $Y = g(X)$,则可用以下两种方法求其分布:

\subsection*{(1) 分布函数法}
由定义直接求:
$$
      F_Y(y) = P\{Y \le y\} = P\{g(X) \le y\} = \int_{g(x) \le y} f_X(x) \, dx.
$$
若 $F_Y(y)$ 连续且可导,则 $f_Y(y) = F_Y'(y)$。

\subsection*{(2) 公式法(单调可导变换)}
若 $y = g(x)$ 在 $(a,b)$ 上严格单调且可导,则存在反函数 $x = h(y)$,其概率密度为
$$
      f_Y(y) = f_X[h(y)] \cdot |h'(y)|, \quad \alpha < y < \beta,
$$
其中
$$
      \alpha = \min\{\lim_{x\to a^+} g(x), \lim_{x\to b^-} g(x)\}, \quad
      \beta = \max\{\lim_{x\to a^+} g(x), \lim_{x\to b^-} g(x)\}.
$$

\section{连续型$\rightarrow$离散型}
若 $X \sim f_X(x)$ 且 $Y = g(X)$ 为离散型变量,先求出 $Y$ 的可能取值 $a_i$,再由
$$
      P\{Y = a_i\} = \int_{g(x)=a_i} f_X(x) \, dx
$$
得出其分布。
\section{两种重要的随机变量变换}

\subsection*{(1) 变换于 $U(0,1)$}
\begin{example}{}{}
      设随机变量 $X$ 的分布函数 $F_X(x)$ 严格单调递增,反函数 $F_X^{-1}(y)$ 存在,令 $Y = F_X(X)$,则 $Y \sim U(0,1)$。
\end{example}
\begin{proof}
      由定义:
      $$
            F_Y(y) = P\{Y \le y\} = P\{F_X(X) \le y\} = P\{X \le F_X^{-1}(y)\} = F_X[F_X^{-1}(y)] = y,
      $$
      对 $0 \le y < 1$ 成立;此外,$F_Y(y) = 0 (y < 0)$,$F_Y(y) = 1 (y \ge 1)$,即
      $$
            F_Y(y) =
            \begin{cases}
                  0, & y < 0,       \\
                  y, & 0 \le y < 1, \\
                  1, & y \ge 1,
            \end{cases}
      $$
      故 $Y \sim U(0,1)$。
\end{proof}


\subsection*{(2) 变换于 $E(1)$}
\begin{example}{}{}
      设 $X$ 的分布函数 $F_X(x)$ 连续且在其密度区间上严格单调,令
      $$
            Y = -\ln[1 - F_X(X)],
      $$
      则 $Y \sim E(1)$。
\end{example}
\begin{proof}
      由定义:
      $$
            P\{Y \le y\} = P\{-\ln[1 - F_X(X)] \le y\} = P\{F_X(X) \le 1 - e^{-y}\}.
      $$
      由于 $F_X(X) \sim U(0,1)$,故
      $$
            F_Y(y) = 1 - e^{-y}, \quad y > 0,
      $$
      即 $Y \sim E(1)$。
\end{proof}
\chapter{多维随机变量及其分布}

\section{离散型问题}
一般不会考

\section{连续型问题}
\begin{enumerate}
    \item 二维均匀分布

          如果$(X,Y)$的概率密度为
          $$f(x,y)=\begin{cases}\dfrac{1}{S_D},&(x,y)\in D,\\0,&\text{其他,}\end{cases}$$
          其中$S_{_{D}}$为区域$D$ 的面积,则称$(X,Y)$在平面有界区域$D$ 上服从均匀分布.

    \item 二维正态分布

          如果(X,Y)的概率密度为
          $$f(x,y)=\frac{1}{2\pi\sigma_1\sigma_2\sqrt{1-\rho^2}}\exp\left\{-\frac{1}{2(1-\rho^2)}\left[\left(\frac{x-\mu_1}{\sigma_1}\right)^2-2\rho\left(\frac{x-\mu_1}{\sigma_1}\right)\left(\frac{y-\mu_2}{\sigma_2}\right)+\left(\frac{y-\mu_2}{\sigma_2}\right)^2\right]\right\},$$

          其中$\mu_1\in R$,$\mu_2\in R$,$\sigma_1>0$,$\sigma_2>0$,$-1<\rho<1$,则称(X,Y)服从参数为$\mu_1$,$\mu_2$,$\sigma_1^2$,$\sigma_2^2$,$\rho$的二维正态分布,记为$(X,Y)\sim N(\mu_1,\mu_2;\sigma_1^2,\sigma_2^2;\rho)$。

\end{enumerate}
\section{求边缘分布、条件分布与独立性问题}
\begin{enumerate}
    \item 边缘分布
          \begin{enumerate}
              \item 求$F_X(x),F_Y(y)$.
                    $$F_{X}(x)=F(x,+\infty) ,\quad F_{Y}(y)=F(+\infty,y).$$
              \item 求$p_{i\cdot},p_{\cdot j}$.
                    $$p_{i\cdot}=\sum_{j}p_{ij},\quad p_{\cdot j}=\sum_{i}p_{ij}.$$
              \item 求$f_X(x),f_Y(y)$.
                    $$f_{X}(x)=\int_{-\infty}^{+\infty}f(x,y)\mathrm{d}y=\int_{-\infty}^{+\infty}f_{Y}(y)f_{X|Y}(x\mid y)\mathrm{d}y,$$
                    $$f_{Y}(y)=\int_{-\infty}^{+\infty}f(x,y)\mathrm{d}x=\int_{-\infty}^{+\infty}f_{X}(x)f_{Y|X}(y\mid x)\mathrm{d}x.$$
          \end{enumerate}
    \item 条件分布
          \begin{enumerate}
              \item 求 $F(x\mid y_j),F(y\mid x_i).$
                    $$F(x\mid y_{j})=\sum_{x_{i}\leq x}P\{X=x_{i}\mid Y=y_{j}\},$$
                    $$F(y\mid x_{i})=\sum_{y_{j}\leq y}P\{Y=y_{j}\mid X=x_{i}\}.$$
              \item 求 $F(x\mid y),F(y\mid x).$
                    $$F(x\mid y)=\int_{-\infty}^{x}f(u\mid y)\mathrm{d}u=\int_{-\infty}^{x}\frac{f(u,y)}{f_{Y}(y)}\mathrm{d}u,$$
                    $$F(y\mid x)=\int_{-\infty}^{y}f(v\mid x)\mathrm{d}v=\int_{-\infty}^{y}\frac{f(x,v)}{f_{X}(x)}\mathrm{d}v.$$
              \item 求$P\{ Y= y_{j}\mid X= x_{i}\}$ , $P\{ X= x_{i}\mid Y= y_{j}\}$ .
                    $$P\{Y=y_{j}\mid X=x_{i}\}=\frac{P\{X=x_{i},Y=y_{j}\}}{P\{X=x_{i}\}}=\frac{p_{ij}}{p_{i}.}\:,$$
                    $$P\{X=x_{i}\mid Y=y_{j}\}=\frac{P\Big\{X=x_{i},Y=y_{j}\Big\}}{P\Big\{Y=y_{j}\Big\}}=\frac{p_{ij}}{p_{\cdot j}}\:.$$
              \item 求$f_{Y\mid X}(y\mid x),f_{X\mid Y}(x\mid y).$
                    $$f_{Y\mid X}(y\mid x)=\frac{f(x,y)}{f_{X}(x)},\quad f_{X|Y}(x\mid y)=\frac{f(x,y)}{f_{Y}(y)}$$
          \end{enumerate}
    \item 判独立
          \begin{enumerate}
              \item $X$与$Y$相互独立 $\Leftrightarrow$ 对任意$x,y$,$F(x,y)=F_X(x)\cdot F_Y(y)$.

                    $X,Y$不独立 $\Leftrightarrow$ 存在$x_0,y_0$,使$A=\{X\leqslant x_0\}$与$B=\{Y\leqslant y_0\}$不独立,即$F(x_0,y_0)\neq F_X(x_0)\cdot F_Y(y_0)$.

                    因此,证明不独立的常用方法:找$x_0,y_0$,使$0<P\{X\leqslant x_0\}$,$P\{Y\leqslant y_0\}<1$,$\{X\leqslant x_0\}\subseteq\{Y\leqslant y_0\}$或$\{Y\leqslant y_0\}\subseteq\{X\leqslant x_0\}$或$\{X\leqslant x_0,Y\leqslant y_0\}=\varnothing$.
              \item 若$(X,Y)$为二维离散型随机变量,$X$与$Y$相互独立 $\Leftrightarrow$ 对任意$i,j$,$p_{ij}=p_i\cdot p_j$.
              \item 若$(X,Y)$为二维连续型随机变量,$X$与$Y$相互独立 $\Leftrightarrow$ 对任意$x,y$,$f(x,y)=f_X(x)f_Y(y)$.
          \end{enumerate}
\end{enumerate}

\section{用分布求概率及反问题}
\begin{enumerate}
    \item $(X,Y)\sim p_{ij}$,则$P\{ ( X, Y) \in D\} = \sum_{( x_{i}, y_{j}) \in D}p_{ij}$.
    \item $(X,Y)\sim f(x,y)$,则$P\{(X,Y)\in D\}=\iint_{D}f(x,y)$d$x$d$y$ .
    \item $(X,Y)$为混合型,则用全概率公式.
    \item 反问题:已知概率反求参数.
\end{enumerate}
\chapter{多维随机变量函数的分布}
\chapter{数字特征}

计算数字特征、判别独立与不相关、用切比雪夫不等式做概率计算

\section{数学期望}
数学期望就是随机变量的取值与取值的概率乘积的和.
\begin{enumerate}
      \item $X$
            \begin{enumerate}
                  \item $X\sim p_{i}\Rightarrow EX=\sum_{i}x_{i}p_{i}\begin{cases}\text{有限项相加,}\\\text{无穷项相加(无穷级数).}\end{cases}$
                  \item $X\sim f(x)\Rightarrow EX=\int_{-\infty}^{+\infty}xf(x)$d$x\begin{cases}\text{有限区间积分(定积分),}\\\text{无穷区间积分(反常积分).}\end{cases}$
            \end{enumerate}
      \item $g(X)$

            g为连续函数(或分段连续函数).
            \begin{enumerate}
                  \item $X\sim p_{i},Y=g(X)\Rightarrow EY=\sum_{i}g(x_{i})p_{i}.$
                  \item $X\sim f( x)$ , $Y= g( X) \Rightarrow EY= \int _{- \infty }^{+ \infty }g( x) f( x)dx$.
            \end{enumerate}
      \item $g(X,Y)$
            \begin{enumerate}
                  \item $( X, Y) \sim p_{ij} , Z= g( X, Y) \Rightarrow EZ= \sum _{i}\sum _{j}g( x_{i}, y_{j}) p_{ij}$.
                  \item $( X, Y) \sim f(x,y) , Z= g( X, Y) \Rightarrow EZ= \int _{- \infty }^{+ \infty }\int _{- \infty }^{+ \infty }g( x, y) f( x, y)dxdy$ .
            \end{enumerate}
      \item 最值
            \begin{enumerate}
                  \item 若 $X_i(i=1,2,\cdots,n;n\geqslant 2)$ 独立同分布,其分布函数为 $F(x)$,概率密度为 $f(x)$,记
                        $$Y = \min\{X_1,X_2,\cdots,X_n\},\ Z = \max\{X_1,X_2,\cdots,X_n\},$$
                        则
                        \begin{enumerate}
                              \item $F_Y(y) = 1 - [1 - F(y)]^n,\ f_Y(y) = n[1 - F(y)]^{n-1}f(y) \Rightarrow EY = \int_{-\infty}^{+\infty} yf_Y(y)dy;$
                              \item $F_Z(z) = [F(z)]^n,\ f_Z(z) = n[F(z)]^{n-1}f(z) \Rightarrow EZ = \int_{-\infty}^{+\infty} zf_Z(z)dz.$
                        \end{enumerate}
                  \item 用好转化公式:
                        $$\max\{X,Y\} = \frac{X+Y+|X-Y|}{2};\quad\min\{X,Y\} = \frac{X+Y-|X-Y|}{2};$$
                        $$\max\{X,Y\} + \min\{X,Y\} = X+Y;$$
                        $$\max\{X,Y\} - \min\{X,Y\} = |X-Y|;\quad\max\{X,Y\} \cdot \min\{X,Y\} = XY.$$
                  \item 用好降维法,令 $Z = X-Y$.
                  \item 用好标准化,令 $U = \frac{X-\mu}{\sigma}$.
            \end{enumerate}
      \item 分解

            若 $X = X_{1} + X_{2} + \cdots + X_{n}$,则 $EX = EX_{1} + EX_{2} + \cdots + EX_{n}$.
      \item 性质
            \begin{enumerate}
                  \item $Ea=a$, $E(EX)=EX$.
                  \item $E(aX+bY)=aEX+bEY$, $E(\sum_{i=1}^{n}a_iX_i)=\sum_{i=1}^{n}a_iEX_i$.
                  \item 若 $X$,$Y$ 相互独立,则 $E(XY)=EXEY$.
            \end{enumerate}
\end{enumerate}
\section{方差}
\begin{enumerate}
      \item X
            \begin{enumerate}
                  \item 定义

                        $DX = E[(X - EX)^2]$,$X$的方差就是$Y = (X - EX)^2$的数学期望.
                  \item 定义法.
                        $$X \sim p_i \Rightarrow DX = E[(X - EX)^2] = \sum_i (x_i - EX)^2 p_i,$$
                        $$X \sim f(x) \Rightarrow DX = E[(X - EX)^2] = \int_{-\infty}^{+\infty} (x - EX)^2 f(x) dx.$$
                  \item 公式法.

                        $DX = E(X^2) - (EX)^2$.
            \end{enumerate}
      \item 最值的方差
            $$E(Y^{2})=\int_{-\infty}^{+\infty}y^{2}f_{Y}(y)\mathrm{d}y\Rightarrow DY=E(Y^{2})-(EY)^{2};$$

            $$E(Z^{2})=\int_{-\infty}^{+\infty}z^{2}f_{Z}(z)\mathrm{d}z\Rightarrow DZ=E(Z^{2})-(EZ)^{2}.$$
      \item 绝对值函数$|aX+bY+c|$的方差

            若 $U=aX+bY+c$,则
            $$EU=aEX+bEY+c,$$
            $$
                  DU=a^2DX+b^2DY(X,Y \text{相互独立}),$$
            $$D(|U|)=E(U^2)-[E(|U|)]^2$$
            $$
                  =DU+(EU)^2-[E(|U|)]^2,$$
            其中 $E(|U|)=\begin{cases}\int_{-\infty}^{+\infty}|u|f(u)du \text{ (连续型)},\\\sum_i|u_i|p_i \text{ (离散型)}.\end{cases}$
      \item 分解随机变量后再求⽅差

            若$X=X_{1}+X_{2}+\cdots+X_{n}$,则$DX=DX_{1}+DX_{2}+\cdots+DX_{n}+2\sum_{1\leqslant i<j\leqslant n}\operatorname{Cov}(X_{i},X_{j})$.

            当$X_{1}$,$X_{2}$,$\cdots$,$X_{n}$相互独立时,有$DX=DX_{1}+DX_{2}+\cdots+DX_{n}$.
      \item 性质
            \begin{enumerate}
                  \item $DX \geqslant 0$, $E(X^2) = DX + (EX)^2 \geqslant (EX)^2$.
                  \item $Dc = 0$($c$ 为常数).

                        $DX = 0 \Leftrightarrow X$ 几乎处处为某个常数 $a$, 即 $P\{X = a\} = 1$.
                  \item $D(aX + b) = a^2DX$.
                  \item $D(X \pm Y) = DX + DY \pm 2\text{Cov}(X, Y)$, $D\left(\sum_{i=1}^{n} a_i X_i\right) = \sum_{i=1}^{n} a_i^2 DX_i + 2 \sum_{1 \leqslant i < j \leqslant n} a_i a_j \text{Cov}(X_i, X_j)$.
            \end{enumerate}
      \item 常用分布的$EX,DX$
            \begin{enumerate}
                  \item 0—1 分布, $EX=p$ , $DX=p-p^{2}=(1-p)p$.
                  \item $X \sim B(n, p)$ , $EX=np$ , $DX=np(1-p)$.
                  \item $X \sim P(\lambda)$ , $EX=\lambda$ , $DX=\lambda$.
                  \item $X \sim G(p)$ , $EX=\frac{1}{p}$ , $DX=\frac{1-p}{p^{2}}$.
                  \item $X \sim U(a, b)$ , $EX=\frac{a+b}{2}$ , $DX=\frac{(b-a)^{2}}{12}$.
                  \item $X \sim E(\lambda)$ , $EX=\frac{1}{\lambda}$ , $DX=\frac{1}{\lambda^{2}}$.
                  \item $X \sim N(\mu, \sigma^{2})$ , $EX=\mu$ , $DX=\sigma^{2}$.
                  \item $X \sim \chi^{2}(n)$ , $EX=n$ , $DX=2n$.
            \end{enumerate}
\end{enumerate}
\section{协方差$Cov(X,Y)$与相关系数$\rho(X,Y)$}
\begin{enumerate}
      \item Cov(X,Y)
            \begin{enumerate}
                  \item 定义.
                        $$\mathrm{Cov}(X,Y)\overset{\Delta}{\operatorname*{\Longrightarrow}}E[(X-EX)(Y-EY)].$$
                  \item 定义法.
                        $$(X,Y) \sim p_{ij} \Rightarrow \operatorname{Cov}(X,Y) = \sum_{i} \sum_{j} (x_i - EX)(y_j - EY)p_{ij},$$
                        $$(X,Y) \sim f(x,y) \Rightarrow \operatorname{Cov}(X,Y) = \int_{-\infty}^{+\infty} \int_{-\infty}^{+\infty} (x - EX)(y - EY)f(x,y) \, dx \, dy.$$
                  \item 公式法.
                        $$\operatorname{Cov}(X,Y) = E(XY) - EXEY .$$
            \end{enumerate}
      \item $\rho(X,Y)$(定义相关系数,表示线性相依程度)

            $\rho_{xr}=\frac{\mathrm{Cov}(X,Y)}{\sqrt{DX}\sqrt{DY}}\begin{cases}=0\Leftrightarrow X,Y\text{不相关,}\\\neq0\Leftrightarrow X,Y\text{相关.}\end{cases}$

            (量纲为 1,无单位 )
      \item 性质
            \begin{enumerate}
                  \item $\mathrm{Cov}(X,Y) = \mathrm{Cov}(Y,X)$ .
                  \item $\mathrm{Cov}(aX,bY) = ab\mathrm{Cov}(X,Y)$.
                  \item $\mathrm{Cov}(X_1 + X_2,Y) = \mathrm{Cov}(X_1,Y) + \mathrm{Cov}(X_2,Y)$.
                  \item $|\rho_{XY}| \le 1$.
                  \item $\rho_{XY}= 1 \Leftrightarrow P\{Y = aX + b\} = 1 (a > 0)$ .

                        $\rho_{XY}= -1 \Leftrightarrow P\{Y = aX + b\} = 1 (a < 0)$.
                  \item 五个充要条件.

                        $\rho_{XY} = 0 \Leftrightarrow \text{Cov}(X,Y) = 0 \Leftrightarrow E(XY) = EXEY$
                        $\Leftrightarrow D(X+Y) = DX + DY \Leftrightarrow D(X-Y) = DX + DY.$
                  \item  $X,Y$ 独立 $\Rightarrow \rho_{XY} = 0$.
                  \item 若 $(X,Y) \sim N(\mu_1, \mu_2; \sigma_1^2, \sigma_2^2; \rho_{XY})$,则 $X,Y$ 独立 $\Leftrightarrow X,Y$ 不相关 ($\rho_{XY} = 0$).
            \end{enumerate}
\end{enumerate}
\section{独立性与不相关性的判定}
\begin{enumerate}
      \item 用分布判独立

            随机变量 $X$ 与 $Y$ 相互独立,指对任意实数 $x, y$,事件 $\{X \leqslant x\}$ 与 $\{Y \leqslant y\}$ 相互独立,即 $X$ 和 $Y$ 的联合分布等于边缘分布相乘:$F(x, y) = F_X(x) \cdot F_Y(y)$.
            \begin{enumerate}
                  \item 若 $(X, Y)$ 是连续型的,则 $X$ 与 $Y$ 相互独立的充要条件是 $f(x, y) = f_X(x) \cdot f_Y(y)$;
                  \item 若 $(X, Y)$ 是离散型的,则 $X$ 与 $Y$ 相互独立的充要条件是
                        $$P\{X = x_i, Y = y_j\} = P\{X = x_i\} \cdot P\{Y = y_j\} \, .$$
            \end{enumerate}

      \item 用数字特征判不相关

            随机变量 $X$ 与 $Y$ 不相关,意指 $X$ 与 $Y$ 之间不存在线性相依性,即 $\rho_{XY} = 0$,其充要条件是
            $$\rho_{XY} = 0 \Leftrightarrow \text{Cov}(X, Y) = 0 \Leftrightarrow E(XY) = EXEY \Leftrightarrow D(X \pm Y) = DX + DY \, .$$
      \item 步骤

            先计算 $\text{Cov}(X, Y)$,然后按下列步骤进行判断或再计算:
            $$\text{Cov}(X, Y) = E(XY) - EXEY \begin{cases} \neq 0 \Leftrightarrow X \text{ 与 } Y \text{ 相关} \Rightarrow X \text{ 与 } Y \text{ 不独立} \\ = 0 \Leftrightarrow X \text{ 与 } Y \text{ 不相关} \text{,通过分布推断} \begin{cases} X, Y \text{ 独立} \\ X, Y \text{ 不独立} \end{cases} \end{cases}$$

      \item 重要结论
            \begin{enumerate}
                  \item 如果 $X$ 与 $Y$ 独立,则 $X, Y$ 不相关,反之不然.
                  \item 如果 $X$ 与 $Y$ 相关,则 $X, Y$ 不独立.
                  \item 如果 $(X, Y)$ 服从二维正态分布,则 $X, Y$ 独立 $\Leftrightarrow X, Y$ 不相关.
                  \item 如果 $X$ 与 $Y$ 均服从 $0-1$ 分布,则 $X, Y$ 独立 $\Leftrightarrow X, Y$ 不相关.
            \end{enumerate}
\end{enumerate}

\section{切比雪夫不等式}
设随机变量$X$的数学期望与方差均存在,则对任意$\varepsilon>0$ ,
$$P\big\{\mid X-EX\mid\geqslant\varepsilon\big\}\leqslant\frac{DX}{\varepsilon^{2}}\:\text{或}\:P\{\big|X-EX\big|<\varepsilon\}\geqslant1-\frac{DX}{\varepsilon^{2}}.$$
\chapter{大数定律与中心极限定理}
\chapter{统计量及其分布}
\section{统计量及其数字特征}
设 $X_{1}, X_{2}, \cdots, X_{n}$ 是来自总体 $X$ 的简单随机样本,则
\begin{enumerate}
    \item 样本均值 $\bar{X}=\frac{1}{n} \sum_{i=1}^{n} X_{i}$.
    \item 样本方差 $S^{2}=\frac{1}{n-1} \sum_{i=1}^{n}\left(X_{i}-\bar{X}\right)^{2}=\frac{1}{n-1}\left(\sum_{i=1}^{n} X_{i}^{2}-n \bar{X}^{2}\right)$.

          样本标准差 $S=\sqrt{\frac{1}{n-1} \sum_{i=1}^{n}\left(X_{i}-\bar{X}\right)^{2}}$.
    \item 样本 $k$ 阶原点矩 $A_{k}=\frac{1}{n} \sum_{i=1}^{n} X_{i}^{k}(k=1,2, \cdots)$.
    \item 样本 $k$ 阶中心矩 $B_{k}=\frac{1}{n} \sum_{i=1}^{n}\left(X_{i}-\bar{X}\right)^{k}(k=2,3, \cdots)$.
    \item 顺序统计量

          将样本 $X_{1}, X_{2}, \cdots, X_{n}$ 的 $n$ 个观测量按其取值从小到大的顺序排列,得
          $$X_{(1)} \leqslant X_{(2)} \leqslant \cdots \leqslant X_{(n)}.$$
          随机变量 $X_{(k)}(k=1,2, \cdots, n)$ 称作第 $k$ 顺序统计量,其中 $X_{(1)}$ 是最小观测量, $X_{(n)}$ 是最大观测量,即
          $$X_{(1)}=\min \left\{X_{1}, X_{2}, \cdots, X_{n}\right\}, \quad X_{(n)}=\max \left\{X_{1}, X_{2}, \cdots, X_{n}\right\}.$$
\end{enumerate}

\section{判别统计量的分布}
定义:统计量的分布称为抽样分布

\begin{enumerate}
    \item 正态分布
          \begin{enumerate}
              \item 概念

                    如果 $X$ 的概率密度为
                    $$f(x) = \frac{1}{\sqrt{2 \pi} \sigma} \mathrm{e}^{-\frac{1}{2} \left( \frac{x - \mu}{\sigma} \right)^2} \quad (-\infty < x < +\infty),$$
                    其中 $-\infty < \mu < +\infty$, $\sigma > 0$, 则称 $X$ 服从参数为 $(\mu, \sigma^2)$ 的正态分布或称 $X$ 为正态变量, 记为 $X \sim N(\mu, \sigma^2)$.
              \item 上$\alpha$分位数

                    若 $X \sim N(0, 1)$, $P\{X > \mu_\alpha\} = \alpha$ ( $0 < \alpha < 1$ ), 则称 $\mu_\alpha$ 为标准正态分布的上 $\alpha$ 分位数
              \item 性质

                    $f(x)$ 的图形关于直线 $x=\mu$对称,即$f(\mu-x)=f(\mu+x)$,并在$x=\mu$处有唯一最大值
                    $$f(\mu)=\frac{1}{\sqrt{2\pi}\sigma}.$$
                    通常称$\mu=0$ , $\sigma=1$时的正态分布$N(0,1)$为标准正态分布,记标准正态分布的概率密度为
                    $\varphi(x)=\frac{1}{\sqrt{2\pi}}\mathrm{e}^{-\frac{1}{2}x^{2}}$,分布函数为$Q(x)=\frac1{\sqrt{2\pi}}\int_{-\infty}^{x}\mathrm{e}^{-\frac{t^{2}}{2}}dt$ .显然$\varphi(x)$为偶函数,且有
                    $$\Phi(0)=\frac{1}{2},\Phi(-x)=1-\Phi(x).$$
          \end{enumerate}
    \item $\chi^2$分布
          \begin{enumerate}
              \item 概念

                    若随机变量 $X_{1}, X_{2}, \cdots, X_{n}$ 相互独立,且都服从标准正态分布,则随机变量 $X = \sum_{i=1}^{n} X_{i}^{2}$ 服从自由度为 $n$ 的 $\chi^{2}$ 分布,记为 $X \sim \chi^{2}(n)$.
              \item 上$\alpha$分位数

                    对给定的 $\alpha (0 < \alpha < 1)$,称满足
                    $$P\{\chi^{2} > \chi_{\alpha}^{2}(n)\} = \int_{\chi_{\alpha}^{2}(n)}^{+\infty} f(x) \, \mathrm{d}x = \alpha$$
                    的 $\chi_{\alpha}^{2}(n)$ 为 $\chi^{2}(n)$ 分布的上 $\alpha$ 分位数(见图). 对于不同的 $\alpha, n$,$\chi^{2}(n)$ 分布上 $\alpha$ 分位数可通过查表求得.
              \item 性质
                    \begin{enumerate}
                        \item  若 $X_{1} \sim \chi^{2}(n_{1})$,$X_{2} \sim \chi^{2}(n_{2})$,$X_{1}$ 与 $X_{2}$ 相互独立,则
                              $$
                                  X_{1} + X_{2} \sim \chi^{2}(n_{1} + n_{2}).$$
                              此结论可推广至有限多个随机变量的和.
                        \item $若X\sim\chi^{2}(n)$,则$EX=n,DX=2n.$
                    \end{enumerate}
          \end{enumerate}
    \item $t$ 分布
          \begin{enumerate}
              \item 概念
                    设随机变量 $X \sim N(0,1)$, $Y \sim \chi^2(n)$, $X$ 与 $Y$ 相互独立, 则随机变量 $t = \frac{X}{\sqrt{Y/n}}$ 服从自由度为 $n$ 的 $t$ 分布, 记为 $t \sim t(n)$.
              \item 上$\alpha$分位数

                    对给定的 $\alpha(0 < \alpha < 1)$, 称满足
                    $$P\{t > t_{\alpha}(n)\} = \alpha$$
                    的 $t_{\alpha}(n)$ 为 $t(n)$ 分布的上 $\alpha$ 分位数.
              \item 性质
                    \begin{enumerate}
                        \item  $t$ 分布概率密度 $f(x)$ 的图形关于 $x = 0$ 对称, 因此
                              $$Et = 0 \quad (n \geqslant 2).$$
                        \item 由 $t$ 分布概率密度 $f(x)$ 图形的对称性, 知 $P\{t > -t_{\alpha}(n)\} = P\{t > t_{1-\alpha}(n)\}$, 故 $t_{1-\alpha}(n) = -t_{\alpha}(n)$. 当 $\alpha$ 值在表中没有时, 可用此式求得上 $\alpha$ 分位数.
                    \end{enumerate}
          \end{enumerate}
    \item $F$ 分布
          \begin{enumerate}
              \item 概念

                    设随机变量 $X \sim \chi^2(n_1)$,$Y \sim \chi^2(n_2)$,且 $X$ 与 $Y$ 相互独立,则 $F = \frac{X / n_1}{Y / n_2}$ 服从自由度为 $(n_1, n_2)$ 的 $F$ 分布,记为 $F \sim F(n_1, n_2)$,其中 $n_1$ 称为第一自由度,$n_2$ 称为第二自由度.$F$ 分布的概率密度 $f(x)$ 的图形.
              \item 上$\alpha$分位数

                    对给定的 $\alpha (0 < \alpha < 1)$,称满足
                    $$P\{F > F_\alpha(n_1, n_2)\} = \alpha$$
                    的 $F_\alpha(n_1, n_2)$ 为 $F(n_1, n_2)$ 分布的上 $\alpha$ 分位数.
              \item 性质
                    \begin{enumerate}
                        \item 若 $F \sim F(n_1, n_2)$,则 $\frac{1}{F} \sim F(n_2, n_1)$.
                        \item $F_{1-\alpha}(n_1, n_2) = \frac{1}{F_\alpha(n_2, n_1)}$.常用来求 $F$ 分布表中未列出的上 $\alpha$ 分位数,显然,有些特殊值可直接得出,如 $1-\alpha = \alpha$,$n_1 = n_2 = n$ 时,有 $F_{0.5}(n, n) = \frac{1}{F_{0.5}(n, n)}$,且 $F_{0.5}(n, n) > 0$,故 $F_{0.5}(n, n) = 1$.
                        \item 若 $t \sim t(n)$,则 $t^2 \sim F(1, n)$.
                    \end{enumerate}
          \end{enumerate}
\end{enumerate}
\section{用正态总体下的常用结论判别分布、计算概率}
设 $X_{1}, X_{2}, \cdots, X_{n}$ 是取自正态总体 $N(\mu, \sigma^{2})$ 的一个样本, $\bar{X}$ , $S^{2}$ 分别是样本均值和样本方差,则

\begin{enumerate}
    \item $\bar{X} \sim N\left(\mu, \frac{\sigma^{2}}{n}\right)$ ,即 $\frac{\bar{X}-\mu}{\frac{\sigma}{\sqrt{n}}} = \frac{\sqrt{n}(\bar{X}-\mu)}{\sigma} \sim N(0,1)$
    \item $\frac{1}{\sigma^{2}} \sum_{i=1}^{n}(X_{i}-\mu)^{2} \sim \chi^{2}(n)$ ;
    \item $\frac{(n-1)S^{2}}{\sigma^{2}} = \sum_{i=1}^{n}\left(\frac{X_{i}-\bar{X}}{\sigma}\right)^{2} \sim \chi^{2}(n-1)$ ( $\mu$ 未知,在 “2.” 中用 $\bar{X}$ 替代 $\mu$ );
    \item $\bar{X}$ 与 $S^{2}$ 相互独立, $\frac{\sqrt{n}(\bar{X}-\mu)}{S} \sim t(n-1)$ ( $\sigma$ 未知,在 “1.” 中用 $S$ 替代 $\sigma$ ). 进一步有
          $$\frac{n(\bar{X}-\mu)^{2}}{S^{2}} \sim F(1, n-1).$$
\end{enumerate}


\chapter{参数估计与假设检验}


\GAIgroupsancheck
\XIANgroupsancheck
\GAOgroupsancheck

\makeatletter
\let\chapter\@std@chapter
\let\@std@chapter\relax
\makeatother

\backmatter
{\small
  \printindex
  \printindex[sym]
}

\end{document}
