\makeatletter
   \def\input@path{{..}} % 搜索上层目录的 LALUbook
\makeatother

\documentclass[
    % colors = false,
    geometry = a4,
]{LALUbook}

\usepackage{mathdots}
\usepackage{booktabs} % Excel 导出的大表格
\usepackage{rotating}
\usepackage{extarrows}
\usepackage{blkarray}
\usepackage{cases}

\usepackage{float}
\usepackage{diagbox}
\usepackage{caption}

\usepackage{pgfplots}
\usetikzlibrary{cd, arrows, arrows.meta, calc, intersections, decorations.pathreplacing, patterns, decorations.markings,angles,quotes, graphs, positioning, shapes.geometric}
\pgfplotsset{compat=newest}

\usepackage[xindy, splitindex]{imakeidx}
\makeindex[
    columns=1,
    program=truexindy,
    intoc=true,
    options=-M texindy -I xelatex -C utf8,
    title={名词索引}
] % 名词索引
\makeindex[
    columns=3,
    program=truexindy,
    intoc=true,
    options=-M numeric-sort -M latex -M latex-loc-fmts -M makeindex -I xelatex -C utf8,
    name=sym,
    title={符号索引}
] % 符号索引

% 标题格式
% \chapter   正常专题
% \LUchapter 强化专题

\newcounter{LUchapter}
\newcounter{LUgreekchap}

\makeatletter
% 此处可按需增改
% \texorpdfstring 的两个参数分别显示在正文中与 PDF 书签中
\newcommand*{\@LUgreek}[1]{%
    \ifcase#1\or\texorpdfstring{$\boldsymbol{\varepsilon}$}{ε}%
    \or\texorpdfstring{$\boldsymbol{\delta}$}{δ}%
    \or\texorpdfstring{$\boldsymbol{\lambda}$}{λ}%
    \or\texorpdfstring{$\boldsymbol{\mu}$}{μ}%
    \or\texorpdfstring{$\boldsymbol{\varphi}$}{φ}%
    \or\texorpdfstring{$\boldsymbol{\theta}$}{θ}%
    \else\@ctrerr\fi%
}
\newcommand*{\LUgreek}[1]{%
    \expandafter\@LUgreek\csname c@#1\endcsname
}
\newcommand*{\LUchapsancheck}{%
\expandafter\@ifundefined{@exist@LUchapter@\arabic{chapter}.\arabic{LUgreekchap}}%
    {\setcounter{LUgreekchap}{1}}
    {\stepcounter{LUgreekchap}}
}
\newcommand*{\LUgroupsancheck}{%
\expandafter\@ifundefined{@exist@LUchapter@\arabic{chapter}}%
    {}
    {\endgroup}
}
\let\@std@chapter\chapter
\renewcommand*{\chapter}{%
    \LUgroupsancheck%
    \@std@chapter
}
\let\@std@chaptermark\chaptermark
\def\chaptermark#1{\def\@LALU@chaptername{#1}\@std@chaptermark{#1}}
\makeatother


% OPD 系列
\newcommand{\OOne}{\textcolor{blue}{\textbf{$O$ (盯住目标)}}}
\newcommand{\DOne}{\textcolor{blue}{\textbf{$D_1$ (常规操作)}}}
\newcommand{\DTwo}{\textcolor{blue}{\textbf{$D_2$ (脱胎换骨)}}}
\newcommand{\DThree}{\textcolor{blue}{\textbf{$D_3$ (移花接木)}}}
\newcommand{\DFour}{\textcolor{blue}{\textbf{$D_4$ (可圈可点)}}}
\newcommand{\DTwoOne}{\textcolor{blue}{\textbf{$D_{21}$ (观察研究对象)}}}
\newcommand{\DTwoTwo}{\textcolor{blue}{\textbf{$D_{22}$ (转换等价表述)}}}
\newcommand{\DTwoThree}{\textcolor{blue}{\textbf{$D_{23}$ (化归经典形式)}}}
\newcommand{\POne}{\textcolor{blue}{\textbf{$P_{1}$ (常规思路)}}}
\newcommand{\POneOne}{\textcolor{blue}{\textbf{$P_{11}$ (正向思路)}}}
\newcommand{\POneTwo}{\textcolor{blue}{\textbf{$P_{12}$ (反向思路)}}}
\newcommand{\POneThree}{\textcolor{blue}{\textbf{$P_{13}$ (双向思路)}}}
\newcommand{\PTwo}{\textcolor{blue}{\textbf{$P_{2}$ (反证思路)}}}
\newcommand{\PThree}{\textcolor{blue}{\textbf{$P_{3}$ (数学归纳)}}}
\newcommand{\PFour}{\textcolor{blue}{\textbf{$P_{4}$ (逆否思路)}}}
\newfontfamily\RomanSymbols{Arial Unicode MS}

\newtcbtheorem[number within=section]{detail}{细节}{laluthmstyle={red}}{det}
\newtcbtheorem[number within=section]{idea}{思路}{laluthmstyle={teal}}{ide}

% 内容总结
\newenvironment{summary}{%
    \hypersetup{bookmarksnumbered=false}%
    \titleformat{\subsection}[block]{\centering\heiti\Large}{}{1em}{}%
    \phantomsection%
    \subsection*{内容总结}%
}{}

\NewDocumentCommand{\LUchapter}{m}{%
\LUgroupsancheck
\begingroup
\LUchapsancheck
\addtocounter{chapter}{-1}
\refstepcounter{LUchapter}
\renewcommand*{\thechapter}{\arabic{chapter}\LUgreek{LUgreekchap}}
% \renewcommand*{\thechapter}{\arabic{chapter}}
\renewcommand*{\theHchapter}{LU.\arabic{LUchapter}}
\ctexset{
    chapter={format={\centering\Huge\bfseries},name={强化专题,},number={\zhnumber{\arabic{LUchapter}}}},
}
\csname @std@chapter\endcsname{#1}
\expandafter\xdef\csname @exist@LUchapter@\arabic{chapter}\endcsname{\null}
\expandafter\xdef\csname @exist@LUchapter@\arabic{chapter}.\arabic{LUgreekchap}\endcsname{\null}
}

\ctexset{
    chapter={format={\centering\Huge\bfseries},name={强化专题,},number={\zhnumber{\arabic{chapter}}}},
    % chapter={format={\centering\Huge\bfseries},name={第,讲},number=\arabic{chapter}},
    section={format={\raggedright\Large\bfseries},name={,},number={\thechapter.\arabic{section}}},
    subsection={format={\raggedright\large\bfseries},name={,},number={\thesection.\arabic{subsection}}},
    subsubsection={format={\raggedright\normalsize\bfseries},name={,},number={\thesubsection.\arabic{subsubsection}}},
}

\title{\heiti 临沂大学 2025--2026 学年 \\ 2026考研数学总结讲义}

\AtEndPreamble{\hypersetup{
    hypertexnames=true,
    pdfauthor={林子立},
    pdftitle={2026考研数学总结讲义},
}}

\begin{document}

\title{2026考研数学总结讲义}
\author{林子立}
\date{\today}
\maketitle

\songti

\pagenumbering{Roman}
\clearpage

\pdfbookmark[0]{目录}{contents}
\tableofcontents

\addtolength{\parskip}{.5em}

\mainmatter
\chapter{行列式}
\XIANchapter{余子式和代数余子式的计算}

\section{计算余子式、代数余子式的线性组合}

% \chapter{习题3}
\clearpage

\begin{question}{}{}
    对于任意给定的 $\varepsilon\in(0,1)$,总存在整数 $N$,当 $n>N$ 时,恒有 $|x_n-a|\le 2\varepsilon$ ,是数列 $\{x_n\}$ 收敛于 $a$ 的( )

    \begin{multicols}{2}  % 并排两列
        \begin{itemize}[label={}]
            \item A 充分条件但非必要条件
            \item B 必要条件但非充分条件
            \item C 充分必要条件
            \item D 既非充分条件又非必要条件
        \end{itemize}
    \end{multicols}
\end{question}
\begin{solution}
    该条件表述为:对任意 $\varepsilon\in(0,1)$,存在 $N$,当 $n>N$ 时 $|x_n-a|\le2\varepsilon$。要判别此条件与收敛 $x_n\to a$ 的等价性:

    若序列收敛于 $a$,则对任意 $\varepsilon>0$ 存在 $N$ 使得当 $n>N$ 有 $|x_n-a|<\varepsilon$,从而对任意 $\varepsilon\in(0,1)$ 同样成立(取同样的 $N$)。反过来,若题中条件成立,则对任意给定的正 $\varepsilon$:若 $\varepsilon\ge1$,不难直接满足;若 $0<\varepsilon<1$,条件已给出 $|x_n-a|\le2\varepsilon$,令 $\varepsilon'=\varepsilon/2\in(0,1)$,由题意存在 $N$ 使得当 $n>N$ 有 $|x_n-a|\le2\varepsilon'=\varepsilon$,故对任意 $\varepsilon>0$ 都能得到相应 $N$,即 $x_n\to a$。因此该条件与收敛等价。

    答案:C(充分必要条件)。
\end{solution}

\begin{question}{}{}
    设 $f^\prime(1)=0,\lim_{x\to1}\frac{f^{\prime}(x)}{\left(x-1\right)^3}=-2$, 则$x=1$为()
    \begin{multicols}{4}  % 并排两列
        \begin{itemize}[label={}]
            \item A 非极值点
            \item B 极大值点
            \item C 极小值点
            \item D 间断点
        \end{itemize}
    \end{multicols}
\end{question}
\begin{solution}
    由极限条件有,当 $x\to1$ 时
    \[
        f'(x)\sim -2(x-1)^3.
    \]
    因此对于 $x>1$(且靠近 1)有 $(x-1)^3>0$,所以 $f'(x)\approx -2(x-1)^3<0$;对于 $x<1$ 有 $(x-1)^3<0$,所以 $f'(x)\approx -2(x-1)^3>0$。即导数在 $x=1$ 处由正变负,函数先增后减,所以在 $x=1$ 处取得局部极大值。

    答案:B(极大值点)。
\end{solution}

\begin{question}{}{}

    设 $\lim _{x\to 0}\frac {\sin 6x+ xf( x) }{x^3}= 0$,则$\lim_{x\to0}\frac{6+f(x)}{x^2}=$
    \begin{multicols}{4}  % 并排两列
        \begin{itemize}[label={}]
            \item A 0
            \item B 6
            \item C 36
            \item D $\infty$
        \end{itemize}
    \end{multicols}
\end{question}
\begin{solution}
    使用 $\sin6x$ 的泰勒展开:
    \[
        \sin6x=6x-\frac{(6x)^3}{6}+o(x^3)=6x-36x^3+o(x^3).
    \]
    将其代入:
    \[
        \frac{\sin6x+xf(x)}{x^3}=\frac{6x-36x^3+xf(x)+o(x^3)}{x^3}
        =\frac{6}{x^2}-36+\frac{f(x)}{x^2}+o(1).
    \]
    极限为 0,故
    \[
        \frac{6+f(x)}{x^2}\to 36.
    \]
    答案:C(36)。
\end{solution}

\begin{question}{}{}

    求极限 $\lim_{x \to +\infty} \frac{\sqrt{5x}-1}{\sqrt{x+2}}$.
\end{question}
\begin{solution}
    按主项比较:
    \[
        \frac{\sqrt{5x}-1}{\sqrt{x+2}}
        = \frac{\sqrt{x}\left(\sqrt5 - \tfrac{1}{\sqrt{x}}\right)}{\sqrt{x}\sqrt{1+2/x}}
        \to \frac{\sqrt5}{1}=\sqrt5.
    \]
    答案:$\sqrt5$。
\end{solution}

\begin{question}{}{}

    求极限 $\lim_{x \to 2} \frac{|x-2|}{x^2-4}$.
\end{question}
\begin{solution}
    化简 ($x\neq2$):
    \[
        \frac{|x-2|}{x^2-4}=\frac{|x-2|}{(x-2)(x+2)}=\frac{1}{|x+2|}.
    \]
    当 $x\to2$,得 $1/4$。

    答案:$\displaystyle \frac14$。
\end{solution}

\begin{question}{}{}

    求极限 $\lim_{x \to 0} \frac{\left(\frac{2+\cos x}{3}\right)^x-1}{x^3}$.
\end{question}
\begin{solution}
    令 $a(x)=\dfrac{2+\cos x}{3}$. 当 $x\to0$,
    \[
        a(x)=1-\frac{x^2}{6}+O(x^4).
    \]
    因此
    \[
        \ln a(x)=-\frac{x^2}{6}+o(x^2).
    \]
    有
    \[
        \left(a(x)\right)^x=e^{x\ln a(x)}=1+x\ln a(x)+o(x\ln a(x))=1-\frac{x^3}{6}+o(x^3).
    \]
    故所求极限为 $-1/6$。
\end{solution}

\begin{question}{}{}

    求极限 $\lim_{x \to 0} \frac{e^{\tan x}-e^{\sin x}}{x^2 \ln(1+x)}$.
\end{question}
\begin{solution}
    先用展开:
    \[
        \tan x = x+\frac{x^3}{3}+o(x^3),\qquad \sin x = x-\frac{x^3}{6}+o(x^3),
    \]
    所以
    \[
        \tan x-\sin x=\frac{x^3}{2}+o(x^3).
    \]
    写
    \[
        e^{\tan x}-e^{\sin x}=e^{\sin x}\big(e^{\tan x-\sin x}-1\big)
        \sim e^{x}\big(\tan x-\sin x\big)\sim 1\cdot\frac{x^3}{2}.
    \]
    分母:
    \[
        x^2\ln(1+x)=x^2\big(x+o(x)\big)=x^3+o(x^3).
    \]
    因此极限为 $(1/2)/1=1/2$。
\end{solution}

\begin{question}{}{}

    求极限 $\lim_{x \to 0} \frac{e^{x^2}-e^{2-2\cos x}}{x^4}$.
\end{question}
\begin{solution}
    先展开:
    \[
        e^{x^2}=1+x^2+\tfrac{x^4}{2}+o(x^4),
    \]
    而
    \[
        2-2\cos x=2-2\big(1-\tfrac{x^2}{2}+\tfrac{x^4}{24}+o(x^4)\big)=x^2-\tfrac{x^4}{12}+o(x^4).
    \]
    于是
    \[
        e^{2-2\cos x}=1+(x^2-\tfrac{x^4}{12})+\tfrac{1}{2}x^4+o(x^4)
        =1+x^2+x^4\Big(-\tfrac{1}{12}+\tfrac{1}{2}\Big)+o(x^4)
        =1+x^2+\tfrac{5}{12}x^4+o(x^4).
    \]
    两者相减得
    \[
        e^{x^2}-e^{2-2\cos x}=\Big(\tfrac{1}{2}-\tfrac{5}{12}\Big)x^4+o(x^4)=\tfrac{1}{12}x^4+o(x^4).
    \]
    所以极限为 $1/12$。
\end{solution}

\begin{question}{}{}

    求极限 $\lim_{x \to +\infty} \left[x - x^2 \ln \left(1 + \frac{1}{x}\right)\right]$.
\end{question}
\begin{solution}
    使用 $\ln(1+u)=u-\tfrac{u^2}{2}+o(u^2)$,取 $u=1/x$:
    \[
        x^2\ln\Big(1+\frac{1}{x}\Big)=x^2\Big(\frac{1}{x}-\frac{1}{2x^2}+o(1/x^2)\Big)=x-\frac{1}{2}+o(1).
    \]
    因此差为约 $\tfrac12$,极限为 $1/2$。
\end{solution}

\begin{question}{}{}

    求极限 $\lim_{x \to 0} \frac{e^{x^2} - \cos x}{x^2}$.
\end{question}
\begin{solution}
    展开:
    \[
        e^{x^2}=1+x^2+\tfrac{x^4}{2}+o(x^4),\qquad \cos x=1-\tfrac{x^2}{2}+\tfrac{x^4}{24}+o(x^4).
    \]
    相减得主项为 $x^2+\tfrac{x^2}{2}=\tfrac{3}{2}x^2$,除以 $x^2$ 得 $3/2$。
\end{solution}

\begin{question}{}{}

    求极限 $\lim_{x \to 0} \frac{[\sin x - \sin (\sin x)] \sin x}{x^4}$.
\end{question}
\begin{solution}
    先展开:
    \[
        \sin x = x-\tfrac{x^3}{6}+o(x^3).
    \]
    令 $s=\sin x=x-\tfrac{x^3}{6}+o(x^3)$,则
    \[
        \sin(\sin x)=s-\tfrac{s^3}{6}+o(s^3)=\Big(x-\tfrac{x^3}{6}\Big)-\tfrac{1}{6}x^3+o(x^3)=x-\tfrac{x^3}{3}+o(x^3).
    \]
    因此
    \[
        \sin x-\sin(\sin x)=\tfrac{x^3}{6}+o(x^3).
    \]
    乘以 $\sin x\sim x$,分子 $\sim \tfrac{x^4}{6}$,故极限为 $1/6$。
\end{solution}

\begin{question}{}{}

    求极限 $\lim_{x \to 0} \left(\frac{2 + e^{\frac{1}{x}}}{1 + e^{\frac{4}{x}}} + \frac{\sin x}{|x|}\right)$.
\end{question}
\begin{solution}
    分左右极限考察。

    当 $x\to0^+$ 时,$1/x\to+\infty$,故 $e^{1/x}\to\infty,\ e^{4/x}\to\infty$,且
    \[
        \frac{2+e^{1/x}}{1+e^{4/x}}=\frac{e^{1/x}\big(1+2e^{-1/x}\big)}{e^{4/x}\big(1+e^{-4/x}\big)}=e^{-3/x}(1+o(1))\to0.
    \]
    同时 $\dfrac{\sin x}{|x|}=\dfrac{\sin x}{x}\to1$。和为 $1$。

    当 $x\to0^-$ 时,$1/x\to-\infty$,所以 $e^{1/x}\to0,\ e^{4/x}\to0$,首项 $\to 2$;而 $\dfrac{\sin x}{|x|}=\dfrac{\sin x}{-x}=-\dfrac{\sin x}{x}\to -1$。和为 $2-1=1$。

    左右极限相等,极限存在且等于 $1$。
\end{solution}

\begin{question}{}{}

    求极限$\lim_{n\to\infty}\left(\frac{1}{\sqrt{n^2+1^2}}+\frac{1}{\sqrt{n^2+2^2}}+\cdots+\frac{1}{\sqrt{n^2+n^2}}\right)$.
\end{question}
\begin{solution}
    写为求和的 Riemann 近似:
    \[
        \sum_{k=1}^n \frac{1}{\sqrt{n^2+k^2}}=\sum_{k=1}^n \frac{1}{n}\cdot\frac{1}{\sqrt{1+(k/n)^2}}.
    \]
    当 $n\to\infty$,该和趋于
    \[
        \int_0^1 \frac{dt}{\sqrt{1+t^2}}.
    \]
    计算不定积分,$\int \frac{dt}{\sqrt{1+t^2}}=\operatorname{arsinh}t=\ln\big(t+\sqrt{1+t^2}\big)$。因此值为
    \[
        \ln\big(1+\sqrt2\big).
    \]
\end{solution}

\begin{question}{}{}

    设$f(x)\in C[a,b]$,证明$\exists\xi\in[a,b]$,使得$\int_{a}^{b}f(x)dx=f(\xi)(b-a)$.
\end{question}
\begin{solution}
    这是积分中值定理(连续函数的平均值定理)。因为 $f$ 在 $[a,b]$ 上连续,令
    \[
        m=\min_{[a,b]} f,\qquad M=\max_{[a,b]} f.
    \]
    则 $m(b-a)\le\int_a^b f(x)\,dx\le M(b-a)$. 由于 $f$ 在 $[a,b]$ 连续,值域为 $[m,M]$,而平均值 $\dfrac{1}{b-a}\int_a^b f(x)\,dx$ 属于 $[m,M]$,由连续性和介值定理存在 $\xi\in[a,b]$ 使得
    \[
        f(\xi)=\frac{1}{b-a}\int_a^b f(x)\,dx,
    \]
    即 $\int_a^b f(x)\,dx=f(\xi)(b-a)$。证毕。
\end{solution}

\begin{question}{}{}

    设$x_{1}=\sqrt{2}$,$x_{n}=\sqrt{2+x_{n-1}}$,$n=2,3,\cdots$. 证明$\{x_{n}\}$极限存在,并求极限.
\end{question}
\begin{solution}
    首先证明单调性与有界性:

    (1) 计算数列初项:$x_1=\sqrt2\approx1.414$,$x_2=\sqrt{2+\sqrt2}\approx1.847\,$。猜测单调递增且有上界 2。

    (2) 若 $x_{n-1}<2$,则 $x_n=\sqrt{2+x_{n-1}}<\sqrt{4}=2$,所以若某一项 $<2$,后项都 $<2$,故有上界 2。且
    \[
        x_{n+1}-x_n=\sqrt{2+x_n}-\sqrt{2+x_{n-1}}=\frac{x_n-x_{n-1}}{\sqrt{2+x_n}+\sqrt{2+x_{n-1}}}.
    \]
    若已知 $x_n>x_{n-1}$,则右端 $>0$,由此可递推证明 $\{x_n\}$ 单调递增(因 $x_2>x_1$ 起)。

    因此序列单调有界,故收敛。设极限为 $L$,取极限于递推式得
    \[
        L=\sqrt{2+L}\quad\Rightarrow\quad L^2=2+L\quad\Rightarrow\quad L^2-L-2=0.
    \]
    解得 $L=\dfrac{1\pm\sqrt{1+8}}{2}=\dfrac{1\pm3}{2}$,因为项均为正且小于 2,取正根 $L=2$。

    结论:极限存在且为 $2$。
\end{solution}
\chapter{矩阵运算}

\chapter{矩阵的秩}
\chapter{线性方程组}
\section{线性方程组理论总结}
\DOne
\begin{enumerate}
    \item 齐次线性方程组$Ax=0$ \DOne
    \item 非齐次线性方程组$Ax=b$ \DOne
\end{enumerate}


\section{线性方程组问题}
\begin{enumerate}
    \item 一般求解问题
    \item 公共解问题
    \item 同解问题
          \DOne+\DTwoTwo
          \begin{detail}{齐次线性方程组\DTwoTwo}{}
          \end{detail}

          \begin{detail}{非齐次线性方程组\DTwoTwo}{}
              设(\RomanSymbols Ⅰ)$A_{m\times n}x=\beta$与(Ⅱ)$B_{s\times n}x=\gamma$均有解,则

              ①(\RomanSymbols Ⅰ)与(Ⅱ)同解

              $\Leftrightarrow$②$A_{m\times n}x=0$与$B_{s\times n}x=0$同解且(\RomanSymbols Ⅰ)与(Ⅱ)有公共解

              $\Leftrightarrow$③$r\left(\begin{bmatrix}A & \beta \\ B & \gamma\end{bmatrix}\right)=r\left(\begin{bmatrix}A \\ B\end{bmatrix}\right)=r(A)=r(B)$

              $\Leftrightarrow$④$[A,\beta]$与$[B,\gamma]$的行向量组等价.
          \end{detail}
\end{enumerate}


\section{线性方程组的几何意义}



\chapter{向量组}

\section{研究具体型向量关系}

\subsection{定义法}
\subsection{求极大线性无关组}
\section{研究抽象型向量关系}
\subsection{定义法}
\subsection{综合问题}
\DOne+\DTwoThree

$$x = \eta^{*} + k_{1} \xi_{1} + \ldots + k_{n-r} \xi_{n-r}.$$
$$
A \xi_{i} = 0.
$$
$$A \eta^{*} = \beta.$$
\section{研究向量组等价}
\section{向量空间}
\subsection{概念}
\subsection{过渡矩阵}
\subsection{坐标变换}
\chapter{特征向量与特征值}
\term{求解利用} $A$的特征值与特征向量

\section{利用特征值命题}
\DOne +\DTwoTwo
\begin{enumerate}
    \item $\lambda_0$是$A$ 的特征值$\Leftrightarrow|\lambda_0E-A|=0$(建立方程求参数或证明行列式 $|\lambda_0E-A|=0$ );
          $\lambda_0$不是$A$ 的特征值$\Leftrightarrow|\lambda_0E-A|\neq0$(矩阵可逆,满秩).
    \item 若$\lambda_1,\lambda_2,\cdots,\lambda_n$是$A$的 $n$个特征值,则

          $$\begin{cases}|A|=\lambda_1\lambda_2\cdots\lambda_n\:,\\\mathrm{tr}\left(A\right)=\lambda_1+\lambda_2+\cdots+\lambda_n\:.\end{cases}$$
    \item \begin{enumerate}
              \item 记住下表
                    \begin{table}[h]
                        \centering
                        \begin{tabular}{|c|c|c|c|c|c|c|}
                            \hline
                            矩阵      & $A$       & $f(A)$       & $A^{-1}$            & $f(A^*)$               & $P^{-1}AP=B$ & $P^{-1}f(A)P=B$ \\
                            \hline
                            特征值     & $\lambda$ & $f(\lambda)$ & $\frac{1}{\lambda}$ & $\frac{|A|}{\lambda} $ & $\lambda$    & $f(\lambda)$    \\
                            \hline
                            对应的特征向量 & $\xi$     & $\xi$        & $\xi$               & $\xi$                  & $P^{-1}\xi$  & $P^{-1}\xi$     \\
                            \hline
                        \end{tabular}
                    \end{table}

                    表中$\lambda$在分母上的,设 $\lambda != 0$
                    \begin{note}{}{}
                        当$\lambda\neq0$ 时,$af(A)\pm bA^{-1}\pm cA^{.}$的特征值为$af\left(\lambda\right)\pm b\frac1\lambda\pm c\frac{|A|}\lambda$,特征向量仍为$\xi$.但
                        $f(A),A^{-1},A^*$ 与$A^T$,$B$ 的线性组合因特征向量不同,无上述规律.
                    \end{note}
              \item $\text{虽然 }A^\mathrm{T}\text{ 的特征值与 }A\text{ 相同,但特征向量不再是 }\xi\text{,要单独计算才能得出 }.$
                    \begin{note}{}{}
                        $ A^\mathrm{T}\text{ 和 }A\text{ 属于不同特征值的特征向量正交 }.$
                    \end{note}
              \item 归零原则.
                    \begin{enumerate}
                        \item 归零准则一: 设$f(x)$为多项式, 若矩阵$A$满足$f(A)=O$, $\lambda$是$A$的任一特征值, 则$\lambda$满足$f(\lambda)=0$.
                        \item 归零准则二:设$n$ 阶方阵 $A$ 的特征多项式为$f(\lambda)=|\lambda E-A|=\lambda^n+a_{n-1}\lambda^{n-1}+\cdots+a_1\lambda+a_0$,则$A$ 的
                              多项式$f(A)$为零矩阵,即$f( A) = A^{n }+ a_{n- 1}A^{n- 1}+ \cdots + a_{1}A+ a_{0}E= O$ .
                    \end{enumerate}
          \end{enumerate}
\end{enumerate}

\section{利用特征向量命题}
\DOne + \DTwoTwo
\begin{enumerate}
    \item $\xi(\neq0)$是$A$的属于$\lambda_{0}$的特征向量$\Leftrightarrow\xi$是$(\lambda_{0}E-A)x=0$的非零解.\DTwoTwo
    \item 重要结论.
          \begin{enumerate}
              \item 单根恰有$1$个线性无关的特征向量.
              \item $k$重特征值$\lambda$至多只有$k$个线性无关的特征向量($k≥2$).
              \item 若$\xi_1,\xi_2$是$A$的属于不同特征值$\lambda_1,\lambda_2$的特征向量,则$\xi_1,\xi_2$线性无关.
              \item 若$\xi_1,\xi_2$是$A$ 的属于同一特征值 $\lambda$ 的特征向量,则当 $k_1k_2\neq0$ 时,非零向量 $k_1\xi_1+k_2\xi_2$仍是 $A$ 的属于特征值$\lambda$的特征向量(常考其中一个系数(如$k_2$)等于0的情形).
              \item 若$\xi_1,\xi_2$是$A$的属于不同特征值$\lambda_1,\lambda_2$的特征向量,则当$k_1\neq0,k_2\neq0$时,$k_1\xi_1+k_2\xi_2$不是$A$的任何特征值的特征向量(常考$k_1=k_2=1$的情形 ).
              \item 若 $\xi$ 是 $A$ 的属于特征值 $\lambda_1$ 的特征向量,$\lambda_1 \neq \lambda_2$,则 $\xi$ 不是 $\lambda_2$ 的特征向量.
              \item 若$A$只有$1$个线性无关的特征向量,即$\sum_{i=1}^{m}[n-r(\lambda_{i}E-A)]=1$,$\lambda_{i}(i=1,2,\cdots,m)$是A的m个不同特征值,则只能有一个$\lambda_{k}(1\leqslant k\leqslant m)$,使$r(\lambda_{k}E-A)=n-1$,而其余$r(\lambda_{i}E-A)=n$,这与$r(\lambda_{i}E-A)<n$矛盾。故A只能有一个$\lambda_{k}$,且此$\lambda_{k}$为n重特征值.
              \item 设$n$阶矩阵$A,B$满足$AB=BA$,且$A$有$n$个互不相同的特征值,则$A$的特征向量都是$B$的特征向量.
              \item 若 $r(A) + r(B) < n$,则 $Ax = 0$,$Bx = 0$ 至少有一个公共非零解 $\xi$.
          \end{enumerate}
\end{enumerate}
\section{利用矩阵方程命题}
\DOne + \DTwoTwo + \DTwoThree

\begin{enumerate}
    \item $AB=O \Rightarrow A[\beta_1, \beta_2, \cdots, \beta_n] = [0, 0, \cdots, 0]$, 即 $A\beta_i = 0\beta_i (i=1, 2, \cdots, n)$, 若 $\beta_i$ 均为非零列向量, 则 $\beta_i$ 为 $A$ 的属于特征值 $\lambda=0$ 的特征向量.
    \item 若任意 $n$ 维列向量 $\xi (\neq 0)$ 均为 $(\lambda E - A)x = 0$ 的解, 则令 $e_1 = \begin{bmatrix} 1 \\ 0 \\ \vdots \\ 0 \end{bmatrix}$, $e_2 = \begin{bmatrix} 0 \\ 1 \\ \vdots \\ 0 \end{bmatrix}$, $\cdots$, $e_n = \begin{bmatrix} 0 \\ \vdots \\ 0 \\ 1 \end{bmatrix}$,且$\boldsymbol{B}=[\boldsymbol{e}_{1},\boldsymbol{e}_{2},\cdots,\boldsymbol{e}_{n}]$,于是$(\lambda\boldsymbol{E}-\boldsymbol{A})\boldsymbol{B}=\boldsymbol{O}$,由于$\boldsymbol{B}$可逆,因此有$\lambda\boldsymbol{E}-\boldsymbol{A}=\boldsymbol{O}$,即$\boldsymbol{A}=\lambda\boldsymbol{E}$.
    \item $AB = C \Rightarrow A[\beta_1, \beta_2, \cdots, \beta_n] = [\gamma_1, \gamma_2, \cdots, \gamma_n] = [\lambda_1\beta_1, \lambda_2\beta_2, \cdots, \lambda_n\beta_n]$, 即 $A\beta_i = \lambda_i\beta_i (i=1, 2, \cdots, n)$, 其中 $\gamma_i = \lambda_i\beta_i$, $\beta_i$ 为非零列向量, 则 $\beta_i$ 为 $A$ 的属于特征值 $\lambda_i$ 的特征向量.
    \item  $AP = PB$, $P$ 可逆 $\Rightarrow P^{-1}AP = B \Rightarrow A \sim B \Rightarrow \lambda_A = \lambda_B$.
    \item $A$ 的每行元素之和均为 $k \Rightarrow A\begin{bmatrix} 1 \\ 1 \\ \vdots \\ 1 \end{bmatrix} = k\begin{bmatrix} 1 \\ 1 \\ \vdots \\ 1 \end{bmatrix}\Rightarrow k$ 是特征值, $\begin{bmatrix} 1 \\ 1 \\ \vdots \\ 1 \end{bmatrix}$ 是 $A$ 的属于特征值 $k$ 的特征向量.
    \item 若 $A$ 可逆, $A$ 的每行元素之和均为 $k$, 则 $A^{-1}$ 的每行元素之和均为 $\frac{1}{k}$.
    \item 若 $A$ 的每行元素之和均为 $k$, 则 $A^n$ 的每行元素之和均为 $k^n$.
\end{enumerate}
\chapter{相似理论}
\section{化归相似对角化的基本局面}
\DOne+\DTwoThree

若 $n$ 阶矩阵 $A$ 有 $n$ 个线性无关的特征向量,则 $A$ 可相似对角化,且有
$$[\xi_1, \xi_2, \cdots, \xi_n]^{-1} A [\xi_1, \xi_2, \cdots, \xi_n] = \begin{bmatrix} \lambda_1 & & \\ & \lambda_2 & \\ & & \ddots \\ & & & \lambda_n \end{bmatrix},$$
牢记这个形式.

\section{用各种条件判$A$能否相似对角化}
\DOne+\DTwoTwo

$\star \star \star$

\begin{enumerate}
    \item 充要条件
          \begin{enumerate}
              \item $A$有$n$个线性无关的特征向量 $\Leftrightarrow A \sim\Lambda$
              \item $n_i = n - r(\lambda_i E - A) \Leftrightarrow A \sim \Lambda$.
          \end{enumerate}
    \item 充分条件
          \begin{enumerate}
              \item $A$是实对称矩阵$\Leftrightarrow A \sim \Lambda$.
              \item $A$有$n$个互异特征值$\Leftrightarrow A \sim \Lambda$.
              \item $A^k=E$($k$为正整数) $\Leftrightarrow A \sim \Lambda$.
              \item $A^2 - (k_1 + k_2)A + k_1 k_2 E = O$ 且 $k_1 \neq k_2 \Rightarrow A \sim \Lambda$.
              \item $r(A) = 1$ 且 $\text{tr}(A) \neq 0 \Rightarrow A \sim \Lambda$.
          \end{enumerate}
    \item 必要条件

          $A \sim \Lambda \Rightarrow r(A) =$ 非零特征值的个数 (重根按重数算).
    \item 否定条件
          \begin{enumerate}
              \item $A \neq O$, $A^k = O$ ($k$ 为大于 1 的整数) $\Rightarrow A$ 不可相似对角化.
              \item $A$ 的特征值全为 $k$, 但 $A \neq kE \Rightarrow A$ 不可相似对角化.
          \end{enumerate}
\end{enumerate}

\section{非对称矩阵$A$与实对称矩阵$A$相似对角化的异同}
\DOne+\DTwoOne

\begin{enumerate}
    \item 非对称矩阵$A$不存在正交矩阵$Q$,使其相似对角化
    \item 实对称矩阵$A$存在正交矩阵$Q$,使其相似对角化
\end{enumerate}
\section{$A$与$B$相似}
\DOne+\DTwoOne+\PFour

$\star \star \star$

\begin{enumerate}
    \item 若$A$相似于$B$,则
          \begin{enumerate}
              \item $|A| = |B|$;
              \item $r(A)=r(B)$;
              \item $tr(A)=tr(B)$;
              \item $\lambda_{A} = \lambda_{B}$ (或 |$\lambda E - A$| = |$\lambda E - B$|);
              \item 属于 $\lambda_{A}$ 的线性无关的特征向量的个数等于属于 $\lambda_{B}$ 的线性无关的特征向量的个数;
              \item $A, B$  的各阶主子式之和分别相等.
          \end{enumerate}

    \item 若$A$相似于$\Lambda$,$B$相似于$\Lambda$,则$A$相似于$B$.
    \item 若$A$相似于$B$,$B$相似于$\Lambda$,则$A$相似于$\Lambda$.
    \item $A$与$B$的相似手段的“三同一不同”.

          若 $P^{-1}AP = B$, 则 $P^{-1}f(A)P = f(B)$, $P^{-1}A^{-1}P = B^{-1}$, $P^{-1}A^{*}P = B^{*}$, 即 $f(A)$ 与 $f(B)$, $A^{-1}$ 与 $B^{-1}$, $A^{*}$ 与 $B^{*}$ 相似的手段相同, 也即 $P^{-1}[af(A) + bA^{-1} + cA^{*}]P = af(B) + bB^{-1} + cB^{*}$. 但 $A^{T}$ 与 $B^{T}$ 相似的手段与上面不同.
\end{enumerate}
\section{相似对角化的应用}
\DOne+\DTwoTwo

\begin{example}{}{}
    已知数列 $\{x_n\}$, $\{y_n\}$, $\{z_n\}$ 满足 $x_0 = -1$, $y_0 = 0$, $z_0 = 2$, 且
    $$\begin{cases}
            x_n = -2x_{n-1} + 2z_{n-1}, \\
            y_n = -2y_{n-1} - 2z_{n-1}, \\
            z_n = -6x_{n-1} - 3y_{n-1} + 3z_{n-1},
        \end{cases}$$
    记 $\alpha_n = \begin{bmatrix} x_n \\ y_n \\ z_n \end{bmatrix}$, 写出满足 $\alpha_n = A\alpha_{n-1}$ 的矩阵 $A$, 并求 $A^n$ 及 $x_n$, $y_n$, $z_n (n=1,2,\cdots)$.
\end{example}
\begin{solution}
    由题设得$\begin{bmatrix}x_n\\y_n\\z_n\end{bmatrix}=\begin{bmatrix}-2&0&2\\0&-2&-2\\-6&-3&3\end{bmatrix}\begin{bmatrix}x_{n-1}\\y_{n-1}\\z_{n-1}\end{bmatrix}$,得矩阵$A=\begin{bmatrix}-2&0&2\\0&-2&-2\\-6&-3&3\end{bmatrix}$满足$\alpha_n=A\alpha_{n-1}$.

    因为

    $|\lambda E-A|=\begin{vmatrix}\lambda+2&0&-2\\0&\lambda+2&2\\6&3&\lambda-3\end{vmatrix}=\lambda(\lambda-1)(\lambda+2)$,

    所以矩阵$A$的特征值为$\lambda_1=0$,$\lambda_2=1$,$\lambda_3=-2$.

    当$\lambda_1=0$时,解方程组$(0E-A)x=0$,得特征向量$\xi_1=\begin{bmatrix}1&-1&1\end{bmatrix}^{T}$;

    当$\lambda_2=1$时,解方程组$(E-A)x=0$,得特征向量$\xi_2=\begin{bmatrix}2&-2&3\end{bmatrix}^{T}$;

    当$\lambda_3=-2$时,解方程组$(-2E-A)x=0$,得特征向量$\xi_3=\begin{bmatrix}-1&2&0\end{bmatrix}^T$.

    令$P=[\xi_1,\xi_2,\xi_3]=\begin{bmatrix}1&2&-1\\-1&-2&2\\1&3&0\end{bmatrix}$,则$P^{-1}AP=\begin{bmatrix}0&0&0\\0&1&0\\0&0&-2\end{bmatrix}$,即$A=P\begin{bmatrix}0&0&0\\0&1&0\\0&0&-2\end{bmatrix}P^{-1}$,从而得$A^n=P\begin{bmatrix}0&0&0\\0&1&0\\0&0&-2\end{bmatrix}^nP^{-1}=\begin{bmatrix}1&2&-1\\-1&-2&2\\1&3&0\end{bmatrix}\begin{bmatrix}0&0&0\\0&1&0\\0&0&(-2)^n\end{bmatrix}\begin{bmatrix}6&3&-2\\-2&-1&1\\1&1&0\end{bmatrix}$

    $$=\begin{bmatrix}-4-(-2)^n&-2-(-2)^n&2\\4-(-2)^{n+1}&2-(-2)^{n+1}&-2\\-6&-3&3\end{bmatrix}$$.

    由递推式$\alpha_n=A\alpha_{n-1}$知$\alpha_n=A^n\alpha_0$,其中$\alpha_0=\begin{bmatrix}-1&0&2\end{bmatrix}^T$,所以

    $\alpha_n=A^n\alpha_0=\begin{bmatrix}-4-(-2)^n&-2-(-2)^n&2\\4-(-2)^{n+1}&2-(-2)^{n+1}&-2\\-6&-3&3\end{bmatrix}\begin{bmatrix}-1\\0\\2\end{bmatrix}=\begin{bmatrix}8+(-2)^n\\-8+(-2)^{n+1}\\12\end{bmatrix}$,

    故$x_n=8+(-2)^n$,$y_n=-8+(-2)^{n+1}$,$z_n=12(n=1,2,\cdots)$.
\end{solution}

\section{正交矩阵及其使用}
\DOne + \DTwoOne

$\star \star \star$

\begin{enumerate}
    \item 若$A$为正交矩阵,则
          $$A^\top A = E \Leftrightarrow A^{-1} = A^\top$$
          $$\Leftrightarrow A \text{ 由规范正交基组成 }$$
          $$\Leftrightarrow A^{\mathrm{T}}\text{是正交矩阵}$$
          $$\Leftrightarrow A^{-1}\text{是正交矩阵}$$
          $$\Leftrightarrow A^{*}\text{是正交矩阵}$$
          $$\Leftrightarrow-A\text{是正交矩阵.}$$
    \item 若$A,B$为同阶正交矩阵,则$AB$为正交矩阵,但$A+B$不一定为正交矩阵.
    \item ${\text{若 }A\text{ 为正交矩阵,则其实特征值的取值范围为}\{-1,1\}}.$
    \item 设$A$为$n$阶非零矩阵,$\begin{cases}\text{若}a_{ij}=A_{ij},\text{则}A^{\mathrm{T}}=A^{*},AA^{\mathrm{T}}=E,\text{且}\mid A\mid=1;\\\text{若}a_{ij}=-A_{ij},\text{则}A^{\mathrm{T}}=-A^*,AA^{\mathrm{T}}=E,\text{且}\mid A\mid=-1.\end{cases}$
\end{enumerate}
\chapter{二次型}
\section{$f=x^TAx$ 中$A$的表示}
\DOne+\DTwoThree

\begin{enumerate}
    \item 给出非对称矩阵 $B$,令$A=\frac{B+B^T}{2}$,则$A=A^\mathrm{T}.$
    \item 通过题设或基本变形显化出 $A.$
\end{enumerate}

\section{配方法与正交变换法的异同}

\begin{enumerate}
    \item 命题语言
          \DTwoTwo
          \begin{enumerate}
              \item 配方法

                    二次型语言:将 $f = x^T A x$ 通过配方法化为标准形,并求出可逆变换矩阵 $C$.

                    矩阵语言:求可逆矩阵 $C$,使得 $C^T A C = \Lambda$.
              \item 正交变换法

                    二次型语言:将 $f = x^T A x$ 通过正交变换法化为标准形,并求出正交矩阵 $Q$.

                    矩阵语言:求正交矩阵 $Q$,使得 $Q^T A Q = \Lambda$.
          \end{enumerate}
    \item 过程与结果的异同
          \DTwoThree

          设$f(x)=x^TAx$.
          \begin{enumerate}
              \item 配方法(可逆线性变换)

                    $x=Cy$,$C$可逆.使得$f\xlongequal{x=\boldsymbol{C}y}y^T\Lambda y$,其中$C^TAC=\Lambda$(使$A$合同于对角矩阵).
              \item 正交变换法(可逆线性变换):

                    $x=Qy$(这里的$Q$不仅可逆,还满足$Q^{-1}=Q^{T}$),使得$f\xlongequal{x=\boldsymbol{Q}y}y^T\Lambda y$,其中$Q^{T}AQ=Q^{-1}AQ=A$.
          \end{enumerate}
          二者区别:在配方法中,$c$只满足可逆,所以$c^{-1}$不一定等于$c^T$,但是在正交变换法中,变换手段$Q$满足$Q^{- 1}= Q^T$ .

          二者相同点:它们的正、负惯性指数是对应相等的.
    \item 惯性指数
          \begin{example}{}{}
              $f(x_{1},x_{2},x_{3})=-2x_{1}x_{2}-2x_{1}x_{3}+6x_{2}x_{3}$的正惯性指数为(  ).
          \end{example}
          \begin{solution}
              令$\begin{cases}x_{1}=y_{1}+y_{2},\\x_{2}=y_{1}-y_{2},\end{cases}$则

              $$f=-2y_{1}^{2}+2y_{2}^{2}+4y_{1}y_{3}-8y_{2}y_{3}
                  =-2(y_{1}-y_{3})^{2}+2(y_{2}-2y_{3})^{2}-6y_{3}^{2},$$

              再令$\begin{cases}z_{1}=y_{1}-y_{3},\\z_{2}=y_{2}-2y_{3},\end{cases}$则

              $$f=-2z_{1}^{2}+2z_{2}^{2}-6z_{3}^{2},$$

              故$f$的正惯性指数为1.
          \end{solution}
\end{enumerate}

\section{伪配方法}
\DTwoThree

“平方和式$A^2+B^2+C^2$”未必就是(拉格朗日)配方法得来的结果,故若非拉格朗日配方法,则称伪配方法.要注意伪配方法的变换矩阵是否有可逆性.
\begin{enumerate}
    \item 如果变换没有可逆性,则有可能改变表达式的几何性质,如封闭性,此时不能得出平方和式正定;
    \item 如果变换是可逆的,则平方和式正定.
\end{enumerate}


\begin{note}{}{}
    对于$f(x_{1},x_{2},x_{3})=(a_{1}x_{1}+a_{2}x_{2}+a_{3}x_{3})^{2}+(b_{1}x_{1}+b_{2}x_{2}+b_{3}x_{3})^{2}+(c_{1}x_{1}+c_{2}x_{2}+c_{3}x_{3})^{2}$的情形,可总结如下做题方法:

    令$f=0$,即$\begin{cases}a_{1}x_{1}+a_{2}x_{2}+a_{3}x_{3}=0,\\b_{1}x_{1}+b_{2}x_{2}+b_{3}x_{3}=0,\\c_{1}x_{1}+c_{2}x_{2}+c_{3}x_{3}=0,\end{cases}$计算$|\boldsymbol{A}|=\begin{vmatrix}a_{1}&a_{2}&a_{3}\\b_{1}&b_{2}&b_{3}\\c_{1}&c_{2}&c_{3}\end{vmatrix}$,若$|\boldsymbol{A}|\neq0$,则$f$正定;若$|\boldsymbol{A}|=0$,则$f$不正定.
\end{note}
\section{正交变换法的传递性}
\DOne+\DTwoThree

若$A$相似于$B$,则$B$相似于$C$,则$A$相似于$C$.这里$B$常为$\Lambda$.

\section{合同的判定与手段}
\DOne+\DTwoThree

\begin{enumerate}
    \item 同阶实对称矩阵$A,B$合同的判定

          用正、负惯性指数:$A,B$合同$\Leftrightarrow p_A=p_B,q_A=q_B$(相同的正、负惯性指数).
    \item 已知$A$,$\Lambda$($\Lambda$是对角矩阵),求可逆矩阵$C$,使得$C^TAC=\Lambda$
    \item 已知$A$,$B$($B$不是对角矩阵),求可逆矩阵$C$,使得$C^TAC=B$
\end{enumerate}
\begin{idea}{求可逆矩阵$C$,使得$C^TAC=\Lambda$}{}
    \begin{enumerate}
        \item 配方 盯着$\Lambda$的对角线元素,提出对应系数
        \item 换元
        \item 求逆
    \end{enumerate}
\end{idea}
\begin{idea}{求可逆矩阵$C$,使得$C^TAC=B$}{}
    \begin{enumerate}
        \item 对$f$配方、换元,写$D_1$
        \item 对$g$配方、换元,写$D_2$
        \item 令$D_1x=D_2y$,求$D_2^{-1}D_1$
    \end{enumerate}
\end{idea}
\section{合同与相似的异同}
\DOne+\DTwoThree

对于实对称矩阵$A$与$B$,相似必合同,反之不成立.
\begin{example}{合同与相似的异同}{合同与相似的异同}
    已知二次型
    $$f(x_{1}, x_{2}, x_{3}) = x_{1}^{2} + 2x_{2}^{2} + 2x_{3}^{2} + 2x_{1}x_{2} - 2x_{1}x_{3}$$
    $$
        g(y_{1}, y_{2}, y_{3}) = y_{1}^{2} + y_{2}^{2} + y_{3}^{2} + 2y_{2}y_{3}$$
    \begin{enumerate}
        \item 求可逆变换 $x = Py$,将 $f(x_{1}, x_{2}, x_{3})$ 化成 $g(y_{1}, y_{2}, y_{3})$.
        \item 是否存在正交变换 $x = Qy$,将 $f(x_{1}, x_{2}, x_{3})$ 化成 $g(y_{1}, y_{2}, y_{3})$?
    \end{enumerate}
\end{example}
\begin{idea}{解题思路 \ref{ex:合同与相似的异同}}{}
    \begin{enumerate}
        \item 求可逆变换用配方法
        \item 判断是否存在正交变换,如果存在必相似,使用相似的充分条件和充要条件
    \end{enumerate}
\end{idea}
\section{正定的判定与应用}
\DOne+\DTwoThree

$\star\star\star$

\begin{enumerate}
    \item 前提
          $A=A^T$($A$是实对称矩阵)
    \item 二次型$f=x^TAx$正定的充要条件 \DTwo

          $n$元二次型$f=x^{T}Ax$正定

          $\Leftrightarrow$对任意的$x\neq 0$,有$x^{T}Ax>0$(定义)

          $\Leftrightarrow A$的特征值$\lambda_{i}>0(i=1,2,\cdots,n)$

          $\Leftrightarrow f$的正惯性指数$p=n$

          $\Leftrightarrow$存在可逆矩阵$D$,使得$A=D^{T}D$

          $\Leftrightarrow A$与$E$合同

          $\Leftrightarrow A$的各阶顺序主子式均大于0.
    \item 二次型$f=x^TAx$正定的必要条件
          \begin{enumerate}
              \item $a_{ii}>0\left(i=1,2,\cdots,n\right).$
              \item $| A| > 0$.
          \end{enumerate}
    \item 重要结论
          \begin{enumerate}
              \item 若$A$正定,则$A^-1,A^{*},A^{m}(m$为正整数$),kA(k>0),C^{\mathrm{T}}AC(C$可逆 )均正定 .

              \item 若$A,B$正定,则$A+B$正定,$\begin{bmatrix}A&O\\O&B\end{bmatrix}$正定.

              \item ${\text{若}A,B}$正定,则$AB$正定的充要条件是$AB= BA$ .
          \end{enumerate}

\end{enumerate}

\section{二次型的最值}
\DOne+\DTwoTwo+\DTwoThree



\GAIchapter{随机事件和概率}
\GAIchapter{一维随机变量及其分布}

\section{判分布}
\DOne + \DTwoTwo
\begin{enumerate}
      \item 判分布函数
            \begin{enumerate}
                  \item 充要条件

                        $F(x)$ 是分布函数 $\Leftrightarrow F(x)$ 是 $x$ 的单调不减且右连续的函数,且 $F(-\infty)=0$,$F(+\infty)=1$.
                  \item 分布函数形式大观.
                        \begin{enumerate}
                              \item 设 $F_i(x)$ 是分布函数,$\lambda_i>0$,$\sum_{i=1}^{n}\lambda_i=1$,则 $\sum_{i=1}^{n}\lambda_iF_i(x)$ 是分布函数.特别地,算术平均值 $\frac{F_1(x)+F_2(x)}{2}$ 是分布函数.
                              \item 设 $F(x)$ 是分布函数,则 $F(x)$ 在 $[x,x+T]$ $(T>0)$ 上的均值 $\frac{1}{T}\int_{x}^{x+T}F(t)dt$ 是分布函数.可见,线性组合 $\sum_{i=1}^{n}\lambda_iF_i(x)$ 及其连续形式均仍是分布函数.
                              \item 几何平均值 $\sqrt{F_1(x)F_2(x)}$ 是分布函数.
                              \item $[F(x)]^n$,$1-[1-F(x)]^n$ 是分布函数.
                        \end{enumerate}
            \end{enumerate}
      \item 判分布律的充要条件

            $\{p_i\}$ 是概率分布 $\Leftrightarrow p_i\geqslant 0$,且 $\sum_{i}p_i=1$.
      \item 判概率密度
            \begin{enumerate}
                  \item 充要条件

                        $f(x)$ 是概率密度 $\Leftrightarrow f(x) \geqslant 0$,且 $\int_{-\infty}^{+\infty} f(x) \mathrm{d}x = 1$.
                  \item 概率密度形式大观
                        \begin{enumerate}
                              \item 设 $f(x)$ 为概率密度,$\lambda_i > 0$,$\sum_{i=1}^{n} \lambda_i = 1$,则 $\sum_{i=1}^{n} \lambda_i f_i(x)$ 是概率密度.特别地,$\frac{1}{2}[f_1(x) + f_2(x)]$ 是概率密度.
                              \item 设 $f(x)$ 为概率密度,则 $f(x)$ 在 $[x, x+T]$($T > 0$)上的均值 $\frac{1}{T} \int_{x}^{x+T} f(t) \mathrm{d}t$ 是概率密度.
                              \item 设 $X_i$ 的分布函数为 $F_i(x)$,概率密度为 $f_i(x)$,则 $\frac{2}{n}\sum_{i=1}^{n}F_i(x)f_i(x)$ 是概率密度.
                              \item 设 $X_i$ 的分布函数为 $F_i(x)$,概率密度为 $f_i(x)$,则 $f_1(x)F_2(x)\cdots F_n(x) + F_1(x)f_2(x)\cdots F_n(x) + \cdots + F_1(x)F_2(x)\cdots f_n(x)$ 是概率密度.
                              \item 设 $F(x)$ 是分布函数,$f(x)$ 是对应的概率密度,则 $n[F(x)]^{n-1}f(x)$,$n[1-F(x)]^{n-1}f(x)$ 是概率密度.
                        \end{enumerate}
            \end{enumerate}
      \item 反问题
            $$\begin{cases}F(-\infty)=0,\\F(+\infty)=1,\\\sum_{i}p_{i}=1,\\\int_{-\infty}^{+\infty}f(x)\mathrm{d}x=1 \end{cases}$$
            建方程,求参数
\end{enumerate}
\section{用分布}
\DOne

\begin{enumerate}
      \item 离散型分布
            \begin{enumerate}
                  \item 0-1分布.

                        $X \sim B(1,p)$, $X$ (伯努利计数变量) $\sim \begin{pmatrix} 1 & 0 \\ p & 1-p \end{pmatrix}$.
                  \item 二项分布.

                        $X \sim B(n,p) \begin{cases} n\text{次试验相互独立;} \\ P(A) = p; \\ \text{只有}A, \overline{A}\text{两种结果}. \end{cases}$
                        记 $X$ 为 $A$ 发生的次数, 则$P\{X = k\} = C_n^k p^k (1-p)^{n-k}$, $k = 0, 1, 2, \cdots, n$,$EX = np$, $DX = np(1-p)$.

                        二项分布 $X \sim B(n,p)$ 还具有如下性质:
                        \begin{enumerate}
                              \item $Y = n - X$ 服从二项分布 $B(n,q)$, 其中 $q = 1 - p$.
                              \item 对固定的 $n$ 和 $p$,随着 $k$ 的增大,$P\{X=k\}$ 先上升到最大值而后下降
                                    \begin{enumerate}
                                          \item 当 $(n+1)p$ 为整数时,$P\{X=(n+1)p\} = P\{X=(n+1)p-1\}$ 最大.
                                          \item 当 $(n+1)p$ 不为整数时,$P\{X=\left[(n+1)p\right]\}$ 最大,其中 $\left[(n+1)p\right]$ 表示 $(n+1)p$ 的整数部分.
                                    \end{enumerate}
                        \end{enumerate}
                  \item 离散型首次冲击分布(几何分布).

                        在伯努利试验序列中 $P(A)=p,P(\bar{A})=1-p$, 首次出现 $A$ 即停止(即首次冲击即失效).令$X$为试
                        验次数,则$P\{X=k\}=p(1-p)^{k-1},k=1,2,\cdots$,其中$P\left\{X=1\right\}$最大,且$EX=\frac1p,DX=\frac{1-p}{p^{2}}.$
                  \item 超几何分布
                        $N$件产品中有$M$件正品, 从中无放回地随机抽取$n$件, 则取到$k$个正品的概率为

                        $P\{X=k\}=\frac{\mathrm{C}_{M}^{k} \mathrm{C}_{N-M}^{n-k}}{\mathrm{C}_{N}^{n}}$, $k$为整数, $\max \{0, n-N+M\} \leqslant k \leqslant \min \{n, M\}$, 且 $E X=\frac{n M}{N}$.
                  \item 泊松分布

                        泊松分布是指某单位时间段, 某场合下, 源源不断的随机质点流的个数, 也常用于描述稀有事件的概率.

                        $$P\{X=k\}=\frac{\lambda^{k}}{k !} \mathrm{e}^{-\lambda}(k=0,1, \cdots ; \lambda>0),$$

                        $\lambda$ 表示强度 $(E X=\lambda)$, 且 $P\{X=[\lambda]\}$ 最大, 其中 $[\lambda]$ 表示对 $\lambda$ 取整.
            \end{enumerate}

      \item 连续型分布
            \begin{enumerate}
                  \item 均匀分布$U(a,b)$.
                        如果随机变量X的概率密度和分布函数分别为
                        $$f(x)=\begin{cases}
                                    \frac{1}{b-a}, & a<x<b,     \\
                                    0,             & \text{其他},
                              \end{cases}$$
                        $$F(x)=\begin{cases}
                                    0,               & x<a,            \\
                                    \frac{x-a}{b-a}, & a\leqslant x<b, \\
                                    1,               & x\geqslant b,
                              \end{cases}$$
                        则称X在区间(a,b)上服从均匀分布,记为$X\sim U(a,b)$.
                  \item 连续型首次冲击分布(指数分布).

                        设随机质点流的计数过程为 $\{N_{t}\}(t\geqslant 0)$,$N_{t}$ 服从参数为 $\lambda t$ 的泊松分布.令 $T_{1}$ 表示第 1 个质点到来的时刻,则当 $t>0$ 时,令 $A=\{T_{1}>t\}$ 表示第 1 个质点在时刻 $t$ 之后到来,$B=\{N_{t}=0\}$ 表示在 $[0,t]$ 时间上有 0 个质点到来,即 $A$ 与 $B$ 是相等事件,故 $P(A)=P(B)$,即
                        $$P\{T_{1}>t\}=P\{N_{t}=0\}=\frac{(\lambda t)^{0}}{0!}e^{-\lambda t}=e^{-\lambda t},$$
                        于是
                        $$
                              F_{T_{1}}(t)=P\{T_{1}\leqslant t\}=1-e^{-\lambda t}, t>0,$$
                        即 $T_{1}$ 服从参数为 $\lambda$ 的指数分布。
                        如果 $X$ 的概率密度和分布函数分别为
                        $$f(x)=\begin{cases}
                                    \lambda e^{-\lambda x}, & x\geqslant 0, (\lambda>0), \\
                                    0,                      & \text{其他}
                              \end{cases}, F(x)=\begin{cases}
                                    1-e^{-\lambda x}, & x\geqslant 0, (\lambda>0), \\
                                    0,                & x<0
                              \end{cases},$$
                        则称 $X$ 服从参数为 $\lambda$ 的指数分布,记为 $X\sim E(\lambda)$.
                        \begin{note}{}{}
                              \begin{enumerate}
                                    \item 当$t,s > 0$时, $P\{X\geq t+s|X\geq t\}=P\{X\geq s\}$称为指数分布的无记忆性.
                                    \item $EX=\frac{1}{\lambda}$称为平均寿命, 也称为平均等待时间, $\lambda$称为失效频率, 它是一个常数, 只有失效频率不变, 元件无损耗, 才有无记忆性.
                                    \item 特别地, 当λ=$\frac{1}{2}$, 即X~f(x)=$\begin{cases}\frac{1}{2}e^{-\frac{x}{2}}, & x≥0\\0, & x<0\end{cases}$时, 也称$X$服从自由度为$2$的$\chi^2$ 分布, 故$E(\frac{1}{2})$与$\chi^2(2)$ 是同一分布.
                                    \item 若$X\sim E(1)$, 则$2X~E(\frac{1}{2})$, $2X\sim \chi^2(2)$ .
                                    \item 若$X\sim E(\lambda)$, 则$2\lambda X~E(\frac{1}{2})$, $2\lambda X \sim\chi^2(2)$ .
                              \end{enumerate}
                        \end{note}
                  \item 自由度为$1$的$t$分布(标准柯西分布).

                        若
                        $$X\sim f(x)=\frac{1}{\pi(1+x^{2})}\:,\:-\infty<x<+\infty\:,$$
                        则称$X$服从自由度为1的$t$分布(标准柯西分布),即$X\sim t(1)$这是用于描述受迫共振的一种分布.
                  \item 正态分布

                        若$X\sim f(x)=\frac{1}{\sqrt{2\pi}\sigma}\mathrm{e}^{-\frac{(x-\mu)^2}{2\sigma^2}}$,$-\infty<x<+\infty$,其中$-\infty<\mu<+\infty$,$\sigma>0$,则称$X$服从参数为$(\mu,\sigma^2)$的正态分布,记为$X\sim N(\mu,\sigma^2)$.
                        \begin{note}{}{}
                              \begin{enumerate}
                                    \item  $\mu=0$,$\sigma=1$时的正态分布为标准正态分布,记为$X\sim N(0,1)$.
                                          $$X\sim\varphi(x)=\frac{1}{\sqrt{2\pi}}\mathrm{e}^{-\frac{x^2}{2}}, \quad \Phi(x)=\int_{-\infty}^x\frac{1}{\sqrt{2\pi}}\mathrm{e}^{-\frac{t^2}{2}}\mathrm{d}t,$$
                                          且
                                          $$Y=\mid X\mid\sim f_{\gamma}(y)=\begin{cases}\dfrac{2}{\sqrt{2\pi}}\mathrm{e}^{-\frac{y^{2}}{2}},&y>0,\\0,&y\leqslant0\end{cases}=\begin{cases}2\varphi(y),&y>0,\\0,&y\leqslant0.\end{cases}$$
                                    \item 计算公式与重要数据.

                                          若$X\sim N(0,1)$,则有
                                          $$
                                                \Phi(-x)=1-\Phi(x); \Phi(0)=\frac{1}{2};$$
                                          $$P\{|X|\leqslant a\}=2\Phi(a)-1(a>0).$$

                                    \item 标准化.

                                          若$X\sim N(\mu,\sigma^2)$,则
                                          $$
                                                \frac{X-\mu}{\sigma}\sim N(0,1),$$
                                          且有
                                          $$F(x)=P\{X\leqslant x\}=\Phi\left(\frac{x-\mu}{\sigma}\right),$$
                                          $$
                                                P\{a\leqslant X\leqslant b\}=\Phi\left(\frac{b-\mu}{\sigma}\right)-\Phi\left(\frac{a-\mu}{\sigma}\right),$$
                                          $$P\{\mu-k\sigma\leqslant X\leqslant\mu+k\sigma\}=2\Phi(k)-1(k>0).$$
                                    \item 含参数的概率密度的结构.

                                          设函数$f(x)=k\mathrm{e}^{-(ax^2+bx+c)}$, $x\in(-\infty,+\infty)(a>0)$,则
                                          $$
                                                ax^2+bx+c=a\left[\left(x+\frac{b}{2a}\right)^2+\frac{4ac-b^2}{4a^2}\right],$$
                                          且$k=\sqrt{\frac{a}{\pi}}\mathrm{e}^{\frac{4ac-b^2}{4a}}$,如$f(x)=k\mathrm{e}^{-\left(\frac{x^2}{4}+\frac{x}{2}+\frac{1}{4}\right)}$,则
                                          $$\frac{x^2}{4}+\frac{x}{2}+\frac{1}{4}=\frac{1}{4}\left[\left(x+\frac{1}{2}\right)^2+\frac{4\cdot\frac{1}{4}\cdot\frac{1}{4}-\left(\frac{1}{2}\right)^2}{4\cdot\left(\frac{1}{4}\right)^2}\right]$$
                                          $$
                                                =\frac{1}{4}(x+1)^2,$$
                                          且$k=\sqrt{\frac{1}{4\pi}}\mathrm{e}^0=\frac{1}{2\sqrt{\pi}}$.
                              \end{enumerate}
                        \end{note}
            \end{enumerate}
      \item 利用分布求概率及反问题
            \begin{enumerate}
                  \item  $X \sim F(x)$,则
                        \begin{enumerate}
                              \item  $P\{X \leqslant a\} = F(a)$;
                              \item  $P\{X < a\} = F(a-0)$;
                              \item  $P\{X = a\} = P\{X \leqslant a\} - P\{X < a\} = F(a) - F(a-0)$;
                              \item  $P\{a < X < b\} = P\{X < b\} - P\{X \leqslant a\} = F(b-0) - F(a)$;
                              \item  $P\{a \leqslant X \leqslant b\} = P\{X \leqslant b\} - P\{X < a\} = F(b) - F(a-0)$.
                        \end{enumerate}
                  \item $X \sim p_{i}$,则
                        $$P\{X \in I\} = \sum_{x_{i} \in I} P\{X = x_{i}\}$$
                  \item $X \sim f(x)$,则
                        $$P\{X \in I\} = \int_{I} f(x) \, dx$$
                  \item 反问题:已知概率反求参数.
            \end{enumerate}
\end{enumerate}

\section{求分布}
\DOne

根据题设条件,建立$F(x) = P\{X\leq x\}$并计算此概率.
\GAIchapter{一维随机变量函数的分布}

\section{离散型$\rightarrow$离散型}
设离散型随机变量 $X$ 的分布为 $P\{X = x_i\} = p_i \ (i=1,2,\cdots)$,若 $Y = g(X)$,则 $Y$ 仍为离散型随机变量,其分布为
$$
      Y \sim
      \begin{pmatrix}
            g(x_1) & g(x_2) & \cdots \\
            p_1    & p_2    & \cdots
      \end{pmatrix}.
$$
若若干个 $g(x_k)$ 取相同值,则合并为一项,并将对应概率相加。

\section{连续型$\rightarrow$连续型(或混合型)}
设连续型随机变量 $X$ 的分布函数与密度分别为 $F_X(x)$、$f_X(x)$,若 $Y = g(X)$,则可用以下两种方法求其分布:

\subsection*{(1) 分布函数法}
由定义直接求:
$$
      F_Y(y) = P\{Y \le y\} = P\{g(X) \le y\} = \int_{g(x) \le y} f_X(x) \, dx.
$$
若 $F_Y(y)$ 连续且可导,则 $f_Y(y) = F_Y'(y)$。

\subsection*{(2) 公式法(单调可导变换)}
若 $y = g(x)$ 在 $(a,b)$ 上严格单调且可导,则存在反函数 $x = h(y)$,其概率密度为
$$
      f_Y(y) = f_X[h(y)] \cdot |h'(y)|, \quad \alpha < y < \beta,
$$
其中
$$
      \alpha = \min\{\lim_{x\to a^+} g(x), \lim_{x\to b^-} g(x)\}, \quad
      \beta = \max\{\lim_{x\to a^+} g(x), \lim_{x\to b^-} g(x)\}.
$$

\section{连续型$\rightarrow$离散型}
若 $X \sim f_X(x)$ 且 $Y = g(X)$ 为离散型变量,先求出 $Y$ 的可能取值 $a_i$,再由
$$
      P\{Y = a_i\} = \int_{g(x)=a_i} f_X(x) \, dx
$$
得出其分布。
\section{两种重要的随机变量变换}

\subsection*{(1) 变换于 $U(0,1)$}
\begin{example}{}{}
      设随机变量 $X$ 的分布函数 $F_X(x)$ 严格单调递增,反函数 $F_X^{-1}(y)$ 存在,令 $Y = F_X(X)$,则 $Y \sim U(0,1)$。
\end{example}
\begin{proof}
      由定义:
      $$
            F_Y(y) = P\{Y \le y\} = P\{F_X(X) \le y\} = P\{X \le F_X^{-1}(y)\} = F_X[F_X^{-1}(y)] = y,
      $$
      对 $0 \le y < 1$ 成立;此外,$F_Y(y) = 0 (y < 0)$,$F_Y(y) = 1 (y \ge 1)$,即
      $$
            F_Y(y) =
            \begin{cases}
                  0, & y < 0,       \\
                  y, & 0 \le y < 1, \\
                  1, & y \ge 1,
            \end{cases}
      $$
      故 $Y \sim U(0,1)$。
\end{proof}


\subsection*{(2) 变换于 $E(1)$}
\begin{example}{}{}
      设 $X$ 的分布函数 $F_X(x)$ 连续且在其密度区间上严格单调,令
      $$
            Y = -\ln[1 - F_X(X)],
      $$
      则 $Y \sim E(1)$。
\end{example}
\begin{proof}
      由定义:
      $$
            P\{Y \le y\} = P\{-\ln[1 - F_X(X)] \le y\} = P\{F_X(X) \le 1 - e^{-y}\}.
      $$
      由于 $F_X(X) \sim U(0,1)$,故
      $$
            F_Y(y) = 1 - e^{-y}, \quad y > 0,
      $$
      即 $Y \sim E(1)$。
\end{proof}
\chapter{多维随机变量及其分布}

\section{离散型问题}
一般不会考

\section{连续型问题}
\begin{enumerate}
    \item 二维均匀分布

          如果$(X,Y)$的概率密度为
          $$f(x,y)=\begin{cases}\dfrac{1}{S_D},&(x,y)\in D,\\0,&\text{其他,}\end{cases}$$
          其中$S_{_{D}}$为区域$D$ 的面积,则称$(X,Y)$在平面有界区域$D$ 上服从均匀分布.

    \item 二维正态分布

          如果(X,Y)的概率密度为
          $$f(x,y)=\frac{1}{2\pi\sigma_1\sigma_2\sqrt{1-\rho^2}}\exp\left\{-\frac{1}{2(1-\rho^2)}\left[\left(\frac{x-\mu_1}{\sigma_1}\right)^2-2\rho\left(\frac{x-\mu_1}{\sigma_1}\right)\left(\frac{y-\mu_2}{\sigma_2}\right)+\left(\frac{y-\mu_2}{\sigma_2}\right)^2\right]\right\},$$

          其中$\mu_1\in R$,$\mu_2\in R$,$\sigma_1>0$,$\sigma_2>0$,$-1<\rho<1$,则称(X,Y)服从参数为$\mu_1$,$\mu_2$,$\sigma_1^2$,$\sigma_2^2$,$\rho$的二维正态分布,记为$(X,Y)\sim N(\mu_1,\mu_2;\sigma_1^2,\sigma_2^2;\rho)$。

\end{enumerate}
\section{求边缘分布、条件分布与独立性问题}
\begin{enumerate}
    \item 边缘分布
          \begin{enumerate}
              \item 求$F_X(x),F_Y(y)$.
                    $$F_{X}(x)=F(x,+\infty) ,\quad F_{Y}(y)=F(+\infty,y).$$
              \item 求$p_{i\cdot},p_{\cdot j}$.
                    $$p_{i\cdot}=\sum_{j}p_{ij},\quad p_{\cdot j}=\sum_{i}p_{ij}.$$
              \item 求$f_X(x),f_Y(y)$.
                    $$f_{X}(x)=\int_{-\infty}^{+\infty}f(x,y)\mathrm{d}y=\int_{-\infty}^{+\infty}f_{Y}(y)f_{X|Y}(x\mid y)\mathrm{d}y,$$
                    $$f_{Y}(y)=\int_{-\infty}^{+\infty}f(x,y)\mathrm{d}x=\int_{-\infty}^{+\infty}f_{X}(x)f_{Y|X}(y\mid x)\mathrm{d}x.$$
          \end{enumerate}
    \item 条件分布
          \begin{enumerate}
              \item 求 $F(x\mid y_j),F(y\mid x_i).$
                    $$F(x\mid y_{j})=\sum_{x_{i}\leq x}P\{X=x_{i}\mid Y=y_{j}\},$$
                    $$F(y\mid x_{i})=\sum_{y_{j}\leq y}P\{Y=y_{j}\mid X=x_{i}\}.$$
              \item 求 $F(x\mid y),F(y\mid x).$
                    $$F(x\mid y)=\int_{-\infty}^{x}f(u\mid y)\mathrm{d}u=\int_{-\infty}^{x}\frac{f(u,y)}{f_{Y}(y)}\mathrm{d}u,$$
                    $$F(y\mid x)=\int_{-\infty}^{y}f(v\mid x)\mathrm{d}v=\int_{-\infty}^{y}\frac{f(x,v)}{f_{X}(x)}\mathrm{d}v.$$
              \item 求$P\{ Y= y_{j}\mid X= x_{i}\}$ , $P\{ X= x_{i}\mid Y= y_{j}\}$ .
                    $$P\{Y=y_{j}\mid X=x_{i}\}=\frac{P\{X=x_{i},Y=y_{j}\}}{P\{X=x_{i}\}}=\frac{p_{ij}}{p_{i}.}\:,$$
                    $$P\{X=x_{i}\mid Y=y_{j}\}=\frac{P\Big\{X=x_{i},Y=y_{j}\Big\}}{P\Big\{Y=y_{j}\Big\}}=\frac{p_{ij}}{p_{\cdot j}}\:.$$
              \item 求$f_{Y\mid X}(y\mid x),f_{X\mid Y}(x\mid y).$
                    $$f_{Y\mid X}(y\mid x)=\frac{f(x,y)}{f_{X}(x)},\quad f_{X|Y}(x\mid y)=\frac{f(x,y)}{f_{Y}(y)}$$
          \end{enumerate}
    \item 判独立
          \begin{enumerate}
              \item $X$与$Y$相互独立 $\Leftrightarrow$ 对任意$x,y$,$F(x,y)=F_X(x)\cdot F_Y(y)$.

                    $X,Y$不独立 $\Leftrightarrow$ 存在$x_0,y_0$,使$A=\{X\leqslant x_0\}$与$B=\{Y\leqslant y_0\}$不独立,即$F(x_0,y_0)\neq F_X(x_0)\cdot F_Y(y_0)$.

                    因此,证明不独立的常用方法:找$x_0,y_0$,使$0<P\{X\leqslant x_0\}$,$P\{Y\leqslant y_0\}<1$,$\{X\leqslant x_0\}\subseteq\{Y\leqslant y_0\}$或$\{Y\leqslant y_0\}\subseteq\{X\leqslant x_0\}$或$\{X\leqslant x_0,Y\leqslant y_0\}=\varnothing$.
              \item 若$(X,Y)$为二维离散型随机变量,$X$与$Y$相互独立 $\Leftrightarrow$ 对任意$i,j$,$p_{ij}=p_i\cdot p_j$.
              \item 若$(X,Y)$为二维连续型随机变量,$X$与$Y$相互独立 $\Leftrightarrow$ 对任意$x,y$,$f(x,y)=f_X(x)f_Y(y)$.
          \end{enumerate}
\end{enumerate}

\section{用分布求概率及反问题}
\begin{enumerate}
    \item $(X,Y)\sim p_{ij}$,则$P\{ ( X, Y) \in D\} = \sum_{( x_{i}, y_{j}) \in D}p_{ij}$.
    \item $(X,Y)\sim f(x,y)$,则$P\{(X,Y)\in D\}=\iint_{D}f(x,y)$d$x$d$y$ .
    \item $(X,Y)$为混合型,则用全概率公式.
    \item 反问题:已知概率反求参数.
\end{enumerate}
\chapter{多维随机变量函数的分布}
\chapter{数字特征}

计算数字特征、判别独立与不相关、用切比雪夫不等式做概率计算

\section{数学期望}
数学期望就是随机变量的取值与取值的概率乘积的和.
\begin{enumerate}
      \item $X$
            \begin{enumerate}
                  \item $X\sim p_{i}\Rightarrow EX=\sum_{i}x_{i}p_{i}\begin{cases}\text{有限项相加,}\\\text{无穷项相加(无穷级数).}\end{cases}$
                  \item $X\sim f(x)\Rightarrow EX=\int_{-\infty}^{+\infty}xf(x)$d$x\begin{cases}\text{有限区间积分(定积分),}\\\text{无穷区间积分(反常积分).}\end{cases}$
            \end{enumerate}
      \item $g(X)$

            g为连续函数(或分段连续函数).
            \begin{enumerate}
                  \item $X\sim p_{i},Y=g(X)\Rightarrow EY=\sum_{i}g(x_{i})p_{i}.$
                  \item $X\sim f( x)$ , $Y= g( X) \Rightarrow EY= \int _{- \infty }^{+ \infty }g( x) f( x)dx$.
            \end{enumerate}
      \item $g(X,Y)$
            \begin{enumerate}
                  \item $( X, Y) \sim p_{ij} , Z= g( X, Y) \Rightarrow EZ= \sum _{i}\sum _{j}g( x_{i}, y_{j}) p_{ij}$.
                  \item $( X, Y) \sim f(x,y) , Z= g( X, Y) \Rightarrow EZ= \int _{- \infty }^{+ \infty }\int _{- \infty }^{+ \infty }g( x, y) f( x, y)dxdy$ .
            \end{enumerate}
      \item 最值
            \begin{enumerate}
                  \item 若 $X_i(i=1,2,\cdots,n;n\geqslant 2)$ 独立同分布,其分布函数为 $F(x)$,概率密度为 $f(x)$,记
                        $$Y = \min\{X_1,X_2,\cdots,X_n\},\ Z = \max\{X_1,X_2,\cdots,X_n\},$$
                        则
                        \begin{enumerate}
                              \item $F_Y(y) = 1 - [1 - F(y)]^n,\ f_Y(y) = n[1 - F(y)]^{n-1}f(y) \Rightarrow EY = \int_{-\infty}^{+\infty} yf_Y(y)dy;$
                              \item $F_Z(z) = [F(z)]^n,\ f_Z(z) = n[F(z)]^{n-1}f(z) \Rightarrow EZ = \int_{-\infty}^{+\infty} zf_Z(z)dz.$
                        \end{enumerate}
                  \item 用好转化公式:
                        $$\max\{X,Y\} = \frac{X+Y+|X-Y|}{2};\quad\min\{X,Y\} = \frac{X+Y-|X-Y|}{2};$$
                        $$\max\{X,Y\} + \min\{X,Y\} = X+Y;$$
                        $$\max\{X,Y\} - \min\{X,Y\} = |X-Y|;\quad\max\{X,Y\} \cdot \min\{X,Y\} = XY.$$
                  \item 用好降维法,令 $Z = X-Y$.
                  \item 用好标准化,令 $U = \frac{X-\mu}{\sigma}$.
            \end{enumerate}
      \item 分解

            若 $X = X_{1} + X_{2} + \cdots + X_{n}$,则 $EX = EX_{1} + EX_{2} + \cdots + EX_{n}$.
      \item 性质
            \begin{enumerate}
                  \item $Ea=a$, $E(EX)=EX$.
                  \item $E(aX+bY)=aEX+bEY$, $E(\sum_{i=1}^{n}a_iX_i)=\sum_{i=1}^{n}a_iEX_i$.
                  \item 若 $X$,$Y$ 相互独立,则 $E(XY)=EXEY$.
            \end{enumerate}
\end{enumerate}
\section{方差}
\begin{enumerate}
      \item X
            \begin{enumerate}
                  \item 定义

                        $DX = E[(X - EX)^2]$,$X$的方差就是$Y = (X - EX)^2$的数学期望.
                  \item 定义法.
                        $$X \sim p_i \Rightarrow DX = E[(X - EX)^2] = \sum_i (x_i - EX)^2 p_i,$$
                        $$X \sim f(x) \Rightarrow DX = E[(X - EX)^2] = \int_{-\infty}^{+\infty} (x - EX)^2 f(x) dx.$$
                  \item 公式法.

                        $DX = E(X^2) - (EX)^2$.
            \end{enumerate}
      \item 最值的方差
            $$E(Y^{2})=\int_{-\infty}^{+\infty}y^{2}f_{Y}(y)\mathrm{d}y\Rightarrow DY=E(Y^{2})-(EY)^{2};$$

            $$E(Z^{2})=\int_{-\infty}^{+\infty}z^{2}f_{Z}(z)\mathrm{d}z\Rightarrow DZ=E(Z^{2})-(EZ)^{2}.$$
      \item 绝对值函数$|aX+bY+c|$的方差

            若 $U=aX+bY+c$,则
            $$EU=aEX+bEY+c,$$
            $$
                  DU=a^2DX+b^2DY(X,Y \text{相互独立}),$$
            $$D(|U|)=E(U^2)-[E(|U|)]^2$$
            $$
                  =DU+(EU)^2-[E(|U|)]^2,$$
            其中 $E(|U|)=\begin{cases}\int_{-\infty}^{+\infty}|u|f(u)du \text{ (连续型)},\\\sum_i|u_i|p_i \text{ (离散型)}.\end{cases}$
      \item 分解随机变量后再求⽅差

            若$X=X_{1}+X_{2}+\cdots+X_{n}$,则$DX=DX_{1}+DX_{2}+\cdots+DX_{n}+2\sum_{1\leqslant i<j\leqslant n}\operatorname{Cov}(X_{i},X_{j})$.

            当$X_{1}$,$X_{2}$,$\cdots$,$X_{n}$相互独立时,有$DX=DX_{1}+DX_{2}+\cdots+DX_{n}$.
      \item 性质
            \begin{enumerate}
                  \item $DX \geqslant 0$, $E(X^2) = DX + (EX)^2 \geqslant (EX)^2$.
                  \item $Dc = 0$($c$ 为常数).

                        $DX = 0 \Leftrightarrow X$ 几乎处处为某个常数 $a$, 即 $P\{X = a\} = 1$.
                  \item $D(aX + b) = a^2DX$.
                  \item $D(X \pm Y) = DX + DY \pm 2\text{Cov}(X, Y)$, $D\left(\sum_{i=1}^{n} a_i X_i\right) = \sum_{i=1}^{n} a_i^2 DX_i + 2 \sum_{1 \leqslant i < j \leqslant n} a_i a_j \text{Cov}(X_i, X_j)$.
            \end{enumerate}
      \item 常用分布的$EX,DX$
            \begin{enumerate}
                  \item 0—1 分布, $EX=p$ , $DX=p-p^{2}=(1-p)p$.
                  \item $X \sim B(n, p)$ , $EX=np$ , $DX=np(1-p)$.
                  \item $X \sim P(\lambda)$ , $EX=\lambda$ , $DX=\lambda$.
                  \item $X \sim G(p)$ , $EX=\frac{1}{p}$ , $DX=\frac{1-p}{p^{2}}$.
                  \item $X \sim U(a, b)$ , $EX=\frac{a+b}{2}$ , $DX=\frac{(b-a)^{2}}{12}$.
                  \item $X \sim E(\lambda)$ , $EX=\frac{1}{\lambda}$ , $DX=\frac{1}{\lambda^{2}}$.
                  \item $X \sim N(\mu, \sigma^{2})$ , $EX=\mu$ , $DX=\sigma^{2}$.
                  \item $X \sim \chi^{2}(n)$ , $EX=n$ , $DX=2n$.
            \end{enumerate}
\end{enumerate}
\section{协方差$Cov(X,Y)$与相关系数$\rho(X,Y)$}
\begin{enumerate}
      \item Cov(X,Y)
            \begin{enumerate}
                  \item 定义.
                        $$\mathrm{Cov}(X,Y)\overset{\Delta}{\operatorname*{\Longrightarrow}}E[(X-EX)(Y-EY)].$$
                  \item 定义法.
                        $$(X,Y) \sim p_{ij} \Rightarrow \operatorname{Cov}(X,Y) = \sum_{i} \sum_{j} (x_i - EX)(y_j - EY)p_{ij},$$
                        $$(X,Y) \sim f(x,y) \Rightarrow \operatorname{Cov}(X,Y) = \int_{-\infty}^{+\infty} \int_{-\infty}^{+\infty} (x - EX)(y - EY)f(x,y) \, dx \, dy.$$
                  \item 公式法.
                        $$\operatorname{Cov}(X,Y) = E(XY) - EXEY .$$
            \end{enumerate}
      \item $\rho(X,Y)$(定义相关系数,表示线性相依程度)

            $\rho_{xr}=\frac{\mathrm{Cov}(X,Y)}{\sqrt{DX}\sqrt{DY}}\begin{cases}=0\Leftrightarrow X,Y\text{不相关,}\\\neq0\Leftrightarrow X,Y\text{相关.}\end{cases}$

            (量纲为 1,无单位 )
      \item 性质
            \begin{enumerate}
                  \item $\mathrm{Cov}(X,Y) = \mathrm{Cov}(Y,X)$ .
                  \item $\mathrm{Cov}(aX,bY) = ab\mathrm{Cov}(X,Y)$.
                  \item $\mathrm{Cov}(X_1 + X_2,Y) = \mathrm{Cov}(X_1,Y) + \mathrm{Cov}(X_2,Y)$.
                  \item $|\rho_{XY}| \le 1$.
                  \item $\rho_{XY}= 1 \Leftrightarrow P\{Y = aX + b\} = 1 (a > 0)$ .

                        $\rho_{XY}= -1 \Leftrightarrow P\{Y = aX + b\} = 1 (a < 0)$.
                  \item 五个充要条件.

                        $\rho_{XY} = 0 \Leftrightarrow \text{Cov}(X,Y) = 0 \Leftrightarrow E(XY) = EXEY$
                        $\Leftrightarrow D(X+Y) = DX + DY \Leftrightarrow D(X-Y) = DX + DY.$
                  \item  $X,Y$ 独立 $\Rightarrow \rho_{XY} = 0$.
                  \item 若 $(X,Y) \sim N(\mu_1, \mu_2; \sigma_1^2, \sigma_2^2; \rho_{XY})$,则 $X,Y$ 独立 $\Leftrightarrow X,Y$ 不相关 ($\rho_{XY} = 0$).
            \end{enumerate}
\end{enumerate}
\section{独立性与不相关性的判定}
\begin{enumerate}
      \item 用分布判独立

            随机变量 $X$ 与 $Y$ 相互独立,指对任意实数 $x, y$,事件 $\{X \leqslant x\}$ 与 $\{Y \leqslant y\}$ 相互独立,即 $X$ 和 $Y$ 的联合分布等于边缘分布相乘:$F(x, y) = F_X(x) \cdot F_Y(y)$.
            \begin{enumerate}
                  \item 若 $(X, Y)$ 是连续型的,则 $X$ 与 $Y$ 相互独立的充要条件是 $f(x, y) = f_X(x) \cdot f_Y(y)$;
                  \item 若 $(X, Y)$ 是离散型的,则 $X$ 与 $Y$ 相互独立的充要条件是
                        $$P\{X = x_i, Y = y_j\} = P\{X = x_i\} \cdot P\{Y = y_j\} \, .$$
            \end{enumerate}

      \item 用数字特征判不相关

            随机变量 $X$ 与 $Y$ 不相关,意指 $X$ 与 $Y$ 之间不存在线性相依性,即 $\rho_{XY} = 0$,其充要条件是
            $$\rho_{XY} = 0 \Leftrightarrow \text{Cov}(X, Y) = 0 \Leftrightarrow E(XY) = EXEY \Leftrightarrow D(X \pm Y) = DX + DY \, .$$
      \item 步骤

            先计算 $\text{Cov}(X, Y)$,然后按下列步骤进行判断或再计算:
            $$\text{Cov}(X, Y) = E(XY) - EXEY \begin{cases} \neq 0 \Leftrightarrow X \text{ 与 } Y \text{ 相关} \Rightarrow X \text{ 与 } Y \text{ 不独立} \\ = 0 \Leftrightarrow X \text{ 与 } Y \text{ 不相关} \text{,通过分布推断} \begin{cases} X, Y \text{ 独立} \\ X, Y \text{ 不独立} \end{cases} \end{cases}$$

      \item 重要结论
            \begin{enumerate}
                  \item 如果 $X$ 与 $Y$ 独立,则 $X, Y$ 不相关,反之不然.
                  \item 如果 $X$ 与 $Y$ 相关,则 $X, Y$ 不独立.
                  \item 如果 $(X, Y)$ 服从二维正态分布,则 $X, Y$ 独立 $\Leftrightarrow X, Y$ 不相关.
                  \item 如果 $X$ 与 $Y$ 均服从 $0-1$ 分布,则 $X, Y$ 独立 $\Leftrightarrow X, Y$ 不相关.
            \end{enumerate}
\end{enumerate}

\section{切比雪夫不等式}
设随机变量$X$的数学期望与方差均存在,则对任意$\varepsilon>0$ ,
$$P\big\{\mid X-EX\mid\geqslant\varepsilon\big\}\leqslant\frac{DX}{\varepsilon^{2}}\:\text{或}\:P\{\big|X-EX\big|<\varepsilon\}\geqslant1-\frac{DX}{\varepsilon^{2}}.$$
\chapter{大数定律与中心极限定理}
\chapter{统计量及其分布}
\section{统计量及其数字特征}
设 $X_{1}, X_{2}, \cdots, X_{n}$ 是来自总体 $X$ 的简单随机样本,则
\begin{enumerate}
    \item 样本均值 $\bar{X}=\frac{1}{n} \sum_{i=1}^{n} X_{i}$.
    \item 样本方差 $S^{2}=\frac{1}{n-1} \sum_{i=1}^{n}\left(X_{i}-\bar{X}\right)^{2}=\frac{1}{n-1}\left(\sum_{i=1}^{n} X_{i}^{2}-n \bar{X}^{2}\right)$.

          样本标准差 $S=\sqrt{\frac{1}{n-1} \sum_{i=1}^{n}\left(X_{i}-\bar{X}\right)^{2}}$.
    \item 样本 $k$ 阶原点矩 $A_{k}=\frac{1}{n} \sum_{i=1}^{n} X_{i}^{k}(k=1,2, \cdots)$.
    \item 样本 $k$ 阶中心矩 $B_{k}=\frac{1}{n} \sum_{i=1}^{n}\left(X_{i}-\bar{X}\right)^{k}(k=2,3, \cdots)$.
    \item 顺序统计量

          将样本 $X_{1}, X_{2}, \cdots, X_{n}$ 的 $n$ 个观测量按其取值从小到大的顺序排列,得
          $$X_{(1)} \leqslant X_{(2)} \leqslant \cdots \leqslant X_{(n)}.$$
          随机变量 $X_{(k)}(k=1,2, \cdots, n)$ 称作第 $k$ 顺序统计量,其中 $X_{(1)}$ 是最小观测量, $X_{(n)}$ 是最大观测量,即
          $$X_{(1)}=\min \left\{X_{1}, X_{2}, \cdots, X_{n}\right\}, \quad X_{(n)}=\max \left\{X_{1}, X_{2}, \cdots, X_{n}\right\}.$$
\end{enumerate}

\section{判别统计量的分布}
定义:统计量的分布称为抽样分布

\begin{enumerate}
    \item 正态分布
          \begin{enumerate}
              \item 概念

                    如果 $X$ 的概率密度为
                    $$f(x) = \frac{1}{\sqrt{2 \pi} \sigma} \mathrm{e}^{-\frac{1}{2} \left( \frac{x - \mu}{\sigma} \right)^2} \quad (-\infty < x < +\infty),$$
                    其中 $-\infty < \mu < +\infty$, $\sigma > 0$, 则称 $X$ 服从参数为 $(\mu, \sigma^2)$ 的正态分布或称 $X$ 为正态变量, 记为 $X \sim N(\mu, \sigma^2)$.
              \item 上$\alpha$分位数

                    若 $X \sim N(0, 1)$, $P\{X > \mu_\alpha\} = \alpha$ ( $0 < \alpha < 1$ ), 则称 $\mu_\alpha$ 为标准正态分布的上 $\alpha$ 分位数
              \item 性质

                    $f(x)$ 的图形关于直线 $x=\mu$对称,即$f(\mu-x)=f(\mu+x)$,并在$x=\mu$处有唯一最大值
                    $$f(\mu)=\frac{1}{\sqrt{2\pi}\sigma}.$$
                    通常称$\mu=0$ , $\sigma=1$时的正态分布$N(0,1)$为标准正态分布,记标准正态分布的概率密度为
                    $\varphi(x)=\frac{1}{\sqrt{2\pi}}\mathrm{e}^{-\frac{1}{2}x^{2}}$,分布函数为$Q(x)=\frac1{\sqrt{2\pi}}\int_{-\infty}^{x}\mathrm{e}^{-\frac{t^{2}}{2}}dt$ .显然$\varphi(x)$为偶函数,且有
                    $$\Phi(0)=\frac{1}{2},\Phi(-x)=1-\Phi(x).$$
          \end{enumerate}
    \item $\chi^2$分布
          \begin{enumerate}
              \item 概念

                    若随机变量 $X_{1}, X_{2}, \cdots, X_{n}$ 相互独立,且都服从标准正态分布,则随机变量 $X = \sum_{i=1}^{n} X_{i}^{2}$ 服从自由度为 $n$ 的 $\chi^{2}$ 分布,记为 $X \sim \chi^{2}(n)$.
              \item 上$\alpha$分位数

                    对给定的 $\alpha (0 < \alpha < 1)$,称满足
                    $$P\{\chi^{2} > \chi_{\alpha}^{2}(n)\} = \int_{\chi_{\alpha}^{2}(n)}^{+\infty} f(x) \, \mathrm{d}x = \alpha$$
                    的 $\chi_{\alpha}^{2}(n)$ 为 $\chi^{2}(n)$ 分布的上 $\alpha$ 分位数(见图). 对于不同的 $\alpha, n$,$\chi^{2}(n)$ 分布上 $\alpha$ 分位数可通过查表求得.
              \item 性质
                    \begin{enumerate}
                        \item  若 $X_{1} \sim \chi^{2}(n_{1})$,$X_{2} \sim \chi^{2}(n_{2})$,$X_{1}$ 与 $X_{2}$ 相互独立,则
                              $$
                                  X_{1} + X_{2} \sim \chi^{2}(n_{1} + n_{2}).$$
                              此结论可推广至有限多个随机变量的和.
                        \item $若X\sim\chi^{2}(n)$,则$EX=n,DX=2n.$
                    \end{enumerate}
          \end{enumerate}
    \item $t$ 分布
          \begin{enumerate}
              \item 概念
                    设随机变量 $X \sim N(0,1)$, $Y \sim \chi^2(n)$, $X$ 与 $Y$ 相互独立, 则随机变量 $t = \frac{X}{\sqrt{Y/n}}$ 服从自由度为 $n$ 的 $t$ 分布, 记为 $t \sim t(n)$.
              \item 上$\alpha$分位数

                    对给定的 $\alpha(0 < \alpha < 1)$, 称满足
                    $$P\{t > t_{\alpha}(n)\} = \alpha$$
                    的 $t_{\alpha}(n)$ 为 $t(n)$ 分布的上 $\alpha$ 分位数.
              \item 性质
                    \begin{enumerate}
                        \item  $t$ 分布概率密度 $f(x)$ 的图形关于 $x = 0$ 对称, 因此
                              $$Et = 0 \quad (n \geqslant 2).$$
                        \item 由 $t$ 分布概率密度 $f(x)$ 图形的对称性, 知 $P\{t > -t_{\alpha}(n)\} = P\{t > t_{1-\alpha}(n)\}$, 故 $t_{1-\alpha}(n) = -t_{\alpha}(n)$. 当 $\alpha$ 值在表中没有时, 可用此式求得上 $\alpha$ 分位数.
                    \end{enumerate}
          \end{enumerate}
    \item $F$ 分布
          \begin{enumerate}
              \item 概念

                    设随机变量 $X \sim \chi^2(n_1)$,$Y \sim \chi^2(n_2)$,且 $X$ 与 $Y$ 相互独立,则 $F = \frac{X / n_1}{Y / n_2}$ 服从自由度为 $(n_1, n_2)$ 的 $F$ 分布,记为 $F \sim F(n_1, n_2)$,其中 $n_1$ 称为第一自由度,$n_2$ 称为第二自由度.$F$ 分布的概率密度 $f(x)$ 的图形.
              \item 上$\alpha$分位数

                    对给定的 $\alpha (0 < \alpha < 1)$,称满足
                    $$P\{F > F_\alpha(n_1, n_2)\} = \alpha$$
                    的 $F_\alpha(n_1, n_2)$ 为 $F(n_1, n_2)$ 分布的上 $\alpha$ 分位数.
              \item 性质
                    \begin{enumerate}
                        \item 若 $F \sim F(n_1, n_2)$,则 $\frac{1}{F} \sim F(n_2, n_1)$.
                        \item $F_{1-\alpha}(n_1, n_2) = \frac{1}{F_\alpha(n_2, n_1)}$.常用来求 $F$ 分布表中未列出的上 $\alpha$ 分位数,显然,有些特殊值可直接得出,如 $1-\alpha = \alpha$,$n_1 = n_2 = n$ 时,有 $F_{0.5}(n, n) = \frac{1}{F_{0.5}(n, n)}$,且 $F_{0.5}(n, n) > 0$,故 $F_{0.5}(n, n) = 1$.
                        \item 若 $t \sim t(n)$,则 $t^2 \sim F(1, n)$.
                    \end{enumerate}
          \end{enumerate}
\end{enumerate}
\section{用正态总体下的常用结论判别分布、计算概率}
设 $X_{1}, X_{2}, \cdots, X_{n}$ 是取自正态总体 $N(\mu, \sigma^{2})$ 的一个样本, $\bar{X}$ , $S^{2}$ 分别是样本均值和样本方差,则

\begin{enumerate}
    \item $\bar{X} \sim N\left(\mu, \frac{\sigma^{2}}{n}\right)$ ,即 $\frac{\bar{X}-\mu}{\frac{\sigma}{\sqrt{n}}} = \frac{\sqrt{n}(\bar{X}-\mu)}{\sigma} \sim N(0,1)$
    \item $\frac{1}{\sigma^{2}} \sum_{i=1}^{n}(X_{i}-\mu)^{2} \sim \chi^{2}(n)$ ;
    \item $\frac{(n-1)S^{2}}{\sigma^{2}} = \sum_{i=1}^{n}\left(\frac{X_{i}-\bar{X}}{\sigma}\right)^{2} \sim \chi^{2}(n-1)$ ( $\mu$ 未知,在 “2.” 中用 $\bar{X}$ 替代 $\mu$ );
    \item $\bar{X}$ 与 $S^{2}$ 相互独立, $\frac{\sqrt{n}(\bar{X}-\mu)}{S} \sim t(n-1)$ ( $\sigma$ 未知,在 “1.” 中用 $S$ 替代 $\sigma$ ). 进一步有
          $$\frac{n(\bar{X}-\mu)^{2}}{S^{2}} \sim F(1, n-1).$$
\end{enumerate}


\chapter{参数估计与假设检验}


\LUgroupsancheck

\makeatletter
\let\chapter\@std@chapter
\let\@std@chapter\relax
\makeatother

\backmatter
{\small
    \printindex
    \printindex[sym]
}

\end{document}
