\chapter{预备知识}

线性代数作为大学的第一门数学课,预修要求并不高. 我们默认读者具有高中数学基础,因此关于集合、映射、向量的基本常识我们不在此赘述. 接下来我们将介绍基本代数结构,以便后续线性空间的引入,然后我们将介绍本书中常见的概念——等价类和最常用的算法之一——高斯-若当消元法.

\section{基本代数结构}

我们选择从基本代数结构谈起,因为在以往的实践中我们深切地体会到直接引入线性空间的跳跃. 因此我们希望从更具象的例子开始,首先引入``代数结构''这一基本概念,然后在下一节中自然地引出线性空间的定义. 为了接下来定义的方便,我们首先介绍集合的笛卡尔积运算:

\begin{definition}{笛卡尔积}{笛卡尔积} \index{dikaerji@笛卡尔积 (Cartesian product)}
    设$A$和$B$是两个非空集合,我们把集合
    \[A \times B = \{(a,b) \mid a \in A, b \in B\}\]
    称为集合 $A$ 和 $B$ 的\term{笛卡尔积}. 如果 $A = B$,我们也可以简记为 $A^2$.
\end{definition}

笛卡尔积的定义是非常直观的,它实际上就是两个集合中的元素两两组合构成的有序对全体. 我们来看一些简单的例子:

\begin{example}{}{}
    \begin{enumerate}
        \item 若 $A = \{1,2\}$,$B = \{3,4\}$,则 $A \times B=\{(1,3),(1,4),(2,3),(2,4)\}$;
        \item 若 $A = B = \mathbf{R}$,则根据定义,$A \times B$ 的元素是形如 $(a,b)$ 的有序对,其中 $a, b \in \mathbf{R}$. 从几何上不难看出,我们可以将 $\mathbf{R} \times \mathbf{R}$ 视为二维平面上的点集(可以简记为 $\mathbf{R}^2$),其中的元素 $(a,b)$ 对应于平面上的一个点,这一点的横坐标为 $a$,纵坐标为 $b$.
    \end{enumerate}
\end{example}

在介绍完笛卡尔积的定义后,我们开始考察一个简单的例子:实数集$\mathbf{R}$,它是一个集合. 在初中我们便知道,在$\mathbf{R}$上我们可以定义加法和乘法两种运算. 本质而言,运算是一种映射(或者更通俗而言,函数):

\begin{center}
    \begin{tabular}{rrcl}
        $+\enspace\colon$      & $\mathbf{R}\times\mathbf{R}$ & $\to$     & $\mathbf{R}$ \\
                               & $(a,b)$                      & $\mapsto$ & $a+b$        \\
        $\times\enspace\colon$ & $\mathbf{R}\times\mathbf{R}$ & $\to$     & $\mathbf{R}$ \\
                               & $(a,b)$                      & $\mapsto$ & $a\times b$
    \end{tabular}
\end{center}

因此 $+$ 和 $\times$ 都以实数的有序对作为函数的自变量,函数值也是一个实数. 或许读者看到这里还是对运算的定义有些许迷茫,但如果我们回忆映射的基本定义 $f \colon A \to B$ 表示给 $A$ 中的任意元素 $a$ 指派一个 $B$ 中的元素 $f(a)$,并将加法乘法写成 $+(2,3) = 5$,$\times(2,3) = 6$,想必就会恍然大悟:$+$ 和 $\times$ 实际上就是函数名,函数做的事情就是输入两个自变量然后进行加法/乘法运算得到结果,并把这个结果指派给自变量作为函数值.

在上述讨论中,我们所做的事情很简单,就是给定一个集合,然后在这一集合的元素之间定义运算. 实际上这就是代数系统的定义:
\begin{definition}{代数系统}{} \index{daishuxitong@代数系统 (algebraic system)}
    一般地,我们把一个非空集合$X$和与$X$相关的若干代数运算$f_1,\ldots,f_k$组成的系统称为\term{代数系统}(简称代数系),记作$\langle X \colon f_1,\ldots,f_k\rangle$.

    在此基础上,我们把定义在集合上的运算具有某些特定性质的一类代数系统称为\term{代数结构}.
\end{definition}

特别注意的是,代数系统上定义的运算必须保证封闭性,也就是运算后的结果必须仍然在集合$X$中. 这事实上早已由映射的方式对运算的定义保证了.

不难理解,代数系统其中蕴含的性质与其中定义的运算具有的性质是关联很大的. 例如从经典分析学的角度实数实际上是是这样一些东西的组合 $\langle\mathbf{R}\colon 0, 1, +, *, \leqslant\rangle$,只是我们在没有歧义时将其简记为了 $\mathbf{R}$. 接下来我们以实数域为例,介绍在代数学中关心的几个运算性质. 我们首先讨论实数域上的加法运算,以下性质对于任意$a,b,c\in\mathbf{R}$都成立:

\begin{enumerate}
    \item 结合律:$(a+b)+c=a+(b+c)$;

    \item 单位元:存在一个元素$0$,使得$a+0=0+a=a$;

    \item 逆元:对于任意$a$,存在一个元素$-a$,使得$a+(-a)=(-a)+a=0$(0为单位元);

    \item 交换律:$a+b=b+a$.
\end{enumerate}

对于乘法运算(可记为$\cdot$或$\times$),单位元一般记为$1$(更一般的可以记为$e$),逆元记为$a^{-1}$. 事实上,我们可以给出更多的例子:
\begin{example}{}{Abel 群}
    \begin{enumerate}
        \item 代数系统$\langle \mathbf{R}\setminus\{0\}\colon\circ\rangle$定义的一般乘法运算

        \item 代数系统$\langle \mathbf{R}^2\colon+\rangle$定义的平面向量的加法
    \end{enumerate}
    均满足上述四条运算性质.
\end{example}

事实上,我们可以对上面的定义做进一步的抽象. 我们可以忽略集合中元素的意义差异(元素可以表示实数,也可以在上述例子中表示平面向量等几何对象),同时也可以忽略运算定义的差异,只关心运算作用于集合元素的性质. 对于一般的代数系统$\langle G\colon\circ\rangle$,我们有如下定义:
\begin{definition}{群}{群}\index{qun@群 (group)}
    若运算$\circ$满足结合律,则称代数系统$\langle G\colon\circ\rangle$为\term{半群}\index{qun!banqun@半群 (semigroup)};若在半群基础上存在单位元,则称之为\term{含幺半群}\index{qun!hanyaobanqun@含幺半群 (monoid)};若在含幺半群基础上每个元素存在逆元,则称之为\term{群};若在群的基础上运算还满足交换律,则称之为\term{Abel 群},也称\term{交换群}\index{qun!abel@Abel 群 (Abelian group), 交换群 (commutative group)}.
\end{definition}

\autoref{def:群} 给出了本节第一个要讨论的代数结构——群的定义. 下面给出了一些具体的例子帮助读者理解上述一系列群的定义,并且我们在后续学习矩阵的时候也会遇到一些群结构,相信这些实例能使读者体会到``在集合上定义运算''的方式的多样与抽象.

\begin{example}{}{}
    \begin{enumerate}
        \item 设$+$,$\times$分别为数的加法与乘法,则$\langle\mathbf{N}^+\colon +\rangle$,$\langle\mathbf{N}^+\colon \times\rangle$,$\langle\mathbf{Z}\colon \times\rangle$均为交换半群;$\langle\mathbf{Q}^+\colon \times\rangle$,$\langle\mathbf{R}^+\colon \times\rangle$均为交换群,且$\mathbf{Q}^+$是$\mathbf{R}^+$关于乘法运算的子群;而$\langle\{1,-1\}\colon \times\rangle$是仅有两个元素的交换群. 一般,含有限个元素的群称为有限群,否则称为无限群;
        \item 设$\mathbf{R}^3$为全体空间几何向量组成的几何,则$\mathbf{R}^3$关于向量加法构成一个交换群,其中单位元$e$为零向量$\vec{0}$,任一向量$\vec{\alpha}$的逆元为$-\vec{\alpha}$;
        \item 设$\mathbf{R}[x]_3=\{a_0+a_1x+a_2x^2\mid a_0,a_1,a_2\in\mathbf{R}\}$,则$\mathbf{R}[x]_3$关于多项式的加法构成交换群,单位元$e$为零多项式,但关于多项式的乘法不构成半群.
        \item 设$G$为某班的学生集,且任意两个人的身高互异,运算$\circ$为比高矮,规定
        \[
        a \circ b = b \circ a = \begin{cases}
            a, & \text{若 $a$ 比 $b$ 高} \\
            b, & \text{若 $b$ 比 $a$ 高}
        \end{cases}; a\circ a=a,\enspace\forall a \in G.
        \]
        该运算为$G$的代数运算,且满足结合律,运算的单位元$e$为最矮的学生,因此$\langle G\colon\circ\rangle$为含幺半群.
    \end{enumerate}
\end{example}

为方便书写,对于\autoref{def:群} 定义的群$\langle G\colon\circ\rangle$,在不引起混淆的情况下我们可以简写为群$G$. 除此之外,我们还需要指出以下两点:
\begin{theorem}{}{群的单位元逆元唯一}
    \begin{enumerate}
        \item 群的单位元唯一;

        \item 群中每个元素的逆元唯一.
    \end{enumerate}
\end{theorem}

\begin{proof}
    \begin{enumerate}
        \item 设$e_1$和$e_2$都是群$G$的单位元,则
              \[e_1=e_1\circ e_2=e_2.\]

        \item 设$b$和$c$都是$a$的逆元,则
              \[b=b\circ e=b\circ(a\circ c)=(b\circ a)\circ c=e\circ c=c.\]
    \end{enumerate}
\end{proof}

其中第一点的证明直接使用了单位元的性质,第二点的证明则使用了结合律和逆元的性质. 这里关于唯一性的证明是非常重要的:我们只需假设要证明唯一的东西有两个,然后说明这两个必然相等即可. 这一思想在之后证明矩阵的逆唯一等问题时也会用到,因此此处特别给出证明强调.

事实上,在很多集合上我们不仅可以定义一种运算,也可以定义两种甚至更多运算,在代数结构中我们仅讨论最多两种运算的情况. 事实上,我们最开始的实数集合定义加法和乘法的例子便可以引入一个新的代数结构——域:
\begin{definition}{域}{} \index{yu@域 (field)}
    我们称代数系统 $\langle F\colon+,\circ\rangle$ 为一个\term{域},如果
    \begin{enumerate}
        \item $\langle F\colon+\rangle$ 是交换群,其单位元记作0;

        \item $\langle F\setminus\{0\}\colon\circ\rangle$ 是交换群;

        \item 运算 $\circ$ 对 $+$ 满足左、右分配律,即
              \begin{gather*}
                  a \circ (b + c) = a \circ b + a \circ c \\
                  (b + c) \circ a = b \circ a + c \circ a
              \end{gather*}
    \end{enumerate}
\end{definition}

显然,实数域 $\mathbf{R}$ 上定义一般的实数加法和乘法后构成一个域. 实际上我们熟悉的例如有理数、实数等集合关于一般的加法和乘法运算都构成域,因此我们会经常使用``有理数域''、``实数域''等说法. 我们称数集对数的加法和乘法构成的域为数域,注意此处运算的定义必须是数学分析中定义的数的加法和乘法,不能是自定义的运算.


\begin{summary}

    本讲为了后续章节讲述方便引入了一些基本概念和算法. 尽管这是一门面向理工科应用的数学课,但我们仍然希望以最自然的方式引入概念,而非填鸭式地轰炸,因此我们首先从大家最熟悉的实数集合开始,讨论在集合上定义运算的方法:我们逐步加强条件,引入了三种基本的代数结构——群、环和域,并且给出了一些例子,并简单讨论了定义代数系统的意义. 事实上,下一讲开始要介绍的线性空间也是一种特殊的代数结构,因此首先引入代数结构对于我们自然展开接下来的讨论有很大的帮助,不至于让读者觉得非常突兀.

    接下来我们也从域的定义入手,构造了$\mathbf{R}^2$上的乘法运算使其构成了一个域,并且可以发现这里的定义与高中学习的复数乘法是完全一致的. 之后我们引入了等价关系的概念,这一概念在后续的讲义中将会多次出现,其重要意义就是将一个集合划分成了几个等价的区域. 最后我们讨论了高斯-若当消元法的一般步骤,这是我们接下来解决线性空间中各类问题绕不开的算法.

\end{summary}
