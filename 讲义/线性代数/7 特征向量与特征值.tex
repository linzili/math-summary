\XIANchapter{特征向量与特征值}

\term{求解利用} $A$ 的特征值与特征向量

% ======================================================
\section{利用特征值命题}
\DOne + \DTwoTwo
\begin{enumerate}

    % ------------------------------------------------------
    \item \textbf{特征值判定}
          \[
              \lambda_0 \text{ 是 }A\text{ 的特征值}
              \;\Longleftrightarrow\;
              |\lambda_0E - A| = 0,
          \]
          \[
              \lambda_0 \text{ 不是特征值}
              \;\Longleftrightarrow\;
              |\lambda_0E - A| \ne 0
              \quad(\text{矩阵可逆、满秩}).
          \]

          % ------------------------------------------------------
    \item \textbf{特征值与行列式、迹的关系}

          若 $A$ 的特征值为 $\lambda_1,\ldots,\lambda_n$,则
          \[
              |A|=\lambda_1\lambda_2\cdots\lambda_n,
              \qquad
              \mathrm{tr}(A)=\lambda_1+\lambda_2+\cdots+\lambda_n.
          \]

          % ------------------------------------------------------
    \item \textbf{矩阵函数与特征值对应表}

          \begin{table}[h]
              \centering
              \begin{tabular}{|c|c|c|c|c|c|c|}
                  \hline
                  矩阵   & $A$       & $f(A)$       & $A^{-1}$            & $A^{*}$ 的 $f$         & $P^{-1}AP=B$ & $P^{-1}f(A)P=B$ \\
                  \hline
                  特征值  & $\lambda$ & $f(\lambda)$ & $\frac{1}{\lambda}$ & $\frac{|A|}{\lambda}$ & $\lambda$    & $f(\lambda)$    \\
                  \hline
                  特征向量 & $\xi$     & $\xi$        & $\xi$               & $\xi$                 & $P^{-1}\xi$  & $P^{-1}\xi$     \\
                  \hline
              \end{tabular}
          \end{table}

          (注:分母出现 $\lambda$ 时均要求 $\lambda\neq 0$)

          \begin{note}{}{}
              若 $\lambda\neq 0$,则
              \[
                  af(A)\pm bA^{-1}\pm cA^{*}
              \]
              的特征值为
              \[
                  a f(\lambda)\pm \frac{b}{\lambda}\pm c\frac{|A|}{\lambda},
              \]
              对应特征向量仍为 $\xi$。但 $f(A),A^{-1},A^{*}$ 与 $A^{T}$ 的线性组合特征向量一般不同,需重新计算。
          \end{note}

          % ------------------------------------------------------
    \item \textbf{关于转置矩阵}

          虽然 $A^{T}$ 与 $A$ 特征值相同,但其特征向量一般不同。

          \begin{note}{}{}
              $A^{T}$ 与 $A$ 属于不同特征值的特征向量相互正交。
          \end{note}

          % ------------------------------------------------------
    \item \textbf{归零原则(常用)}

          \begin{enumerate}
              \item \textbf{归零准则一(对任意矩阵)}
                    若 $f(A)=O$,则 $A$ 的任意特征值 $\lambda$ 满足
                    \[
                        f(\lambda)=0.
                    \]

              \item \textbf{归零准则二(凯莱–哈密顿定理)}
                    若 $A$ 的特征多项式为
                    \[
                        f(\lambda)=|\lambda E-A|
                        =\lambda^n+a_{n-1}\lambda^{n-1}+\cdots+a_0,
                    \]
                    则
                    \[
                        f(A)=A^n+a_{n-1}A^{n-1}+\cdots +a_0E=O.
                    \]
          \end{enumerate}

\end{enumerate}

% ======================================================
\section{利用特征向量命题}
\DOne + \DTwoTwo
\begin{enumerate}

    % ------------------------------------------------------
    \item \textbf{特征向量判定}
          \[
              \xi\neq 0 \text{ 是 }A\text{ 的属于 }\lambda_0\text{ 的特征向量}
              \;\Longleftrightarrow\;
              (\lambda_0E-A)\xi=0.
          \]

          % ------------------------------------------------------
    \item \textbf{重要结论}

          \begin{enumerate}
              \item 单根特征值恰有 $1$ 个线性无关特征向量。
              \item $k$ 重特征值 $\lambda$ 至多有 $k$ 个线性无关特征向量。
              \item 不同特征值对应的特征向量必线性无关。
              \item 同一特征值下,若 $\xi_1,\xi_2$ 为特征向量,则 $k_1\xi_1+k_2\xi_2$($k_1k_2\neq 0$)仍是特征向量。
              \item 不同特征值下 $\xi_1,\xi_2$,线性组合 $k_1\xi_1+k_2\xi_2$ 不是任何特征向量($k_1k_2\neq 0$)。
              \item 若 $\xi$ 属于 $\lambda_1$,则不可能属于其它 $\lambda_2\neq\lambda_1$。
              \item 若 $A$ 只有 1 个线性无关特征向量,则唯一特征值必须是 $n$ 重特征值。
              \item 若 $AB=BA$ 且 $A$ 有 $n$ 个互异特征值,则 $A$ 的全部特征向量都是 $B$ 的特征向量。
              \item 若 $r(A)+r(B)<n$,则 $Ax=0,Bx=0$ 至少有一个公共非零解。
          \end{enumerate}

\end{enumerate}

% ======================================================
\section{利用矩阵方程命题}
\DOne+\DTwoTwo+\DTwoThree
\begin{enumerate}

    % ------------------------------------------------------
    \item $AB=O$
          \[
              AB=O\Rightarrow
              A[\beta_1,\ldots,\beta_n]=[0,\ldots,0],
          \]
          若每个 $\beta_i\neq 0$,则 $\beta_i$ 为 $A$ 属于特征值 $0$ 的特征向量。

          % ------------------------------------------------------
    \item 若任意 $\xi\ne 0$ 都是 $(\lambda E-A)x=0$ 的解
          取标准基 $e_1,\ldots,e_n$,得
          \[
              (\lambda E-A)B=O,\quad B=[e_1,\ldots,e_n] \text{ 可逆},
          \]
          故
          \[
              \lambda E-A=O \Rightarrow A=\lambda E.
          \]

          % ------------------------------------------------------
    \item 若 $AB=C$
          令 $B=[\beta_1,\ldots,\beta_n], C=[\gamma_1,\ldots,\gamma_n]$,若
          \[
              \gamma_i=\lambda_i\beta_i,
          \]
          则
          \[
              A\beta_i=\lambda_i\beta_i,
          \]
          所以 $\beta_i$ 是 $A$ 的属于 $\lambda_i$ 的特征向量。

          % ------------------------------------------------------
    \item 若 $AP=PB$,$P$ 可逆
          \[
              P^{-1}AP=B \Rightarrow A\sim B \Rightarrow \lambda_A=\lambda_B.
          \]

          % ------------------------------------------------------
    \item 若 $A$ 每行元素之和均为 $k$
          \[
              A \begin{bmatrix}1\\1\\ \vdots\\1\end{bmatrix}
              =
              k\begin{bmatrix}1\\1\\ \vdots\\1\end{bmatrix},
          \]
          因此上述向量为特征向量,特征值为 $k$。

          % ------------------------------------------------------
    \item 若 $A$ 可逆且每行元素之和为 $k$
          则 $A^{-1}$ 的每行元素之和为
          \[
              \frac{1}{k}.
          \]

          % ------------------------------------------------------
    \item 若 $A$ 每行元素之和为 $k$
          则 $A^n$ 每行元素之和为
          \[
              k^n.
          \]

\end{enumerate}