\XIANchapter{矩阵运算}
\section{证明 $A = O$ 的常用方法}

若需证明矩阵 $A$ 为零矩阵,可从以下角度出发:

\begin{enumerate}
      \item \textbf{秩判别法}
            若 $r(A) = 0$,则 $A = O$。

      \item \textbf{伴随矩阵判别法}
            若 $r(A) < n-1$,则 $A^{*} = O$,从而 $A = O$。

      \item \textbf{迹与 Frobenius 范数判别法}
            若
            \[
                  \mathrm{tr}(AA^{\top}) = 0,
            \]
            则 $AA^{\top} = O$,从而 $A = O$。

      \item \textbf{行向量与列向量均正交}
            若 $A$ 的行向量都与某矩阵 $B$ 的列向量正交,则
            \[
                  AB =
                  \begin{bmatrix}
                        \alpha_1 \\ \vdots \\ \alpha_n
                  \end{bmatrix}
                  \begin{bmatrix}
                        \beta_1,\cdots,\beta_m
                  \end{bmatrix}
                  = O.
            \]
            从而可推出 $A = O$。
\end{enumerate}

\section{计算 $A^n$ 的常用技巧}

设矩阵 $A$ 为 $m \times n$ 矩阵。若 $A$ 为方阵,常用以下方法计算 $A^n$。

\subsection{$r(A)=1$ 的情形}

若 $A$ 为 $n$ 阶方阵,且 $r(A)=1$,则必可写成外积形式
\[
      A = \alpha \beta^{\top},\qquad \alpha,\beta\neq O.
\]
于是
\[
      A^n = (\alpha\beta^{\top})^n
      = \alpha (\beta^{\top}\alpha)^{\,n-1} \beta^{\top}
      = \left(\mathrm{tr}(A)\right)^{n-1} A.
\]

\textit{结论:若 $r(A)=1$,则 $A^n = [\mathrm{tr}(A)]^{n-1}A$。}

\subsection{试算 $A^2$ 或 $A^3$,寻找规律}

\begin{enumerate}
      \item 若 $A^2 = kA$,则
            \[
                  A^n = k^{n-1} A.
            \]

      \item 若 $A^2 = kE$,则
            \[
                  A^{2n} = k^{\,n} E,\qquad
                  A^{2n+1} = k^{\,n} A.
            \]
\end{enumerate}

\subsection{分解法 $A = B + C$(且 $BC = CB$)}

若 $A = B + C$ 且 $BC = CB$,可用二项式展开:
\[
      A^{n}
      = \sum_{k=0}^{n} \binom{n}{k} B^{\,n-k} C^{\,k}.
\]

特别地:

\begin{enumerate}
      \item 若 $B = E$,则
            \[
                  A^n
                  = E + nC + \frac{n(n-1)}2 C^{2} + \cdots + C^{n}.
            \]

      \item 若 $BC = CB = O$,则
            \[
                  A^n = B^n + C^n.
            \]
\end{enumerate}

\subsection{利用初等矩阵求 $P_1^m A P_2^n$}

若 $P_1,P_2$ 为初等矩阵,则 $P_1^m A P_2^n$ 表示:

\begin{enumerate}
      \item 先对 $A$ 作 $m$ 次与 $P_1$ 相同的初等行变换;
      \item 再对所得矩阵作 $n$ 次与 $P_2$ 相同的初等列变换。
\end{enumerate}

可进一步推广到
\[
      P_1^{T} A P_2^{2},\quad
      P_1^{-2} A P_2^{T},\quad
      P_1^{3}A P_2^{-3},
\]
等形式的分析。

\subsection{相似理论计算 $A^n$}

\begin{enumerate}
      \item 若 $A \sim B$,即存在可逆矩阵 $P$ 使
            \[
                  P^{-1} A P = B,
            \]
            则
            \[
                  A = PBP^{-1},\qquad A^n = P B^n P^{-1}.
            \]

      \item 若矩阵 $A$ 满足
            \[
                  P^{-1} A P = A,
            \]
            则 $A$ 与自身相似,且
            \[
                  A^n = P A^n P^{-1}.
            \]
\end{enumerate}

\section{$A^TA,A^*,A^{-1}$与初等矩阵}

\begin{enumerate}

      %-----------------------------------------------------
      \item  $A^\mathrm{T}A$

            这里 $A$ 为 $n$ 阶实矩阵,称 $A^\mathrm{T}A$ 为格拉姆矩阵。
            以 3 阶矩阵为例,记 $A = [\boldsymbol{a}_1, \boldsymbol{a}_2, \boldsymbol{a}_3]$,则
            \[
                  A^\mathrm{T}
                  =\begin{bmatrix}
                        \boldsymbol{a}_1^\mathrm{T} \\
                        \boldsymbol{a}_2^\mathrm{T} \\
                        \boldsymbol{a}_3^\mathrm{T}
                  \end{bmatrix},
                  \qquad
                  A^\mathrm{T}A
                  =\begin{bmatrix}
                        \|\boldsymbol{a}_1\|^2               & (\boldsymbol{a}_1, \boldsymbol{a}_2) & (\boldsymbol{a}_1, \boldsymbol{a}_3) \\
                        (\boldsymbol{a}_2, \boldsymbol{a}_1) & \|\boldsymbol{a}_2\|^2               & (\boldsymbol{a}_2, \boldsymbol{a}_3) \\
                        (\boldsymbol{a}_3, \boldsymbol{a}_1) & (\boldsymbol{a}_3, \boldsymbol{a}_2) & \|\boldsymbol{a}_3\|^2
                  \end{bmatrix}.
            \]

            \begin{enumerate}
                  \item 若 $A^\mathrm{T}A =
                              \begin{bmatrix}
                                    1 & 0 & 0 \\0&2&0\\0&0&3
                              \end{bmatrix}$,
                        则 $\|\boldsymbol{a}_1\|=1$,$\|\boldsymbol{a}_2\|=\sqrt2$,
                        $\|\boldsymbol{a}_3\|=\sqrt3$,且三列向量两两正交。
                  \item 若 $\mathrm{tr}(A^\mathrm{T}A)=0$,
                        则三列向量的范数平方和为零,故均为零向量,$A=\boldsymbol{0}$。
                  \item $A^\mathrm{T}Ax=\boldsymbol{0}$ 与 $Ax=\boldsymbol{0}$ 同解。
                  \item $r(A)=r(A^\mathrm{T})=r(A^\mathrm{T}A)=r(AA^\mathrm{T})$。
                  \item 设 $A^\mathrm{T}A=B$($B$ 正定矩阵),求 $A$?
            \end{enumerate}

            由于 $B$ 正定,存在可逆矩阵 $P$ 使 $P^\mathrm{T}BP=E$。
            且
            \[
                  x^\mathrm{T}A^\mathrm{T}Ax=x^\mathrm{T}Bx,
                  \quad
                  (Ax)^\mathrm{T}(Ax)=x^\mathrm{T}Bx.
            \]
            配方法得 $y=Ax$,又 $x=Py$,故 $A=P^{-1}$。

            %-----------------------------------------------------
      \item $A^{*}$

            \begin{enumerate}
                  \item 定义
                        \[
                              A^{*}=
                              \begin{bmatrix}
                                    A_{11} & A_{21} & \cdots & A_{n1} \\
                                    A_{12} & A_{22} & \cdots & A_{n2} \\
                                    \vdots & \vdots &        & \vdots \\
                                    A_{1n} & A_{2n} & \cdots & A_{nn}
                              \end{bmatrix},
                        \]
                        其中 $A_{ij}$ 是 $a_{ij}$ 的代数余子式。

                  \item 公式($A,B$ 为 $n$ 阶矩阵,5、6、7需 $A$ 可逆):
                        \begin{enumerate}
                              \item $AA^{*}=A^{*}A=|A|E$。
                              \item $|A^{*}|=|A|^{\,n-1}$。
                              \item $(A^{\mathrm{T}})^{*}=(A^{*})^{\mathrm{T}}$。
                              \item $(kA)^*=k^{n-1}A^*$,$(-A)^*=(-1)^{n-1}A^*$。
                              \item $A^{-1}=\dfrac1{|A|}A^{*}$。
                              \item $A^{*}=|A|A^{-1}$。
                              \item $(A^{*})^{-1}=\dfrac1{|A|}A=(A^{-1})^{*}$。
                              \item $(A^{*})^{*}=|A|^{\,n-2}A$。
                              \item $|(A^{*})^{*}|=|A|^{(n-1)^2}$。
                              \item $(AB)^{*}=B^{*}A^{*}$。
                              \item 若 $r(A)=n-1$,则 $(A^*)^n=[\mathrm{tr}(A^*)]^{n-1}A^*$。
                        \end{enumerate}

                  \item 秩
                        \[
                              r(A^*)=
                              \begin{cases}
                                    n, & r(A)=n,   \\
                                    1, & r(A)=n-1, \\
                                    0, & r(A)<n-1.
                              \end{cases}
                        \]
            \end{enumerate}

            %-----------------------------------------------------
      \item $A^{-1}$

            \begin{enumerate}
                  \item 定义

                        若存在 $B$ 使 $AB=BA=E$,则 $A^{-1}=B$。

                  \item 性质
                        \begin{enumerate}
                              \item $(A^{-1})^{-1}=A$。
                              \item $(AB)^{-1}=B^{-1}A^{-1}$(穿脱原则)。
                              \item $(kA)^{-1}=\dfrac1k A^{-1}$。
                              \item $(A^{T})^{-1}=(A^{-1})^{T}$。
                              \item $|A^{-1}|=\dfrac1{|A|}$。
                        \end{enumerate}

                  \item 求 $A^{-1}$
                        \begin{enumerate}
                              \item 具体型
                                    \begin{enumerate}
                                          \item $A^{-1}=\dfrac1{|A|}A^{*}$。
                                          \item $[A,E]\xrightarrow{\text{初等行变换}}[E,A^{-1}]$。
                                    \end{enumerate}

                              \item 抽象型
                                    \begin{enumerate}
                                          \item 利用题设构造 $AB=E$ 得 $A^{-1}$。
                                          \item 若 $A=BC$ 且 $B,C$ 可逆,则 $A^{-1}=C^{-1}B^{-1}$。
                                    \end{enumerate}
                        \end{enumerate}

                  \item $A_{n\times n}$ 可逆的全部等价条件

                        $A_{n \times n}$ 可逆 $\Leftrightarrow$ 存在方阵 $B$ 使 $AB=BA=E$

                        $\Leftrightarrow$ 存在 $B$ 使 $BA=E$

                        $\Leftrightarrow$ 存在 $B$ 使 $AB=E$

                        $\Leftrightarrow |A|\neq 0$ ——与行列式联系

                        $\Leftrightarrow A$ 的等价标准形为 $E$

                        $\Leftrightarrow$ $A$ 可经初等行变换化为 $E$

                        $\Leftrightarrow$ $A$ 可经初等列变换化为 $E$

                        $\Leftrightarrow A$ 可分解为初等矩阵乘积

                        $\Leftrightarrow r(A)=n$

                        $\Leftrightarrow \forall C_{n\times s},\, r(AC)=r(C)$

                        $\Leftrightarrow \forall C_{n\times n},\, r(AC)=r(C)$

                        $\Leftrightarrow \forall C_{n\times n},\, r(CA)=r(C)$

                        $\Leftrightarrow \forall C_{s\times n},\, r(CA)=r(C)$

                        $\Leftrightarrow Ax=0$ 只有零解

                        $\Leftrightarrow \forall \beta,\,Ax=\beta$ 有唯一解

                        $\Leftrightarrow \forall \beta,\,Ax=\beta$ 有解

                        $\Leftrightarrow AX=O$ 只有零解

                        $\Leftrightarrow \forall C,\,AX=C$ 有唯一解

                        $\Leftrightarrow \forall C,\,AX=C$ 有解

                        $\Leftrightarrow$ 列向量组线性无关

                        $\Leftrightarrow$ 行向量组线性无关

                        $\Leftrightarrow$ 列向量组为 $\mathbb R^n$ 的基(数一)

                        $\Leftrightarrow$ 行向量组为 $\mathbb R^n$ 的基

                        $\Leftrightarrow 0$ 不是 $A$ 的特征值

                        $\Leftrightarrow A^\mathrm{T}A$ 正定

                        $\Leftrightarrow AA^\mathrm{T}$ 正定

                        $\Leftrightarrow A^\mathrm{T}A$ 的特征值全正

                        $\Leftrightarrow A^\mathrm{T}A$ 与 $E$ 合同

                        $\Leftrightarrow A^\mathrm{T}A$ 正惯性指数为 $n$

                        $\Leftrightarrow A^\mathrm{T}A$ 顺序主子式全正

            \end{enumerate}

            %-----------------------------------------------------
      \item 初等矩阵

            \begin{enumerate}

                  \item 初等矩阵性质
                        \begin{enumerate}
                              \item $|E_{ij}|=-1,\ |E_{ij}(k)|=1,\ |E_i(k)|=k$。
                              \item $E_{ij}^\top=E_{ij}$,$E_{ij}(k)^\top=E_{ji}(k)$,$E_i(k)^\top=E_i(k)$。
                              \item $E_{ij}^{-1}=E_{ij}$,$E_{ij}(k)^{-1}=E_{ij}(-k)$。
                              \item
                                    \(
                                    E_{ij}^*=|E_{ij}|E_{ij}^{-1}=-E_{ij}
                                    \)
                                    ;
                                    \[
                                          E_{ij}(k)^*=|E_{ij}(k)|E_{ij}(k)^{-1}=E_{ij}(-k),
                                          \qquad
                                          E_i(k)^*=|E_i(k)|\,E_i(k)^{-1}=kE_i(1/k).
                                    \]
                        \end{enumerate}

                  \item 初等变换的使用场合
                        \begin{enumerate}
                              \item 行列式关系:
                                    $E_{ij}A=B \Rightarrow |B|=-|A|$;
                                    $E_i(k)A=B \Rightarrow |B|=k|A|$;
                                    $E_{ij}(k)A=B \Rightarrow |B|=|A|$。

                              \item 求逆矩阵:
                                    $[A,E]\to[E,A^{-1}]$ 或
                                    $\begin{bmatrix}A\\E\end{bmatrix}\to\begin{bmatrix}E\\A^{-1}\end{bmatrix}$。

                              \item 求矩阵秩。

                              \item 解线性方程组:
                                    $[A,\beta]\to[PA,P\beta]$。

                              \item 解矩阵方程:
                                    $AX=B\Rightarrow X=A^{-1}B$;
                                    $XA=B\Rightarrow X=BA^{-1}$;
                                    若 $AP=PB$,用列变换求 $A=PBP^{-1}$。

                              \item 求极大线性无关组:行化阶梯形后按列取主元列。

                              \item 求等价标准形
                                    若 $PAQ=\begin{bmatrix}E_r&0\\0&0\end{bmatrix}$
                                    则为等价标准形;$P,Q$ 由扩展矩阵构造。

                              \item 满秩分解
                                    \[
                                          A
                                          =P^{-1}
                                          \begin{bmatrix}
                                                E_r & 0 \\0&0
                                          \end{bmatrix}
                                          Q^{-1}
                                          =P^{-1}
                                          \begin{bmatrix}
                                                E_r \\0
                                          \end{bmatrix}
                                          \begin{bmatrix}
                                                E_r & 0
                                          \end{bmatrix}
                                          Q^{-1}.
                                    \]

                              \item(仅数学一)求基和维数。
                        \end{enumerate}

            \end{enumerate}

\end{enumerate}
\section{分块矩阵}

\begin{enumerate}

      %-------------------- 1 定义 --------------------
      \item \textbf{定义}

            用横线和竖线将矩阵划分成若干小块,每一块称为子块,把子块视作原矩阵的元素,得到分块矩阵。

            矩阵 $A$ 按行分块:
            \[
                  A =
                  \begin{bmatrix}
                        a_{11} & a_{12} & \cdots & a_{1n} \\
                        a_{21} & a_{22} & \cdots & a_{2n} \\
                        \vdots & \vdots & \ddots & \vdots \\
                        a_{m1} & a_{m2} & \cdots & a_{mn}
                  \end{bmatrix}
                  =
                  \begin{bmatrix}
                        A_1 \\A_2\\\vdots\\A_m
                  \end{bmatrix},
                  \qquad
                  A_i=[a_{i1},a_{i2},\cdots,a_{in}],\ i=1,\dots,m.
            \]

            矩阵 $B$ 按列分块:
            \[
                  B=
                  \begin{bmatrix}
                        b_{11} & b_{12} & \cdots & b_{1n} \\
                        b_{21} & b_{22} & \cdots & b_{2n} \\
                        \vdots & \vdots & \ddots & \vdots \\
                        b_{m1} & b_{m2} & \cdots & b_{mn}
                  \end{bmatrix}
                  =
                  \begin{bmatrix}
                        B_1,\ B_2,\ \cdots,\ B_n
                  \end{bmatrix},
                  \qquad
                  B_j=[b_{1j},b_{2j},\cdots,b_{mj}]^\mathrm{T}.
            \]

            %-------------------- 2 运算 --------------------
      \item \textbf{运算}
            \begin{enumerate}

                  \item \textbf{转置}
                        \[
                              \begin{bmatrix}A & B\\ C & D\end{bmatrix}^\mathrm{T}
                              =
                              \begin{bmatrix}A^\mathrm{T} & C^\mathrm{T}\\ B^\mathrm{T} & D^\mathrm{T}\end{bmatrix}.
                        \]

                  \item \textbf{加法}
                        \[
                              \begin{bmatrix}A_1&A_2\\A_3&A_4\end{bmatrix}
                              +
                              \begin{bmatrix}B_1&B_2\\B_3&B_4\end{bmatrix}
                              =
                              \begin{bmatrix}A_1+B_1 & A_2+B_2\\ A_3+B_3 & A_4+B_4\end{bmatrix}.
                        \]

                  \item \textbf{数乘}
                        \[
                              k\begin{bmatrix}A & B \\ C & D\end{bmatrix}
                              =
                              \begin{bmatrix}
                                    kA & kB \\
                                    kC & kD
                              \end{bmatrix}.
                        \]

                  \item \textbf{乘法}
                        \[
                              \begin{bmatrix}A&B\\C&D\end{bmatrix}
                              \begin{bmatrix}X&Y\\Z&W\end{bmatrix}
                              =
                              \begin{bmatrix}
                                    AX+B Z & AY + BW \\
                                    CX+D Z & CY + DW
                              \end{bmatrix}.
                        \]

                  \item \textbf{分块对角矩阵的幂}
                        若 $A$、$B$ 分别为 $m$、$n$ 阶方阵,则
                        \[
                              \begin{bmatrix}A&O\\O&B\end{bmatrix}^k
                              =
                              \begin{bmatrix}A^k&O\\O&B^k\end{bmatrix}.
                        \]

                  \item \textbf{分块下三角矩阵的逆}
                        若
                        \[
                              A=\begin{bmatrix}B&O\\ D&C\end{bmatrix},
                              \quad
                              B,\ C\text{均可逆},
                        \]
                        则
                        \[
                              A^{-1}=
                              \begin{bmatrix}
                                    B^{-1}         & O      \\
                                    -C^{-1}DB^{-1} & C^{-1}
                              \end{bmatrix}.
                        \]

                  \item \textbf{主对角/副对角分块矩阵的逆}

                        主对角线分块:
                        \[
                              A=
                              \begin{bmatrix}
                                    A_1 \\&A_2\\&&\ddots\\&&&A_s
                              \end{bmatrix}
                              \Rightarrow
                              A^{-1}=
                              \begin{bmatrix}
                                    A_1^{-1} \\&A_2^{-1}\\&&\ddots\\&&&A_s^{-1}
                              \end{bmatrix}.
                        \]

                        副对角线分块:
                        \[
                              A=
                              \begin{bmatrix}
                                        &         &     & A_1 \\
                                        &         & A_2 &     \\
                                        & \iddots &     &     \\
                                    A_s &         &     &
                              \end{bmatrix}
                              \Rightarrow
                              A^{-1}=
                              \begin{bmatrix}
                                             &          &         & A_s^{-1} \\
                                             &          & \iddots &          \\
                                             & A_2^{-1} &         &          \\
                                    A_1^{-1} &          &         &
                              \end{bmatrix}.
                        \]

                  \item \textbf{$AB=O$ 的重要结论}

                        若 $A_{m\times n}B_{n\times s}=O$,令
                        \[
                              B=[\beta_1,\dots,\beta_s],
                        \]
                        则
                        \[
                              A\beta_i=0,\quad i=1,\dots,s,
                        \]
                        故 $\beta_i$ 均为齐次方程组 $Ax=0$ 的解,且
                        \[
                              r(A)+r(B)\le n.
                        \]

                  \item \textbf{$AB=C$ 的重要结论}

                        $C$ 的行向量可由 $B$ 的行向量线性组合表示;
                        $C$ 的列向量可由 $A$ 的列向量线性组合表示。

                        \begin{enumerate}
                              \item 列向量由 $A$ 的列向量组合:
                                    \[
                                          [A B,\ A] \to [O,\ A],\qquad
                                          [A,\ A B] \to [A,\ O].
                                    \]

                              \item 行向量由 $B$ 的行向量组合:
                                    \[
                                          \begin{bmatrix}AB\\B\end{bmatrix}
                                          \to
                                          \begin{bmatrix}O\\B\end{bmatrix},
                                          \quad
                                          r\!\left(\begin{bmatrix}AB\\B\end{bmatrix}\right)=r(B).
                                    \]
                                    若 $r(AB)=r(B)$,则 $ABx=0$ 与 $Bx=0$ 同解。
                        \end{enumerate}

                  \item \textbf{考研常见思路}
                        \begin{enumerate}
                              \item 熟记上面 (1)–(9);
                              \item 分块矩阵仍满足通用矩阵公式;
                              \item 遇到新分块矩阵,优先考虑上述技巧或初等变换。
                        \end{enumerate}

            \end{enumerate}
\end{enumerate}
\section{求解矩阵方程}

\begin{enumerate}

      %-------------------- 1 定义 --------------------
      \item \textbf{定义}

            含有未知矩阵的方程称为矩阵方程。

            %-------------------- 2 化简 --------------------
      \item \textbf{化简方法}

            解矩阵方程前,应先利用恒等变形将其化为
            \[
                  AX=B,\qquad XA=B,\qquad AXB=C.
            \]

            常用化简手段:

            \begin{enumerate}
                  \item 消公因式:若 $CA=CB$ 且 $C$ 可逆,则 $A=B$。
                  \item 提取公因式:$CA+CB=C(A+B)$。
                  \item 移项:将未知项移到一侧。
                  \item 利用公式:
                        \begin{enumerate}
                              \item $AA^*=|A|E$;若 $A$ 可逆,则 $A^*=|A|A^{-1}$;$(A^*)^*=|A|^{n-2}A$。
                              \item $A^2-E=(A+E)(A-E)$;$A^3\pm E=(A\pm E)(A^2\mp A+E)$。
                              \item $(A^T B^T)=(BA)^T$;$(A^*B^*)=(BA)^*$;可逆矩阵满足 $(AB)^{-1}=B^{-1}A^{-1}$。
                              \item $(A^{-1})^*=(A^*)^{-1}$;$(A^{-1})^T=(A^T)^{-1}$;$(A^*)^T=(A^T)^*$。
                        \end{enumerate}
            \end{enumerate}

            %-------------------- 3 求解 --------------------
      \item \textbf{求解方法}

            \begin{enumerate}

                  \item 若 $A$ 或 $A,B$ 可逆:

                        \[
                              AX=B\Rightarrow X=A^{-1}B,\qquad
                              XA=B\Rightarrow X=BA^{-1},
                              \qquad
                              AXB=C\Rightarrow X=A^{-1}CB^{-1}.
                        \]

                  \item 若 $A$ 不可逆:

                        对方程 $AX=B$,将 $X$ 按列分块:
                        \[
                              X=[\xi_1,\xi_2,\dots,\xi_n],\qquad
                              B=[\beta_1,\beta_2,\dots,\beta_n],
                        \]
                        则
                        \[
                              A\xi_i=\beta_i.
                        \]
                        逐列解线性方程组即可。

                  \item 若无法化为标准形式:

                        设未知矩阵 $X=(x_{ij})$,直接代入得到关于 $x_{ij}$ 的线性方程组(待定元素法)。

                  \item 充分必要条件:

                        设 $A$ 为 $m\times n$,$B$ 为 $m\times s$,则
                        \[
                              AX=B \text{ 有解} \quad\Longleftrightarrow\quad r(A)=r([A,B]).
                        \]

            \end{enumerate}

\end{enumerate}