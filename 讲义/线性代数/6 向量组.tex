\XIANchapter{向量组}

% ======================================================
\section{研究具体型向量关系}

\begin{enumerate}
    % ------------------------------------------------------
    \item 研究 $\beta$ 与 $\alpha_1,\alpha_2,\ldots,\alpha_n$ 的关系

          \begin{enumerate}
              \item 建立方程组
                    \[
                        \begin{bmatrix}\alpha_1 & \alpha_2 & \cdots & \alpha_n\end{bmatrix}
                        \begin{bmatrix}x_1\\ x_2\\ \vdots\\ x_n\end{bmatrix}=\beta.
                    \]

              \item 化为行阶梯形
                    \[
                        [A,\beta]=[\alpha_1,\alpha_2,\ldots,\alpha_n,\beta]
                        \xrightarrow{\text{行变换}}
                        [I,0].
                    \]

              \item 结论讨论
                    \begin{enumerate}
                        \item $r(A)\neq r([A,\beta]) \;\Longleftrightarrow\;$ 无解 $\Longleftrightarrow$ $\beta$ 不能表示为其线性组合;
                        \item $r(A)=r([A,\beta])=n \;\Longleftrightarrow\;$ 唯一解 $\Longleftrightarrow$ 表示法唯一;
                        \item $r(A)=r([A,\beta])<n \;\Longleftrightarrow\;$ 无穷多解 $\Longleftrightarrow$ 表示法不唯一.
                    \end{enumerate}
          \end{enumerate}

          % ------------------------------------------------------
    \item 研究向量组 $\alpha_1,\alpha_2,\ldots,\alpha_n$

          \begin{enumerate}
              \item 若向量个数 $>$ 维数,则必线性相关.

              \item 若向量个数 = 维数,则用行列式判断:
                    \[
                        |\alpha_1,\alpha_2,\ldots,\alpha_n|=0 \Longleftrightarrow \text{相关},
                        \qquad
                        \neq 0 \Longleftrightarrow \text{无关}.
                    \]

              \item 若向量个数 $<$ 维数,则化 $A=[\alpha_1,\ldots,\alpha_n]$ 为行阶梯形:
                    \begin{enumerate}
                        \item $r(A)<n \Longleftrightarrow$ 向量组线性相关;
                        \item $r(A)=n \Longleftrightarrow$ 向量组线性无关;
                        \item 若线性相关,欲求 $\alpha_s$ 的表示式,回到第 1 步建立方程组即可.
                    \end{enumerate}
          \end{enumerate}

          % ------------------------------------------------------
    \item 求极大线性无关组

          给定列向量组 $\alpha_1,\ldots,\alpha_n$,步骤如下:

          \begin{enumerate}
              \item 构造矩阵 $A=[\alpha_1,\alpha_2,\ldots,\alpha_n]$;
              \item 作行变换 $A\to B$(行阶梯形);
              \item 阶梯形中枢的位置给出秩 $r$,取出对应的 $r$ 列即可构成一组极大线性无关组.
          \end{enumerate}

\end{enumerate}


% ======================================================
\section{研究抽象型向量关系}

\begin{enumerate}
    \item 定义法(最基本方法)

          若已知某些线性关系,欲判定某组向量是否线性无关,则写出
          \[
              k_1a_1+k_2a_2+\cdots+k_na_n=0,
          \]
          检查是否只有 $k_1=\cdots=k_n=0$.

    \item 综合问题
          多与方程组、矩阵秩、特征值等知识结合,需要在具体题目中综合判断.
\end{enumerate}


% ======================================================
\section{研究向量组等价}

给定向量组(I)$\alpha_1,\ldots,\alpha_s$ 与(II)$\beta_1,\ldots,\beta_t$,若两组向量处于同一维数空间:

若(I)中每个向量都可由(II)线性表示,且(II)中每个向量都可由(I)线性表示,则称两向量组等价.

等价的判定命题:

\begin{enumerate}
    \item 秩相等且可单方向表示
          \[
              r(\alpha_1,\ldots,\alpha_s)=r(\beta_1,\ldots,\beta_t),
          \]
          且互相能表示.

    \item 三秩相等
          \[
              r(\alpha_1,\ldots,\alpha_s)
              =
              r(\beta_1,\ldots,\beta_t)
              =
              r(\alpha_1,\ldots,\alpha_s,\beta_1,\ldots,\beta_t).
          \]
\end{enumerate}


% ======================================================
\section{向量空间}

\begin{enumerate}
    % ------------------------------------------------------
    \item 向量空间与基

          若在向量空间 $V$ 中有向量组 $\alpha_1,\ldots,\alpha_r$ 满足:
          \begin{enumerate}
              \item $\alpha_1,\ldots,\alpha_r$ 线性无关;
              \item $V$ 中任意向量可由它们线性表示;
          \end{enumerate}
          则该组向量是 $V$ 的一组基,$r$ 称为向量空间 $V$ 的维数.

          任意向量 $\xi\in V$ 可唯一表示为
          \[
              \xi=x_1\alpha_1+\cdots+x_r\alpha_r.
          \]
          $(x_1,\ldots,x_r)$ 称为 $\xi$ 在基中的坐标.

          因此
          \[
              V=\{\;x_1\alpha_1+x_2\alpha_2+\cdots+x_r\alpha_r\mid x_i\in\mathbb{R}\;\}.
          \]

          % ------------------------------------------------------
    \item 过渡矩阵

          若 $V$ 的两组基为 $\eta_1,\ldots,\eta_n$ 与 $\xi_1,\ldots,\xi_n$,且
          \[
              [\eta_1,\ldots,\eta_n]=[\xi_1,\ldots,\xi_n]\,C,
          \]
          则 $C$ 为从基 $\{\xi\}$ 过渡到基 $\{\eta\}$ 的过渡矩阵.

          % ------------------------------------------------------
    \item 坐标变换

          同一向量 $\alpha$ 在不同基下有:
          \[
              \alpha=[\xi_1,\ldots,\xi_n]x=[\eta_1,\ldots,\eta_n]y.
          \]
          代入过渡矩阵可得
          \[
              x=Cy.
          \]
          称为坐标变换公式.
\end{enumerate}