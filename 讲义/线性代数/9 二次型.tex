\chapter{二次型}
\section{$f=x^TAx$ 中$A$的表示}
\DOne+\DTwoThree

\begin{enumerate}
    \item 给出非对称矩阵 $B$,令$A=\frac{B+B^T}{2}$,则$A=A^\mathrm{T}.$
    \item 通过题设或基本变形显化出 $A.$
\end{enumerate}

\section{配方法与正交变换法的异同}

\begin{enumerate}
    \item 命题语言
          \DTwoTwo
          \begin{enumerate}
              \item 配方法

                    二次型语言:将 $f = x^T A x$ 通过配方法化为标准形,并求出可逆变换矩阵 $C$.

                    矩阵语言:求可逆矩阵 $C$,使得 $C^T A C = \Lambda$.
              \item 正交变换法

                    二次型语言:将 $f = x^T A x$ 通过正交变换法化为标准形,并求出正交矩阵 $Q$.

                    矩阵语言:求正交矩阵 $Q$,使得 $Q^T A Q = \Lambda$.
          \end{enumerate}
    \item 过程与结果的异同
          \DTwoThree

          设$f(x)=x^TAx$.
          \begin{enumerate}
              \item 配方法(可逆线性变换)

                    $x=Cy$,$C$可逆.使得$f\xlongequal{x=\boldsymbol{C}y}y^T\Lambda y$,其中$C^TAC=\Lambda$(使$A$合同于对角矩阵).
              \item 正交变换法(可逆线性变换):

                    $x=Qy$(这里的$Q$不仅可逆,还满足$Q^{-1}=Q^{T}$),使得$f\xlongequal{x=\boldsymbol{Q}y}y^T\Lambda y$,其中$Q^{T}AQ=Q^{-1}AQ=A$.
          \end{enumerate}
          二者区别:在配方法中,$c$只满足可逆,所以$c^{-1}$不一定等于$c^T$,但是在正交变换法中,变换手段$Q$满足$Q^{- 1}= Q^T$ .

          二者相同点:它们的正、负惯性指数是对应相等的.
    \item 惯性指数
          \begin{example}{}{}
              $f(x_{1},x_{2},x_{3})=-2x_{1}x_{2}-2x_{1}x_{3}+6x_{2}x_{3}$的正惯性指数为(  ).
          \end{example}
          \begin{solution}
              令$\begin{cases}x_{1}=y_{1}+y_{2},\\x_{2}=y_{1}-y_{2},\end{cases}$则

              $$f=-2y_{1}^{2}+2y_{2}^{2}+4y_{1}y_{3}-8y_{2}y_{3}
                  =-2(y_{1}-y_{3})^{2}+2(y_{2}-2y_{3})^{2}-6y_{3}^{2},$$

              再令$\begin{cases}z_{1}=y_{1}-y_{3},\\z_{2}=y_{2}-2y_{3},\end{cases}$则

              $$f=-2z_{1}^{2}+2z_{2}^{2}-6z_{3}^{2},$$

              故$f$的正惯性指数为1.
          \end{solution}
\end{enumerate}

\section{伪配方法}
\DTwoThree

“平方和式$A^2+B^2+C^2$”未必就是(拉格朗日)配方法得来的结果,故若非拉格朗日配方法,则称伪配方法.要注意伪配方法的变换矩阵是否有可逆性.
\begin{enumerate}
    \item 如果变换没有可逆性,则有可能改变表达式的几何性质,如封闭性,此时不能得出平方和式正定;
    \item 如果变换是可逆的,则平方和式正定.
\end{enumerate}


\begin{note}{}{}
    对于$f(x_{1},x_{2},x_{3})=(a_{1}x_{1}+a_{2}x_{2}+a_{3}x_{3})^{2}+(b_{1}x_{1}+b_{2}x_{2}+b_{3}x_{3})^{2}+(c_{1}x_{1}+c_{2}x_{2}+c_{3}x_{3})^{2}$的情形,可总结如下做题方法:

    令$f=0$,即$\begin{cases}a_{1}x_{1}+a_{2}x_{2}+a_{3}x_{3}=0,\\b_{1}x_{1}+b_{2}x_{2}+b_{3}x_{3}=0,\\c_{1}x_{1}+c_{2}x_{2}+c_{3}x_{3}=0,\end{cases}$计算$|\boldsymbol{A}|=\begin{vmatrix}a_{1}&a_{2}&a_{3}\\b_{1}&b_{2}&b_{3}\\c_{1}&c_{2}&c_{3}\end{vmatrix}$,若$|\boldsymbol{A}|\neq0$,则$f$正定;若$|\boldsymbol{A}|=0$,则$f$不正定.
\end{note}
\section{正交变换法的传递性}
\DOne+\DTwoThree

若$A$相似于$B$,则$B$相似于$C$,则$A$相似于$C$.这里$B$常为$\Lambda$.

\section{合同的判定与手段}
\DOne+\DTwoThree

\begin{enumerate}
    \item 同阶实对称矩阵$A,B$合同的判定

          用正、负惯性指数:$A,B$合同$\Leftrightarrow p_A=p_B,q_A=q_B$(相同的正、负惯性指数).
    \item 已知$A$,$\Lambda$($\Lambda$是对角矩阵),求可逆矩阵$C$,使得$C^TAC=\Lambda$
    \item 已知$A$,$B$($B$不是对角矩阵),求可逆矩阵$C$,使得$C^TAC=B$
\end{enumerate}
\begin{idea}{求可逆矩阵$C$,使得$C^TAC=\Lambda$}{}
    \begin{enumerate}
        \item 配方 盯着$\Lambda$的对角线元素,提出对应系数
        \item 换元
        \item 求逆
    \end{enumerate}
\end{idea}
\begin{idea}{求可逆矩阵$C$,使得$C^TAC=B$}{}
    \begin{enumerate}
        \item 对$f$配方、换元,写$D_1$
        \item 对$g$配方、换元,写$D_2$
        \item 令$D_1x=D_2y$,求$D_2^{-1}D_1$
    \end{enumerate}
\end{idea}
\section{合同与相似的异同}
\DOne+\DTwoThree

对于实对称矩阵$A$与$B$,相似必合同,反之不成立.
\begin{example}{合同与相似的异同}{合同与相似的异同}
    已知二次型
    $$f(x_{1}, x_{2}, x_{3}) = x_{1}^{2} + 2x_{2}^{2} + 2x_{3}^{2} + 2x_{1}x_{2} - 2x_{1}x_{3}$$
    $$
        g(y_{1}, y_{2}, y_{3}) = y_{1}^{2} + y_{2}^{2} + y_{3}^{2} + 2y_{2}y_{3}$$
    \begin{enumerate}
        \item 求可逆变换 $x = Py$,将 $f(x_{1}, x_{2}, x_{3})$ 化成 $g(y_{1}, y_{2}, y_{3})$.
        \item 是否存在正交变换 $x = Qy$,将 $f(x_{1}, x_{2}, x_{3})$ 化成 $g(y_{1}, y_{2}, y_{3})$?
    \end{enumerate}
\end{example}
\begin{idea}{解题思路 \ref{ex:合同与相似的异同}}{}
    \begin{enumerate}
        \item 求可逆变换用配方法
        \item 判断是否存在正交变换,如果存在必相似,使用相似的充分条件和充要条件
    \end{enumerate}
\end{idea}
\section{正定的判定与应用}
\DOne+\DTwoThree

$\star\star\star$

\begin{enumerate}
    \item 前提
          $A=A^T$($A$是实对称矩阵)
    \item 二次型$f=x^TAx$正定的充要条件 \DTwo

          $n$元二次型$f=x^{T}Ax$正定

          $\Leftrightarrow$对任意的$x\neq 0$,有$x^{T}Ax>0$(定义)

          $\Leftrightarrow A$的特征值$\lambda_{i}>0(i=1,2,\cdots,n)$

          $\Leftrightarrow f$的正惯性指数$p=n$

          $\Leftrightarrow$存在可逆矩阵$D$,使得$A=D^{T}D$

          $\Leftrightarrow A$与$E$合同

          $\Leftrightarrow A$的各阶顺序主子式均大于0.
    \item 二次型$f=x^TAx$正定的必要条件
          \begin{enumerate}
              \item $a_{ii}>0\left(i=1,2,\cdots,n\right).$
              \item $| A| > 0$.
          \end{enumerate}
    \item 重要结论
          \begin{enumerate}
              \item 若$A$正定,则$A^-1,A^{*},A^{m}(m$为正整数$),kA(k>0),C^{\mathrm{T}}AC(C$可逆 )均正定 .

              \item 若$A,B$正定,则$A+B$正定,$\begin{bmatrix}A&O\\O&B\end{bmatrix}$正定.

              \item ${\text{若}A,B}$正定,则$AB$正定的充要条件是$AB= BA$ .
          \end{enumerate}

\end{enumerate}

\section{二次型的最值}
\DOne+\DTwoTwo+\DTwoThree

设 $A$ 为实对称矩阵,令
\[
    f(x)=x^T A x,\qquad x\in\mathbb R^n.
\]
若 $A=Q\Lambda Q^T$($Q$ 正交, $\Lambda=\mathrm{diag}(\lambda_i)$),令 $x=Qy$ 得
\[
    f(x)=\sum_{i=1}^n\lambda_i y_i^2.
\]
记 $\lambda_{\min}=\min_i\lambda_i,\ \lambda_{\max}=\max_i\lambda_i$,则
\[
    \boxed{\;\lambda_{\min}\|x\|^2 \le x^TAx \le \lambda_{\max}\|x\|^2\;},
\]
两端同时除以 $x^Tx$ 得 Rayleigh 商界:
\[
    \lambda_{\min}\le \frac{x^TAx}{x^Tx}\le \lambda_{\max}.
\]
等号成立当且仅当 $x$ 为对应的特征向量.

给定对称 $A$ 与对称且正定的 $B$,考题常要求求
\[
    \max_{x\ne0}\frac{x^TAx}{x^TBx}\quad\text{或}\quad\min_{x\ne0}\frac{x^TAx}{x^TBx}.
\]

若 $B$ 为对称正定矩阵,则可作分解 $B=P^TP$.记 $y=Px$,则有
$\frac{x^TAx}{x^TBx}\xlongequal[x=P^{-1}y]{y=Px} \frac{(P^{-1}y)^TAP^{-1}y}{y^Ty}=\frac{y^T(PAP^{-1})y}{y^Ty}=\frac{y^TCy}{y^Ty}$,
其中 $C=PAP^{-1}$ 为对称矩阵.

于是,$\dfrac{x^TAx}{x^TBx}$ 的最大值、最小值,分别等于 $C$ 的最大、最小特征值;相应的 $x$ 可由 $y=Px$ 反变换得到.