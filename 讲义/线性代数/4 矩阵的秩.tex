\XIANchapter{矩阵的秩}

\section{定义}

设 $A$ 为 $m\times n$ 矩阵。$A$ 中阶数最大的非零子式的阶数称为矩阵 $A$ 的秩,记为 $r(A)$。

亦即:若存在 $k$ 阶子式不为零,而所有 $k+1$ 阶子式(若存在)均为零,则
\[
    r(A)=k.
\]

特别地,若 $A$ 为 $n\times n$ 方阵,则
\[
    r(A)=n \;\Longleftrightarrow\; |A|\neq 0 \;\Longleftrightarrow\; A\ \text{可逆}.
\]

\section{公式}

\begin{enumerate}
    \item $0\le r(A_{m\times n})\le \min\{m,n\}$.
    \item $r(kA)=r(A)$($k\neq 0$)。
    \item $r(A)=r(PA)=r(AQ)=r(PAQ)$,其中 $P,Q$ 可逆。
    \item 若 $A$ 为 $m\times n$,$B$ 为 $n\times s$,则:

          \begin{itemize}
              \item 若 $r(A)=n$(列满秩),则 $r(AB)=r(B)$;
              \item 若 $r(B)=n$(行满秩),则 $r(AB)=r(A)$。
          \end{itemize}

    \item 若 $CA=CB$ 且 $C$ 列满秩,则 $A=B$;若 $AC=BC$ 且 $C$ 行满秩,则 $A=B$。
    \item $r(AB)\le \min\{r(A),r(B)\}$.
    \item $r(A+B)\le r([A,B])\le r(A)+r(B)$.
    \item 若 $A$ 为 $m\times n$,$B$ 为 $n\times s$,则
          \[
              r(AB)\ge r(A)+r(B)-n.
          \]
    \item $r(ABC)\ge r(AB)+r(BC)-r(B)$.
    \item
          \[
              r(A)=r(A^\mathrm{T})=r(AA^\mathrm{T})=r(A^\mathrm{T}A).
          \]
    \item
          \[
              r(A^*)=
              \begin{cases}
                  n, & r(A)=n,   \\[3pt]
                  1, & r(A)=n-1, \\[3pt]
                  0, & r(A)<n-1.
              \end{cases}
          \]
    \item 若 $A^2-(k_1+k_2)A+k_1k_2E=O$,$k_1\neq k_2$,则
          \[
              r(A-k_1E)+r(A-k_2E)=n.
          \]
    \item 对于齐次方程 $Ax=0$($A$ 为 $m\times n$),其基础解系含向量个数
          \[
              s=n-r(A).
          \]
    \item 方程组 $A_{m\times n}x=0$ 与 $B_{s\times n}x=0$ 同解,当且仅当
          \[
              r(A)=r\!\begin{pmatrix}A\\ B\end{pmatrix}=r(B).
          \]
    \item $r(\text{I}) = r(\text{II}) = r(\text{I},\text{II})$
          当且仅当向量组(I)与向量组(II)等价。
    \item 若 $A\sim \Lambda$,$\lambda_i$ 为 $n_i$ 重特征值,则
          \[
              n_i = n-r(\lambda_iE-A).
          \]
    \item 若 $A\sim \Lambda$,则 $r(A)$ 等于所有非零特征值的重数之和(按重数计算)。
\end{enumerate}