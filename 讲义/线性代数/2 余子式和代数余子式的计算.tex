\XIANchapter{余子式和代数余子式的计算}
\section{用矩阵}

当$|A|\neq0$时,
$$A^*=|A|A^{-1}.$$
由于$A^{*}$由$A_{ij}$组成,用上式求出$A^{*}$,即得到所有的$A_{ij}$.但要注意,此方法要求$|A|\neq0$,这既是前提,也是一种限制.

\section{用特征值}
设 $A$ 为 3 阶方阵,当 $A$ 为可逆矩阵时,记其特征值为 $\lambda_{1}$,$\lambda_{2}$,$\lambda_{3}$,则 $A^{-1}$ 的特征值为 $\lambda_{1}^{-1}$,$\lambda_{2}^{-1}$,$\lambda_{3}^{-1}$,且由 $A^{*} = |A|A^{-1} = \lambda_{1}\lambda_{2}\lambda_{3}A^{-1}$,可知 $A^{*}$ 的特征值为
$$\lambda_{1}^{*} = \lambda_{1}\lambda_{2}\lambda_{3} \cdot \lambda_{1}^{-1} = \lambda_{2}\lambda_{3}, \quad \lambda_{2}^{*} = \lambda_{1}\lambda_{2}\lambda_{3} \cdot \lambda_{2}^{-1} = \lambda_{1}\lambda_{3}, \quad \lambda_{3}^{*} = \lambda_{1}\lambda_{2}\lambda_{3} \cdot \lambda_{3}^{-1} = \lambda_{1}\lambda_{2},$$
故由
$$
    A^{*} = \begin{bmatrix}
        A_{11} & A_{21} & A_{31} \\
        A_{12} & A_{22} & A_{32} \\
        A_{13} & A_{23} & A_{33}
    \end{bmatrix},
$$
知 $A_{11} + A_{22} + A_{33} = \operatorname{tr}(A^{*}) = \lambda_{1}^{*} + \lambda_{2}^{*} + \lambda_{3}^{*} = \lambda_{2}\lambda_{3} + \lambda_{1}\lambda_{3} + \lambda_{1}\lambda_{2}$.