\XIANchapter{线性方程组}

\section{线性方程组理论总结}

\DOne
\begin{enumerate}

    %==========================
    % 1. 齐次线性方程组
    %==========================
    \item 齐次线性方程组 $Ax=0$ \DOne

          \begin{enumerate}
              % 性质
              \item 解的性质

                    对于齐次线性方程组
                    \[
                        A_{m\times n}x=0,
                    \]
                    若 $\xi_1,\xi_2$ 是解,则任意线性组合
                    \[
                        x=k_1\xi_1+k_2\xi_2 \quad (k_1,k_2\in\mathbb{R})
                    \]
                    仍为其解。

                    % 基础解系
              \item 基础解系与通解
                    \begin{enumerate}
                        \item 若 $r(A)<n$,设 $\xi_1,\dots,\xi_s$ 满足:
                              \begin{enumerate}
                                  \item $\xi_1,\dots,\xi_s$ 线性无关;
                                  \item 任一解均可由它们线性表示,即 $s=n-r(A)$;
                              \end{enumerate}
                              则称 $\xi_1,\dots,\xi_s$ 为 $Ax=0$ 的一个基础解系。

                        \item 齐次线性方程组有非零解 $\Longleftrightarrow r(A)<n$。
                              通解为:
                              \[
                                  x = k_1\xi_1 + \cdots + k_{n-r}\xi_{\,n-r}, \qquad r=r(A).
                              \]
                    \end{enumerate}

                    % 与系数的关系
              \item 解与系数的关系

                    对于齐次线性方程组
                    \[
                        \begin{cases}
                            a_{11}x_1+\cdots+a_{1n}x_n=0, \\
                            \vdots                        \\
                            a_{m1}x_1+\cdots+a_{mn}x_n=0,
                        \end{cases}
                    \]
                    若 $x=\beta=[b_1,\dots,b_n]^T$ 是其解,则
                    \[
                        \alpha_i\beta=0,\qquad \alpha_i=[a_{i1},\dots,a_{in}],
                    \]
                    即 **系数矩阵的行向量与解向量正交**。
          \end{enumerate}

          %==========================
          % 2. 非齐次线性方程组
          %==========================
    \item 非齐次线性方程组 $Ax=b$ \DOne

          \begin{enumerate}
              \item 解的性质

                    若 $\eta_1,\eta_2$ 都是 $Ax=b$ 的解,则两者差
                    \[
                        \eta_1-\eta_2
                    \]
                    必是齐次方程组 $Ax=0$ 的解。

              \item 解的情形
                    \begin{enumerate}
                        \item $r(A)=r([A,b])=n$:有唯一解;
                        \item $r(A)=r([A,b])<n$:有无穷多解;
                        \item $r(A)\ne r([A,b])$:无解。
                    \end{enumerate}

              \item 通解

                    设 $r(A)=r<n$,齐次方程 $Ax=0$ 的基础解系为
                    \[
                        \xi_1,\xi_2,\dots,\xi_{n-r},
                    \]
                    非齐次方程 $Ax=b$ 的一个特解为 $\eta^\cdot$,则通解为
                    \[
                        x=\eta^\cdot+k_1\xi_1+\cdots+k_{n-r}\xi_{\,n-r}.
                    \]
          \end{enumerate}

\end{enumerate}


%==========================================================
% 第二章:线性方程组问题
%==========================================================
\section{线性方程组问题}

\begin{enumerate}

    \item 一般求解问题

          %-----------------------------------------
    \item 公共解问题
          %-----------------------------------------
          \begin{enumerate}
              \item 齐次线性方程组的公共非零解
                    \begin{enumerate}
                        \item 对于 $Ax=0$ 与 $Bx=0$,求公共解实质上是联立求解。
                        \item 两者有公共非零解的充要条件是
                              \[
                                  \begin{bmatrix}A\\B\end{bmatrix}x=0
                              \]
                              有非零解,即
                              \[
                                  r\!\begin{pmatrix}A\\B\end{pmatrix}<n.
                              \]
                    \end{enumerate}

              \item 非齐次线性方程组公共解
                    \begin{enumerate}
                        \item “公共解 = 联立解” 是核心思想。
                        \item 若 (I) $Ax=\beta$ 与 (II) $Bx=\gamma$ 均有解,则存在公共解的充要条件为
                              \[
                                  r\!\begin{pmatrix}A\\B\end{pmatrix}
                                  =
                                  r\!\begin{pmatrix}A&\beta\\B&\gamma\end{pmatrix}.
                              \]
                    \end{enumerate}
          \end{enumerate}

          %-----------------------------------------
    \item 同解问题 \DOne+\DTwoTwo
          %-----------------------------------------
          \begin{enumerate}

              % 齐次方程同解
              \item 齐次线性方程组
                    \begin{enumerate}
                        \item $Ax=0$ 与 $Bx=0$ 同解的充要条件:
                              \[
                                  r\!\begin{pmatrix}A\\B\end{pmatrix}=r(A)=r(B).
                              \]
                        \item 其行向量组必须等价。
                        \item 存在矩阵 $P,Q$ 使
                              \[
                                  B=PA,\qquad A=QB.
                              \]
                    \end{enumerate}

                    % 非齐次同解
              \item 非齐次线性方程组

                    设 (I) $Ax=\beta$,(II) $Bx=\gamma$ 均有解,则

                    \[
                        \text{(I) 与 (II) 同解}
                        \Longleftrightarrow
                        \begin{cases}
                            Ax=0 \text{ 与 } Bx=0 \text{同解}, \\
                            \text{(I),(II) 有公共解},
                        \end{cases}
                    \]

                    进一步等价于

                    \[
                        r\!\begin{pmatrix}A&\beta\\B&\gamma\end{pmatrix}
                        =
                        r\!\begin{pmatrix}A\\B\end{pmatrix}
                        =
                        r(A)=r(B),
                    \]

                    即 $[A,\beta]$ 与 $[B,\gamma]$ 的行向量组等价。

          \end{enumerate}

\end{enumerate}


%==========================================================
% 第三章:几何意义
%==========================================================
\section{线性方程组的几何意义}

设
\[
    \begin{cases}
        a_1x+b_1y+c_1z=d_1, \\
        a_2x+b_2y+c_2z=d_2, \\
        a_3x+b_3y+c_3z=d_3.
    \end{cases}
\]

记
\[
    A=\begin{bmatrix}
        a_1 & b_1 & c_1 \\
        a_2 & b_2 & c_2 \\
        a_3 & b_3 & c_3
    \end{bmatrix},
    \qquad
    \overline{A}=\begin{bmatrix}
        a_1 & b_1 & c_1 & d_1 \\
        a_2 & b_2 & c_2 & d_2 \\
        a_3 & b_3 & c_3 & d_3
    \end{bmatrix}.
\]

$\Pi_i$ 表示第 $i$ 个平面 $a_ix+b_iy+c_iz=d_i$;
$\boldsymbol{\alpha}_i=[a_i,b_i,c_i]$ 为平面法向量(即 $A$ 的行向量);
$\boldsymbol{\beta}_i=[a_i,b_i,c_i,d_i]$ 为 $\overline{A}$ 的行向量。