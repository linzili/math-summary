\XIANchapter{行列式}

\section{具体型行列式的计算:$a_{ij}$已给出}
\begin{enumerate}
    \item 化为基本型行列式
          \DTwoThree
          \begin{enumerate}
              \item 主对角线行列式
                    $$\begin{vmatrix}
                            a_{11} & a_{12} & \cdots & a_{1n} \\
                            0      & a_{22} & \cdots & a_{2n} \\
                            \vdots & \vdots &        & \vdots \\
                            0      & 0      & \cdots & a_{nn}
                        \end{vmatrix}=
                        \begin{vmatrix}
                            a_{11} & 0      & \cdots & 0      \\
                            a_{21} & a_{22} & \cdots & 0      \\
                            \vdots & \vdots &        & \vdots \\
                            a_{n1} & a_{n2} & \cdots & a_{nn}
                        \end{vmatrix}=
                        \begin{vmatrix}
                            a_{11} & 0      & \cdots & 0      \\
                            0      & a_{22} & \cdots & 0      \\
                            \vdots & \vdots &        & \vdots \\
                            0      & 0      & \cdots & a_{nn}
                        \end{vmatrix}=\prod_{i=1}^na_{ii}.$$
              \item 副对角线行列式
                    $$\begin{aligned}
                            \begin{vmatrix}
                                a_{11} & a_{12} & \cdots & a_{1,n-1} & a_{1n} \\
                                a_{21} & a_{22} & \cdots & a_{2,n-1} & 0      \\
                                \vdots & \vdots &        & \vdots    & \vdots \\
                                a_{n1} & 0      & \cdots & 0         & 0
                            \end{vmatrix} & =
                            \begin{vmatrix}
                                0      & \cdots & 0         & a_{1n} \\
                                0      & \cdots & a_{2,n-1} & a_{2n} \\
                                \vdots &        & \vdots    & \vdots \\
                                a_{n1} & \cdots & a_{n,n-1} & a_{nn}
                            \end{vmatrix}=
                            \begin{vmatrix}
                                0      & \cdots & 0         & a_{1n} \\
                                0      & \cdots & a_{2,n-1} & 0      \\
                                \vdots &        & \vdots    & \vdots \\
                                a_{n1} & \cdots & 0         & 0
                            \end{vmatrix}                                                                     \\
                                                                             & =(-1)^{\frac{n(n-1)}{2}}a_{1n}a_{2,n-1}\cdots a_{n1}.
                        \end{aligned}$$
              \item 拉普拉斯展开式

                    设$A$为$m$阶矩阵,$B$为$n$阶矩阵,则
                    $$\begin{gathered}
                            \begin{vmatrix}
                                A & O \\
                                O & B
                            \end{vmatrix}=
                            \begin{vmatrix}
                                A & C \\
                                O & B
                            \end{vmatrix}=
                            \begin{vmatrix}
                                A & O \\
                                C & B
                            \end{vmatrix}=|A||B|, \\
                            \begin{vmatrix}
                                O & A \\
                                B & O
                            \end{vmatrix}=
                            \begin{vmatrix}
                                C & A \\
                                B & O
                            \end{vmatrix}=
                            \begin{vmatrix}
                                O & A \\
                                B & C
                            \end{vmatrix}=(-1)^{mn}|A||B|.
                        \end{gathered}$$
              \item 范德蒙德行列式
                    $$\begin{vmatrix}
                            1         & 1         & \cdots & 1         \\
                            x_1       & x_2       & \cdots & x_n       \\
                            x_1^2     & x_2^2     & \cdots & x_n^2     \\
                            \vdots    & \vdots    &        & \vdots    \\
                            x_1^{n-1} & x_2^{n-1} & \cdots & x_n^{n-1}
                        \end{vmatrix}=\prod_{1\leq i<j\leq n}(x_j-x_i),n\geq2.$$
          \end{enumerate}
          \begin{note}{行列式计算思路}{行列式计算思路}
              \begin{enumerate}
                  \item 若所给行列式就是基本形或接近基本形,则直接套公式或经过简单处理化成基本形后套公式.
                  \item 简单处理的手段: \DTwoOne
                        \begin{enumerate}
                            \item 按零元素多的行或列展开;
                            \item 用行列式的性质对差别最小的“对应位置元素”进行处理,尽可能多地化出零元素,再按此行或列展开;
                            \item 对于行和或列和相等的情形,将所有列加到第1列或将所有行加到第1行,提出公因式,再用上述方法,等等.
                        \end{enumerate}
                  \item 具体型行列式的元素中若含$x$,则其为$x$的多项式.
              \end{enumerate}
          \end{note}
    \item 加边法
          \DTwoThree
          对于某些一开始不宜使用“互换”“倍乘”“倍加”性质的行列式,可以考虑使用加边法:$n$阶行列式中添加一行、一列升至$n+1$阶行列式.若添加在第 $1$ 列,且添加的是$[ 1, 0,...,0]^\mathrm{T}$,则第 $1$ 行其余元素可以任意添加,行列式的值不变,即
          $$D_n=
              \begin{vmatrix}
                  a_{11} & a_{12} & \cdots & a_{1n} \\
                  a_{21} & a_{22} & \cdots & a_{2n} \\
                  \vdots & \vdots &        & \vdots \\
                  a_{n1} & a_{n2} & \cdots & a_{nn}
              \end{vmatrix}=
              \begin{vmatrix}
                  1      & *      & *      & \cdots & *      \\
                  0      & a_{11} & a_{12} & \cdots & a_{1n} \\
                  0      & a_{21} & a_{22} & \cdots & a_{2n} \\
                  \vdots & \vdots & \vdots &        & \vdots \\
                  0      & a_{n1} & a_{n2} & \cdots & a_{nn}
              \end{vmatrix},$$
    \item 递推法
          \DOne+\DTwoOne

          \begin{enumerate}
              \item 建立递推公式,即建立$D_n$ 与$D_{n-1}$的关系,有些复杂的题甚至要建立$D_n$,$D_{n-1}$与$D_{n-2}$的关系.
              \item$D_{n-1}$与$D_n$要有完全相同的元素分布规律,只是$D_n-1$ 比 $D_n$ 低了一阶.
          \end{enumerate}
    \item 数学归纳法
          \PThree

          涉及$n$阶行列式的证明型计算问题,即告知行列式计算结果,让考生证明之,可考虑数学归纳法.
          \begin{enumerate}
              \item 第一数学归纳法 (适用于 $F(D_{n},D_{n-1})=0$):
                    \begin{enumerate}
                        \item 验证当 $n=1$ 时,命题成立;
                        \item 假设当 $n=k$ (≥2)时,命题成立;
                        \item 证明当 $n=k+1$ 时,命题成立.
                    \end{enumerate}
                    则命题对任意正整数 $n$ 成立.
              \item 第二数学归纳法 (适用于 $F(D_{n},D_{n-1},D_{n-2})=0$):
                    \begin{enumerate}
                        \item 验证当 $n=1$ 和 $n=2$ 时,命题成立;
                        \item 假设当 $n<k$ 时,命题成立;
                        \item 证明当 $n=k$ ($≥3$)时,命题成立.
                    \end{enumerate}
                    则命题对任意正整数 $n$ 成立.

          \end{enumerate}
\end{enumerate}
\section{抽象型行列式的计算:$a_{ij}$未给出}

\begin{enumerate}
    \item 用行列式的性质

          用行列式的性质将所求行列式进一步化成已知行列式.
    \item 用矩阵知识
          \begin{enumerate}
              \item 设$C=AB,A,B$为同阶方阵,则$|C|=|AB|=|A||B|.$
              \item 设$C=A+B,A,B$ 为同阶方阵,则 $|C|=|A+B|$,但由于$|A+B|$ 不一定等于$|A|+|B|$,故需对$|A+B|$作恒等变形,转化为矩阵乘积的行列式.这里的恒等变形一般是:
                    \begin{enumerate}
                        \item 由题设条件,如 $E=AA^\mathrm{T}$;
                        \item 用$E=AA^{-1}$等.
                    \end{enumerate}
          \end{enumerate}
\end{enumerate}