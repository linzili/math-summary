\GAOchapter{数列极限}

\section{含 $f(x_n, x_{n+1})$ 的等式关系}
\DTwoThree

\begin{enumerate}
    \item 若 $x_{n+1} = f(x_n)$,且 $f(x)$ 可导,且存在常数 $k<1$ 使得
          \[
              |f'(x)| \le k,
          \]
          则由 \textbf{压缩映射原理},数列 $\{x_n\}$ 收敛。

    \item 若 $x_{n+1} = f(x_n)$,但 $f(x)$ 不易求导或 $|f'(x)| \nleq k < 1$,可从以下角度分析:
          \begin{enumerate}
              \item 比较 $x_{n+1}$ 与 $x_n$ 的大小,判断单调性;
              \item 作差:$x_{n+1} - x_n$,根据符号判断单调性;
              \item 作商:$\dfrac{x_{n+1}}{x_n}$(当 $x_{n+1}$ 与 $x_n$ 同号时),根据与 1 的大小比较单调性;
              \item 结合题设提示(往往在第(1)问给出),判断有界性或单调性。
          \end{enumerate}
\end{enumerate}

\section{含 $f(x_n, x_{n+1})$ 的不等式关系}

若题设给出不等式关系,可通过比较 $x_n$ 与 $x_{n+1}$ 的大小,
确定 $\{x_n\}$ 的上、下界,从而判断其是否单调有界。

\section{初值 $x_1$ 对递推式收敛性的影响}

若递推式为 $x_{n+1}=f(x_n)$,且初值 $x_1$ 仅给出取值范围,
需根据 $x_1$ 所在区间分情况讨论收敛或发散情形。

\section{双通项数列问题($a_n, b_n$ 型)}
\DTwoOne+\DTwoThree

\begin{enumerate}
    \item 将 $a_n,b_n$ 满足的式子联立,消去其中一个,化为单通项递推问题。
          常用技巧:
          \begin{enumerate*}[itemjoin=\quad]
              \item 恒等变形;
              \item 无穷小比阶;
              \item 放缩比较;
              \item 函数单调性。
          \end{enumerate*}

    \item 令 $c_n = \dfrac{a_n}{b_n}$,转化为求 $\lim\limits_{n\to\infty} c_n$。
          常用技巧:
          \begin{enumerate*}[itemjoin=\quad]
              \item 单调有界准则;
              \item 夹逼准则;
              \item 极限保号性。
          \end{enumerate*}
\end{enumerate}

\section{复合函数的极限}
\DTwoOne

\begin{enumerate}
    \item \textbf{因变量极限定理}

          设 $y=f[g(x)]$,令 $u=g(x)$,若
          \[
              \begin{cases}
                  \displaystyle \lim_{x\to x_0} g(x) = u_0, \\[3pt]
                  \displaystyle \lim_{u\to u_0} f(u) = a,   \\[3pt]
                  g(x)\ne u_0 \text{ 当 } x\ne x_0,
              \end{cases}
          \]
          则有
          \[
              \lim_{x\to x_0} f[g(x)] = a.
          \]

    \item \textbf{中间变量极限定理}

          若 $\{u_n\}$ 取自有限区间 $I$,且 $f(x)$ 在 $I$ 上严格单调,若
          \[
              \lim_{n\to\infty} f(u_n) \text{ 存在},
          \]
          则 $\lim_{n\to\infty} u_n$ 也存在。
\end{enumerate}