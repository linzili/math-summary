\GAOchapter{一元函数微分学的计算}

\section{泰勒展开法}
\DTwoThree

若 $f(x)$ 是 $e^x$、$\ln(1+x)$、$\sin x$、$\cos x$、$\frac{1}{1+x}$ 等函数或其“亲戚”,
可通过适当的恒等变形,将其化为已知展开式形式,再利用\textbf{泰勒展开的唯一性}求得 $f^{(n)}(x_0)$。

\begin{enumerate}
    \item \textbf{通分的逆运算(瓦解敌人,各个击破)}
          \[
              \frac{1}{x(x+1)} = \frac{1}{x} - \frac{1}{x+1}.
          \]

    \item \textbf{对数运算性质}
          \[
              \ln(2+x) = \ln\!\left[2\!\left(1+\frac{x}{2}\right)\right]
              = \ln 2 + \ln\!\left(1+\frac{x}{2}\right).
          \]

    \item \textbf{三角恒等式}
          \[
              \sin^2 x = \frac{1-\cos 2x}{2}.
          \]

    \item \textbf{广义化:} 若 $x \to 0$,可作类似极限形式的推广分析。

    \item \textbf{“偏导数化”:} 若函数中含多元关系,可视作 $x$ 的偏导形式进行处理。
\end{enumerate}

\section{莱布尼茨公式法}
\DTwoThree

若函数含有可有限次求导的多项式部分,如
\[
    f(x)\cdot (a x^2 + b x + c),
\]
例如 $e^x(1+x^2)$,则用\textbf{莱布尼茨乘积求导公式}更为高效。
因为 $(a x^2 + b x + c)^{(3)} = 0$,使用该公式后仅保留前三项。

常用 $n$ 阶导数公式($n$ 为正整数)如下:
\[
    \begin{aligned}
        (a^x)^{(n)}                      & = a^x (\ln a)^n,                       \\
        \left(\frac{1}{1+x}\right)^{(n)} & = \frac{(-1)^n n!}{(1+x)^{n+1}},       \\
        [\ln(1+x)]^{(n)}                 & = \frac{(-1)^{n-1}(n-1)!}{(1+x)^n},    \\
        (\sin x)^{(n)}                   & = \sin\!\left(x+n\frac{\pi}{2}\right), \\
        (\cos x)^{(n)}                   & = \cos\!\left(x+n\frac{\pi}{2}\right).
    \end{aligned}
\]

\section{求导转化法}
\DTwoThree
\begin{enumerate}
    \item 若函数既非“亲戚”,又不便于恒等变形,可先求一阶、二阶导,再转化为熟悉形式。
          例如:
          \[
              y = \arctan x, \quad y' = \frac{1}{1+x^2} \Rightarrow y'(1+x^2)=1,
          \]
          从而转化为“莱布尼茨公式法”情形。

    \item 必须熟记前述莱布尼茨法中的常用五个公式,并掌握其\textbf{递推规律}。
\end{enumerate}

\section{特殊点的高阶导数}
\begin{enumerate}
    \item 分段函数的分段点;
    \item 含绝对值的函数。
\end{enumerate}


\section{奇偶与周期函数的高阶导数}

\begin{enumerate}
    \item 若 $f(x)$ 为奇函数:
          \[
              \begin{cases}
                  f^{(2n)}(x)\text{ 为奇函数}, \\
                  f^{(2n+1)}(x)\text{ 为偶函数}.
              \end{cases}
          \]
    \item 若 $f(x)$ 为偶函数:
          \[
              \begin{cases}
                  f^{(2n)}(x)\text{ 为偶函数}, \\
                  f^{(2n+1)}(x)\text{ 为奇函数}.
              \end{cases}
          \]
    \item 若 $f(x)$ 为周期函数,则 $f^{(n)}(x)$ 亦为周期函数。
\end{enumerate}


\section{隐函数的二阶导}

若 $F(x, y) = 0$,且 $y = y(x)$,则对 $x$ 求导:
\[
    F_x + F_y y' = 0 \Rightarrow y' = -\frac{F_x}{F_y}.
\]
再对该式关于 $x$ 求导,得:
\[
    y'' = -\frac{F_{xx} + 2F_{xy}y' + F_{yy}(y')^2}{F_y}.
\]


\section{参数方程的二阶导}

若
\[
    \begin{cases}
        x = x(t), \\
        y = y(t),
    \end{cases}
\]
则
\[
    \frac{\mathrm{d}y}{\mathrm{d}x}
    = \frac{\mathrm{d}y/\mathrm{d}t}{\mathrm{d}x/\mathrm{d}t}
    = \frac{y'(t)}{x'(t)} = \varphi(t),
\]
进而
\[
    \frac{\mathrm{d}^2y}{\mathrm{d}x^2}
    = \frac{\mathrm{d}\varphi/\mathrm{d}t}{\mathrm{d}x/\mathrm{d}t}
    = \frac{\varphi'(t)}{x'(t)}.
\]


\section{反函数的二阶导}

若 $y=f(x)$ 单调且二阶可导,且 $f'(x)\ne0$,则存在反函数 $x=\varphi(y)$。

设
\[
    f'(x)=y'_x,\quad \varphi'(y)=x'_y,
\]
则
\[
    y'_x=\frac{1}{x'_y}, \qquad
    y''_{xx}=-\frac{x''_{yy}}{(x'_y)^3}.
\]
反过来,有
\[
    x'_y = \frac{1}{y'_x}, \qquad
    x''_{yy} = -\frac{y''_{xx}}{(y'_x)^3}.
\]