\GAOchapter{多元函数积分学的预备知识}

\section{方向导数}

\begin{enumerate}
    \item \textbf{定义}

          设函数 $f(x,y)$ 在点 $(a,b)$ 及其邻域内有定义,单位向量
          $l = (\cos\alpha, \cos\beta)$ 给出方向.若极限
          \[
              \lim_{t \to 0^+} \frac{f(a + t\cos\alpha,\, b + t\cos\beta) - f(a,b)}{t}
          \]
          存在,则称其为函数 $f(x,y)$ 在点 $(a,b)$ 沿方向 $l$ 的\textbf{方向导数},记作
          \[
              \left.\frac{\partial f}{\partial l}\right|_{(a,b)}.
          \]

    \item \textbf{计算公式}

          若 $f(x,y)$ 在点 $(a,b)$ 可微,则沿任意方向 $l = (\cos\alpha, \cos\beta)$ 的方向导数存在,且
          \[
              \left.\frac{\partial f}{\partial l}\right|_{(a,b)}
              = f_x'(a,b)\cos\alpha + f_y'(a,b)\cos\beta.
          \]

          该式表明方向导数是偏导数的线性组合.
\end{enumerate}

\section{梯度的概念}

设 $f(x,y)$ 在点 $(a,b)$ 处可微,定义其\textbf{梯度(Gradient)}为
\[
    \nabla f(a,b) = (f_x'(a,b),\, f_y'(a,b)).
\]

梯度的几何意义如下:
\begin{enumerate}
    \item 梯度方向是函数增长最快的方向;
    \item 梯度的模表示函数在该点处的最大方向导数:
          \[
              \max_{|l|=1}\left.\frac{\partial f}{\partial l}\right|_{(a,b)} = |\nabla f(a,b)|.
          \]
\end{enumerate}

\section{多元函数的泰勒多项式}

若 $f(x,y)$ 在点 $X_0(x_0,y_0)$ 处二阶偏导连续,记
\[
    \Delta X = (\Delta x,\Delta y) = (x - x_0, y - y_0),
\]
则其二阶泰勒多项式为
\[
    f(x,y) \approx f(x_0,y_0)
    + (f_x', f_y')|_{X_0}
    \begin{pmatrix}\Delta x\\\Delta y\end{pmatrix}
    + \frac{1}{2}
    (\Delta x,\Delta y)
    \begin{pmatrix}
        f_{xx}'' & f_{xy}'' \\
        f_{yx}'' & f_{yy}''
    \end{pmatrix}_{X_0}
    \begin{pmatrix}\Delta x\\\Delta y\end{pmatrix}.
\]

若高阶项趋于零,则函数在该点可用此近似.

\section{空间曲线的切线与法平面}

设空间曲线
\[
    L:\begin{cases}
        x = x(t), \\
        y = y(t), \\
        z = z(t),
    \end{cases}
    \quad t \in [\alpha, \beta],
\]
其中 $x(t),y(t),z(t)$ 一阶连续可导,且不全为零.

\begin{enumerate}
    \item \textbf{切向量}
          \[
              \boldsymbol{\tau} = (x'(t_0), y'(t_0), z'(t_0)).
          \]
    \item \textbf{切线方程}
          \[
              \frac{x - x(t_0)}{x'(t_0)}
              = \frac{y - y(t_0)}{y'(t_0)}
              = \frac{z - z(t_0)}{z'(t_0)}.
          \]
    \item \textbf{法平面方程}
          \[
              x'(t_0)[x - x(t_0)] + y'(t_0)[y - y(t_0)] + z'(t_0)[z - z(t_0)] = 0.
          \]
\end{enumerate}

\section{空间曲面的切平面与法线}

设光滑曲面
\[
    F(x,y,z) = 0,
\]
且在点 $(a,b,c)$ 处有
\[
    F_x'^2 + F_y'^2 + F_z'^2 \ne 0.
\]

\begin{enumerate}
    \item \textbf{法向量}
          \[
              \boldsymbol{n} = (F_x'(a,b,c),\, F_y'(a,b,c),\, F_z'(a,b,c)).
          \]

    \item \textbf{切平面方程}
          \[
              F_x'(a,b,c)(x - a)
              + F_y'(a,b,c)(y - b)
              + F_z'(a,b,c)(z - c)
              = 0.
          \]

    \item \textbf{法线方程}
          \[
              \frac{x - a}{F_x'(a,b,c)}
              = \frac{y - b}{F_y'(a,b,c)}
              = \frac{z - c}{F_z'(a,b,c)}.
          \]
\end{enumerate}

\textbf{几何意义:}
法向量垂直于曲面,切平面在该点处与曲面“最贴近”,法线即过该点且垂直于切平面的直线.