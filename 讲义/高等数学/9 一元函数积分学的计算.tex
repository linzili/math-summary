\GAOchapter{一元函数积分学的计算}

\section{恒等变形法}

通过代数变形,将被积函数化为基本积分公式中的形式,从而直接求得原函数。

常用方法:
\begin{enumerate}
    \item 配方、通分、分解因式;
    \item 利用奇偶性、对称性;
    \item 利用三角恒等变换或平方差公式;
    \item 拆分常数倍:$\displaystyle \int (a f(x) + b g(x))dx = a\int f(x)dx + b\int g(x)dx$。
\end{enumerate}

\section{第一类换元法(凑微分法)}

\[
    \int f[g(x)]g'(x)\,dx = \int f[g(x)]\,d[g(x)] = \int f(u)\,du.
\]

若被积函数是 $g(x)$ 的函数与 $g'(x)$ 的乘积,则“凑微分”后即可直接积分。

\textbf{技巧总结:}
\begin{enumerate}
    \item 常见凑法:若分母是 $x^2+1$、$1-x^2$、$x\sqrt{1-x^2}$ 等;
    \item 若被积函数可写成 $f'(x)/f(x)$,则 $\displaystyle \int \frac{f'(x)}{f(x)}dx = \ln|f(x)| + C$。
\end{enumerate}

\section{第二类换元法(代换法)}

\[
    \int f(x)\,dx = \int f[g(u)]g'(u)\,du.
\]

当被积函数复杂、含根号或分式时,可令 $x=g(u)$,使 $f(x)$ 化为较简单的 $h(u)$。

\textbf{常见换元:}
\begin{enumerate}
    \item 去根号:如 $x=a\sin\theta,\ a\tan\theta,\ a\sec\theta$;
    \item 分母变单项:如 $\sqrt[n]{\frac{ax+b}{cx+d}} = u$;
    \item 含对称结构:如 $x+\frac{1}{x}=t$;
    \item 对称区间换元:$x=a-t$、$x=\frac{1}{t}$ 等。
\end{enumerate}

\textbf{核心目标:} 通过换元使被积函数\textbf{简化为多项式或基本初等函数形式}。

\section{分部积分法}

\[
    \int u\,dv = uv - \int v\,du.
\]

\textbf{口诀:} “一降一不变,优先降幂项”。

\begin{enumerate}
    \item 若为 $\int f(x)g(x)\,dx$,选取 $u$ 为\textbf{易微分、难积分}的部分;
    \item 若为单个函数(如 $x e^x,\, x\sin x$ 等),仍可设 $dv$ 为整项的可积分部分;
    \item 常见四类应用:
          \begin{enumerate}
              \item \textbf{再现法:} 积分两边含原式,建立方程;
              \item \textbf{抵消法:} 通过两次积分消除复杂项;
              \item \textbf{递推法:} 构造 $I_n$ 递推公式;
              \item \textbf{定积分结合法:} 特殊上限下限抵消项。
          \end{enumerate}
\end{enumerate}

\section{有理函数的积分}

形如 $\displaystyle \int \frac{P_n(x)}{Q_m(x)}dx$($n<m$)的积分称为有理函数积分。

\textbf{常用步骤:}
若 $n\ge m$,先作多项式除法;若 $n<m$,再作部分分式分解。

\textbf{四类基本积分:}
\begin{enumerate}
    \item $\displaystyle \int\frac{A}{ax+b}\,dx = \frac{A}{a}\ln|ax+b| + C$;
    \item $\displaystyle \int\frac{A}{(ax+b)^k}\,dx = \frac{A}{a(1-k)}(ax+b)^{1-k} + C \quad (k>0,\ k\ne1)$;
    \item $\displaystyle \int\frac{Bx+C}{px^2+qx+r}\,dx\ (q^2-4pr<0)$:
          \[
              \text{如 } \int\frac{x+1}{x^2+x+1}dx = \frac{1}{2}\int\frac{2x+1}{x^2+x+1}dx + \frac{1}{2}\int\frac{dx}{\left(x+\frac{1}{2}\right)^2+\left(\frac{\sqrt{3}}{2}\right)^2}.
          \]
    \item $\displaystyle \int\frac{Bx+C}{(px^2+qx+r)^k}\,dx\ (q^2-4pr<0,k>0,k\ne1)$:
          \[
              \text{如 } \int\frac{x+1}{(x^2+x+1)^2}dx = \frac{1}{2}\int\frac{2x+1}{(x^2+x+1)^2}dx + \frac{1}{2}\int\frac{dx}{\left[\left(x+\frac{1}{2}\right)^2+\left(\frac{\sqrt{3}}{2}\right)^2\right]^2}.
          \]
\end{enumerate}

\section{三角有理式的积分}

形如 $\displaystyle \int R(\sin x, \cos x)\,dx$ 的积分称为三角有理式积分。

\subsection{全角换元法}
\begin{enumerate}
    \item 若 $R(-\sin x, \cos x) = -R(\sin x, \cos x)$,令 $t = \cos x$;
    \item 若 $R(\sin x, -\cos x) = -R(\sin x, \cos x)$,令 $t = \sin x$。
\end{enumerate}

\subsection{半角万能换元法}
设 $t=\tan\frac{x}{2}$,则
\[
    \sin x = \frac{2t}{1+t^2}, \quad \cos x = \frac{1-t^2}{1+t^2}, \quad dx = \frac{2\,dt}{1+t^2}.
\]
代入后化为 $t$ 的有理函数积分。

\section{定积分的计算(牛顿–莱布尼茨公式)}

若 $F'(x)=f(x)$,则:
\[
    \int_a^b f(x)\,dx = F(b)-F(a).
\]

\subsection{反常积分的牛顿–莱布尼茨公式}
\begin{enumerate}
    \item $\displaystyle \int_a^{+\infty} f(x)\,dx = \lim_{x\to+\infty}F(x)-F(a)$;
    \item $\displaystyle \int_{-\infty}^b f(x)\,dx = F(b)-\lim_{x\to-\infty}F(x)$;
    \item 若 $x=c$ 为瑕点,则
          \[
              \int_a^b f(x)\,dx = \int_a^c f(x)\,dx + \int_c^b f(x)\,dx.
          \]
\end{enumerate}

\section{变限积分函数的求导}

\begin{enumerate}
    \item \textbf{直接求导型:}
          \[
              \left[\int_{a}^{\varphi(x)} f(t)\,dt\right]' = f[\varphi(x)]\varphi'(x),\quad
              \left[\int_{\varphi_1(x)}^{\varphi_2(x)} f(t)\,dt\right]' = f[\varphi_2(x)]\varphi_2'(x) - f[\varphi_1(x)]\varphi_1'(x).
          \]
    \item \textbf{换元求导型:} 先换元,再代入求导。
    \item \textbf{拆分求导型:} 若带绝对值或分段区间,需拆分处理。
    \item \textbf{换序积分型:} 累次积分中先交换积分次序以简化计算。
\end{enumerate}

\section{分段函数的积分}

\begin{enumerate}
    \item \textbf{不定积分:} 各段分别求原函数,检查分段点连续性;
    \item \textbf{定积分:} 按区间分段后相加;
    \item \textbf{变限积分:} 随 $x$ 变化分情况讨论,$F(x)=\int_a^x f(t)dt$ 为累加函数。
\end{enumerate}

\section{几何法}

定积分具有面积意义,可快速计算几何型积分:
\[
    \int_{-a}^{a}\sqrt{a^2-x^2}\,dx = \frac{\pi a^2}{2}, \quad
    \int_{0}^{a}\sqrt{x(2a-x)}\,dx = \frac{\pi a^2}{4}.
\]
由此还可得:
\[
    \int_{0}^{2a}\sqrt{x(2a-x)}\,dx = \frac{\pi a^2}{2}, \quad
    \int_{a}^{2a}\sqrt{x(2a-x)}\,dx = \frac{\pi a^2}{4}.
\]

\textbf{应用提示:} 若被积函数可化为圆弧、椭圆或抛物线截面形式,可直接用面积公式代替积分计算。