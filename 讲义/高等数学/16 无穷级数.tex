\GAOchapter{无穷级数}

\section{级数 $\sum_{n=1}^{\infty} u_n$ 的敛散性判别}

\subsection{第一步:计算 $\displaystyle \lim_{n \to \infty} u_n$}

若 $\lim_{n \to \infty} u_n \neq 0$,则级数发散;若等于 0,则需进一步判别。

\begin{example}{}{}
    判别级数 $\displaystyle \sum_{n=2}^{\infty} \left(1-\frac{1}{n}\right)^n$ 的敛散性。
\end{example}
\begin{solution}
    $\displaystyle \lim_{n\to\infty}\left(1-\frac{1}{n}\right)^n=e^{-1}\neq0$,故该级数发散。
\end{solution}

\begin{example}{}{}
    判别级数 $\displaystyle \sum_{n=2}^{\infty}\left(2-\frac{2}{n}\right)^n\ln\left(\frac{1}{2^n}+1\right)$ 的敛散性。
\end{example}
\begin{solution}
    $\displaystyle \lim_{n\to\infty}\left(2-\frac{2}{n}\right)^n\ln\left(\frac{1}{2^n}+1\right)
        =\lim_{n\to\infty}\left(1-\frac{1}{n}\right)^n=e^{-1}\neq0$,
    故该级数发散。
\end{solution}


\subsection{第二步:研究 $u_n$ 的结构特征}

\begin{enumerate}
    \item \textbf{见到 $f(n)$ 型通项}

          \begin{enumerate}
              \item \textbf{恒等变形与放缩技巧}
                    \begin{itemize}
                        \item 若含 $a^n - b^n$,提取 $a^n$ 得 $a^n[1 - (b/a)^n]$。
                        \item 含 $\ln$ 时:
                              \[
                                  \ln b - \ln a = \ln\frac{b}{a}, \quad
                                  \ln(1+n) < n, \quad \ln n < n.
                              \]
                        \item 含 $e^{f(n)}$ 或复杂 $f(n)\pm g(n)$ 时,作 \textbf{泰勒展开},与 $\frac{1}{n^p}$ 比较阶。
                    \end{itemize}

              \item \textbf{含 $(-1)^n$ 时}
                    \begin{itemize}
                        \item 若 $(-1)^n$ 不影响正负性,可去掉。
                        \item 若影响,考虑交错级数判别(莱布尼茨判别法)。
                        \item 例如:
                              \[
                                  \sum_{n=2}^{\infty}\frac{(-1)^n}{\sqrt{n}+(-1)^n}
                                  =\sum_{n=2}^{\infty}\frac{(-1)^n\sqrt{n}}{n-1}-\sum_{n=2}^{\infty}\frac{1}{n-1},
                              \]
                              发散。
                    \end{itemize}
          \end{enumerate}

    \item \textbf{见到 $f(n)$ 与 $f'(n)$}
          \begin{itemize}
              \item 用 \textbf{拉格朗日中值定理}。
              \item 或用求和恒等式:
                    \[
                        \sum_{k=1}^{n} [f(k+1)-f(k)] = f(n+1)-f(1)。
                    \]
          \end{itemize}

    \item \textbf{见到 $f(n)-f(n-1)$}
          优先考虑:
          \begin{itemize}
              \item \textbf{有理化}(处理分子);
              \item \textbf{通分}(处理分母)。
          \end{itemize}

    \item \textbf{见到 $f(a_n)$}
          \begin{itemize}
              \item 若 $\{a_n\}$ 收敛于 $a>0$,则对充分大 $n$,有 $a_n>\tfrac{a}{2}>0$。
              \item 若 $\lim n^2 a_n = a>0$,则 $|a_n|\le \tfrac{M}{n^2}$。
              \item 若 $\lim n^2(a_n-b_n)=k<\infty$,则 $\sum (a_n-b_n)$ 收敛。
              \item 若 $\lim_{n\to\infty}a_n=p$,则
                    \[
                        \sum\frac{1}{n^{a_n}}
                        \begin{cases}
                            \text{发散,} & p<1, \\
                            \text{收敛,} & p>1, \\
                            \text{不定,} & p=1。
                        \end{cases}
                    \]
          \end{itemize}

    \item \textbf{见到 $f(a_n,a_{n+1})$}
          \begin{itemize}
              \item 若给出 $f(a_n,a_{n+1})$ 的关系式,可尝试写为 $S_n$ 或差分形式 $f(a_{n+1})-f(a_n)$。
          \end{itemize}

    \item \textbf{见到 $f(a_n,b_n)$}
          \begin{itemize}
              \item 建立关系:
                    \[
                        a_n b_n = n a_n \cdot \frac{b_n}{n}, \quad
                        a_n = (a_n-b_n)+b_n,
                    \]
                    或令 $\frac{b_n}{a_n}=c_n$,转化为单变量函数 $f(c_n)$。
          \end{itemize}

    \item \textbf{见到 $f(a_n, n^p)$}
          常用不等式:
          \[
              |ab|\le \frac{a^2+b^2}{2}。
          \]

    \item \textbf{见到 $f(a_n,S_n)$}
          \begin{itemize}
              \item 代入定义 $a_n=S_n-S_{n-1}$;
              \item 写出 $\sum a_n = S_n$,检查是否望项消去;
              \item $\sum a_n$ 收敛 $\Rightarrow S_n$ 有界;
              \item $\sum a_n$ 收敛 $\Leftarrow S_n$ 有界且 $a_n\to 0$。
          \end{itemize}

    \item \textbf{见到 $(-1)^n - 1$ 型通项}
          \begin{itemize}
              \item 可作 \textbf{泰勒展开} 分项讨论;
              \item 若可能,造出交错形式,用 \textbf{莱布尼茨判别法};
              \item 若失败,尝试 \textbf{绝对收敛判别法};
              \item 拆项法:若
                    \[
                        \sum\frac{f(n)\pm g(n)}{h(n)} = \sum\frac{f(n)}{h(n)} \pm \sum\frac{g(n)}{h(n)},
                    \]
                    分别讨论敛散性。
          \end{itemize}
\end{enumerate}


\subsection{常用判别法总结}
\begin{itemize}
    \item \textbf{比较判别法}($u_n>0$)
          \[
              \text{若 }0<u_n<v_n\text{ 且 }\sum v_n\text{ 收敛,则 }\sum u_n\text{ 收敛。}
          \]
    \item \textbf{比值判别法}:若 $\displaystyle \lim_{n\to\infty}\frac{u_{n+1}}{u_n}=q$,
          \[
              \begin{cases}
                  q<1 & \Rightarrow \text{收敛}, \\
                  q>1 & \Rightarrow \text{发散}, \\
                  q=1 & \Rightarrow \text{不定。}
              \end{cases}
          \]
    \item \textbf{根值判别法}:若 $\displaystyle \lim_{n\to\infty}\sqrt[n]{|u_n|}=q$,
          结论同上。
    \item \textbf{积分判别法}:若 $u_n=f(n)$ 且 $f(x)\ge 0$ 单调递减,
          \[
              \sum_{n=1}^{\infty} f(n)\text{ 收敛} \iff \int_1^{\infty} f(x)\,dx\text{ 收敛。}
          \]
    \item \textbf{莱布尼茨判别法}(交错级数):
          若 $a_n>0$ 且 $a_n\downarrow 0$,则 $\sum (-1)^{n-1}a_n$ 收敛。
\end{itemize}

\section{求幂级数的和函数}

\subsection{确定收敛域}

\begin{enumerate}
    \item \textbf{具体型问题}
          \begin{enumerate}
              \item 对于一般幂级数 $\displaystyle \sum_{n=0}^{\infty} a_n x^n$:

                    \textbf{收敛半径:}
                    \[
                        R =
                        \begin{cases}
                            \dfrac{1}{\rho}, & \rho = \lim\limits_{n\to\infty}\left|\dfrac{a_{n+1}}{a_n}\right|
                            \text{ 或 } \rho = \lim\limits_{n\to\infty}\sqrt[n]{|a_n|},                          \\[0.8em]
                            +\infty,         & \rho = 0,                                                        \\[0.3em]
                            0,               & \rho = +\infty.
                        \end{cases}
                    \]

                    \textbf{收敛区间与收敛域:}
                    在区间 $(-R, R)$ 内绝对收敛。
                    当 $x = \pm R$ 时需单独讨论敛散性,故收敛域可能为
                    $(-R, R)$、$[-R, R)$、$(-R, R]$ 或 $[-R, R]$。

              \item 对于缺项幂级数或一般函数项级数 $\sum u_n(x)$:

                    \begin{enumerate}
                        \item 取绝对值写作 $\sum |u_n(x)|$;
                        \item 用比值(或根值)判别法求出收敛区间 $(a,b)$;
                        \item 再讨论 $x=a,b$ 时的敛散性,从而确定收敛域。
                    \end{enumerate}
          \end{enumerate}

    \item \textbf{抽象型问题——阿贝尔定理的应用}
          \begin{enumerate}
              \item 若 $\sum a_n x^n$ 在 $x=x_1(\ne 0)$ 处收敛,则当 $|x|<|x_1|$ 时绝对收敛;
                    若在 $x=x_2(\ne 0)$ 处发散,则当 $|x|>|x_2|$ 时发散。

              \item 根据阿贝尔定理:
                    \[
                        \begin{cases}
                            \text{若在 } x_1 \text{ 处收敛}   & \Rightarrow R \ge |x_1 - x_0|, \\
                            \text{若在 } x_1 \text{ 处发散}   & \Rightarrow R \le |x_1 - x_0|, \\
                            \text{若在 } x_1 \text{ 处条件收敛} & \Rightarrow R = |x_1 - x_0|.
                        \end{cases}
                    \]

              \item 已知 $\sum a_n (x-x_1)^n$ 的敛散性,讨论 $\sum b_m (x-x_2)^m$:
                    \begin{enumerate}
                        \item 平移或提出 $(x-x_0)^k$,收敛半径不变;
                        \item \textbf{逐项求导:}收敛半径不变,收敛域可能缩小;
                        \item \textbf{逐项积分:}收敛半径不变,收敛域可能扩大。
                    \end{enumerate}
          \end{enumerate}
\end{enumerate}

\subsection{求和函数的方法}

\begin{enumerate}
    \item \textbf{先积后导、先导后积法}
          \begin{enumerate}
              \item $\sum (a n + b)x^n$:先积分再求导;
              \item $\sum \dfrac{x^n}{a n + b}$:先求导再积分;
              \item $\sum \dfrac{c n^2 + d n + e}{a n + b}x^n$:拆为若干简单级数相加。
          \end{enumerate}

    \item \textbf{常用幂级数公式}
          \begin{multicols}{2}
              \begin{enumerate}
                  \item $\displaystyle \sum_{n=0}^{\infty} x^n = \frac{1}{1-x}$, ($|x|<1$);
                  \item $\displaystyle \sum_{n=1}^{\infty} n x^{n-1} = \frac{1}{(1-x)^2}$;
                  \item $\displaystyle \sum_{n=2}^{\infty} n(n-1)x^{n-2} = \frac{2}{(1-x)^3}$;
                  \item $\displaystyle \sum_{n=1}^{\infty} (-1)^{n-1}\frac{x^{n}}{n} = \ln(1+x)$;
                  \item $\displaystyle \sum_{n=1}^{\infty} \frac{x^n}{n} = -\ln(1-x)$;
                  \item $\displaystyle \sum_{n=0}^{\infty} \frac{x^{2n+1}}{2n+1} = \frac{1}{2}\ln\frac{1+x}{1-x}$;
                  \item $\displaystyle \sum_{n=0}^{\infty} \frac{(-1)^n x^{2n+1}}{2n+1} = \arctan x$;
                  \item $\displaystyle \sum_{n=0}^{\infty} \frac{x^n}{n!} = e^x$;
                  \item $\displaystyle \sum_{n=0}^{\infty} \frac{x^{2n}}{(2n)!} = \cosh x = \frac{e^x+e^{-x}}{2}$;
                  \item $\displaystyle \sum_{n=0}^{\infty} \frac{(-1)^n x^{2n+1}}{(2n+1)!} = \sin x$;
                  \item $\displaystyle \sum_{n=0}^{\infty} \frac{(-1)^n x^{2n}}{(2n)!} = \cos x$.
              \end{enumerate}
          \end{multicols}
\end{enumerate}

\subsection{微分方程法求和函数}

\begin{enumerate}
    \item \textbf{已给微分方程型:}
          \begin{enumerate}
              \item 验证级数满足某微分方程;
              \item 求通解;
              \item 根据初值条件确定常数;
              \item 代入特定 $x$(如 $0,\frac{1}{2},1$)求具体和。
          \end{enumerate}

    \item \textbf{由通项关系建立微分方程型:}
          \begin{enumerate}
              \item 根据 $a_{n+1}$、$a_n$、$a_{n-1}$ 的关系式建立方程;
              \item 求解通解;
              \item 展开为 $\sum a_n x^n$ 并确定 $a_n$ 通项。
          \end{enumerate}
\end{enumerate}

\textbf{总结:}

\begin{itemize}
    \item 求收敛域首选比值法或根值法;
    \item 边界常考:$\sum \frac{x^n}{n}$、$\sum \frac{(-1)^n x^n}{n}$;
    \item 熟记导数与积分规律:求导 $\Rightarrow$ 乘 $n$;积分 $\Rightarrow$ 除 $n$;
    \item 收敛半径不变的三种操作:提因式、逐项求导、逐项积分;
    \item 求和函数常从 $\sum x^n$ 推导。
\end{itemize}

\section{函数展开成幂级数}

\begin{enumerate}
    \item \textbf{含对数函数的展开}

          \begin{enumerate}
              \item $\ln(a + bx)$ 型:
                    \[
                        \ln(a + bx) = \ln a + \ln\left(1 + \frac{b}{a}x\right), \quad a > 0.
                    \]
                    若 $|x| < \frac{a}{b}$,则
                    \[
                        \ln(1 + \tfrac{b}{a}x) = \sum_{n=1}^{\infty}(-1)^{n-1}\frac{1}{n}\left(\frac{b}{a}x\right)^n.
                    \]
              \item $\ln(1 + ax + bx^2)$ 型:
                    \[
                        \ln(1 + ax + bx^2)
                        = \ln(1 + cx) + \ln(1 + dx),
                    \]
                    其中 $a = c + d,\ b = cd.$
          \end{enumerate}

    \item \textbf{含分式的展开}

          \begin{enumerate}
              \item $\dfrac{1}{a + bx}$ 型:
                    \[
                        \frac{1}{a + bx}
                        = \frac{1}{a} \cdot \frac{1}{1 + \frac{b}{a}x}
                        = \frac{1}{a}\sum_{n=0}^{\infty}(-1)^n\left(\frac{b}{a}x\right)^n, \quad |x| < \frac{a}{b}.
                    \]
              \item $\dfrac{1}{(x+a)(x+b)}$ 型:
                    \[
                        \frac{1}{(x+a)(x+b)}
                        = \frac{1}{b - a}\left(\frac{1}{x+a} - \frac{1}{x+b}\right).
                    \]
          \end{enumerate}

    \item \textbf{含三角平方函数的化简}

          \[
              \sin^2x = \frac{1 - \cos 2x}{2}, \qquad
              \cos^2x = \frac{1 + \cos 2x}{2}.
          \]

          (常用于将 $\sin^2 x$、$\cos^2 x$ 转化为含单角的可积形式。)
\end{enumerate}

\section{傅里叶级数}

\subsection{周期为 $2l$ 的傅里叶级数定义}

设 $f(x)$ 是以 $2l$ 为周期的可积函数,则称
\[
    a_n = \frac{1}{l}\int_{-l}^{l} f(x)\cos\frac{n\pi x}{l}\,\mathrm{d}x, \quad
    b_n = \frac{1}{l}\int_{-l}^{l} f(x)\sin\frac{n\pi x}{l}\,\mathrm{d}x
\]
为 $f(x)$ 的傅里叶系数。

则其傅里叶级数为
\[
    f(x) \sim \frac{a_0}{2}
    + \sum_{n=1}^{\infty}\left(a_n\cos\frac{n\pi x}{l}
    + b_n\sin\frac{n\pi x}{l}\right).
\]

\subsection{狄利克雷收敛定理}

若周期为 $2l$ 的函数 $f(x)$ 在区间 $[-l,l]$ 上满足:

\begin{enumerate}
    \item 连续或仅有有限个第一类间断点;
    \item 至多具有有限个极值点,
\end{enumerate}

则其傅里叶级数在 $[-l,l]$ 上处处收敛,和函数为
\[
    S(x) =
    \begin{cases}
        f(x),                      & x\ \text{为连续点}, \\[0.4em]
        \dfrac{f(x-0)+f(x+0)}{2},  & x\ \text{为间断点}, \\[0.8em]
        \dfrac{f(-l+0)+f(l-0)}{2}, & x = \pm l.
    \end{cases}
\]
\subsection{正弦级数与余弦级数}

\begin{enumerate}
    \item 当 $f(x)$ 为\textbf{奇函数}时:
          \[
              f(x) \sim \sum_{n=1}^{\infty} b_n \sin\frac{n\pi x}{l},
              \quad b_n = \frac{2}{l}\int_0^l f(x)\sin\frac{n\pi x}{l}\,\mathrm{d}x.
          \]

    \item 当 $f(x)$ 为\textbf{偶函数}时:
          \[
              f(x) \sim \frac{a_0}{2}
              + \sum_{n=1}^{\infty} a_n \cos\frac{n\pi x}{l},
              \quad a_n = \frac{2}{l}\int_0^l f(x)\cos\frac{n\pi x}{l}\,\mathrm{d}x.
          \]
\end{enumerate}

\subsection{区间 $[0,l]$ 上函数的傅里叶展开}

若 $f(x)$ 仅定义在 $[0,l]$ 上,则先作\textbf{周期延拓}:

\begin{enumerate}
    \item \textbf{周期奇延拓(正弦级数)}

          \[
              F(x) =
              \begin{cases}
                  f(x),   & 0 < x \le l, \\
                  -f(-x), & -l < x < 0,  \\
                  0,      & x = 0,
              \end{cases}
          \]
          再令 $F(x)$ 为以 $2l$ 为周期的函数,则
          \[
              f(x) \sim \sum_{n=1}^{\infty} b_n \sin\frac{n\pi x}{l},
              \quad b_n = \frac{2}{l}\int_0^l f(x)\sin\frac{n\pi x}{l}\,\mathrm{d}x.
          \]

    \item \textbf{周期偶延拓(余弦级数)}

          \[
              F(x) =
              \begin{cases}
                  f(x),  & 0 \le x \le l, \\
                  f(-x), & -l < x < 0,
              \end{cases}
          \]
          再令 $F(x)$ 为以 $2l$ 为周期的函数,则
          \[
              f(x) \sim \frac{a_0}{2}
              + \sum_{n=1}^{\infty} a_n \cos\frac{n\pi x}{l},
              \quad a_n = \frac{2}{l}\int_0^l f(x)\cos\frac{n\pi x}{l}\,\mathrm{d}x.
          \]
\end{enumerate}

\subsection{总结}

\begin{itemize}
    \item 傅里叶级数是将函数在区间上展开为 $\sin$ 与 $\cos$ 线性组合;
    \item 奇函数 $\Rightarrow$ 正弦级数;偶函数 $\Rightarrow$ 余弦级数;
    \item 延拓思想是将 $[0,l]$ 上函数扩展为周期函数;
    \item 计算核心:求 $a_n$、$b_n$ 两类系数;
    \item 在间断点取左右极限的平均值;
    \item 傅里叶级数是高等数学与信号分析的桥梁。
\end{itemize}