\GAOchapter{一元函数微分学的应用(二)——中值定理、微分等式与微分不等式}

\section{寻找元函数}
\POneTwo+\POneThree+\DTwoThree

\subsection{一阶乘积求导公式的逆用}
\[
    (uv)' = u'v + uv'
\]
常见形式:
\begin{enumerate}
    \item $[f(x) x^n]' = x^{n-1}[x f'(x) + n f(x)]$;
    \item $[f(x) e^{nx}]' = e^{nx}[f'(x) + n f(x)]$;
    \item $[f(x) e^{x^n}]' = e^{x^n}[f'(x) + n x^{n-1} f(x)]$;
    \item $[f(x) e^{\varphi(x)}]' = e^{\varphi(x)}[f'(x) + f(x)\varphi'(x)]$;
    \item $\displaystyle \left\{ f(x)e^{\int_0^x [f(t)]^{n-1}dt} \right\}' = e^{\int_0^x [f(t)]^{n-1}dt}\{f'(x) + [f(x)]^n\}$;
    \item $[f(x)f'(x)]' = [f'(x)]^2 + f(x)f''(x)$;
    \item $[f(x)g(x)]' = f'(x)g(x) + f(x)g'(x)$;
    \item $[f(x)\arctan x]' = f'(x)\arctan x + \dfrac{f(x)}{1+x^2}$;
    \item $[f(x)\sin x]' = f'(x)\sin x + f(x)\cos x = [f'(x)\tan x + f(x)]\cos x$.
\end{enumerate}

\subsection{二阶乘积求导公式的逆用}
\[
    (uv)'' = u''v + 2u'v' + uv'', \quad [f(x)e^x]'' = e^x[f''(x) + 2f'(x) + f(x)].
\]

\subsection{一阶商求导公式的逆用}
\[
    \left(\frac{u}{v}\right)' = \frac{u'v - uv'}{v^2}
\]
\begin{enumerate}
    \item $\displaystyle \left[\frac{f(x)}{x}\right]' = \frac{f'(x)x - f(x)}{x^2}$
          见到 $f'(x)x - f(x)$,$x \ne 0$,令 $F(x)=\frac{f(x)}{x}$;
    \item $\displaystyle \left[\frac{f'(x)}{f(x)}\right]' = \frac{f''(x)f(x) - [f'(x)]^2}{f^2(x)}$
          见到 $f''(x)f(x) - [f'(x)]^2$,令 $F(x)=\frac{f'(x)}{f(x)}$;
    \item $[\ln f(x)]'' = \left[\frac{f'(x)}{f(x)}\right]' = \frac{f''(x)f(x) - [f'(x)]^2}{f^2(x)}$
          见到 $f''(x)f(x) - [f'(x)]^2$,$f(x)>0$ 时可令 $F(x)=\ln f(x)$.
\end{enumerate}

\subsection{祖孙三代传承法}
若欲证结论“差辈分”,如
\[
    f''(\xi)=f(\xi)\quad\text{或}\quad f'(\xi)=\int_0^\xi f(t)\,dt,
\]
则补齐辈分作恒等变形:

\paragraph{例1:}
\[
    f''(\xi)-f'(\xi)+f'(\xi)-f(\xi)=0,
\]
令 $F'(x)=[f''(x)-f'(x)]e^x + [f'(x)-f(x)]e^x$,则
$F(x)=[f'(x)-f(x)]e^x$.

\paragraph{例2:}
\[
    f'(\xi)-f(\xi)+f(\xi)-\int_0^\xi f(t)\,dt = 0,
\]
令
\[
    F'(x)=\big([f'(x)-f(x)]+[f(x)-\int_0^x f(t)\,dt]\big)e^x,
\]
则 $F(x)=\left[f(x)-\int_0^x f(t)\,dt\right]e^x$.

\subsection{小伎俩(障眼法)}
\begin{enumerate}[label=(\arabic*)]
    \item \textbf{简单化:}
          $2\xi-1\rightarrow x^2-x,\ f'(ξ)f(ξ)\rightarrow\frac{1}{2}f^2(x),\
              \dfrac{f'(ξ)}{f(ξ)}\rightarrow\ln f(x),\ \dfrac{1}{ξ}\rightarrow\ln x$.

    \item \textbf{升辈降辈:}
          $f(x)\to f'(x)$ 或 $f(x)\to \int_0^x f(t)dt$.

    \item \textbf{平移:}
          \[
              [f(x)(x+1)^2]'=(x+1)\big[f'(x)(x+1)+2f(x)\big],
          \]
          \[
              [f(x)e^{(x-1)^2}]'=e^{(x-1)^2}[f'(x)+2(x-1)f(x)].
          \]

    \item \textbf{恒等变形:}
          \begin{enumerate}
              \item 移项:令右端为 0;
              \item 乘除:如 $\big[f'(x)g(x)-f(x)g'(x)\big]_{x=\xi}=0 \Rightarrow \dfrac{f'(ξ)}{g'(ξ)}=\dfrac{f(ξ)}{g(ξ)}$.
          \end{enumerate}
\end{enumerate}

\subsection{积分型题设}
当题设或结论出现 $\displaystyle \int_a^b f(x)\,dx = c$、$\int_a^\xi f(x)\,dx$ 等形式时:
\begin{enumerate}
    \item 令被积函数为辅助函数;
    \item 令 $F(x)=\int_a^x f(t)\,dt$ 或 $F(x)=\int_x^b f(t)\,dt$;
    \item 用推广积分中值定理:
          $\displaystyle \int_a^b f(x)\,dx=f(\xi)(b-a)$;
    \item 用分部积分:$\displaystyle \int_a^b u\,d\nu=u\nu\Big|_a^b-\int_a^b\nu\,du$;
    \item 或作泰勒展开再积分.
\end{enumerate}


\section{证明 $f'(\xi)=0$}
\DTwoTwo+\DTwoThree

常用三种情形:
\begin{enumerate}
    \item 区间内部最值点:费马定理;
    \item 区间端点值相等:罗尔定理;
    \item 区间端点导数异号:导数介值定理.
\end{enumerate}


\section{证明含高阶导数的等式或不等式}
\DTwoTwo+\DTwoThree

\subsection{利用中值定理}
\begin{enumerate}
    \item 拉格朗日中值定理:
          若 $f(x)$ 在 $[a,b]$ 上连续且在 $(a,b)$ 内可导,则
          $\exists \xi \in (a,b)$,使得
          \[
              f'(\xi)=\frac{f(b)-f(a)}{b-a}.
          \]
    \item 推论:$f'(x)=0\Rightarrow f(x)$ 在区间上为常数.
    \item 写作 $f(x)-f(a)=f'(\xi)(x-a)$,常用于出现“$f-f$”或“$f-f'$”关系;
    \item 写作 $\dfrac{f(b)-f(a)}{b-a}=f'(\xi)$,常用于“切线斜率”;
    \item 写作 $f(x)-f(a)=f'(\theta x)x$,$0<\theta<1$,用于极限估值;
    \item 写作 $\int_a^x f(t)\,dt=f(\xi)(x-a)$,用于积分中值定理.
\end{enumerate}

\subsection{柯西中值定理}
当式子可化为 $\dfrac{f(b)-f(a)}{g(b)-g(a)}$ 或 $\dfrac{f'(\xi)}{g'(\xi)}$ 时使用.
\begin{enumerate}
    \item 若一函数取具体形式,可利用 $\displaystyle f(\xi)=\left[\int_a^x f(t)\,dt\right]'_{x=\xi}$;
    \item 若函数值含 $0$ 或 $1$,可利用 $f(a)=0$、$e^0=1$、$\int_a^a f=0$ 等;
    \item 若 $f(a)=0$,可作
          \[
              f(\xi)=f'(\eta)(\xi-a),\ \eta\in(a,\xi),
          \]
          再代回主式,得:
          \[
              f'(\eta)(b^2-a^2)=\frac{2\xi}{\xi-a}\int_a^b f(x)\,dx.
          \]
\end{enumerate}


\section{用泰勒公式}
\DTwoThree+\DThree

\subsection{常用泰勒展开式或形式展开式大观}

\begin{enumerate}
    \item \textbf{第一组:基本代数型展开式}
          \begin{enumerate}
              \item $\sqrt{1\pm x} = 1 \pm \dfrac{1}{2}x - \dfrac{1}{8}x^2 + \cdots, \quad |x| < 1$;
              \item $\dfrac{1}{\sqrt{1\pm x}} = 1 \mp \dfrac{1}{2}x + \dfrac{3}{8}x^2 + \cdots, \quad |x| < 1$;
              \item $\dfrac{1}{1+x} = 1 - x + x^2 - x^3 + \cdots = \displaystyle\sum_{n=0}^{\infty}(-1)^n x^n, \quad |x| < 1$;
              \item $\dfrac{1}{1-x} = 1 + x + x^2 + x^3 + \cdots = \displaystyle\sum_{n=0}^{\infty} x^n, \quad |x| < 1$;
              \item $\dfrac{1}{(1-x)^2} = 1 + 2x + 3x^2 + \cdots = \displaystyle\sum_{n=0}^{\infty}(n+1)x^n, \quad |x| < 1$;
              \item $\dfrac{1}{(1+x)^2} = 1 - 2x + 3x^2 - 4x^3 + \cdots = \displaystyle\sum_{n=0}^{\infty}(-1)^n (n+1)x^n, \quad |x| < 1$.
          \end{enumerate}

    \item \textbf{第二组:指数型展开式}
          \begin{enumerate}
              \item $e^x = \displaystyle\sum_{n=0}^{\infty} \dfrac{x^n}{n!}$;
              \item $a^x = e^{x\ln a} = \displaystyle\sum_{n=0}^{\infty} \dfrac{(\ln a)^n x^n}{n!}$;
              \item $\sinh x = \dfrac{e^x - e^{-x}}{2} = \displaystyle\sum_{n=0}^{\infty} \dfrac{x^{2n+1}}{(2n+1)!}$;
              \item $\cosh x = \dfrac{e^x + e^{-x}}{2} = \displaystyle\sum_{n=0}^{\infty} \dfrac{x^{2n}}{(2n)!}$.
          \end{enumerate}

    \item \textbf{第三组:三角与反三角展开式}
          \begin{enumerate}
              \item $\sin x = x - \dfrac{x^3}{3!} + \dfrac{x^5}{5!} - \cdots = \displaystyle\sum_{n=0}^{\infty} (-1)^n \dfrac{x^{2n+1}}{(2n+1)!}$;
              \item $\cos x = 1 - \dfrac{x^2}{2!} + \dfrac{x^4}{4!} - \cdots = \displaystyle\sum_{n=0}^{\infty} (-1)^n \dfrac{x^{2n}}{(2n)!}$;
              \item $\tan x = x + \dfrac{1}{3}x^3 + \dfrac{2}{15}x^5 + \cdots, \quad |x| < \dfrac{\pi}{2}$;
              \item $\arcsin x = x + \dfrac{1}{6}x^3 + \dfrac{3}{40}x^5 + \cdots, \quad |x| < 1$;
              \item $\arctan x = x - \dfrac{1}{3}x^3 + \dfrac{1}{5}x^5 - \cdots = \displaystyle\sum_{n=0}^{\infty} (-1)^n \dfrac{x^{2n+1}}{2n+1}, \quad |x| \le 1$.
          \end{enumerate}

    \item \textbf{第四组:对数型展开式}
          \begin{enumerate}
              \item $\ln(1+x) = x - \dfrac{1}{2}x^2 + \dfrac{1}{3}x^3 - \dfrac{1}{4}x^4 + \cdots = \displaystyle\sum_{n=1}^{\infty}(-1)^{n+1}\dfrac{x^n}{n}, \quad -1 < x \le 1$;
              \item $\ln(1-x) = -\displaystyle\sum_{n=1}^{\infty} \dfrac{x^n}{n}, \quad -1 \le x < 1$;
              \item $\ln x = \ln(1+(x-1)) = (x-1) - \dfrac{1}{2}(x-1)^2 + \dfrac{1}{3}(x-1)^3 - \cdots = \displaystyle\sum_{n=1}^{\infty}(-1)^{n+1}\dfrac{(x-1)^n}{n}, \quad 0 < x \le 2$;
              \item $\ln(a+x) = \ln a + \ln\!\left(1+\dfrac{x}{a}\right) = \ln a + \dfrac{x}{a} - \dfrac{1}{2a^2}x^2 + \cdots, \quad a>0,\ -a<x\le a$;
              \item $\ln\dfrac{1+x}{1-x} = 2\left(x + \dfrac{x^3}{3} + \dfrac{x^5}{5} + \cdots\right) = 2\displaystyle\sum_{n=0}^{\infty} \dfrac{x^{2n+1}}{2n+1}, \quad |x| < 1$;
              \item $\ln\dfrac{x+1}{x-1} = 2\displaystyle\sum_{n=0}^{\infty} \dfrac{1}{(2n+1)x^{2n+1}}, \quad |x| > 1$;
              \item $\ln\!\big(x+\sqrt{x^2+1}\big) = x - \dfrac{1}{6}x^3 + \dfrac{3}{40}x^5 + \cdots, \quad |x| < 1$.
          \end{enumerate}
\end{enumerate}

\subsection{本质与用途}
泰勒公式是函数的 $n$ 阶近似:
\[
    f(x)=\sum_{k=0}^n \frac{f^{(k)}(x_0)}{k!}(x-x_0)^k + R_n(x),
\]
其中余项可取:
\[
    R_n(x)=o\big((x-x_0)^n\big)\quad\text{或}\quad R_n(x)=\frac{f^{(n+1)}(\xi)}{(n+1)!}(x-x_0)^{n+1}.
\]

\textbf{常用场景:}
\begin{enumerate}
    \item 求极限;
    \item 判断无穷小阶;
    \item 求 $f^{(n)}(x_0)$;
    \item 证明估值问题.
\end{enumerate}

\textbf{$x_0$ 与 $x$ 的选取:}
\begin{enumerate}
    \item $x_0$ 取使导数简单的点,如 0、1、端点;
    \item $x$ 取任意点或关于 $x_0$ 的对称点.
\end{enumerate}

\section{讨论 $f(x)=0$ 的根的个数}

\begin{enumerate}
    \item \textbf{零点定理及其推广:}
          若 $f(x)$ 在 $[a,b]$ 上连续,且 $f(a)f(b)<0$,则方程 $f(x)=0$ 在 $(a,b)$ 内至少有一个根.
    \item \textbf{反证思想:}
          若要证 $f(x)$ 存在零点,可设 $f(x)$ 无零点,按条件推矛盾;反之亦然.
    \item \textbf{导数工具:}
          用 $f'(x)$ 研究函数单调性、极值性态,以判断零点个数.
    \item \textbf{罗尔定理推论:}
          若 $f^{(n)}(x)=0$ 至多有 $k$ 个根,则 $f(x)=0$ 至多有 $k+n$ 个根.
    \item \textbf{实系数奇次方程:}
          方程 $x^{2n+1}+a_1x^{2n}+\cdots+a_{2n}x+a_{2n+1}=0$ 至少有一个实根.
    \item \textbf{渐近性态:}
          若 $\lim_{x\to+\infty}f(x)$ 与 $\lim_{x\to-\infty}f(x)$ 异号,则方程至少有一个实根.
\end{enumerate}

\section{证明不等式}

\begin{enumerate}
    \item \textbf{用单调性:}
          \begin{enumerate}
              \item 若 $\displaystyle\lim_{x\to a^+}F(x)\ge0$ 且 $F'(x)\ge0$,则 $(a,b)$ 内 $F(x)\ge0$;
              \item 若 $\displaystyle\lim_{x\to b^-}F(x)\ge0$ 且 $F'(x)\le0$,则 $(a,b)$ 内 $F(x)\ge0$.
          \end{enumerate}

    \item \textbf{用最值:}
          若 $F(x)$ 在 $(a,b)$ 有最小值 $m$,则 $F(x)\ge m$;若有最大值 $M$,则 $F(x)\le M$.

    \item \textbf{用凹凸性:}
          若 $F''(x)>0$,则:
          \begin{enumerate}
              \item $\displaystyle \frac{F(x_1)+F(x_2)}{2}\ge F\!\left(\frac{x_1+x_2}{2}\right)$;
              \item 对任意 $\lambda_1,\lambda_2>0$ 且 $\lambda_1+\lambda_2=1$,有
                    $\lambda_1F(x_1)+\lambda_2F(x_2)\ge F(\lambda_1x_1+\lambda_2x_2)$;
              \item 对任意 $x\ne x_0$,有 $F(x)\ge F(x_0)+F'(x_0)(x-x_0)$.
          \end{enumerate}
          若 $F''(x)<0$,则上述不等式方向反向.

    \item \textbf{用拉格朗日中值定理:}
          若 $F'(x)\ge A$(或 $\le A$),则
          $F(b)-F(a)\ge A(b-a)$(或 $\le A(b-a)$).

    \item \textbf{用柯西中值定理:}
          若 $\dfrac{F'(x)}{G'(x)}\ge A$(或 $\le A$),则
          $\dfrac{F(b)-F(a)}{G(b)-G(a)}\ge A$(或 $\le A$).

    \item \textbf{用带余项的泰勒公式:}
          若 $F''(x)$ 存在且恒 $>0$(或 $<0$),则在 $x_0$ 处展开:
          \[
              F(x)=F(x_0)+F'(x_0)(x-x_0)+\frac{1}{2}F''(\xi)(x-x_0)^2,
          \]
          其中 $\xi$ 介于 $x$ 与 $x_0$,从而有
          $F(x)\ge F(x_0)+F'(x_0)(x-x_0)$(或反向不等式).
\end{enumerate}

\section{求解含参等式或不等式问题}

\begin{enumerate}
    \item \textbf{导数中不含参数:}
          辅助函数 $f'(x)$ 不含参数.研究函数性态时不考虑参数,最后再根据参数取值讨论与 $x$ 轴位置关系.
    \item \textbf{导数中含参数:}
          辅助函数 $f'(x)$ 含参数.研究性态时需分类讨论参数取值.常见形式:
          \begin{enumerate}
              \item 函数型:$f(x)=k$ 或 $f(x,k)$;
              \item 参数方程型:
                    \[
                        \begin{cases}
                            x = x(t,k), \\
                            y = y(t,k).
                        \end{cases}
                    \]
          \end{enumerate}
\end{enumerate}