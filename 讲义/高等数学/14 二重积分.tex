\GAOchapter{二重积分}

\section{和式极限与二重积分的定义}

设 $D = \{(x, y) \mid a \le x \le b,\, c \le y \le d\}$,若极限存在,则称为 $f(x,y)$ 在 $D$ 上的二重积分:
\[
    \iint_{D} f(x, y) \, d\sigma
    = \lim_{n \to \infty} \sum_{i=1}^{n} \sum_{j=1}^{n}
    f\!\left(a + \frac{b-a}{n}i,\; c + \frac{d-c}{n}j\right)
    \cdot \frac{b-a}{n} \cdot \frac{d-c}{n}.
\]

\section{交换积分次序的技巧与必要性}

\begin{itemize}
    \item 若题中给出累次积分:
          \[
              \int_{a}^{b} \left[\int_{g_1(x)}^{g_2(x)} f(x,y)\,dy\right] dx,
          \]
          当 \textbf{内层积分} 属于以下类型时,应优先考虑交换积分次序:
          \begin{enumerate}
              \item 可积但不可求积型;
              \item 计算困难型。
          \end{enumerate}

    \item 常见“可积不可求积”形式:
          \[
              \int \frac{\sin x}{x}\,dx, \quad
              \int \frac{\cos x}{x}\,dx, \quad
              \int \frac{1}{\ln x}\,dx, \quad
              \int \sin x^2\,dx, \quad
              \int e^{x^2}\,dx, \quad
              \int \frac{e^x}{x}\,dx, \quad
              \int \frac{\tan x}{x}\,dx.
          \]
          见到这些形式的被积函数时,一般都是\textbf{需要交换积分次序}的信号。
\end{itemize}

\section{积分的保号性}

\begin{enumerate}
    \item 若连续函数 $f(x,y) \ge 0$ 且不恒为 $0$,则
          \[
              \iint\limits_D f(x,y)\, d\sigma > 0.
          \]
    \item 若连续函数 $f(x,y)$ 满足:
          \[
              \forall D, \quad \iint\limits_D f(x,y)\, d\sigma = 0,
          \]
          则必有 $f(x,y) \equiv 0$。
\end{enumerate}

\section{对称性法则}

\subsection{普通对称性}
设 $D$ 与 $f(x,y)$ 同时具备一定对称性,则可大幅简化计算:

\begin{enumerate}
    \item 关于 $y$ 轴对称:
          \[
              \iint\limits_D f(x,y)\,d\sigma =
              \begin{cases}
                  2\iint\limits_{D_1} f(x,y)\,d\sigma, & f(x,y)=f(-x,y),  \\[4pt]
                  0,                                   & f(x,y)=-f(-x,y),
              \end{cases}
          \]
          其中 $D_1$ 为 $D$ 的右半部分。
    \item 关于 $x$ 轴对称:
          \[
              \iint\limits_D f(x,y)\,d\sigma =
              \begin{cases}
                  2\iint\limits_{D_1} f(x,y)\,d\sigma, & f(x,y)=f(x,-y),  \\[4pt]
                  0,                                   & f(x,y)=-f(x,-y).
              \end{cases}
          \]
    \item 关于原点对称:
          \[
              \iint\limits_D f(x,y)\,d\sigma =
              \begin{cases}
                  2\iint\limits_{D_1} f(x,y)\,d\sigma, & f(x,y)=f(-x,-y),  \\[4pt]
                  0,                                   & f(x,y)=-f(-x,-y).
              \end{cases}
          \]
    \item 关于直线 $y = x$ 对称:
          \[
              \iint\limits_D f(x,y)\,d\sigma =
              \begin{cases}
                  2\iint\limits_{D_1} f(x,y)\,d\sigma, & f(x,y)=f(y,x),  \\[4pt]
                  0,                                   & f(x,y)=-f(y,x).
              \end{cases}
          \]
\end{enumerate}

\subsection{轮换对称性}
若交换 $x$ 与 $y$ 后区域 $D$ 不变(即关于 $y=x$ 对称),则:
\[
    \iint\limits_D f(x,y)\,d\sigma = \iint\limits_D f(y,x)\,d\sigma.
\]

\section{二重积分常用结论(单位圆区域)}

\[
    \begin{aligned}
         & \iint_{x^2+y^2\le1}(x^2+y^2)\,d\sigma = \frac{\pi}{2}, \qquad
        \iint_{x^2+y^2\le1}\sqrt{x^2+y^2}\,d\sigma = \frac{2\pi}{3},                  \\[6pt]
         & \iint_{x^2+y^2\le1}\sqrt{1-(x^2+y^2)}\,d\sigma = \frac{2\pi}{3}, \qquad
        \iint_{x^2+y^2\le1}\!\!\bigl(1-\sqrt{x^2+y^2}\bigr)\,d\sigma = \frac{\pi}{3}, \\[6pt]
         & \iint_{x^2+y^2\le1}\!\!\left(\frac{x^2}{a^2}+\frac{y^2}{b^2}\right)d\sigma
        = \frac{\pi}{4}\!\left(\frac{1}{a^2}+\frac{1}{b^2}\right).
    \end{aligned}
\]


\section{二重积分的计算方法}

\subsection{直角坐标法}
\begin{enumerate}
    \item $X$ 型区域:
          \[
              \iint\limits_D f(x,y)\,d\sigma = \int_a^b \!dx \int_{\varphi_1(x)}^{\varphi_2(x)}\! f(x,y)\,dy.
          \]
    \item $Y$ 型区域:
          \[
              \iint\limits_D f(x,y)\,d\sigma = \int_c^d \!dy \int_{\psi_1(y)}^{\psi_2(y)}\! f(x,y)\,dx.
          \]
\end{enumerate}

\subsection{极坐标法}

设
\[
    \begin{cases}
        x = r\cos\theta, \\
        y = r\sin\theta,
    \end{cases}
    \quad d\sigma = r\,dr\,d\theta,
\]
则
\[
    \iint\limits_D f(x,y)\,d\sigma
    = \int_{\alpha}^{\beta}\!d\theta \int_{r_1(\theta)}^{r_2(\theta)}\! f(r\cos\theta, r\sin\theta)\,r\,dr.
\]
常见情况:
\begin{itemize}
    \item 极点 $O$ 在区域外;
    \item 极点在边界上;
    \item 极点在区域内部(常为圆或扇形区域)。
\end{itemize}

\subsection{二重积分的换元法}

\begin{enumerate}
    \item \textbf{一元换元回顾:}
          \[
              \int_a^b f(x)\,dx = \int_{\alpha}^{\beta} f[\varphi(t)]\,\varphi'(t)\,dt.
          \]
    \item \textbf{二重积分换元:}
          \[
              \iint_{D_{xy}} f(x,y)\,dx\,dy
              = \iint_{D_{uv}} f[x(u,v),y(u,v)]\,
              \left|\frac{\partial(x,y)}{\partial(u,v)}\right| du\,dv,
          \]
          其中
          \[
              \frac{\partial(x,y)}{\partial(u,v)} =
              \begin{vmatrix}
                  \dfrac{\partial x}{\partial u} & \dfrac{\partial x}{\partial v} \\[4pt]
                  \dfrac{\partial y}{\partial u} & \dfrac{\partial y}{\partial v}
              \end{vmatrix} \ne 0.
          \]

    \item \textbf{极坐标换元:}
          \[
              \begin{cases}
                  x = r\cos\theta, \\
                  y = r\sin\theta,
              \end{cases}
              \quad
              \left|\dfrac{\partial(x,y)}{\partial(r,\theta)}\right|
              = r.
          \]
          因此:
          \[
              \iint_{D_{xy}} f(x,y)\,dx\,dy
              = \iint_{D_{r\theta}} f(r\cos\theta, r\sin\theta)\,r\,dr\,d\theta.
          \]
\end{enumerate}