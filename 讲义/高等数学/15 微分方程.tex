\GAOchapter{微分方程}
\section{求解微分方程并研究解的性质}
\subsection{一阶微分方程的求解}
\DTwoThree

若题目中出现 $y'$ 或 $dy=\cdots dx$,则通常属于以下类型:

\begin{enumerate}
    % ========== 可分离变量型 ==========
    \item \textbf{可分离变量型}(或可换元化为此形式)
          \begin{enumerate}
              \item 若可写为 $y' = f(x)g(y)$,
                    则分离变量得:
                    \[
                        \frac{dy}{g(y)} = f(x)\,dx,
                        \qquad
                        \int \frac{dy}{g(y)} = \int f(x)\,dx.
                    \]

              \item 若可写为 $y' = f(ax+by+c)$,
                    令 $u=ax+by+c$,则 $u' = a + b f(u)$,
                    于是:
                    \[
                        \frac{du}{a + b f(u)} = dx,
                        \qquad
                        \int \frac{du}{a + b f(u)} = \int dx.
                    \]
          \end{enumerate}

          % ========== 齐次型 ==========
    \item \textbf{齐次型}(或可换元化为此形式)
          \begin{enumerate}
              \item 若 $y' = f\!\left(\frac{y}{x}\right)$,
                    令 $\displaystyle \frac{y}{x} = u$,即 $y=ux,\ y' = u + x u'$,
                    代入得:
                    \[
                        x\,\frac{du}{dx} + u = f(u)
                        \quad\Rightarrow\quad
                        \frac{du}{f(u)-u} = \frac{dx}{x}.
                    \]

              \item 若 $\displaystyle \frac{1}{y'} = f\!\left(\frac{x}{y}\right)$,
                    令 $\displaystyle \frac{x}{y} = u$,即 $x=uy,\ x' = u + y u'$,
                    代入得:
                    \[
                        y\,\frac{du}{dy} + u = f(u)
                        \quad\Rightarrow\quad
                        \frac{du}{f(u)-u} = \frac{dy}{y}.
                    \]

              \item 若 $y' = f\!\left(\frac{ax+by+c}{a_1x+b_1y+c_1}\right)$,则:
                    \begin{enumerate}
                        \item 若 $c^2 + c_1^2 = 0$,可化为 $y' = g\!\left(\frac{y}{x}\right)$;
                        \item 若 $c^2 + c_1^2 \neq 0$ 且 $\frac{a}{a_1}=\frac{b}{b_1}$,
                              可化为 $y' = g(ax+by)$;
                        \item 若 $c^2 + c_1^2 \neq 0$ 且 $\frac{a}{a_1}\neq\frac{b}{b_1}$,
                              由
                              \[
                                  \begin{cases}
                                      ax+by+c=0, \\
                                      a_1x+b_1y+c_1=0
                                  \end{cases}
                                  \quad\Rightarrow\quad (x_0,y_0)
                              \]
                              令
                              \[
                                  \begin{cases}
                                      x = X + x_0, \\
                                      y = Y + y_0,
                                  \end{cases}
                              \]
                              则方程化为
                              \[
                                  y' = f\!\left(\frac{aX+bY}{a_1X+b_1Y}\right)
                                  \Rightarrow
                                  \frac{dY}{dX} = g\!\left(\frac{Y}{X}\right).
                              \]
                    \end{enumerate}
          \end{enumerate}

          % ========== 一阶线性型 ==========
    \item \textbf{一阶线性型}(或可换元化为此形式)

          若 $y' + p(x)y = q(x)$,
          则通解为:
          \[
              y = e^{-\int p(x)\,dx}
              \left[
                  \int e^{\int p(x)\,dx}\, q(x)\,dx + C
                  \right].
          \]

          % ========== 伯努利型 ==========
    \item \textbf{伯努利方程}

          若 $y' + p(x)y = q(x) y^n \ (n\neq0,1)$,
          变形、换元如下:
          \[
              y^{-n} y' + p(x) y^{1-n} = q(x),
              \qquad
              \text{令 } z = y^{1-n},
          \]
          则
          \[
              \frac{1}{1-n}\frac{dz}{dx} + p(x)z = q(x),
          \]
          为一阶线性方程,可按前述方法求解.
\end{enumerate}

\subsection{二阶可降阶微分方程的求解}
\DTwoThree

若二阶方程中缺少某个变量($x$ 或 $y$),可通过设 $y' = p$ 将其降为一阶方程.

\begin{enumerate}
    % ============================================================
    \item \textbf{缺 $y$ 的情形}:
          方程可写成
          \[
              y'' = f(x, y') \quad \text{或} \quad y'' = f(y').
          \]
          \begin{enumerate}[label=(\roman*)]
              \item 令 $y' = p$,则 $y'' = p' = \dfrac{dp}{dx}$,
                    方程化为一阶方程
                    \[
                        \frac{dp}{dx} = f(x, p)
                        \quad \text{或} \quad
                        \frac{dp}{dx} = f(p).
                    \]
              \item 若其通解为 $p = \varphi(x, C_1)$,即
                    \[
                        y' = \varphi(x, C_1),
                    \]
                    再对 $x$ 积分得:
                    \[
                        \boxed{
                            y = \int \varphi(x, C_1)\,dx + C_2.
                        }
                    \]
          \end{enumerate}

          % ============================================================
    \item \textbf{缺 $x$ 的情形}:
          方程可写成
          \[
              y'' = f(y, y').
          \]
          \begin{enumerate}[label=(\roman*)]
              \item 令 $y' = p$,则
                    \[
                        y'' = \frac{dp}{dx}
                        = \frac{dp}{dy} \cdot \frac{dy}{dx}
                        = p\,\frac{dp}{dy}.
                    \]
                    代入原方程得一阶方程:
                    \[
                        p\,\frac{dp}{dy} = f(y, p).
                    \]
              \item 若其通解为 $p = \varphi(y, C_1)$,
                    则由 $p = \dfrac{dy}{dx}$ 得
                    \[
                        \frac{dy}{dx} = \varphi(y, C_1),
                        \qquad
                        \text{即 } \frac{dy}{\varphi(y, C_1)} = dx.
                    \]
              \item 两边积分得:
                    \[
                        \boxed{
                            \int \frac{dy}{\varphi(y, C_1)} = x + C_2,
                        }
                    \]
                    即为原方程的通解.
          \end{enumerate}
\end{enumerate}
\subsection{高阶常系数线性微分方程的求解}
\DTwoThree

一般形式为
\[
    y'' + p y' + q y = f(x),
\]
其中 $p,q$ 为常数,$f(x)$ 为已知函数。

\subsubsection*{微分算子法}

约定:
\[
    D=\frac{d}{dx},\quad
    Dy=\frac{dy}{dx},\quad
    D^2y=\frac{d^2y}{dx^2}.
\]
于是微分方程可写为
\[
    (D^{2}+pD+q)y=f(x),
\]
记算子多项式为
\[
    F(D)=D^{2}+pD+q.
\]
则特解可形式写为:
\[
    y^{*}=\frac{1}{F(D)}\,f(x).
\]

并约定 $\frac{1}{D}$ 表示积分操作,例如
\[
    D\sin x=\cos x,\qquad
    \frac{1}{D}\sin x=-\cos x\ (C=0).
\]

\subsubsection*{1.\quad$\frac{1}{F(D)}e^{\alpha x}$ 型}

\[
    F(D)e^{\alpha x}=F(\alpha)e^{\alpha x}.
\]

\begin{enumerate}
    \item  若 $F(\alpha)\neq 0$,则
          \[
              y^{*}=\frac{1}{F(\alpha)}e^{\alpha x}.
          \]

    \item  若 $F(\alpha)=0,\ F'(\alpha)\neq 0$(一次重复根):
          \[
              y^{*}=x\frac{1}{F'(\alpha)}e^{\alpha x}.
          \]

    \item  若 $F(\alpha)=F'(\alpha)=0,\ F''(\alpha)\neq 0$(二次重复根):
          \[
              y^{*}=x^{2}\frac{1}{F''(\alpha)}e^{\alpha x}.
          \]
\end{enumerate}

\paragraph*{例题}
\begin{enumerate}
    \item $y'' + y' - 2y = 2$

          \[
              y^* = \frac{2e^{0x}}{F(0)}
              =\frac{2}{-2}=-1.
          \]

    \item $y'' + y' - 2y = e^x$

          \[
              F(1)=0,\qquad F'(1)=3.
          \]
          \[
              y^*=x\frac{1}{3}e^x=\frac13 xe^x.
          \]

    \item $y'' - 2y' + y = e^x$

          \[
              F(1)=0,\quad F'(1)=0,\quad F''(1)=2.
          \]
          \[
              y^{*}=\frac12 x^{2}e^{x}.
          \]
\end{enumerate}


\subsubsection*{2.\quad$\frac{1}{F(D)}\sin\beta x,\ \frac{1}{F(D)}\cos\beta x$ 型}

\paragraph{(1) 当 $F(D)=D^{2}+q$}

\[
    F(i\beta)=q-\beta^{2}.
\]

\begin{enumerate}
    \item 若 $q-\beta^{2}\neq 0$:
          \[
              y^*=\frac{1}{q-\beta^{2}}\sin\beta x,\qquad
              y^*=\frac{1}{q-\beta^{2}}\cos\beta x.
          \]

    \item  若 $q-\beta^{2}=0$(共振):
          \[
              y^{*}=x\frac{1}{F'(D)}\sin\beta x.
          \]
\end{enumerate}

\paragraph{(2) 当 $F(D)=D^{2}+pD+q$}

代入 $D^{2}=-\beta^{2}$ 得
\[
    F(D)\big|_{D^{2}=\beta i^{2}}=pD+q-\beta^{2}.
\]

\paragraph*{例题}
\begin{enumerate}
    \item $y''-y=\sin x$

          \[
              F(i)= -1 -1 = -2\neq 0.
          \]
          \[
              y^{*}=-\frac12\sin x.
          \]

    \item $y''+4y=\sin 2x$

          \[
              F(2i)= -4+4=0 \Rightarrow \text{共振}.
          \]
          \[
              y^{*}=x\frac{1}{2D}\sin 2x
              =-\frac14 x\cos 2x.
          \]

    \item $y''-3y'+2y=-\frac12\cos 2x$

          \[
              F(D)\to -4-3D+2=-(3D+2)
          \]
          \[
              y^{*}=\frac12\cdot\frac{1}{3D+2}\cos 2x
              =\frac{1}{40}(3\sin 2x+\cos 2x).
          \]
\end{enumerate}

\subsubsection*{3.\quad$\frac{1}{F(D)}P_k(x)$(多项式)}

将
\[
    \frac{1}{F(D)}
\]
展开成 $k$ 次泰勒多项式:
\[
    Q_k(D)=b_0+b_1D+\cdots+b_kD^k.
\]

则特解为:
\[
    y^{*}=Q_k(D)\,P_k(x).
\]

\paragraph*{例题}
$y''+y'=x^{2}+1$

\[
    y^{*}=\frac{1}{D(D+1)}(x^{2}+1)
    =\frac{1}{D}(1-D+D^{2})(x^{2}+1).
\]

计算得:
\[
    y^{*}=\frac13 x^{3}-x^{2}+3x-1.
\]


\subsubsection*{4.\quad$\frac{1}{F(D)}e^{ax}v(x)$ 型(平移算子)}

\[
    y^{*}=\frac{1}{F(D)}e^{ax}v(x)
    =e^{ax}\cdot\frac{1}{F(D+a)}v(x).
\]

\paragraph*{例题}
\begin{enumerate}
    \item $y''+4y'+5y=e^{-2x}\sin x$

          \[
              y^{*}=e^{-2x}\frac{1}{(D-2)^{2}+4(D-2)+5}\sin x
              =e^{-2x}\frac{1}{D^{2}+1}\sin x.
          \]
          \[
              y^{*}=-\frac12 x e^{-2x}\cos x.
          \]

    \item $y''-3y'+2y=2xe^x$

          \[
              y^{*}=2e^x\cdot\frac{1}{(D+1)^2-3(D+1)+2}x
              =2e^x\cdot\frac{1}{D(D-1)}x.
          \]

          \[
              y^{*}
              =2e^x\Bigl(-\frac{1}{D}-1\Bigr)x
              =-x(x+2)e^{x}.
          \]
\end{enumerate}

\begin{enumerate}
    % ============================================================
    \item \textbf{二阶非齐次方程:}
          \[
              y'' + p y' + q y = f(x).
          \]
          \begin{enumerate}[label=(\roman*)]
              \item 写出特征方程:
                    \[
                        \lambda^2 + p\lambda + q = 0,
                        \quad \Rightarrow \quad
                        \lambda_1,\,\lambda_2.
                    \]
              \item 得齐次方程通解:
                    \[
                        y_h =
                        \begin{cases}
                            C_1 e^{\lambda_1 x} + C_2 e^{\lambda_2 x}, & \lambda_1 \neq \lambda_2, \\[0.5em]
                            (C_1 + C_2 x)e^{\lambda x},                & \lambda_1 = \lambda_2.
                        \end{cases}
                    \]
              \item 设特解为 $\dot{y}$,代入原方程求待定系数;
              \item 写出通解:
                    \[
                        \boxed{y = y_h + \dot{y}.}
                    \]
          \end{enumerate}

          % ============================================================
    \item \textbf{右端项可分解时:}
          \[
              y'' + p y' + q y = f_1(x) + f_2(x).
          \]
          \begin{enumerate}[label=(\roman*)]
              \item 对每一项分别求特解:
                    \[
                        \begin{cases}
                            y'' + p y' + q y = f_1(x) \quad \Rightarrow \quad \text{特解 } \dot{y}_1, \\[0.3em]
                            y'' + p y' + q y = f_2(x) \quad \Rightarrow \quad \text{特解 } \dot{y}_2.
                        \end{cases}
                    \]
              \item 总特解为两者之和:
                    \[
                        \boxed{\dot{y} = \dot{y}_1 + \dot{y}_2.}
                    \]
              \item 写出总通解:
                    \[
                        y = y_h + \dot{y}.
                    \]
          \end{enumerate}

          % ============================================================
    \item \textbf{欧拉方程(Cauchy–Euler 型)}
          \[
              x^2 y'' + p x y' + q y = f(x).
          \]
          \begin{enumerate}[label=(\roman*)]
              \item 当 $x>0$ 时,令 $x = e^t$,则 $t = \ln x$,有
                    \[
                        \frac{dy}{dx} = \frac{1}{x}\frac{dy}{dt},
                        \quad
                        \frac{d^2y}{dx^2}
                        = \frac{1}{x^2}\left( \frac{d^2y}{dt^2} - \frac{dy}{dt} \right).
                    \]
                    代入得
                    \[
                        \frac{d^2y}{dt^2} + (p - 1)\frac{dy}{dt} + qy = f(e^t).
                    \]
                    解此方程后,用 $t = \ln x$ 回代.
              \item 当 $x<0$ 时,令 $x = -e^t$,同理可解.
          \end{enumerate}

          % ============================================================
    \item \textbf{$n$阶齐次线性方程:}
          \[
              y^{(n)} + a_1 y^{(n-1)} + \cdots + a_n y = 0.
          \]
          其特征方程为
          \[
              \lambda^n + a_1 \lambda^{n-1} + \cdots + a_n = 0.
          \]
          根据根的类型写通解:
          \begin{enumerate}[label=(\roman*)]
              \item 若 $\lambda$ 为\textbf{单实根}:
                    \[
                        y = C e^{\lambda x};
                    \]
              \item 若 $\lambda$ 为\textbf{$k$重实根}:
                    \[
                        y = (C_1 + C_2 x + \cdots + C_k x^{k-1}) e^{\lambda x};
                    \]
              \item 若 $\lambda = \alpha \pm \beta i$ 为\textbf{单复根}:
                    \[
                        y = e^{\alpha x}(C_1 \cos \beta x + C_2 \sin \beta x);
                    \]
              \item 若 $\lambda = \alpha \pm \beta i$ 为\textbf{二重复根}:
                    \[
                        y = e^{\alpha x}(C_1 \cos \beta x + C_2 \sin \beta x
                        + C_3 x \cos \beta x + C_4 x \sin \beta x).
                    \]
          \end{enumerate}
\end{enumerate}

\subsection{用换元法求解微分方程}

\DTwoThree
\begin{enumerate}
    \item \textbf{根据表达式形式直接换元}
          \begin{enumerate}
              \item 若出现 $f(x\pm y)$,令 $t = x \pm y$;
              \item 若出现 $f(xy)$,令 $t = xy$;
              \item 若出现 $f\!\left(\dfrac{y}{x}\right)$,令 $t = \dfrac{y}{x}$;
              \item 若出现 $f(x^{2}\pm y^{2})$,令 $t = x^{2} \pm y^{2}$.
          \end{enumerate}
          (说明:这类方程常见于可分离或齐次方程.换元后需写出 $dt$ 与 $dx,dy$ 的关系以进行分离.)

    \item \textbf{逆用求导公式进行换元}

          若方程中某部分可看作已知复合函数的导数形式,可通过“逆用求导”来换元简化.
          \[
              \begin{cases}
                  \text{见到 } f^{\prime}[g(x)] \cdot g^{\prime}(x) \ \Rightarrow \ \left(f[g(x)]\right)^{\prime},      & \text{令 } u = f[g(x)];   \\
                  \text{见到 } f^{\prime}(x) g(x) + f(x) g^{\prime}(x) \ \Rightarrow \ \left(f(x) g(x)\right)^{\prime}, & \text{令 } u = f(x) g(x).
              \end{cases}
          \]
          (说明:此法相当于“识别导数结构”,遇到链式或积的求导形式时可直接降阶或积分.)

    \item \textbf{交换自变量与因变量}

          当微分方程关于 $x$ 的形式复杂、而关于 $y$ 的形式更简单时,可以交换 $x,y$ 的角色,
          即令 $x$ 为 $y$ 的函数,用 $\dfrac{dx}{dy}$ 代替 $\dfrac{dy}{dx}$.

          (说明:尤其当方程“缺 $x$”时,令 $p = \dfrac{dy}{dx}$,转为 $p\dfrac{dp}{dy} = f(y,p)$ 类型;若“缺 $y$”,则令 $p = \dfrac{dy}{dx}$,转为 $p' = f(x,p)$.)
\end{enumerate}

\section{建立微分方程并求解}
寻找信息点$A$与信息点$B$,根据题设关系,建立方程

\DThree
\subsection{用极限、导数、积分表达式建方程}
\DTwoTwo+\DThree
\begin{enumerate}
    \item 信息点$A,B$为极限、导数、积分表达式且为等量关系时,令$A=B$,建立方程。
    \item 信息点为$f(x),g(x)$及$f^\prime(x),g^{\prime}(x)$的等量关系组,用求导、消元建方程.如:

          $$\begin{cases}f'(x)+xf'(-x)=x,\\f'(-x)-xf'(x)=-x\end{cases}\Rightarrow f'(x)=\frac{x+x^2}{1+x^2}\Rightarrow f(x)=x+\frac{1}{2}\ln(1+x^2)-\arctan x+C\:.$$
    \item 信息点为关于$x,y$的恒等式,如$f(xy)=yf(x)+xf(y)$,代入特殊点,并写$f^\prime(x)$定义.
\end{enumerate}
\subsection{用几何量表达式建方程}
\DThree

\begin{enumerate}
    \item 用曲线的切线斜率.
          $$k = f'(x_0) = \tan \alpha.$$
    \item 用两曲线 $f(x)$ 与 $g(x)$ 的公切线斜率.
          $$f'(x_0) = g'(x_0).$$
    \item 用截距.
          $$Y - y = y'(X - x) \begin{cases} \text{令 } Y = 0, 则 X = x - \frac{y}{y'} (\text{x轴上的截距}); \\ \text{令 } X = 0, 则 Y = y - xy' (\text{y轴上的截距}). \end{cases}$$
          如, 令 $X = Y$, 建等式(方程).
    \item 用面积.
          $$
              \int_a^b f(x) \, \mathrm{d}x.$$
    \item 用体积.

          $$V_{x}=\int_{a}^{b}\pi f^{2}(x)\mathrm{d}x\:,\:V_{y}=\int_{a}^{b}2\pi x\big|f(x)\big|\mathrm{d}x\:.$$

    \item 用平均值.


          $$\overline{f}=\frac{1}{b-a}\int_{a}^{b}f(x)\mathrm{d}x=f(\xi)\:.$$

    \item 用弧长.

          $$s=\int_{a}^{b}\sqrt{1+\left(y_{x}^{\prime}\right)^{2}}\mathrm{d}x\:.$$

    \item 用侧面积.


          $$S=\int_{a}^{b}2\pi\big|y(x)\big|\sqrt{1+(y_{x}^{\prime})^{2}}\mathrm{d}x\:.$$

    \item 用曲率.
          $$k=\frac{|y^{\prime\prime}|}{\left[1+(y^{\prime})^2\right]^{\frac{3}{2}}}.$$

    \item 用形心.

          $$\overline{x}=\frac{\iint_Dx\:\mathrm{d}\sigma}{\iint_D\mathrm{d}\sigma}\:,\:\overline{y}=\frac{\iint_Dy\:\mathrm{d}\sigma}{\iint_D\mathrm{d}\sigma}\:.$$
\end{enumerate}
\subsection{用变化率建方程}
\DThree
\begin{enumerate}
    \item  信息点A的变化率与B成比例,令$\frac{\mathrm{d}A}{\mathrm{d}t}=\pm kB$.

          物理背景:位移、速度、加速度,$F=ma$.
    \item 见到“$P$点的运动方向始终指向$Q$点”,立即寻找◯信息点$A:P$点处的切线斜率$.\textcircled{2}$信息点$B:PQ$

          连线与水平线夹角的正切值.令(1)=2(注意正负).
\end{enumerate}


