\GAOchapter{函数极限与连续}

\section{判定类型,做好计算}

\begin{enumerate}
    %-------------------------------------------------------
    \item \textbf{未定式整体判定} \DTwoThree

          一般地,见到以下形式的极限:
          \[
              \frac{?}{0}, \quad \frac{0}{?}, \quad \frac{\infty}{?}, \quad
              \frac{?}{\infty}, \quad ?\cdot\infty, \quad
              \infty\cdot?, \quad \infty-?, \quad \infty^{?}, \quad ?^{\infty},
          \]
          可直接判断它们分别对应的七种未定式类型:
          \[
              \frac{0}{0}, \quad \frac{\infty}{\infty}, \quad 0\cdot\infty, \quad
              \infty-\infty, \quad \infty^{0}, \quad 0^{0}, \quad 1^{\infty}.
          \]
          若题设形式不属于这七类,则通常不是未定式问题。

          %-------------------------------------------------------
    \item \textbf{未定式局部判定} \DTwoThree

          例如:
          \[
              \lim_{n\to\infty}\frac{1+x}{1+nx^{2n}}=
              \begin{cases}
                  0,   & x=\pm1, \\
                  1+x, & |x|<1,  \\
                  0,   & |x|>1.
              \end{cases}
          \]
          关键在于判定局部项 $nx^{2n}$ 的极限性质。常见局部极限总结如下:

          \[
              \lim_{n\to\infty}|x|^n=
              \begin{cases}
                  \infty, & |x|>1, \\
                  1,      & |x|=1, \\
                  0,      & |x|<1;
              \end{cases}
              \quad
              \lim_{x\to0^+}x^a=
              \begin{cases}
                  0,       & a>0, \\
                  1,       & a=0, \\
                  +\infty, & a<0;
              \end{cases}
          \]
          \[
              \lim_{n\to\infty}nx^{2n}=
              \begin{cases}
                  +\infty, & |x|\ge1, \\
                  0,       & |x|<1;
              \end{cases}
              \quad
              \lim_{n\to\infty}e^{nx}=
              \begin{cases}
                  +\infty, & x>0, \\
                  1,       & x=0, \\
                  0,       & x<0;
              \end{cases}
          \]
          \[
              \lim_{n\to\infty}n^x=
              \begin{cases}
                  +\infty, & x>0, \\
                  1,       & x=0, \\
                  0,       & x<0.
              \end{cases}
          \]

          %-------------------------------------------------------
    \item \textbf{常用的无穷小量阶的比较}

          \begin{enumerate}
              \item \textbf{普通函数型:}

                    当 $x\to0$ 时,
                    \[
                        \sin x \sim x, \quad \tan x \sim x, \quad
                        \arcsin x \sim x, \quad \arctan x \sim x,
                    \]
                    \[
                        e^x - 1 \sim x, \quad \ln(1+x) \sim x, \quad
                        \ln(x + \sqrt{1+x^2}) \sim x,
                    \]
                    \[
                        a^x - 1 = e^{x\ln a}-1 \sim x\ln a \ (a>0,a\ne1),
                        \quad 1 - \cos x \sim \tfrac{1}{2}x^2,
                    \]
                    \[
                        1 - \cos^{\alpha}x \sim \tfrac{\alpha}{2}x^2 \ (\alpha\ne0),
                        \quad (1+x)^{\alpha}-1 \sim \alpha x,
                        \quad (1+x)^x - 1 \sim x^2.
                    \]

              \item \textbf{差函数型:}

                    当 $x\to0$ 时,
                    \[
                        x - \sin x \sim \tfrac{1}{6}x^3, \quad
                        x - \arcsin x \sim -\tfrac{1}{6}x^3, \quad
                        x - \tan x \sim -\tfrac{1}{3}x^3, \quad
                        x - \arctan x \sim \tfrac{1}{3}x^3,
                    \]
                    \[
                        x - \ln(1+x) \sim \tfrac{1}{2}x^2,
                        \quad e^x - 1 - x \sim \tfrac{1}{2}x^2.
                    \]

                    可通过恒等变形“创造”差函数,如:
                    \[
                        \begin{cases}
                            x - \ln(1+\tan x) = (x - \tan x) + (\tan x - \ln(1+\tan x)); \\[4pt]
                            \sin x + \ln(1 - \sin x) = -[-\sin x - \ln(1 - \sin x)];     \\[4pt]
                            f(x) - \tan x = (f(x) - x) + (x - \tan x).
                        \end{cases}
                    \]

              \item \textbf{复合函数型:}

                    若 $f(x)\sim ax^m,\ g(x)\sim bx^n$ 且 $ab\ne0$,则
                    \[
                        f[g(x)] \sim ab^m x^{mn}.
                    \]

              \item \textbf{变上限积分型:}
                    \begin{enumerate}
                        \item 若 $f(x)\sim ax^m$,则
                              \[
                                  \int_0^x f(t)\,\mathrm{d}t \sim \int_0^x a t^m\,\mathrm{d}t.
                              \]
                        \item 若 $\lim_{x\to0}f(x)=A\ne0$, $h(x)\to0$,则
                              \[
                                  \int_0^{h(x)} f(t)\,\mathrm{d}t \sim A\,h(x).
                              \]
                    \end{enumerate}

              \item \textbf{复合+变上限积分型:}
                    \[
                        f(x)\sim ax^m,\ g(x)\sim bx^n
                        \Rightarrow
                        \int_0^{g(x)}f(t)\,\mathrm{d}t \sim \int_0^{bx^n}a t^m\,\mathrm{d}t.
                    \]

              \item \textbf{带头大哥型:}

                    若 $\alpha=o(\beta)$,则
                    \[
                        \textcircled{1}\ \alpha+\beta\sim\beta, \qquad
                        \textcircled{2}\ \alpha+\beta\text{ 与 }\beta\text{同号}, \qquad
                        \textcircled{3}\ \alpha\beta=o(\beta^2).
                    \]
          \end{enumerate}

          %-------------------------------------------------------
    \item \textbf{常用无穷大量阶的比较}

          \[
              \begin{cases}
                  \text{当 }x\to+\infty: & \ln^p x \ll x^q \ll a^x \ll x^x,        \\[4pt]
                  \text{当 }n\to\infty:  & \ln^p n \ll n^q \ll a^n \ll n! \ll n^n,
              \end{cases}\quad (p,q>0,a>1)
          \]
          因此:
          \[
              \lim_{n\to\infty}\frac{\ln^p n}{n^q}=0,\quad
              \lim_{n\to\infty}\frac{n^q}{a^n}=0,\quad
              \lim_{n\to\infty}\frac{a^n}{n!}=0,\quad
              \lim_{n\to\infty}\frac{n!}{n^n}=0.
          \]

          %-------------------------------------------------------
    \item \textbf{涉及 $\infty$ 的计算问题} \DTwoThree

          关于 $\infty-\infty$ 型,注意以下四点:

          \begin{enumerate}
              \item 若 $f(x)$ 在 $|x|$ 足够大时有定义,则
                    \[
                        \lim_{x\to\infty}f(x) \text{ 存在}
                        \iff
                        \lim_{x\to+\infty}f(x) = \lim_{x\to-\infty}f(x).
                    \]

              \item 若出现差式 $f(x)-f(x)$(如三角、对数、反三角函数差),
                    可用\textbf{拉格朗日中值定理}变形再求极限。

              \item 若出现幂次差:
                    \[
                        [f_1(x)]^{g(x)}-[f_2(x)]^{g(x)} = [f_2(x)]^{g(x)}
                        \left(\left[\frac{f_1(x)}{f_2(x)}\right]^{g(x)}-1\right),
                    \]
                    \[
                        [f(x)]^{g_1(x)}-[f(x)]^{g_2(x)} = [f(x)]^{g_2(x)}
                        \left([f(x)]^{g_1(x)-g_2(x)}-1\right).
                    \]

              \item 若见 $\lim_{x\to\infty}[f(x)-ax]$,
                    常通过\textbf{恒等变形}将其改写为乘除形式处理。
          \end{enumerate}
\end{enumerate}

\section{判定连续与间断}

\begin{enumerate}
    %---------------------------------------------------
    \item \textbf{常见备选点判定}

          常见函数的无定义点或分段点如下表所示:

          \begin{enumerate}
              \item $\displaystyle \mathrm{e}^{\frac{1}{x}} \Rightarrow x=0$ 为无定义点;
              \item $\displaystyle \frac{1}{\int_{1}^{x}|\sin t|\,\mathrm{d}t} \Rightarrow x=\pm1$ 为无定义点;
              \item $\displaystyle \frac{1}{\sin x} \Rightarrow x=k\pi \ (k=0,\pm1,\pm2,\cdots)$ 为无定义点;
              \item $\displaystyle \frac{1}{\arctan x} \Rightarrow x=0$ 为无定义点;
              \item $\displaystyle \frac{1}{\tan\!\left(x-\frac{\pi}{4}\right)},\ 0<x<2\pi
                        \Rightarrow x=\frac{\pi}{4},\frac{3\pi}{4},\frac{5\pi}{4},\frac{7\pi}{4}$ 为无定义点;
              \item $\displaystyle \frac{1}{|x|(x^2-1)} \Rightarrow x=0,\pm1$ 为无定义点;
              \item $[x] \Rightarrow x=n\ (n=0,\pm1,\pm2,\cdots)$ 为分段点;
              \item $\displaystyle |x|^{\frac{1}{(1-x)(x-2)}} \Rightarrow x=0,1,2$ 为无定义点。
          \end{enumerate}

          \vspace{0.5em}
          \textit{技巧提示:}
          若题目中出现分母、根号、对数或分段定义函数,应先检查其定义域边界与分段点,这些往往是可能的间断点。

          %---------------------------------------------------
    \item \textbf{计算三个关键值}

          判定连续性需依次计算以下三个值(或判断其是否存在):
          \[
              \lim_{x\to x_0^-}f(x), \quad
              \lim_{x\to x_0^+}f(x), \quad
              f(x_0).
          \]
          然后进行比较。

          %---------------------------------------------------
    \item \textbf{根据定义作出结论}

          若以上三者不全相等或有不存在的情况,按下列规则分类:

          \begin{enumerate}
              \item \textbf{跳跃间断点(第一类间断)}:
                    \[
                        \lim_{x\to x_0^-}f(x) \neq \lim_{x\to x_0^+}f(x),
                    \]
                    且左右极限都存在。

              \item \textbf{可去间断点}:
                    \[
                        \lim_{x\to x_0}f(x) \text{ 存在}, \quad
                        \text{但 } f(x_0)\text{ 未定义或 }\lim f(x)\neq f(x_0).
                    \]
                    若通过重新定义 $f(x_0)=\lim f(x)$ 可使函数连续。

              \item \textbf{无穷间断点}:
                    \[
                        \lim_{x\to x_0^\pm}f(x)=\pm\infty.
                    \]
                    函数趋于无穷大,图像在此处“竖直渐近”。

              \item \textbf{振荡间断点(第二类间断)}:
                    \[
                        \lim_{x\to x_0}f(x) \text{ 不存在且无穷振荡}.
                    \]
                    例如 $\sin\frac{1}{x}$ 在 $x=0$ 处。
          \end{enumerate}
\end{enumerate}
\section{研究 $x \to \cdot$ 时 $f(x)$ 的微观性态}

\begin{enumerate}
    %--------------------------
    \item \textbf{定义法} \DTwoTwo

          极限的 $\varepsilon$–$\delta$ 定义:
          \[
              \lim_{x \to x_0} f(x) = A
              \;\Leftrightarrow\;
              \forall \varepsilon > 0,\ \exists \delta > 0,\
              \text{当 } 0 < |x - x_0| < \delta \text{ 时},\ |f(x) - A| < \varepsilon.
          \]

          ✅ \textit{说明:}
          表明当 $x$ 无限接近 $x_0$ 时,$f(x)$ 可以被控制在 $A$ 的任意邻域内。

          %--------------------------
    \item \textbf{局部保号性} \DTwoTwo

          \begin{enumerate}
              \item 若 $f(x) \to A \ (x \to x_0)$ 且 $A > 0$(或 $A < 0$),
                    则存在 $\delta > 0$,使得当 $0 < |x - x_0| < \delta$ 时,
                    $f(x) > 0$(或 $f(x) < 0$)。

              \item 若在 $x_0$ 的某去心邻域内 $f(x) \ge 0$(或 $\le 0$),
                    且 $\displaystyle \lim_{x \to x_0} f(x) = A$,
                    则 $A \ge 0$(或 $A \le 0$)。
          \end{enumerate}

          ✅ \textit{应用技巧:}
          在求极限符号问题(如 $\lim f(x)/g(x)$ 是否为正)时,常结合保号性与等价无穷小判断符号。

          %--------------------------
    \item \textbf{夹逼准则(两边夹法)} \DTwoTwo+\DTwoThree

          若存在函数 $g(x)$、$h(x)$ 使:
          \begin{enumerate}
              \item $h(x) \le f(x) \le g(x)$;
              \item $\displaystyle \lim_{x \to x_0} h(x) = \lim_{x \to x_0} g(x) = A$;
          \end{enumerate}
          则 $\displaystyle \lim_{x \to x_0} f(x) = A$。

          ✅ \textit{常见应用:}
          \[
              \lim_{x \to 0} x^2 \sin\frac{1}{x} = 0, \quad
              \lim_{x \to 0} x \sin\frac{1}{x^2} = 0.
          \]

          %--------------------------
    \item \textbf{单调有界准则} \DTwoThree

          若存在 $\delta > 0$,使得:

          \begin{enumerate}
              \item $f(x)$ 在区间 $(x_0, x_0 + \delta)$ 内单调且有界,
                    则右极限 $\displaystyle \lim_{x \to x_0^+} f(x)$ 存在;
              \item $f(x)$ 在区间 $(x_0 - \delta, x_0)$ 内单调且有界,
                    则左极限 $\displaystyle \lim_{x \to x_0^-} f(x)$ 存在。
          \end{enumerate}

          同理,若 $f(x)$ 在 $(a, +\infty)$ 或 $(-\infty, b)$ 上单调有界,
          则 $\displaystyle \lim_{x \to +\infty} f(x)$ 或 $\lim_{x \to -\infty} f(x)$ 存在。

          ✅ \textit{典型函数:}
          \[
              f(x) = 1 - \frac{1}{x},\quad f(x)=\arctan x,\quad f(x)=1 - \frac{1}{2^x}.
          \]
\end{enumerate}