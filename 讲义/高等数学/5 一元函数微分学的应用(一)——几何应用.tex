\GAOchapter{一元函数微分学的应用(一)——几何应用}

\section{切线、法线与截距}
\DTwoTwo

设 $y = f(x)$ 在 $x_0$ 处可导,切点为 $(x_0, y_0)$。

\begin{enumerate}
    \item \textbf{切线方程:}
          \[
              y - y_0 = f'(x_0)(x - x_0).
          \]
    \item \textbf{法线方程:}
          \[
              y - y_0 = -\frac{1}{f'(x_0)}(x - x_0).
          \]
    \item \textbf{各截距公式:}
          \[
              \begin{aligned}
                  \text{x轴切线截距:} & \quad x_0 - \frac{y_0}{f'(x_0)}, \\
                  \text{y轴切线截距:} & \quad y_0 - x_0 f'(x_0),         \\
                  \text{x轴法线截距:} & \quad x_0 + y_0 f'(x_0),         \\
                  \text{y轴法线截距:} & \quad y_0 + \frac{x_0}{f'(x_0)}.
              \end{aligned}
          \]
\end{enumerate}


\section{单调性、极值、凹凸性与拐点}
\DTwoTwo
\subsection{单调性判别}

设 $f(x)$ 在 $[a,b]$ 上连续、在 $(a,b)$ 内可导:
\begin{enumerate}
    \item 若 $f'(x) \ge 0$ 且仅在有限点取等号,则 $f(x)$ 在 $[a,b]$ 上\textbf{严格递增};
    \item 若 $f'(x) \le 0$ 且仅在有限点取等号,则 $f(x)$ 在 $[a,b]$ 上\textbf{严格递减}。
\end{enumerate}

\subsection{极值的定义与判别}

若存在 $x_0$ 的某邻域,使得
\[
    f(x) \le f(x_0) \ (\text{或 } f(x) \ge f(x_0)),
\]
则称 $x_0$ 为 $f(x)$ 的\textbf{极大值点}(或极小值点)。

\textbf{常用判别:}
\begin{enumerate}
    \item 一阶导数法:$f'(x_0) = 0$;
    \item 二阶导数法:若 $f''(x_0) > 0$,则为极小值;若 $f''(x_0) < 0$,则为极大值。
\end{enumerate}

\subsection{凹凸性的定义与判别}

\begin{enumerate}
    \item 定义法:
          \[
              f\!\left(\frac{x_1+x_2}{2}\right)
              \begin{cases}
                  < \frac{f(x_1)+f(x_2)}{2}, & \text{凹函数;} \\[4pt]
                  > \frac{f(x_1)+f(x_2)}{2}, & \text{凸函数.}
              \end{cases}
          \]
    \item 导数法:
          若 $f''(x) > 0$,则函数在该区间上\textbf{凹向上(凸)};
          若 $f''(x) < 0$,则函数在该区间上\textbf{凹向下(凹)}。
\end{enumerate}

\subsection{拐点}

连续曲线的凹弧与凸弧的分界点称为\textbf{拐点}。

\textbf{判别条件:}
\[
    f''(x_0) = 0 \quad \text{且 } f''(x) \text{ 在 } x_0 \text{ 附近变号}.
\]

\subsection{重要结论总结}

\[
    \boxed{
        \begin{aligned}
            \text{有极值点}        & \Leftrightarrow f'(x) \text{ 有零点},     \\[3pt]
            \text{有拐点}         & \Leftrightarrow f''(x) \text{ 有零点且变号}, \\[3pt]
            f'(x) \text{ 无零点}  & \Rightarrow f(x) \text{ 单调性不变},        \\[3pt]
            f''(x) \text{ 无零点} & \Rightarrow f'(x) \text{ 单调性不变.}
        \end{aligned}
    }
\]

\section{渐近线}

\textbf{定义:} 曲线 $y=f(x)$ 的渐近线是指曲线无限接近的一条直线。

\begin{enumerate}
    \item \textbf{竖直渐近线:}
          若 $\displaystyle \lim_{x\to a^\pm}f(x)=\infty$,则 $x=a$ 为竖直渐近线。

    \item \textbf{水平渐近线:}
          若 $\displaystyle \lim_{x\to\infty}f(x)=A$ 或 $\lim_{x\to-\infty}f(x)=B$,
          则 $y=A$ 或 $y=B$ 为水平渐近线。

    \item \textbf{斜渐近线:}
          若 $\displaystyle \lim_{x\to\infty}\frac{f(x)}{x}=a$ 且 $\lim_{x\to\infty}[f(x)-ax]=b$,
          则 $y=ax+b$ 为斜渐近线。
\end{enumerate}

\section{最值与值域求法}
\DTwoThree
\begin{enumerate}
    \item 若 $f(x)$ 在 $[a,b]$ 上连续,则\textbf{最值只可能出现在:}
          \[
              \text{驻点、导数不存在点、区间端点}.
          \]
    \item 若 $f(x)$ 在 $(a,b)$ 内连续,且仅有一个极值点 $x_0$,
          则 $x_0$ 即为全区间的最值点。
    \item 若难以直接判断最值,可使用:
          \begin{enumerate}
              \item 平方和放缩法;
              \item 三角代换法;
              \item 单调性区间法;
              \item 取值范围不等式(如 $a^2+b^2\ge2ab$).
          \end{enumerate}
\end{enumerate}