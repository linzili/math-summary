\GAOchapter{一元函数积分学的概念与性质}

\section{求连和 $\sum_{k=1}^{n} a_k$、连积 $\prod_{k=1}^{n} a_k$ 的极限}

连积常可取对数化为连和:
\[
    \prod_{k=1}^{n} a_k = e^{\sum_{k=1}^{n} \ln a_k}.
\]

\subsection{基本型(能凑成 $\frac{i}{n}$)}

若通项中能化为含 $\frac{i}{n}$ 的形式,则可用定积分定义处理。典型结构:

\begin{enumerate}
    \item $n + i = n\left(1 + \frac{i}{n}\right)$;
    \item $n^2 + i^2 = n^2\left[1 + \left(\frac{i}{n}\right)^2\right]$;
    \item $n^2 + ni = n^2\left(1 + \frac{i}{n}\right)$。
\end{enumerate}

于是可直接套用定积分定义:
\[
    \lim_{n \to \infty} \sum_{i=1}^{n} f\!\left(0+\frac{1-0}{n}i\right) \frac{1-0}{n}
    = \int_{0}^{1} f(x)\,dx,
\]
或
\[
    \lim_{n \to \infty} \sum_{i=0}^{n-1} f\!\left(0+\frac{1-0}{n}i\right) \frac{1-0}{n}
    = \int_{0}^{1} f(x)\,dx.
\]

\subsection{放缩型(凑不成 $\frac{i}{n}$)}

\begin{enumerate}
    \item \textbf{夹逼准则法:}
          若通项含 $n^2+i$ 等无法化为 $\frac{i}{n}$ 的形式,可尝试对通项放缩后夹逼。

    \item \textbf{放缩后再凑 $\frac{i}{n}$:}
          如通项 $\frac{i^2+1}{n^2}$,虽凑不成 $\frac{i}{n}$,但可放缩为:
          \[
              \left(\frac{i}{n}\right)^2 < \frac{i^2+1}{n^2} < \left(\frac{i+1}{n}\right)^2,
          \]
          从而凑出 $\frac{i}{n}$,再按定积分处理。
\end{enumerate}

\subsection{变量上限型}

若通项含 $\frac{x}{n}$,则:
\[
    \lim_{n \to \infty} \sum_{i=1}^{n} f\!\left(0+\frac{x-0}{n}i\right) \frac{x-0}{n}
    = \int_{0}^{x} f(t)\,dt.
\]

\subsection{其他分割与取高方式}

若分割方法或取样点不同,依然可按定积分定义处理。例如:

\begin{enumerate}
    \item \textbf{极坐标分割型:}
          \begin{example}
              求极限 $\lim_{n\to\infty}\sum_{k=1}^{n}\frac{\pi}{4n}\cos^{2}\frac{k\pi}{4n}$。
          \end{example}
          \begin{solution}
              \[
                  \lim_{n\to\infty}\sum_{k=1}^{n}\cos^{2}\!\left(\frac{\pi k}{4n}\right)\frac{\pi}{4n}
                  = \int_{0}^{\pi/4}\cos^{2}\theta\,d\theta
                  = \frac{1}{4}+\frac{\pi}{8}.
              \]
          \end{solution}

    \item 取中点:$\frac{\frac{k-1}{n}+\frac{k}{n}}{2}$;
    \item 取凸组合:$\lambda_1\frac{k-1}{n}+\lambda_2\frac{k}{n}$,$\lambda_1+\lambda_2=1$;
    \item 取几何均值:$\sqrt{\frac{k-1}{n}\cdot\frac{k}{n}}$;
    \item 取均方根:$\sqrt{\frac{(\frac{k-1}{n})^2+(\frac{k}{n})^2}{2}}$;
    \item 取调和均值:$\frac{2}{\frac{1}{\frac{k-1}{n}}+\frac{1}{\frac{k}{n}}}$。
\end{enumerate}

\section{判断反常积分的敛散性}

\subsection{基本结论}

\begin{enumerate}
    \item $\displaystyle \int_0^{1}\frac{dx}{x^{p}}$
          \[
              \begin{cases}
                  \text{收敛}, & 0<p<1,  \\
                  \text{发散}, & p\ge 1;
              \end{cases}
          \]
          \quad
          $\displaystyle \int_0^{1}\frac{\ln x}{x^{p}}dx$
          \[
              \begin{cases}
                  \text{收敛}, & 0\le p<1, \\
                  \text{发散}, & p\ge 1.
              \end{cases}
          \]

    \item $\displaystyle \int_1^{+\infty}\frac{dx}{x^{p}}$
          \[
              \begin{cases}
                  \text{收敛}, & p>1,    \\
                  \text{发散}, & p\le 1;
              \end{cases}
          \]
          \quad
          $\displaystyle \int_1^{+\infty}\frac{\ln x}{x^{p}}dx$
          \[
              \begin{cases}
                  \text{收敛}, & p>1,    \\
                  \text{发散}, & p\le 1.
              \end{cases}
          \]
\end{enumerate}

\subsection{常用判定方法}

\begin{enumerate}
    \item \textbf{积分可拆性:}
          \[
              \int_{a}^{b} f(x)\,dx = \int_{a}^{c} f(x)\,dx + \int_{c}^{b} f(x)\,dx.
          \]

    \item \textbf{分母设置法:}
          将 $x^p=\frac{1}{x^{-p}}$,$\ln^q x=\frac{1}{\ln^{-q}x}$ 化简。

    \item \textbf{换元法:}
          常用 $1-x=t$ 或 $\ln x=t$。

    \item \textbf{分部积分法:}
          例如:
          \[
              \int_{1}^{+\infty}\frac{\sin x}{x}dx
              = -\frac{\cos x}{x}\Big|_1^{+\infty} - \int_{1}^{+\infty}\frac{\cos x}{x^{2}}dx.
          \]
          因 $\displaystyle \int_{1}^{+\infty}\left|\frac{\cos x}{x^{2}}\right|dx\le\int_{1}^{+\infty}\frac{1}{x^{2}}dx<\infty$,
          故原积分收敛但不绝对收敛。

    \item \textbf{等价代换:}
          当 $x\to 0$ 时,$\ln(1+x)\sim x$,$\arctan x\sim x$。

    \item \textbf{绝对值判敛法:}
          若 $\int |f(x)|dx$ 收敛,则 $\int f(x)dx$ 绝对收敛。
\end{enumerate}

\subsection{含参反常积分敛散性结论大观}

\begin{enumerate}
    \item $\displaystyle \int_{1}^{2}\frac{dx}{x\ln^{p}x}$
          \[
              \begin{cases}
                  \text{收敛}, & 0<p<1,  \\
                  \text{发散}, & p\ge 1.
              \end{cases}
          \]

    \item $\displaystyle \int_{2}^{+\infty}\frac{dx}{x\ln^{p}x}$
          与 $\sum_{n=2}^{\infty}\frac{1}{n\ln^{p}n}$ 同敛散:
          \[
              \begin{cases}
                  \text{收敛}, & p>1,    \\
                  \text{发散}, & p\le 1.
              \end{cases}
          \]

    \item $\displaystyle \int_{1}^{+\infty}\frac{dx}{x\ln^{p}x}$ 必发散。

    \item $\displaystyle \int_{k}^{+\infty} e^{-\alpha x}\ln^{p}x\,dx$
          或 $\displaystyle \int_{A}^{+\infty} e^{-\alpha x}x^{q}\,dx$:
          \[
              \begin{cases}
                  \alpha>0 \Rightarrow \text{收敛}, \\
                  \alpha<0 \Rightarrow \text{发散}.
              \end{cases}
          \]

    \item $\displaystyle \int_{1}^{2}\frac{dx}{x^{p}\ln^{q}x}$
          \[
              \begin{cases}
                  \text{收敛}, & 0<q<1,  \\
                  \text{发散}, & q\ge 1.
              \end{cases}
          \]

    \item $\displaystyle \int_{2}^{+\infty}\frac{dx}{x^{p}\ln^{q}x}$
          与 $\sum_{n=2}^{\infty}\frac{1}{n^{p}\ln^{q}n}$ 同敛散:
          \[
              \begin{cases}
                  \text{收敛}, & p>1,        \\
                  \text{发散}, & p<1,        \\
                  \text{收敛}, & p=1,q>1,    \\
                  \text{发散}, & p=1,q\le 1.
              \end{cases}
          \]

    \item $\displaystyle \int_{0}^{1}\frac{dx}{|\ln x|^{p}}$
          \[
              \begin{cases}
                  \text{收敛}, & p<1,    \\
                  \text{发散}, & p\ge 1.
              \end{cases}
          \]

    \item $\displaystyle \int_{1}^{2}\frac{dx}{|\ln x|^{p}}$
          \[
              \begin{cases}
                  \text{收敛}, & 0<p<1,  \\
                  \text{发散}, & p\ge 1.
              \end{cases}
          \]

    \item $\displaystyle \int_{2}^{\infty}\frac{dx}{|\ln x|^{p}}$ 必发散。
\end{enumerate}