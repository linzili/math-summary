\GAOchapter{一元函数积分学的应用(三)——物理应用}

\begin{enumerate}
    \item \textbf{变力沿直线做功}

          设沿 $x$ 轴方向的力为 $F(x)$,物体从 $x=a$ 移动到 $x=b$,则变力所做的功为
          $$
              W = \int_{a}^{b} F(x)\,\mathrm{d}x.
          $$
          功的微元为
          $$
              \mathrm{d}W = F(x)\,\mathrm{d}x.
          $$
          \textit{物理意义:} $F(x)$ 为“力—位移”关系曲线下的面积.

          \vspace{0.5em}

    \item \textbf{变速直线运动的位移}

          若质点的速度为 $v(t)$,则从时刻 $t_1$ 到 $t_2$ 的位移为
          $$
              s = \int_{t_1}^{t_2} v(t)\,\mathrm{d}t,
          $$
          微元形式为 $\mathrm{d}s = v(t)\,\mathrm{d}t$.

          \textit{推论:} 若已知加速度 $a(t)$,则速度变化量为 $\displaystyle v_2 - v_1 = \int_{t_1}^{t_2} a(t)\,\mathrm{d}t$.

          \vspace{0.5em}

    \item \textbf{抽水做功(典型应用)}

          把容器中水全部抽出所做的功为
          $$
              W = \rho g \int_{a}^{b} x A(x)\,\mathrm{d}x,
          $$
          其中:
          \begin{itemize}
              \item $\rho$ —— 水的密度;
              \item $g$ —— 重力加速度;
              \item $A(x)$ —— 水平截面积;
              \item $x$ —— 水被提升的高度.
          \end{itemize}
          功的微元:
          $$
              \mathrm{d}W = \rho g\,x\,A(x)\,\mathrm{d}x.
          $$
          \textit{思路:先确定截面积 $A(x)$,其余均为常量.}

          \vspace{0.5em}

    \item \textbf{静水压力}

          垂直浸没在水中的平板 $ABCD$(见图)一侧所受的水压力为
          $$
              P = \rho g \int_{a}^{b} x [f(x) - h(x)]\,\mathrm{d}x,
          $$
          其中:
          \begin{itemize}
              \item $x$ —— 水深;
              \item $f(x)-h(x)$ —— 平板在深度 $x$ 处的宽度;
              \item $\rho g x$ —— 该处压强.
          \end{itemize}
          压力微元为
          $$
              \mathrm{d}P = \rho g x [f(x)-h(x)]\,\mathrm{d}x.
          $$
          \textit{几何意义:水压力等于压强与受力面积的积分.}

          \vspace{0.5em}

    \item \textbf{引力作用}

          一根长度为 $l$、线密度为常数 $\mu$ 的细棒位于 $x\in[-l,0]$,
          在其右端距离为 $a$ 处有质点 $M(m)$,引力常量为 $G$,则 $M$ 与细棒间的引力为
          $$
              F = \int_{-l}^{0} \frac{Gm\mu}{(a-x)^2}\,\mathrm{d}x
              = Gm\mu\!\left(\frac{1}{a-l} - \frac{1}{a}\right).
          $$
          \textit{此类问题的关键是:用微元表示“分布体–质点”之间的作用,然后积分求总量.}

          \vspace{0.5em}

    \item \textbf{综合:用微元法建立物理量积分表达式}

          \textit{步骤总结:}
          \begin{enumerate}[label=(\arabic*)]
              \item 明确研究对象(力、速度、压力、质量等);
              \item 选取微元并建立物理关系式;
              \item 写出微元量($\mathrm{d}W$、$\mathrm{d}P$、$\mathrm{d}F$、$\mathrm{d}m$ 等);
              \item 确定积分区间并积分;
              \item 结合题意写出物理解释.
          \end{enumerate}
          \textit{核心思想:} 用积分表示“连续分布量”的总和.
\end{enumerate}