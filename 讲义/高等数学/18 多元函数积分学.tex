\GAOchapter{多元函数积分学}
\section{计算三重积分}

\subsection{和式极限定义}

设区域
\[
      \Omega = \{(x, y, z) \mid a \le x \le b,\, c \le y \le d,\, e \le z \le f\},
\]
则三重积分定义为
\[
      \iiint_{\Omega} g(x, y, z) \, \mathrm{d}v
      = \lim_{n \to \infty}
      \sum_{i=1}^{n} \sum_{j=1}^{n} \sum_{k=1}^{n}
      g\!\left(a + \tfrac{b-a}{n}i,\, c + \tfrac{d-c}{n}j,\, e + \tfrac{f-e}{n}k\right)
      \frac{(b-a)(d-c)(f-e)}{n^3}.
\]

\subsection{积分次序的交换}

将所给积分次序还原为
\[
      \iiint_\Omega f(x,y,z)\, \mathrm{d}v,
\]
再根据区域特征或函数形式,选择新的积分次序以便计算(如先 $z$ 后 $x,y$ 或先 $x$ 后 $y,z$)。

\subsection{积分的保号性}

\begin{enumerate}
      \item 若 $f(x,y,z)$ 在 $\Omega$ 上连续且 $f \ge 0$ 且不恒为零,则
            \[
                  \iiint_{\Omega} f(x,y,z)\, \mathrm{d}v > 0.
            \]
      \item 若连续函数 $f(x,y,z)$ 满足对任意有界闭区域 $\Omega$,
            \[
                  \iiint_{\Omega} f(x,y,z)\, \mathrm{d}v = 0,
            \]
            则必有 $f(x,y,z) \equiv 0$ 于该区域。
\end{enumerate}

\subsection{对称性的应用}

与二重积分完全类似。

\begin{enumerate}
      \item \textbf{普通对称性.}
            \begin{enumerate}
                  \item 若 $\Omega$ 关于 $xOz$ 面对称,则
                        \[
                              \iiint_{\Omega} f(x, y, z)\, \mathrm{d}v =
                              \begin{cases}
                                    2\iiint_{\Omega_1} f(x,y,z)\, \mathrm{d}v, & f(x,y,z) = f(x,-y,z),  \\[4pt]
                                    0,                                         & f(x,y,z) = -f(x,-y,z),
                              \end{cases}
                        \]
                        其中 $\Omega_1$ 为 $\Omega$ 在 $xOz$ 面右侧部分。
                  \item 若 $\Omega$ 关于三个坐标面对称,$\Omega_1$ 为第一卦限部分,则
                        \[
                              \iiint_{\Omega} f(x,y,z)\, \mathrm{d}v =
                              \begin{cases}
                                    8\iiint_{\Omega_1} f(x,y,z)\, \mathrm{d}v, & f(x,y,z)=f(-x,-y,-z),  \\[4pt]
                                    0,                                         & f(x,y,z)=-f(-x,-y,-z).
                              \end{cases}
                        \]
            \end{enumerate}

      \item \textbf{轮换对称性.}
            若 $\Omega$ 在交换 $x,y$ 后不变,则
            \[
                  \iiint_{\Omega} f(x,y,z)\, \mathrm{d}v
                  = \iiint_{\Omega} f(y,x,z)\, \mathrm{d}v.
            \]
            特别地,若 $\Omega=\{x^2+y^2+z^2\le R^2\}$,则
            \[
                  I = \iiint_{\Omega} f(x)\, \mathrm{d}v
                  = \iiint_{\Omega} f(y)\, \mathrm{d}v
                  = \iiint_{\Omega} f(z)\, \mathrm{d}v
                  = \tfrac{1}{3} \iiint_{\Omega} [f(x)+f(y)+f(z)]\, \mathrm{d}v.
            \]
\end{enumerate}

\subsection{直角坐标系下的积分方法}

\begin{enumerate}
      \item \textbf{先一后二法(投影穿线法)}
            \begin{enumerate}
                  \item 适用:区域 $\Omega$ 由下曲面 $z=z_1(x,y)$ 与上曲面 $z=z_2(x,y)$ 所围。
                  \item 计算公式:
                        \[
                              \iiint_{\Omega} f(x,y,z)\, \mathrm{d}v
                              = \iint_{D_{xy}} \mathrm{d}\sigma
                              \int_{z_1(x,y)}^{z_2(x,y)} f(x,y,z)\, \mathrm{d}z.
                        \]
            \end{enumerate}

      \item \textbf{先二后一法(定限截面法)}
            \begin{enumerate}
                  \item 适用:$\Omega$ 是旋转体或按 $z$ 分层。
                  \item 计算公式:
                        \[
                              \iiint_{\Omega} f(x,y,z)\, \mathrm{d}v
                              = \int_{a}^{b} \mathrm{d}z
                              \iint_{D_z} f(x,y,z)\, \mathrm{d}\sigma.
                        \]
            \end{enumerate}
\end{enumerate}

\subsection{柱面坐标系积分法}

若积分区域适于极坐标表示,令
\[
      \begin{cases}
            x = r\cos\theta, \\
            y = r\sin\theta,
      \end{cases}
\]
则
\[
      \iiint_{\Omega} f(x,y,z)\, \mathrm{d}x\mathrm{d}y\mathrm{d}z
      = \iiint_{\Omega} f(r\cos\theta,r\sin\theta,z)\, r\, \mathrm{d}r\,\mathrm{d}\theta\,\mathrm{d}z.
\]

\subsection{球面坐标系积分法}

\begin{enumerate}
      \item \textbf{适用场合:}
            \begin{enumerate}
                  \item 被积函数含有 $f(x^2 + y^2 + z^2)$;
                  \item 积分区域为球体或锥体的部分。
            \end{enumerate}
      \item \textbf{坐标变换:}
            \[
                  \begin{cases}
                        x = r\sin\varphi\cos\theta, \\
                        y = r\sin\varphi\sin\theta, \\
                        z = r\cos\varphi,
                  \end{cases}
                  \quad
                  \mathrm{d}v = r^2 \sin\varphi\, \mathrm{d}r\, \mathrm{d}\varphi\, \mathrm{d}\theta.
            \]
      \item \textbf{积分形式:}
            \[
                  \iiint_{\Omega} f(x,y,z)\, \mathrm{d}v
                  = \int_{\theta_1}^{\theta_2} \!\mathrm{d}\theta
                  \int_{\varphi_1(\theta)}^{\varphi_2(\theta)} \!\mathrm{d}\varphi
                  \int_{r_1(\varphi,\theta)}^{r_2(\varphi,\theta)}\!
                  f(r\sin\varphi\cos\theta, r\sin\varphi\sin\theta, r\cos\varphi)
                  r^2 \sin\varphi\, \mathrm{d}r.
            \]
\end{enumerate}
\subsection{重积分的应用}

\begin{enumerate}
      \item \textbf{体积:}
            \[
                  V = \iiint_{\Omega} 1\, \mathrm{d}v.
            \]

      \item \textbf{重心(质心):}
            若体密度为 $\rho(x,y,z)$,则
            \[
                  \overline{x} = \frac{\iiint_{\Omega} x\rho\, \mathrm{d}v}{\iiint_{\Omega} \rho\, \mathrm{d}v},\quad
                  \overline{y} = \frac{\iiint_{\Omega} y\rho\, \mathrm{d}v}{\iiint_{\Omega} \rho\, \mathrm{d}v},\quad
                  \overline{z} = \frac{\iiint_{\Omega} z\rho\, \mathrm{d}v}{\iiint_{\Omega} \rho\, \mathrm{d}v}.
            \]

      \item \textbf{引力:}
            对物体外一点 $M_0(x_0,y_0,z_0)$,质量为 $m$,体密度 $\rho(x,y,z)$,
            \[
                  \begin{aligned}
                        F_x & = Gm \iiint_{\Omega} \frac{\rho(x,y,z)(x-x_0)}{[(x-x_0)^2+(y-y_0)^2+(z-z_0)^2]^{3/2}}\, \mathrm{d}v, \\
                        F_y & = Gm \iiint_{\Omega} \frac{\rho(x,y,z)(y-y_0)}{[(x-x_0)^2+(y-y_0)^2+(z-z_0)^2]^{3/2}}\, \mathrm{d}v, \\
                        F_z & = Gm \iiint_{\Omega} \frac{\rho(x,y,z)(z-z_0)}{[(x-x_0)^2+(y-y_0)^2+(z-z_0)^2]^{3/2}}\, \mathrm{d}v.
                  \end{aligned}
            \]

      \item \textbf{转动惯量:}
            \[
                  \begin{aligned}
                        I_x & = \iiint_{\Omega} \rho(y^2+z^2)\, \mathrm{d}v,     \\
                        I_y & = \iiint_{\Omega} \rho(x^2+z^2)\, \mathrm{d}v,     \\
                        I_z & = \iiint_{\Omega} \rho(x^2+y^2)\, \mathrm{d}v,     \\
                        I_O & = \iiint_{\Omega} \rho(x^2+y^2+z^2)\, \mathrm{d}v.
                  \end{aligned}
            \]
\end{enumerate}
\section{计算第一型曲面积分}

\subsection{定义}

设 $f(x,y,z)$ 定义在光滑曲面 $\Sigma$ 上,若将 $\Sigma$ 分割为若干小面元 $\Delta S_i$,取 $\Sigma_i$ 上一点 $M_i(x_i,y_i,z_i)$,当面元最大直径趋于零时,若极限
\[
      \lim_{\lambda \to 0} \sum_i f(x_i,y_i,z_i)\Delta S_i
\]
存在,则称其为函数 $f(x,y,z)$ 在曲面 $\Sigma$ 上的\textbf{第一型曲面积分},记作
\[
      \iint_{\Sigma} f(x,y,z)\, \mathrm{d}S.
\]

其物理意义:当 $f(x,y,z)\ge0$ 表示面密度时,该积分即为曲面薄片的质量。


\subsection{代入曲面方程}

若曲面方程为 $\Sigma: z = z(x,y)$,则需将其代入被积函数:
\[
      f(x,y,z) \;\Rightarrow\; f(x,y,z(x,y)),
\]
并结合面积微元
\[
      \mathrm{d}S = \sqrt{1 + z_x'^2 + z_y'^2}\, \mathrm{d}x\, \mathrm{d}y,
\]
从而化为二重积分形式:
\[
      \iint_{\Sigma} f(x,y,z)\, \mathrm{d}S
      = \iint_{D_{xy}} f(x,y,z(x,y)) \sqrt{1 + z_x'^2 + z_y'^2}\, \mathrm{d}x\, \mathrm{d}y.
\]


\subsection{几何意义}

当 $f(x,y,z)\equiv1$ 时,
\[
      \iint_{\Sigma} \mathrm{d}S
\]
即为曲面 $\Sigma$ 的面积。


\subsection{形心公式的应用}

若曲面 $\Sigma$ 的形心(或质心)坐标为 $(\overline{x},\overline{y},\overline{z})$,则
\[
      \overline{x} = \frac{\iint_{\Sigma} x\, \mathrm{d}S}{\iint_{\Sigma} \mathrm{d}S}, \quad
      \overline{y} = \frac{\iint_{\Sigma} y\, \mathrm{d}S}{\iint_{\Sigma} \mathrm{d}S}, \quad
      \overline{z} = \frac{\iint_{\Sigma} z\, \mathrm{d}S}{\iint_{\Sigma} \mathrm{d}S}.
\]
若 $\Sigma$ 为规则对称图形(形心坐标已知,面积易求),可得
\[
      \iint_{\Sigma} x\, \mathrm{d}S = \overline{x} \, S_{\Sigma}, \quad
      \iint_{\Sigma} y\, \mathrm{d}S = \overline{y} \, S_{\Sigma}, \quad
      \iint_{\Sigma} z\, \mathrm{d}S = \overline{z} \, S_{\Sigma}.
\]


\subsection{对称性}

\begin{enumerate}
      \item \textbf{普通对称性.}
            若 $\Sigma$ 关于 $xOz$ 面对称,则
            \[
                  \iint_{\Sigma} f(x,y,z)\, \mathrm{d}S =
                  \begin{cases}
                        2 \iint_{\Sigma_1} f(x,y,z)\, \mathrm{d}S, & f(x,y,z) = f(x,-y,z),  \\[4pt]
                        0,                                         & f(x,y,z) = -f(x,-y,z),
                  \end{cases}
            \]
            其中 $\Sigma_1$ 为 $\Sigma$ 在 $xOz$ 面右侧部分。
            其他关于 $yOz$ 或 $xOy$ 的情形类似。

      \item \textbf{轮换对称性.}
            若曲面 $\Sigma: z=z(x,y)$ 在交换 $x,y$ 后不变,则
            \[
                  \iint_{\Sigma} f(x,y,z)\, \mathrm{d}S
                  = \iint_{\Sigma} f(y,x,z)\, \mathrm{d}S.
            \]
            若 $\Sigma$ 关于 $x,y,z$ 完全对称,则
            \[
                  \iint_{\Sigma} f(x,y,z)\, \mathrm{d}S
                  = \frac{1}{3} \iint_{\Sigma} [f(x,y,z) + f(y,z,x) + f(z,x,y)]\, \mathrm{d}S.
            \]
\end{enumerate}


\subsection{物理应用}

\begin{enumerate}
      \item \textbf{重心(质心)或形心.}
            对光滑曲面薄片 $\Sigma$,面密度为 $\rho(x,y,z)$,则
            \[
                  \overline{x} = \frac{\iint_{\Sigma} x\rho(x,y,z)\, \mathrm{d}S}{\iint_{\Sigma} \rho(x,y,z)\, \mathrm{d}S}, \quad
                  \overline{y} = \frac{\iint_{\Sigma} y\rho(x,y,z)\, \mathrm{d}S}{\iint_{\Sigma} \rho(x,y,z)\, \mathrm{d}S}, \quad
                  \overline{z} = \frac{\iint_{\Sigma} z\rho(x,y,z)\, \mathrm{d}S}{\iint_{\Sigma} \rho(x,y,z)\, \mathrm{d}S}.
            \]

      \item \textbf{转动惯量.}
            面密度为 $\rho(x,y,z)$ 的曲面薄片 $\Sigma$,其关于 $x$、$y$、$z$ 轴及原点的转动惯量分别为:
            \[
                  \begin{aligned}
                        I_x & = \iint_{\Sigma} (y^2 + z^2)\rho(x,y,z)\, \mathrm{d}S,       \\
                        I_y & = \iint_{\Sigma} (z^2 + x^2)\rho(x,y,z)\, \mathrm{d}S,       \\
                        I_z & = \iint_{\Sigma} (x^2 + y^2)\rho(x,y,z)\, \mathrm{d}S,       \\
                        I_O & = \iint_{\Sigma} (x^2 + y^2 + z^2)\rho(x,y,z)\, \mathrm{d}S.
                  \end{aligned}
            \]
\end{enumerate}

\section{计算第一型曲面积分}

\subsection{定义与物理意义}

设函数 $f(x,y,z)$ 定义在空间曲面 $\Sigma$ 上。
第一型曲面积分的物理意义是:若 $f(x,y,z) \ge 0$ 表示曲面上各点的面密度,则
\[
      \iint_{\Sigma} f(x,y,z)\,\mathrm{d}S
\]
表示该物质曲面的总质量。
其定义方式与二重积分、三重积分类似,都是“分割—近似—求和—取极限”的结果。


\subsection{代入曲面方程}

若曲面 $\Sigma$ 可表示为 $z = z(x,y)$,则需将其代入被积函数中化简:
\[
      f(x,y,z) \;\Rightarrow\; f\big(x,\,y,\,z(x,y)\big).
\]


\subsection{几何意义}

若 $f(x,y,z) \equiv 1$,则
\[
      \iint_{\Sigma} \mathrm{d}S = S_{\Sigma},
\]
即为曲面 $\Sigma$ 的面积。

\subsection{利用形心公式}

由形心定义:
\[
      \overline{x} = \frac{\iint_{\Sigma} x\,\mathrm{d}S}{\iint_{\Sigma}\mathrm{d}S},
\]
可得
\[
      \iint_{\Sigma} x\,\mathrm{d}S = \overline{x} \cdot S_{\Sigma}.
\]
当 $\Sigma$ 为规则图形(形心坐标 $\overline{x}$ 已知且面积易求)时,此公式尤为方便。


\subsection{利用对称性质}

\begin{enumerate}
      \item \textbf{普通对称性:}
            若 $\Sigma$ 关于 $xOz$ 面对称,则
            \[
                  \iint_{\Sigma} f(x,y,z)\,\mathrm{d}S =
                  \begin{cases}
                        2\displaystyle\iint_{\Sigma_1} f(x,y,z)\,\mathrm{d}S, & f(x,y,z)=f(x,-y,z),  \\[6pt]
                        0,                                                    & f(x,y,z)=-f(x,-y,z),
                  \end{cases}
            \]
            其中 $\Sigma_1$ 为 $\Sigma$ 在 $xOz$ 面右侧部分。
            其余坐标面对称情况类似。

      \item \textbf{轮换对称性:}
            若曲面 $\Sigma: z=z(x,y)$ 在交换 $x,y$ 后保持不变,则
            \[
                  \iint_{\Sigma} f(x,y,z)\,\mathrm{d}S = \iint_{\Sigma} f(y,x,z)\,\mathrm{d}S.
            \]
            若 $\Sigma$ 关于 $x,y,z$ 三坐标完全对称,则
            \[
                  \iint_{\Sigma} f(x,y,z)\,\mathrm{d}S
                  = \frac{1}{3}\iint_{\Sigma}\!\big[f(x,y,z)+f(y,z,x)+f(z,x,y)\big]\,\mathrm{d}S.
            \]
\end{enumerate}


\subsection{物理应用}

\begin{enumerate}
      \item \textbf{形心(重心)计算:}
            对面密度为 $\rho(x,y,z)$ 的光滑曲面 $\Sigma$,其形心 $(\overline{x},\overline{y},\overline{z})$ 为
            \[
                  \overline{x} = \frac{\iint_{\Sigma} x\rho(x,y,z)\,\mathrm{d}S}{\iint_{\Sigma}\rho(x,y,z)\,\mathrm{d}S},\quad
                  \overline{y} = \frac{\iint_{\Sigma} y\rho(x,y,z)\,\mathrm{d}S}{\iint_{\Sigma}\rho(x,y,z)\,\mathrm{d}S},\quad
                  \overline{z} = \frac{\iint_{\Sigma} z\rho(x,y,z)\,\mathrm{d}S}{\iint_{\Sigma}\rho(x,y,z)\,\mathrm{d}S}.
            \]

      \item \textbf{转动惯量:}
            对同一曲面,其关于各坐标轴与原点的转动惯量分别为:
            \[
                  I_x = \iint_{\Sigma}(y^2+z^2)\rho(x,y,z)\,\mathrm{d}S,\quad
                  I_y = \iint_{\Sigma}(z^2+x^2)\rho(x,y,z)\,\mathrm{d}S,
            \]
            \[
                  I_z = \iint_{\Sigma}(x^2+y^2)\rho(x,y,z)\,\mathrm{d}S,\quad
                  I_O = \iint_{\Sigma}(x^2+y^2+z^2)\rho(x,y,z)\,\mathrm{d}S.
            \]
\end{enumerate}


\subsection*{小结}

\begin{itemize}
      \item 第一型曲面积分计算的是曲面“面积型”量(如面积、质量、形心、惯量等);
      \item 对称性与形心法常用于快速求积分;
      \item 若 $f(x,y,z)=1$,积分即为曲面面积。
\end{itemize}

\section{计算第二型线面积分}
\subsection{第二型曲线积分}

\begin{enumerate}
      \item \textbf{定义与物理意义(做功)}

            设向量场
            \[
                  \mathbf{F}(x,y)=P(x,y)\mathbf{i}+Q(x,y)\mathbf{j}
                  \quad\text{或}\quad
                  \mathbf{F}(x,y,z)=P(x,y,z)\mathbf{i}+Q(x,y,z)\mathbf{j}+R(x,y,z)\mathbf{k},
            \]
            定义在有向曲线 $L$(或空间曲线 $\Gamma$)上,则第二型曲线积分表示变力 $\mathbf{F}$ 沿该曲线从起点到终点所做的功:
            \[
                  \int_{L} P\,dx + Q\,dy
                  \quad\text{或}\quad
                  \int_{\Gamma} P\,dx + Q\,dy + R\,dz.
            \]

            与定积分、二重积分、三重积分、第一型曲线与曲面积分不同,
            第二型曲线积分是\textbf{向量场沿有向曲线的积分}(非几何量)。

            \[
                  \begin{cases}
                        \text{平面:} & \displaystyle\int_L (P,Q)\cdot(dx,dy) = \int_L P\,dx + Q\,dy,                        \\[6pt]
                        \text{空间:} & \displaystyle\int_\Gamma (P,Q,R)\cdot(dx,dy,dz) = \int_\Gamma P\,dx + Q\,dy + R\,dz.
                  \end{cases}
            \]


      \item \textbf{计算方法}

            \begin{enumerate}
                  \item \textbf{对称性法(类对称)}

                        若 $L^*$ 可分为关于某直线类对称的两部分 $L_1,L_2$,且对称点处 $P$ 绝对值相等,则
                        \[
                              \int_{L^*} P\,dx =
                              \begin{cases}
                                    2\displaystyle\int_{L_1} P\,dx, & P(x,y)=P(-x,y),  \\[6pt]
                                    0,                              & P(x,y)=-P(-x,y),
                              \end{cases}
                              \qquad
                              \int_{L^*} Q\,dy =
                              \begin{cases}
                                    0,                              & Q(x,y)=Q(-x,y),  \\[6pt]
                                    2\displaystyle\int_{L_1} Q\,dy, & Q(x,y)=-Q(-x,y).
                              \end{cases}
                        \]

                  \item \textbf{参数化法——一投二代三计算(化为定积分)}

                        若 $L$ 的参数方程为
                        \[
                              \begin{cases}
                                    x = x(t), \\
                                    y = y(t),
                              \end{cases}\quad t\in[\alpha,\beta],
                        \]
                        则
                        \[
                              \int_{L} P\,dx + Q\,dy
                              = \int_{\alpha}^{\beta}\!\big[P(x(t),y(t))x'(t) + Q(x(t),y(t))y'(t)\big]\,dt.
                        \]
                        起点终点由参数 $\alpha,\beta$ 对应,顺序须与曲线方向一致。

                  \item \textbf{格林公式(将曲线积分化为二重积分)}

                        设平面区域 $D$ 由分段光滑的正向闭曲线 $L$ 围成,且 $P,Q$ 在 $D$ 上具有连续一阶偏导数,则
                        \[
                              \oint_{L} P\,dx + Q\,dy = \iint_{D}\!\left(\frac{\partial Q}{\partial x}-\frac{\partial P}{\partial y}\right)\!d\sigma.
                        \]
                        \begin{enumerate}
                              \item 若 $L$ 为闭曲线且内部无奇点,可直接使用格林公式。
                              \item 若有奇点但除奇点外 $\dfrac{\partial Q}{\partial x}=\dfrac{\partial P}{\partial y}$,可换路径封闭。
                              \item 若非封闭曲线且 $\dfrac{\partial Q}{\partial x}=\dfrac{\partial P}{\partial y}$,可在区域内换一条起终点相同的简路径计算。
                              \item 若非封闭曲线且两偏导不等,可补线成闭合曲线 $L=L_{AB}+C_{BA}$,应用格林公式后再减去补线部分。
                        \end{enumerate}

                  \item \textbf{积分与路径无关(保守场条件)}

                        若 $P,Q$ 在单连通区域 $D$ 内具有一阶连续偏导,则下列命题等价:
                        \begin{enumerate}[label=(\alph*)]
                              \item $\int_{L_{AB}} P\,dx+Q\,dy$ 与路径无关;
                              \item 对任意闭曲线 $\oint_L P\,dx+Q\,dy=0$;
                              \item 存在函数 $u(x,y)$ 使 $du = P\,dx + Q\,dy$;
                              \item $\mathbf{F}=(P,Q)$ 为某函数的梯度场;
                              \item $\dfrac{\partial P}{\partial y}=\dfrac{\partial Q}{\partial x}$。
                        \end{enumerate}

                        \textbf{求原函数法:}
                        \[
                              u(x,y) = \int_{(x_0,y_0)}^{(x,y)} P\,dx + Q\,dy,
                        \]
                        或沿折线路径计算:
                        \[
                              u(x,y) = \int_{x_0}^{x} P(x,y_0)\,dx + \int_{y_0}^{y} Q(x,y)\,dy.
                        \]
                        若 $\frac{\partial Q}{\partial x}=\frac{\partial P}{\partial y}$ 不成立,则 $u(x,y)$ 不存在。

                        \textbf{凑微分法:}
                        若能写出 $P\,dx+Q\,dy=d[u(x,y)]$,则
                        \[
                              \int_{L_{AB}} P\,dx + Q\,dy = u(B)-u(A).
                        \]

                  \item \textbf{两类曲线积分关系式}
                        \[
                              \int_{\Gamma} P\,dx + Q\,dy + R\,dz
                              = \int_{\Gamma} (P\cos\alpha + Q\cos\beta + R\cos\gamma)\,ds,
                        \]
                        其中 $(\cos\alpha,\cos\beta,\cos\gamma)$ 为 $\Gamma$ 上点处的单位切向量。

                  \item \textbf{空间曲线的两种计算法}
                        \begin{enumerate}
                              \item \textbf{参数法(一投二代三计算):}
                                    \[
                                          \Gamma:\begin{cases}
                                                x=x(t), \\
                                                y=y(t), \\
                                                z=z(t),\quad t\in[a,b],
                                          \end{cases}
                                    \]
                                    \[
                                          \int_{\Gamma} P\,dx+Q\,dy+R\,dz
                                          = \int_a^b [P x'(t)+Q y'(t)+R z'(t)]\,dt.
                                    \]

                              \item \textbf{斯托克斯公式:}
                                    若 $\Gamma=\partial\Sigma$ 为曲面 $\Sigma$ 的正向边界,则
                                    \[
                                          \oint_{\Gamma} P\,dx+Q\,dy+R\,dz
                                          = \iint_{\Sigma}
                                          \begin{vmatrix}
                                                dy\,dz                       & dz\,dx                       & dx\,dy                       \\
                                                \dfrac{\partial}{\partial x} & \dfrac{\partial}{\partial y} & \dfrac{\partial}{\partial z} \\
                                                P                            & Q                            & R
                                          \end{vmatrix}
                                          = \iint_{\Sigma}
                                          \begin{vmatrix}
                                                \cos\alpha                   & \cos\beta                    & \cos\gamma                   \\
                                                \dfrac{\partial}{\partial x} & \dfrac{\partial}{\partial y} & \dfrac{\partial}{\partial z} \\
                                                P                            & Q                            & R
                                          \end{vmatrix} dS.
                                    \]
                                    其中 $(\cos\alpha,\cos\beta,\cos\gamma)$ 为 $\Sigma$ 的单位外法线方向余弦。

                              \item 若 $\operatorname{rot}\mathbf{F}=0$(无旋场),则积分与路径无关,可换路径计算。
                        \end{enumerate}
            \end{enumerate}
\end{enumerate}

\subsection*{小结}

\begin{itemize}
      \item 第二型曲线积分体现\textbf{向量场沿曲线的累积作用}(常为功、环流等);
      \item 计算常用方法:参数化、一投二代三计算、对称性、格林公式;
      \item 若 $\nabla\times\mathbf{F}=0$,则积分与路径无关;
      \item 与第一型曲线积分关系:$\displaystyle\int_\Gamma (P,Q,R)\cdot(dx,dy,dz) = \int_\Gamma (P\cos\alpha+Q\cos\beta+R\cos\gamma)\,ds.$
\end{itemize}

\subsection{第二型曲面积分}

\subsubsection{定义与物理意义(通量)}

设向量场
\[
      \mathbf{F}(x, y, z) = P(x, y, z)\mathbf{i} + Q(x, y, z)\mathbf{j} + R(x, y, z)\mathbf{k},
\]
定义在光滑有向曲面 $\Sigma$ 上。
第二型曲面积分表示向量场 $\mathbf{F}$ 通过曲面 $\Sigma$ 的\textbf{通量}:
\[
      \iint_{\Sigma} P\,\mathrm{d}y\,\mathrm{d}z + Q\,\mathrm{d}z\,\mathrm{d}x + R\,\mathrm{d}x\,\mathrm{d}y
      = \iint_{\Sigma} \mathbf{F} \cdot (\mathrm{d}y\,\mathrm{d}z,\, \mathrm{d}z\,\mathrm{d}x,\, \mathrm{d}x\,\mathrm{d}y).
\]
它反映了向量场穿过曲面的量,与第一型曲面积分(面积量)不同,应注意区分。


\subsubsection{计算方法:一投二代三计算}

若曲面 $\Sigma$ 可写为 $z = z(x, y)$,则:
\[
      \iint_{\Sigma} R(x, y, z)\,\mathrm{d}x\,\mathrm{d}y
      = \pm \iint_{D_{xy}} R[x, y, z(x, y)]\,\mathrm{d}x\,\mathrm{d}y,
\]
其中 $D_{xy}$ 为 $\Sigma$ 在 $xOy$ 平面的投影区域。符号“$\pm$”根据法向量方向确定:
- 当上侧为正($\cos\gamma > 0$)取“+”;
- 当下侧为正($\cos\gamma < 0$)取“-”。

若曲面垂直于投影面,则该积分为零。
若曲面在投影中有重叠部分,应先剖分为互不重叠的曲面片再计算。

\subsubsection{转换投影法}

\paragraph{1. 法向量表达式}
若曲面 $\Sigma: z = z(x, y)$,则其单位法向量为
\[
      \mathbf{n} = \pm \frac{1}{\sqrt{1 + z_x^2 + z_y^2}} \, (-z_x, -z_y, 1),
\]
上侧为正取“+”,下侧为正取“-”。

\paragraph{2. 转换投影公式}
设 $P, Q, R$ 在 $\Sigma$ 上连续,且 $z = z(x, y)$ 有连续一阶偏导,则
\[
      \iint_{\Sigma} P\,\mathrm{d}y\,\mathrm{d}z + Q\,\mathrm{d}z\,\mathrm{d}x + R\,\mathrm{d}x\,\mathrm{d}y
      = \pm \iint_{D_{xy}} \big(-P\,z_x - Q\,z_y + R\big)\,\mathrm{d}x\,\mathrm{d}y.
\]


\subsubsection{“类”对称性}

若 $\Sigma^*$ 关于某平面对称,且对称点处 $R(x, y, z)$ 绝对值相等,则
\[
      \iint_{\Sigma^*} R(x, y, z)\,\mathrm{d}x\,\mathrm{d}y
      = \begin{cases}
            2 \displaystyle\iint_{\Sigma_1} R(x, y, z)\,\mathrm{d}x\,\mathrm{d}y, & R(x, y, z)\mathrm{d}x\,\mathrm{d}y \text{ 同号}, \\[6pt]
            0,                                                                    & R(x, y, z)\mathrm{d}x\,\mathrm{d}y \text{ 异号},
      \end{cases}
\]
其中 $\Sigma_1$ 为 $\Sigma^*$ 的一侧部分。
对 $P\,\mathrm{d}y\,\mathrm{d}z$、$Q\,\mathrm{d}z\,\mathrm{d}x$ 可类比处理。


\subsubsection{高斯公式(散度定理)}

设空间闭区域 $\Omega$ 由光滑闭曲面 $\Sigma$ 围成,且 $\Sigma$ 外侧为正。若 $P, Q, R$ 在 $\Omega$ 上具有连续一阶偏导数,则有
\[
      \iint_{\Sigma} P\,\mathrm{d}y\,\mathrm{d}z + Q\,\mathrm{d}z\,\mathrm{d}x + R\,\mathrm{d}x\,\mathrm{d}y
      = \iiint_{\Omega} \left(
      \frac{\partial P}{\partial x} + \frac{\partial Q}{\partial y} + \frac{\partial R}{\partial z}
      \right)\mathrm{d}v.
\]

\paragraph{常用技巧:}
\begin{enumerate}
      \item \textbf{封闭曲面、无奇点:} 直接用高斯公式;
      \item \textbf{封闭曲面、有奇点,且 $\mathrm{div}\,\mathbf{F}=0$:} 可换为包含奇点的其他封闭曲面;
      \item \textbf{非封闭曲面,且 $\mathrm{div}\,\mathbf{F}=0$:} 可换路径但边界需相同;
      \item \textbf{非封闭曲面,$\mathrm{div}\,\mathbf{F}\neq0$:} 可补面封闭(加面减面法);
      \item \textbf{若 $\mathrm{div}\,\mathbf{F}=0$ 对任意闭曲面成立:} 可建立方程求未知函数 $f(x)$。
\end{enumerate}

\subsubsection{两类曲面积分的关系}

设曲面 $\Sigma$ 有单位法向量 $\mathbf{n} = (\cos\alpha, \cos\beta, \cos\gamma)$,则
\[
      \iint_{\Sigma} P\,\mathrm{d}y\,\mathrm{d}z + Q\,\mathrm{d}z\,\mathrm{d}x + R\,\mathrm{d}x\,\mathrm{d}y
      = \iint_{\Sigma} (P\cos\alpha + Q\cos\beta + R\cos\gamma)\,\mathrm{d}S.
\]
右式即为第一型曲面积分的形式,表示通量的几何意义。


\subsubsection{总结}

\begin{itemize}
      \item \textbf{第一型曲面积分:} 几何量(面积、质量);
      \item \textbf{第二型曲面积分:} 向量场通量;
      \item \textbf{高斯公式:} 封闭曲面 $\leftrightarrow$ 体积分;
      \item \textbf{斯托克斯公式:} 曲线 $\leftrightarrow$ 曲面;
      \item 若 $\mathrm{div}\,\mathbf{F}=0$(无源场),则通量与选取曲面无关。
\end{itemize}
