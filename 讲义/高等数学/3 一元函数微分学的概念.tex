\GAOchapter{一元函数微分学的概念}

\section{微分——一阶泰勒公式}
\DTwo

当 $f'(x_0)$ 存在时,函数 $f(x)$ 在点 $x_0$ 处的微分为
\[
    d[f(x)]\big|_{x=x_0} = f'(x_0)\Delta x = f'(x_0)dx,
\]
即
\[
    f(x) - f(x_0) = f'(x_0)\Delta x + o(1)\Delta x = f'(x_0)\Delta x + o(\Delta x) \quad (\Delta x \to 0).
\]
此即为 \textbf{微分公式},亦称 \textbf{一阶泰勒公式}.

\section{导数——因变量差与自变量差的比值极限}
\DTwoTwo

函数 $f(x)$ 在点 $x_0$ 处的导数为
\[
    f'(x_0) = \lim_{\Delta x \to 0} \frac{\Delta f}{\Delta x}
    = \lim_{\Delta x \to 0} \frac{f(x_0 + \Delta x) - f(x_0)}{\Delta x}.
\]
由上式可得:
\[
    f(x_0 + \Delta x) - f(x_0)
    = f'(x_0)\Delta x + o(\Delta x), \quad \Delta x \to 0.
\]
其中 $f'(x_0)$ 为一次项系数,也称 \textbf{微分系数}.
它反映了因变量增量与自变量增量的一次项比值的极限.若该极限存在,则称 $f(x)$ 在 $x_0$ 处可导.

\section{$f(x)$ 与 $|f(x)|$ 的连续与可导关系}
\DTwoOne+\DFourThree

\begin{enumerate}
    \item 若 $f(x)$ 在 $x_0$ 处连续,则 $|f(x)|$ 在 $x_0$ 处亦连续;反之不成立.
    \item 若 $f(x)$ 在 $x_0$ 处可导,则:
          \begin{enumerate}
              \item 若 $f(x_0) \neq 0$,则 $|f(x)|$ 在 $x_0$ 处可导,且
                    \[
                        [|f(x)|]'_{x=x_0} =
                        \begin{cases}
                            f'(x_0),  & f(x_0) > 0, \\
                            -f'(x_0), & f(x_0) < 0.
                        \end{cases}
                    \]
              \item 若 $f(x_0) = 0$,则:
                    \[
                        \begin{cases}
                            f'(x_0) = 0 \Rightarrow |f(x)| \text{ 在 } x_0 \text{ 处可导,且 } [|f(x)|]'_{x=x_0}=0, \\[5pt]
                            f'(x_0) \ne 0 \Rightarrow |f(x)| \text{ 在 } x_0 \text{ 处不可导.}
                        \end{cases}
                    \]
          \end{enumerate}
\end{enumerate}

\section{导函数 $f'(x)$ 的性质总结}
\DTwoOne

\begin{enumerate}
    \item 关于 $f'(x)$ 的连续性:
          \begin{enumerate}
              \item 若导函数 $f'(x)$ 存在,则当 $f'(x)$ 在某点极限存在时,该点处导函数连续;
              \item 若导函数在某点存在,则该点 \textbf{不可能是第一类间断点};
              \item 若 $f(x)$ 可导,则 $f'(x)$ 可能连续,也可能含有振荡间断点.
          \end{enumerate}

    \item 关于 $\lim\limits_{x\to+\infty} f'(x)$:
          \begin{enumerate}
              \item 若 $\lim\limits_{x\to+\infty} f(x)$ 存在,$\lim\limits_{x\to+\infty} f'(x)$ 不一定存在.
                    例如:$f(x)=\dfrac{\sin x^2}{x}$,
                    则 $f'(x)=2\cos x^2 - \dfrac{\sin x^2}{x^2}$,其极限不存在.
              \item 若 $f(x)$ 在 $(0,+\infty)$ 可导,且 $y=f(x)$ 在 $x\to+\infty$ 时有斜渐近线,
                    $\lim\limits_{x\to+\infty} f'(x)$ 也不一定存在.
                    例如:$f(x)=x+\dfrac{\sin x^2}{x}$,
                    有斜渐近线 $y=x$,但 $f'(x)=1+2\cos x^2-\dfrac{\sin x^2}{x^2}$,其极限不存在.
          \end{enumerate}
\end{enumerate}

\section{函数在一点求导的注意事项}
\DTwoTwo
\begin{enumerate}
    \item \textbf{关于 $f'(x_0)$ 与 $f'(x)$ 的区别}
          \begin{enumerate}
              \item $f'(x_0)$ 表示函数在点 $x_0$ 处的导数;
              \item $f'(x)$ 表示用求导法则求得的导函数表达式;
              \item 若 $f'(x)$ 在 $x_0$ 处无定义,仅说明求导法则在此点不适用,
                    并不意味着 $f(x)$ 在 $x_0$ 处不可导,此时应当\textbf{使用定义求导}.
          \end{enumerate}

    \item \textbf{绝对值型函数的可导性}
          \begin{enumerate}
              \item 若 $F(x)=f(x)g(x)$,其中 $f(x)$ 在 $x_0$ 处连续但不可导,$g(x)$ 在 $x_0$ 处可导,
                    则
                    \[
                        F(x) \text{ 在 } x_0 \text{ 处可导 } \Longleftrightarrow g(x_0)=0.
                    \]
                    特别地,若 $F(x)=|x-x_0|g(x)$,且 $g(x)$ 在 $x_0$ 处可导,
                    则 $F(x)$ 在 $x_0$ 处可导 $\Longleftrightarrow g(x_0)=0$.
              \item 由 $|f(x)| = \sqrt{f^2(x)}$ 得
                    \[
                        [|f(x)|]' = [\sqrt{f^2(x)}]'
                        = \frac{1}{2\sqrt{f^2(x)}} \cdot 2f(x)f'(x)
                        = \frac{f(x)f'(x)}{|f(x)|}.
                    \]
          \end{enumerate}

    \item \textbf{分段函数在分段点处的可导性}

          \begin{example}{}{}
              下列函数中,在 $x=0$ 处不可导的是( ).

              \vspace{0.5em}
              \begin{tabular}{ll}
                  (A) & $f(x)=|x|\tan|x|$        \\[0.3em]
                  (B) & $f(x)=|x|\tan\sqrt{|x|}$ \\[0.3em]
                  (C) & $f(x)=\sqrt{\cos|x|}$    \\[0.3em]
                  (D) & $f(x)=\cos\sqrt{|x|}$
              \end{tabular}
          \end{example}

          \begin{solution}
              \begin{enumerate}
                  \item[(A)]
                        \[
                            f'(0)=\lim_{x\to0}\frac{f(x)-f(0)}{x}
                            =\lim_{x\to0}\frac{|x|\tan|x|}{x}
                            =\lim_{x\to0}\frac{x^2}{x}=0.
                        \]

                  \item[(B)]
                        \[
                            f'(0)=\lim_{x\to0}\frac{|x|\tan\sqrt{|x|}}{x}
                            =\lim_{x\to0}\left(\frac{|x|}{x}\cdot\tan\sqrt{|x|}\right)=0.
                        \]

                  \item[(C)]
                        \[
                            f'(0)=\lim_{x\to0}\frac{\sqrt{\cos|x|}-1}{x}
                            =\lim_{x\to0}\frac{\cos|x|-1}{x(\sqrt{\cos|x|}+1)}
                            =\lim_{x\to0}\frac{-\frac{1}{2}x^2}{2x}=0.
                        \]

                  \item[(D)]
                        \[
                            f'(0)=\lim_{x\to0}\frac{\cos\sqrt{|x|}-1}{x}
                            =\lim_{x\to0}\frac{-\frac{1}{2}|x|}{x},
                        \]
                        左、右极限符号相反,极限不存在,因此 $f'(0)$ 不存在.
              \end{enumerate}
              \textbf{故选:} (D).
          \end{solution}
\end{enumerate}