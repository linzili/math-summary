\chapter{习题3}
\clearpage

\begin{question}{}{}
    对于任意给定的 $\varepsilon\in(0,1)$,总存在整数 $N$,当 $n>N$ 时,恒有 $|x_n-a|\le 2\varepsilon$ ,是数列 $\{x_n\}$ 收敛于 $a$ 的( )

    \begin{multicols}{2}  % 并排两列
        \begin{itemize}[label={}]
            \item A 充分条件但非必要条件
            \item B 必要条件但非充分条件
            \item C 充分必要条件
            \item D 既非充分条件又非必要条件
        \end{itemize}
    \end{multicols}
\end{question}
\begin{solution}
    该条件表述为:对任意 $\varepsilon\in(0,1)$,存在 $N$,当 $n>N$ 时 $|x_n-a|\le2\varepsilon$。要判别此条件与收敛 $x_n\to a$ 的等价性:

    若序列收敛于 $a$,则对任意 $\varepsilon>0$ 存在 $N$ 使得当 $n>N$ 有 $|x_n-a|<\varepsilon$,从而对任意 $\varepsilon\in(0,1)$ 同样成立(取同样的 $N$)。反过来,若题中条件成立,则对任意给定的正 $\varepsilon$:若 $\varepsilon\ge1$,不难直接满足;若 $0<\varepsilon<1$,条件已给出 $|x_n-a|\le2\varepsilon$,令 $\varepsilon'=\varepsilon/2\in(0,1)$,由题意存在 $N$ 使得当 $n>N$ 有 $|x_n-a|\le2\varepsilon'=\varepsilon$,故对任意 $\varepsilon>0$ 都能得到相应 $N$,即 $x_n\to a$。因此该条件与收敛等价。

    答案:C(充分必要条件)。
\end{solution}

\begin{question}{}{}
    设 $f^\prime(1)=0,\lim_{x\to1}\frac{f^{\prime}(x)}{\left(x-1\right)^3}=-2$, 则$x=1$为()
    \begin{multicols}{4}  % 并排两列
        \begin{itemize}[label={}]
            \item A 非极值点
            \item B 极大值点
            \item C 极小值点
            \item D 间断点
        \end{itemize}
    \end{multicols}
\end{question}
\begin{solution}
    由极限条件有,当 $x\to1$ 时
    \[
        f'(x)\sim -2(x-1)^3.
    \]
    因此对于 $x>1$(且靠近 1)有 $(x-1)^3>0$,所以 $f'(x)\approx -2(x-1)^3<0$;对于 $x<1$ 有 $(x-1)^3<0$,所以 $f'(x)\approx -2(x-1)^3>0$。即导数在 $x=1$ 处由正变负,函数先增后减,所以在 $x=1$ 处取得局部极大值。

    答案:B(极大值点)。
\end{solution}

\begin{question}{}{}

    设 $\lim _{x\to 0}\frac {\sin 6x+ xf( x) }{x^3}= 0$,则$\lim_{x\to0}\frac{6+f(x)}{x^2}=$
    \begin{multicols}{4}  % 并排两列
        \begin{itemize}[label={}]
            \item A 0
            \item B 6
            \item C 36
            \item D $\infty$
        \end{itemize}
    \end{multicols}
\end{question}
\begin{solution}
    使用 $\sin6x$ 的泰勒展开:
    \[
        \sin6x=6x-\frac{(6x)^3}{6}+o(x^3)=6x-36x^3+o(x^3).
    \]
    将其代入:
    \[
        \frac{\sin6x+xf(x)}{x^3}=\frac{6x-36x^3+xf(x)+o(x^3)}{x^3}
        =\frac{6}{x^2}-36+\frac{f(x)}{x^2}+o(1).
    \]
    极限为 0,故
    \[
        \frac{6+f(x)}{x^2}\to 36.
    \]
    答案:C(36)。
\end{solution}

\begin{question}{}{}

    求极限 $\lim_{x \to +\infty} \frac{\sqrt{5x}-1}{\sqrt{x+2}}$.
\end{question}
\begin{solution}
    按主项比较:
    \[
        \frac{\sqrt{5x}-1}{\sqrt{x+2}}
        = \frac{\sqrt{x}\left(\sqrt5 - \tfrac{1}{\sqrt{x}}\right)}{\sqrt{x}\sqrt{1+2/x}}
        \to \frac{\sqrt5}{1}=\sqrt5.
    \]
    答案:$\sqrt5$。
\end{solution}

\begin{question}{}{}

    求极限 $\lim_{x \to 2} \frac{|x-2|}{x^2-4}$.
\end{question}
\begin{solution}
    化简 ($x\neq2$):
    \[
        \frac{|x-2|}{x^2-4}=\frac{|x-2|}{(x-2)(x+2)}=\frac{1}{|x+2|}.
    \]
    当 $x\to2$,得 $1/4$。

    答案:$\displaystyle \frac14$。
\end{solution}

\begin{question}{}{}

    求极限 $\lim_{x \to 0} \frac{\left(\frac{2+\cos x}{3}\right)^x-1}{x^3}$.
\end{question}
\begin{solution}
    令 $a(x)=\dfrac{2+\cos x}{3}$. 当 $x\to0$,
    \[
        a(x)=1-\frac{x^2}{6}+O(x^4).
    \]
    因此
    \[
        \ln a(x)=-\frac{x^2}{6}+o(x^2).
    \]
    有
    \[
        \left(a(x)\right)^x=e^{x\ln a(x)}=1+x\ln a(x)+o(x\ln a(x))=1-\frac{x^3}{6}+o(x^3).
    \]
    故所求极限为 $-1/6$。
\end{solution}

\begin{question}{}{}

    求极限 $\lim_{x \to 0} \frac{e^{\tan x}-e^{\sin x}}{x^2 \ln(1+x)}$.
\end{question}
\begin{solution}
    先用展开:
    \[
        \tan x = x+\frac{x^3}{3}+o(x^3),\qquad \sin x = x-\frac{x^3}{6}+o(x^3),
    \]
    所以
    \[
        \tan x-\sin x=\frac{x^3}{2}+o(x^3).
    \]
    写
    \[
        e^{\tan x}-e^{\sin x}=e^{\sin x}\big(e^{\tan x-\sin x}-1\big)
        \sim e^{x}\big(\tan x-\sin x\big)\sim 1\cdot\frac{x^3}{2}.
    \]
    分母:
    \[
        x^2\ln(1+x)=x^2\big(x+o(x)\big)=x^3+o(x^3).
    \]
    因此极限为 $(1/2)/1=1/2$。
\end{solution}

\begin{question}{}{}

    求极限 $\lim_{x \to 0} \frac{e^{x^2}-e^{2-2\cos x}}{x^4}$.
\end{question}
\begin{solution}
    先展开:
    \[
        e^{x^2}=1+x^2+\tfrac{x^4}{2}+o(x^4),
    \]
    而
    \[
        2-2\cos x=2-2\big(1-\tfrac{x^2}{2}+\tfrac{x^4}{24}+o(x^4)\big)=x^2-\tfrac{x^4}{12}+o(x^4).
    \]
    于是
    \[
        e^{2-2\cos x}=1+(x^2-\tfrac{x^4}{12})+\tfrac{1}{2}x^4+o(x^4)
        =1+x^2+x^4\Big(-\tfrac{1}{12}+\tfrac{1}{2}\Big)+o(x^4)
        =1+x^2+\tfrac{5}{12}x^4+o(x^4).
    \]
    两者相减得
    \[
        e^{x^2}-e^{2-2\cos x}=\Big(\tfrac{1}{2}-\tfrac{5}{12}\Big)x^4+o(x^4)=\tfrac{1}{12}x^4+o(x^4).
    \]
    所以极限为 $1/12$。
\end{solution}

\begin{question}{}{}

    求极限 $\lim_{x \to +\infty} \left[x - x^2 \ln \left(1 + \frac{1}{x}\right)\right]$.
\end{question}
\begin{solution}
    使用 $\ln(1+u)=u-\tfrac{u^2}{2}+o(u^2)$,取 $u=1/x$:
    \[
        x^2\ln\Big(1+\frac{1}{x}\Big)=x^2\Big(\frac{1}{x}-\frac{1}{2x^2}+o(1/x^2)\Big)=x-\frac{1}{2}+o(1).
    \]
    因此差为约 $\tfrac12$,极限为 $1/2$。
\end{solution}

\begin{question}{}{}

    求极限 $\lim_{x \to 0} \frac{e^{x^2} - \cos x}{x^2}$.
\end{question}
\begin{solution}
    展开:
    \[
        e^{x^2}=1+x^2+\tfrac{x^4}{2}+o(x^4),\qquad \cos x=1-\tfrac{x^2}{2}+\tfrac{x^4}{24}+o(x^4).
    \]
    相减得主项为 $x^2+\tfrac{x^2}{2}=\tfrac{3}{2}x^2$,除以 $x^2$ 得 $3/2$。
\end{solution}

\begin{question}{}{}

    求极限 $\lim_{x \to 0} \frac{[\sin x - \sin (\sin x)] \sin x}{x^4}$.
\end{question}
\begin{solution}
    先展开:
    \[
        \sin x = x-\tfrac{x^3}{6}+o(x^3).
    \]
    令 $s=\sin x=x-\tfrac{x^3}{6}+o(x^3)$,则
    \[
        \sin(\sin x)=s-\tfrac{s^3}{6}+o(s^3)=\Big(x-\tfrac{x^3}{6}\Big)-\tfrac{1}{6}x^3+o(x^3)=x-\tfrac{x^3}{3}+o(x^3).
    \]
    因此
    \[
        \sin x-\sin(\sin x)=\tfrac{x^3}{6}+o(x^3).
    \]
    乘以 $\sin x\sim x$,分子 $\sim \tfrac{x^4}{6}$,故极限为 $1/6$。
\end{solution}

\begin{question}{}{}

    求极限 $\lim_{x \to 0} \left(\frac{2 + e^{\frac{1}{x}}}{1 + e^{\frac{4}{x}}} + \frac{\sin x}{|x|}\right)$.
\end{question}
\begin{solution}
    分左右极限考察。

    当 $x\to0^+$ 时,$1/x\to+\infty$,故 $e^{1/x}\to\infty,\ e^{4/x}\to\infty$,且
    \[
        \frac{2+e^{1/x}}{1+e^{4/x}}=\frac{e^{1/x}\big(1+2e^{-1/x}\big)}{e^{4/x}\big(1+e^{-4/x}\big)}=e^{-3/x}(1+o(1))\to0.
    \]
    同时 $\dfrac{\sin x}{|x|}=\dfrac{\sin x}{x}\to1$。和为 $1$。

    当 $x\to0^-$ 时,$1/x\to-\infty$,所以 $e^{1/x}\to0,\ e^{4/x}\to0$,首项 $\to 2$;而 $\dfrac{\sin x}{|x|}=\dfrac{\sin x}{-x}=-\dfrac{\sin x}{x}\to -1$。和为 $2-1=1$。

    左右极限相等,极限存在且等于 $1$。
\end{solution}

\begin{question}{}{}

    求极限$\lim_{n\to\infty}\left(\frac{1}{\sqrt{n^2+1^2}}+\frac{1}{\sqrt{n^2+2^2}}+\cdots+\frac{1}{\sqrt{n^2+n^2}}\right)$.
\end{question}
\begin{solution}
    写为求和的 Riemann 近似:
    \[
        \sum_{k=1}^n \frac{1}{\sqrt{n^2+k^2}}=\sum_{k=1}^n \frac{1}{n}\cdot\frac{1}{\sqrt{1+(k/n)^2}}.
    \]
    当 $n\to\infty$,该和趋于
    \[
        \int_0^1 \frac{dt}{\sqrt{1+t^2}}.
    \]
    计算不定积分,$\int \frac{dt}{\sqrt{1+t^2}}=\operatorname{arsinh}t=\ln\big(t+\sqrt{1+t^2}\big)$。因此值为
    \[
        \ln\big(1+\sqrt2\big).
    \]
\end{solution}

\begin{question}{}{}

    设$f(x)\in C[a,b]$,证明$\exists\xi\in[a,b]$,使得$\int_{a}^{b}f(x)dx=f(\xi)(b-a)$.
\end{question}
\begin{solution}
    这是积分中值定理(连续函数的平均值定理)。因为 $f$ 在 $[a,b]$ 上连续,令
    \[
        m=\min_{[a,b]} f,\qquad M=\max_{[a,b]} f.
    \]
    则 $m(b-a)\le\int_a^b f(x)\,dx\le M(b-a)$. 由于 $f$ 在 $[a,b]$ 连续,值域为 $[m,M]$,而平均值 $\dfrac{1}{b-a}\int_a^b f(x)\,dx$ 属于 $[m,M]$,由连续性和介值定理存在 $\xi\in[a,b]$ 使得
    \[
        f(\xi)=\frac{1}{b-a}\int_a^b f(x)\,dx,
    \]
    即 $\int_a^b f(x)\,dx=f(\xi)(b-a)$。证毕。
\end{solution}

\begin{question}{}{}

    设$x_{1}=\sqrt{2}$,$x_{n}=\sqrt{2+x_{n-1}}$,$n=2,3,\cdots$. 证明$\{x_{n}\}$极限存在,并求极限.
\end{question}
\begin{solution}
    首先证明单调性与有界性:

    (1) 计算数列初项:$x_1=\sqrt2\approx1.414$,$x_2=\sqrt{2+\sqrt2}\approx1.847\,$。猜测单调递增且有上界 2。

    (2) 若 $x_{n-1}<2$,则 $x_n=\sqrt{2+x_{n-1}}<\sqrt{4}=2$,所以若某一项 $<2$,后项都 $<2$,故有上界 2。且
    \[
        x_{n+1}-x_n=\sqrt{2+x_n}-\sqrt{2+x_{n-1}}=\frac{x_n-x_{n-1}}{\sqrt{2+x_n}+\sqrt{2+x_{n-1}}}.
    \]
    若已知 $x_n>x_{n-1}$,则右端 $>0$,由此可递推证明 $\{x_n\}$ 单调递增(因 $x_2>x_1$ 起)。

    因此序列单调有界,故收敛。设极限为 $L$,取极限于递推式得
    \[
        L=\sqrt{2+L}\quad\Rightarrow\quad L^2=2+L\quad\Rightarrow\quad L^2-L-2=0.
    \]
    解得 $L=\dfrac{1\pm\sqrt{1+8}}{2}=\dfrac{1\pm3}{2}$,因为项均为正且小于 2,取正根 $L=2$。

    结论:极限存在且为 $2$。
\end{solution}