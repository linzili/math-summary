\GAIchapter{多维随机变量函数的分布}

\section{多维$\rightarrow$一维}
\begin{enumerate}
      \item (离散型,离散型)$\rightarrow$离散型
            \begin{enumerate}
                  \item $(X,Y)\sim p_{ij}$ , $Z=g(X,Y)$ ,有$Z\sim q_i$.
                  \item $X \sim p_k$, $Y \sim q_k$, $X$, $Y$ 独立且取值在某一集合, 可考 $Z = X + Y$, $XY$, $\max\{X, Y\}$, $\min\{X, Y\}$ 等, 这是重点
            \end{enumerate}
      \item (连续型,连续型)$\rightarrow$连续型
            \begin{enumerate}
                  \item 分布函数法.

                        设$( X, Y) \thicksim f( x, y)$ , $Z= g( X, Y)$,则
                        $$F_{z}(z)=P\{g(X,Y)\leqslant z\}=\iint_{g(x,y)\leqslant z}f(x,y)\mathrm{d}x\mathrm{d}y\:,$$
                        $$f_z(z)=F_z^{\prime}(z).$$
                  \item 换元法.

                        设 $(X,Y) \sim f(x,y)$,若 $\begin{cases} u=u(x,y), \\ v=v(x,y) \end{cases}$ 具有一阶连续偏导数,且存在唯一的反函数 $\begin{cases} x=x(u,v), \\ y=y(u,v), \end{cases}$ 又
                        $$J=\frac{\partial(x,y)}{\partial(u,v)}=\left| \begin{array}{cc} \frac{\partial x}{\partial u} & \frac{\partial x}{\partial v} \\ \frac{\partial y}{\partial u} & \frac{\partial y}{\partial v} \end{array} \right| \neq 0 \text{,}$$

                        则 $(X,Y)$ 的函数 $\begin{cases} U=U(X,Y), \\ V=V(X,Y) \end{cases}$ 的联合概率密度为 $f_{U,V}(u,v)=f[x(u,v),y(u,v)]\cdot|J|$.

                        $U$ 的概率密度为 $f_U(u)=\int_{-\infty}^{+\infty} f_{U,V}(u,v)\text{d}v$, $V$ 的概率密度为 $f_V(v)=\int_{-\infty}^{+\infty} f_{U,V}(u,v)\text{d}u$.
                  \item 最值函数的分布.
                        \begin{enumerate}
                              \item $\max\{X,Y\}$的分布

                                    设$(X,Y)\sim F(x,y)$,则$Z=\max\{X,Y\}$ 的分布函数为
                                    $$F_{\text{max}}(z)=P\{\text{max}\{X,Y\} \leqslant z\}=P\{X \leqslant z,Y \leqslant z\}=F(z,z)$$
                                    当 $X$ 与 $Y$ 相互独立时,
                                    $$F_{\text{max}}(z)=F_X(z) \cdot F_Y(z)$$
                              \item $\min\{X,Y\}$的分布

                                    设$(X,Y)\sim F(x,y)$,则$Z=\min\{X,Y\}$ 的分布函数为
                                    $$F_{\text{min}}(z)=P\{\text{min}\{X,Y\} \leqslant z\}=P\{\{X \leqslant z\} \cup \{Y \leqslant z\}\}$$
                                    $$=P\{X \leqslant z\}+P\{Y \leqslant z\}-P\{X \leqslant z,Y \leqslant z\}$$
                                    $$=F_X(z)+F_Y(z)-F(z,z)$$
                                    当 $X$ 与 $Y$ 相互独立时,
                                    $$F_{\text{min}}(z)=F_X(z)+F_Y(z)-F_X(z)F_Y(z)=1-[1-F_X(z)][1-F_Y(z)]$$
                        \end{enumerate}
                        推广到 $n$ 个相互独立的随机变量 $X_1,X_2,\cdots,X_n$  的情况,即
                        $$F_{\text{max}}(z)=F_{X_1}(z)F_{X_2}(z)\cdots F_{X_n}(z)$$
                        $$F_{\text{min}}(z)=1-[1-F_{X_1}(z)][1-F_{X_2}(z)]\cdots[1-F_{X_n}(z)]$$
                        特别地,当 $X_i(i=1,2,\cdots,n)$ 相互独立且有相同的分布函数 $F(x)$ 与概率密度 $f(x)$ 时,
                        $$F_{\text{max}}(z)=[F(z)]^n,\ f_{\text{max}}(z)=n[F(z)]^{n-1}f(z)$$
                        $$F_{\text{min}}(z)=1-[1-F(z)]^n,\ f_{\text{min}}(z)=n[1-F(z)]^{n-1}f(z)$$
            \end{enumerate}
      \item (离散型,连续型)$\rightarrow$连续型

            设$X\sim p_i,Y\sim f_Y(y),Z=g(X,Y)$(常考$X\pm Y,XY$等),则
            \begin{enumerate}
                  \item $X,Y$独立时,可用分布函数法及全概率公式求$F_z(z);$
                  \item $X,Y$不独立时,用分布函数法求$F_z(z).$
            \end{enumerate}
\end{enumerate}
\section{一维$\rightarrow$多维}
\begin{enumerate}
      \item 离散型$\rightarrow$(离散型,离散型)
            $$X\sim p_i,\begin{cases}U=g(X),\\V=h(X)\end{cases}\Rightarrow(U,V)\sim q_{ij}.$$
      \item 连续型$\rightarrow$(离散型,离散型)
            $$X\sim f(x),\begin{cases}U=g(X),\\V=h(X)\end{cases}\Rightarrow(U,V)\sim p_{ij}.$$
\end{enumerate}
\section{多维$\rightarrow$多维}

\begin{enumerate}
      \item (离散型,离散型)$\rightarrow$(离散型,离散型)
            $$(X,Y)\sim p_{ij},\begin{cases}U=g(X,Y),\\V=h(X,Y)\end{cases}\Rightarrow(U,V)\sim q_{ij}.$$
      \item (连续型,连续型)$\rightarrow$(离散型,离散型)或(连续型,连续型)
            $$(X,Y)\sim f(x,y),\begin{cases}U=g(X,Y),\\V=h(X,Y)\end{cases}\Rightarrow(U,V)\sim p_{ij}\text{或}f_{U,V}(u,v).$$
      \item (离散型,连续型)$\rightarrow$(离散型,离散型)
            $$X\sim p_i,\:Y\sim f_Y(y),\:\begin{cases}U=g(X,Y),\\V=h(X,Y)\end{cases}\Rightarrow(U,V)\sim q_{ij}.$$
\end{enumerate}