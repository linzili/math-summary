\GAIchapter{一维随机变量函数的分布}

\section{离散型$\rightarrow$离散型}
设离散型随机变量 $X$ 的分布为 $P\{X = x_i\} = p_i \ (i=1,2,\cdots)$,若 $Y = g(X)$,则 $Y$ 仍为离散型随机变量,其分布为
$$
      Y \sim
      \begin{pmatrix}
            g(x_1) & g(x_2) & \cdots \\
            p_1    & p_2    & \cdots
      \end{pmatrix}.
$$
若若干个 $g(x_k)$ 取相同值,则合并为一项,并将对应概率相加。

\section{连续型$\rightarrow$连续型(或混合型)}
设连续型随机变量 $X$ 的分布函数与密度分别为 $F_X(x)$、$f_X(x)$,若 $Y = g(X)$,则可用以下两种方法求其分布:

\subsection*{(1) 分布函数法}
由定义直接求:
$$
      F_Y(y) = P\{Y \le y\} = P\{g(X) \le y\} = \int_{g(x) \le y} f_X(x) \, dx.
$$
若 $F_Y(y)$ 连续且可导,则 $f_Y(y) = F_Y'(y)$。

\subsection*{(2) 公式法(单调可导变换)}
若 $y = g(x)$ 在 $(a,b)$ 上严格单调且可导,则存在反函数 $x = h(y)$,其概率密度为
$$
      f_Y(y) = f_X[h(y)] \cdot |h'(y)|, \quad \alpha < y < \beta,
$$
其中
$$
      \alpha = \min\{\lim_{x\to a^+} g(x), \lim_{x\to b^-} g(x)\}, \quad
      \beta = \max\{\lim_{x\to a^+} g(x), \lim_{x\to b^-} g(x)\}.
$$

\section{连续型$\rightarrow$离散型}
若 $X \sim f_X(x)$ 且 $Y = g(X)$ 为离散型变量,先求出 $Y$ 的可能取值 $a_i$,再由
$$
      P\{Y = a_i\} = \int_{g(x)=a_i} f_X(x) \, dx
$$
得出其分布。
\section{两种重要的随机变量变换}

\subsection*{(1) 变换于 $U(0,1)$}
\begin{example}{}{}
      设随机变量 $X$ 的分布函数 $F_X(x)$ 严格单调递增,反函数 $F_X^{-1}(y)$ 存在,令 $Y = F_X(X)$,则 $Y \sim U(0,1)$。
\end{example}
\begin{proof}
      由定义:
      $$
            F_Y(y) = P\{Y \le y\} = P\{F_X(X) \le y\} = P\{X \le F_X^{-1}(y)\} = F_X[F_X^{-1}(y)] = y,
      $$
      对 $0 \le y < 1$ 成立;此外,$F_Y(y) = 0 (y < 0)$,$F_Y(y) = 1 (y \ge 1)$,即
      $$
            F_Y(y) =
            \begin{cases}
                  0, & y < 0,       \\
                  y, & 0 \le y < 1, \\
                  1, & y \ge 1,
            \end{cases}
      $$
      故 $Y \sim U(0,1)$。
\end{proof}


\subsection*{(2) 变换于 $E(1)$}
\begin{example}{}{}
      设 $X$ 的分布函数 $F_X(x)$ 连续且在其密度区间上严格单调,令
      $$
            Y = -\ln[1 - F_X(X)],
      $$
      则 $Y \sim E(1)$。
\end{example}
\begin{proof}
      由定义:
      $$
            P\{Y \le y\} = P\{-\ln[1 - F_X(X)] \le y\} = P\{F_X(X) \le 1 - e^{-y}\}.
      $$
      由于 $F_X(X) \sim U(0,1)$,故
      $$
            F_Y(y) = 1 - e^{-y}, \quad y > 0,
      $$
      即 $Y \sim E(1)$。
\end{proof}