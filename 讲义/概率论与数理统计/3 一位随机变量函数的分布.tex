\chapter{一位随机变量函数的分布}

\section{离散型$\rightarrow$离散型}
设$X$为离散型随机变量,其概率分布为$p_i=P\{X=x_i\}(i=1,2,\cdotp\cdotp\cdotp)$,则$X$的函数$Y=g(X)$也是离散型随机变量,其概率分布为$P\{Y=g(x_i)\}=p_i$ ,即

$$Y\sim\begin{pmatrix}g(x_1)&g(x_2)&...\\p_1&p_2&...\end{pmatrix}.$$

如果有若干个$g(x_k)$相同,则合并诸项为一项$g(x_k)$,并将相应概率相加作为 $Y$取 $g(x_k)$值的概率.
\section{连续型$\rightarrow$连续型(或混合型)}
设$X$为连续型随机变量,其分布函数、概率密度分别为$F_x(x)$与$f_x(x)$,随机变量$Y=g(X)$是$X$的函数,则$Y$的分布函数或概率密度可用下面两种方法求得.
\begin{enumerate}
    \item 分布函数法

          直接由定义求$Y$的分布函数

          $$F_{Y}(y)=P\{Y\leqslant y\}=P\{g(X)\leqslant y\}=\int_{g(x)\leqslant y}f_{X}(x)\mathrm{d}x\:.$$

          如果$F_Y(y)$连续,且除有限个点外,$F_Y^\prime(y)$存在且连续,则 $Y$的概率密度$f_Y(y)=F_Y^\prime(y)$ .
    \item 公式法 \DTwoThree

          根据上面的分布函数法,若 $y = g(x)$ 在 $(a, b)$ 上是关于 $x$ 的严格单调可导函数,则存在 $x = h(y)$ 是 $y = g(x)$ 在 $(a, b)$ 上的可导反函数.
          \begin{enumerate}
              \item 若 $y = g(x)$ 严格单调增加,则 $x = h(y)$ 也严格单调增加,即 $h'(y) > 0$,且
                    $$F_y(y) = P\{Y \leqslant y\} = P\{g(X) \leqslant y\} = P\{X \leqslant h(y)\} = \int_{-\infty}^{h(y)} f_X(x) \, dx$$
                    故 $f_y(y) = F_y'(y) = f_X[h(y)] \cdot h'(y)$.
              \item 若 $y = g(x)$ 严格单调减少,则 $x = h(y)$ 也严格单调减少,即 $h'(y) < 0$,且
                    $$
                        F_y(y) = P\{Y \leqslant y\} = P\{g(X) \leqslant y\} = P\{X \geqslant h(y)\} = \int_{h(y)}^{+\infty} f_X(x) \, dx$$
                    故 $f_y(y) = F_y'(y) = -f_X[h(y)] \cdot h'(y) = f_X[h(y)] \cdot [-h'(y)]$.
          \end{enumerate}

          综上,
          $$f_y(y) = \begin{cases} f_X[h(y)] \cdot |h'(y)|, & \alpha < y < \beta, \\ 0, & \text{其他,} \end{cases}$$

          其中 $\alpha = \min \left\{ \lim_{x \to a^+} g(x), \lim_{x \to b^-} g(x) \right\}$, $\beta = \max \left\{ \lim_{x \to a^+} g(x), \lim_{x \to b^-} g(x) \right\}$.

\end{enumerate}
\section{连续型$\rightarrow$离散型}
若$X\thicksim f_X(x)$,且$Y=g(X)$是离散型随机变量.首先确定$Y$的可能取值$a$,然后通过计算概率$P\{Y=a\}$求得$Y$的概率分布.

\section{两种重要的随机变量变换}

\begin{enumerate}
    \item 变换于$U(0,1)$
          \begin{example}{}{}
              设随机变量$X$的分布函数$F_x(x)$ 是严格单调增加函数,其反函数$F_x^{-1}(y)$存在,$Y=F_X(X)$.证明:$Y$服从区间$(0,1)$上的均匀分布.
          \end{example}
          \begin{proof}
              $Y = F_{X}(X)$ 是在区间$(0,1)$上取值的随机变量,故

              当$y<0$时,$F_{Y}(y)=0$;

              当$y\ge 1$时,$F_{Y}(y)=1$;

              当$0\ge y<1$时,
              $$F_{Y}(y)=P\{Y\le y\}=P\{F_{X}(X)\le y\}=P\{X\le F_{X}^{-1}(y)\}=F_{X}[F_{X}^{-1}(y)]=y.$$
              综上所述,$Y = F_{X}(X)$ 的分布函数为
              $$
                  F_{Y}(y)=\begin{cases}0,&y<0,\\y,&0\le y<1,\\1,&y\ge 1,\end{cases}$$
              这就是在区间$(0,1)$上的均匀分布函数,所以$Y\sim U(0,1)$.
          \end{proof}
    \item 变换于$E(1)$
          \begin{example}{}{}
              设随机变量$X$的分布函数$F_x(x)$连续,且$F_x(x)$在$X$的正概率密度区间上严格单调.$Y=-\ln[1-F_{\chi}(X)]$,证明随机变量 $Y$服从指数分布 $E(1).$
          \end{example}
          \begin{proof}
              由于$F_{X}(x)$连续,因此$Y$为单调不减且非负的连续函数,又对任意的$x>0$,有

              $$P\{Y\leqslant x\}=P\{-\ln[1-F_{X}(X)]\leqslant x\}=P\{F_{X}(X)\leqslant 1-e^{-x}\},$$

              且随机变量$F_{X}(X)$服从$U(0,1)$,于是

              $$P\{Y\leqslant x\}=P\{F_{X}(X)\leqslant 1-e^{-x}\}=1-e^{-x},x>0,$$

              故$Y$服从指数分布$E(1)$.
          \end{proof}
\end{enumerate}