\chapter{数字特征}

计算数字特征、判别独立与不相关、用切比雪夫不等式做概率计算

\section{数学期望}
数学期望就是随机变量的取值与取值的概率乘积的和.
\begin{enumerate}
      \item $X$
            \begin{enumerate}
                  \item $X\sim p_{i}\Rightarrow EX=\sum_{i}x_{i}p_{i}\begin{cases}\text{有限项相加,}\\\text{无穷项相加(无穷级数).}\end{cases}$
                  \item $X\sim f(x)\Rightarrow EX=\int_{-\infty}^{+\infty}xf(x)$d$x\begin{cases}\text{有限区间积分(定积分),}\\\text{无穷区间积分(反常积分).}\end{cases}$
            \end{enumerate}
      \item $g(X)$

            g为连续函数(或分段连续函数).
            \begin{enumerate}
                  \item $X\sim p_{i},Y=g(X)\Rightarrow EY=\sum_{i}g(x_{i})p_{i}.$
                  \item $X\sim f( x)$ , $Y= g( X) \Rightarrow EY= \int _{- \infty }^{+ \infty }g( x) f( x)dx$.
            \end{enumerate}
      \item $g(X,Y)$
            \begin{enumerate}
                  \item $( X, Y) \sim p_{ij} , Z= g( X, Y) \Rightarrow EZ= \sum _{i}\sum _{j}g( x_{i}, y_{j}) p_{ij}$.
                  \item $( X, Y) \sim f(x,y) , Z= g( X, Y) \Rightarrow EZ= \int _{- \infty }^{+ \infty }\int _{- \infty }^{+ \infty }g( x, y) f( x, y)dxdy$ .
            \end{enumerate}
      \item 最值
            \begin{enumerate}
                  \item 若 $X_i(i=1,2,\cdots,n;n\geqslant 2)$ 独立同分布,其分布函数为 $F(x)$,概率密度为 $f(x)$,记
                        $$Y = \min\{X_1,X_2,\cdots,X_n\},\ Z = \max\{X_1,X_2,\cdots,X_n\},$$
                        则
                        \begin{enumerate}
                              \item $F_Y(y) = 1 - [1 - F(y)]^n,\ f_Y(y) = n[1 - F(y)]^{n-1}f(y) \Rightarrow EY = \int_{-\infty}^{+\infty} yf_Y(y)dy;$
                              \item $F_Z(z) = [F(z)]^n,\ f_Z(z) = n[F(z)]^{n-1}f(z) \Rightarrow EZ = \int_{-\infty}^{+\infty} zf_Z(z)dz.$
                        \end{enumerate}
                  \item 用好转化公式:
                        $$\max\{X,Y\} = \frac{X+Y+|X-Y|}{2};\quad\min\{X,Y\} = \frac{X+Y-|X-Y|}{2};$$
                        $$\max\{X,Y\} + \min\{X,Y\} = X+Y;$$
                        $$\max\{X,Y\} - \min\{X,Y\} = |X-Y|;\quad\max\{X,Y\} \cdot \min\{X,Y\} = XY.$$
                  \item 用好降维法,令 $Z = X-Y$.
                  \item 用好标准化,令 $U = \frac{X-\mu}{\sigma}$.
            \end{enumerate}
      \item 分解

            若 $X = X_{1} + X_{2} + \cdots + X_{n}$,则 $EX = EX_{1} + EX_{2} + \cdots + EX_{n}$.
      \item 性质
            \begin{enumerate}
                  \item $Ea=a$, $E(EX)=EX$.
                  \item $E(aX+bY)=aEX+bEY$, $E(\sum_{i=1}^{n}a_iX_i)=\sum_{i=1}^{n}a_iEX_i$.
                  \item 若 $X$,$Y$ 相互独立,则 $E(XY)=EXEY$.
            \end{enumerate}
\end{enumerate}
\section{方差}
\begin{enumerate}
      \item X
            \begin{enumerate}
                  \item 定义

                        $DX = E[(X - EX)^2]$,$X$的方差就是$Y = (X - EX)^2$的数学期望.
                  \item 定义法.
                        $$X \sim p_i \Rightarrow DX = E[(X - EX)^2] = \sum_i (x_i - EX)^2 p_i,$$
                        $$X \sim f(x) \Rightarrow DX = E[(X - EX)^2] = \int_{-\infty}^{+\infty} (x - EX)^2 f(x) dx.$$
                  \item 公式法.

                        $DX = E(X^2) - (EX)^2$.
            \end{enumerate}
      \item 最值的方差
            $$E(Y^{2})=\int_{-\infty}^{+\infty}y^{2}f_{Y}(y)\mathrm{d}y\Rightarrow DY=E(Y^{2})-(EY)^{2};$$

            $$E(Z^{2})=\int_{-\infty}^{+\infty}z^{2}f_{Z}(z)\mathrm{d}z\Rightarrow DZ=E(Z^{2})-(EZ)^{2}.$$
      \item 绝对值函数$|aX+bY+c|$的方差

            若 $U=aX+bY+c$,则
            $$EU=aEX+bEY+c,$$
            $$
                  DU=a^2DX+b^2DY(X,Y \text{相互独立}),$$
            $$D(|U|)=E(U^2)-[E(|U|)]^2$$
            $$
                  =DU+(EU)^2-[E(|U|)]^2,$$
            其中 $E(|U|)=\begin{cases}\int_{-\infty}^{+\infty}|u|f(u)du \text{ (连续型)},\\\sum_i|u_i|p_i \text{ (离散型)}.\end{cases}$
      \item 分解随机变量后再求⽅差

            若$X=X_{1}+X_{2}+\cdots+X_{n}$,则$DX=DX_{1}+DX_{2}+\cdots+DX_{n}+2\sum_{1\leqslant i<j\leqslant n}\operatorname{Cov}(X_{i},X_{j})$.

            当$X_{1}$,$X_{2}$,$\cdots$,$X_{n}$相互独立时,有$DX=DX_{1}+DX_{2}+\cdots+DX_{n}$.
      \item 性质
            \begin{enumerate}
                  \item $DX \geqslant 0$, $E(X^2) = DX + (EX)^2 \geqslant (EX)^2$.
                  \item $Dc = 0$($c$ 为常数).

                        $DX = 0 \Leftrightarrow X$ 几乎处处为某个常数 $a$, 即 $P\{X = a\} = 1$.
                  \item $D(aX + b) = a^2DX$.
                  \item $D(X \pm Y) = DX + DY \pm 2\text{Cov}(X, Y)$, $D\left(\sum_{i=1}^{n} a_i X_i\right) = \sum_{i=1}^{n} a_i^2 DX_i + 2 \sum_{1 \leqslant i < j \leqslant n} a_i a_j \text{Cov}(X_i, X_j)$.
            \end{enumerate}
      \item 常用分布的$EX,DX$
            \begin{enumerate}
                  \item 0—1 分布, $EX=p$ , $DX=p-p^{2}=(1-p)p$.
                  \item $X \sim B(n, p)$ , $EX=np$ , $DX=np(1-p)$.
                  \item $X \sim P(\lambda)$ , $EX=\lambda$ , $DX=\lambda$.
                  \item $X \sim G(p)$ , $EX=\frac{1}{p}$ , $DX=\frac{1-p}{p^{2}}$.
                  \item $X \sim U(a, b)$ , $EX=\frac{a+b}{2}$ , $DX=\frac{(b-a)^{2}}{12}$.
                  \item $X \sim E(\lambda)$ , $EX=\frac{1}{\lambda}$ , $DX=\frac{1}{\lambda^{2}}$.
                  \item $X \sim N(\mu, \sigma^{2})$ , $EX=\mu$ , $DX=\sigma^{2}$.
                  \item $X \sim \chi^{2}(n)$ , $EX=n$ , $DX=2n$.
            \end{enumerate}
\end{enumerate}
\section{协方差$Cov(X,Y)$与相关系数$\rho(X,Y)$}
\begin{enumerate}
      \item Cov(X,Y)
            \begin{enumerate}
                  \item 定义.
                        $$\mathrm{Cov}(X,Y)\overset{\Delta}{\operatorname*{\Longrightarrow}}E[(X-EX)(Y-EY)].$$
                  \item 定义法.
                        $$(X,Y) \sim p_{ij} \Rightarrow \operatorname{Cov}(X,Y) = \sum_{i} \sum_{j} (x_i - EX)(y_j - EY)p_{ij},$$
                        $$(X,Y) \sim f(x,y) \Rightarrow \operatorname{Cov}(X,Y) = \int_{-\infty}^{+\infty} \int_{-\infty}^{+\infty} (x - EX)(y - EY)f(x,y) \, dx \, dy.$$
                  \item 公式法.
                        $$\operatorname{Cov}(X,Y) = E(XY) - EXEY .$$
            \end{enumerate}
      \item $\rho(X,Y)$(定义相关系数,表示线性相依程度)

            $\rho_{xr}=\frac{\mathrm{Cov}(X,Y)}{\sqrt{DX}\sqrt{DY}}\begin{cases}=0\Leftrightarrow X,Y\text{不相关,}\\\neq0\Leftrightarrow X,Y\text{相关.}\end{cases}$

            (量纲为 1,无单位 )
      \item 性质
            \begin{enumerate}
                  \item $\mathrm{Cov}(X,Y) = \mathrm{Cov}(Y,X)$ .
                  \item $\mathrm{Cov}(aX,bY) = ab\mathrm{Cov}(X,Y)$.
                  \item $\mathrm{Cov}(X_1 + X_2,Y) = \mathrm{Cov}(X_1,Y) + \mathrm{Cov}(X_2,Y)$.
                  \item $|\rho_{XY}| \le 1$.
                  \item $\rho_{XY}= 1 \Leftrightarrow P\{Y = aX + b\} = 1 (a > 0)$ .

                        $\rho_{XY}= -1 \Leftrightarrow P\{Y = aX + b\} = 1 (a < 0)$.
                  \item 五个充要条件.

                        $\rho_{XY} = 0 \Leftrightarrow \text{Cov}(X,Y) = 0 \Leftrightarrow E(XY) = EXEY$
                        $\Leftrightarrow D(X+Y) = DX + DY \Leftrightarrow D(X-Y) = DX + DY.$
                  \item  $X,Y$ 独立 $\Rightarrow \rho_{XY} = 0$.
                  \item 若 $(X,Y) \sim N(\mu_1, \mu_2; \sigma_1^2, \sigma_2^2; \rho_{XY})$,则 $X,Y$ 独立 $\Leftrightarrow X,Y$ 不相关 ($\rho_{XY} = 0$).
            \end{enumerate}
\end{enumerate}
\section{独立性与不相关性的判定}
\begin{enumerate}
      \item 用分布判独立

            随机变量 $X$ 与 $Y$ 相互独立,指对任意实数 $x, y$,事件 $\{X \leqslant x\}$ 与 $\{Y \leqslant y\}$ 相互独立,即 $X$ 和 $Y$ 的联合分布等于边缘分布相乘:$F(x, y) = F_X(x) \cdot F_Y(y)$.
            \begin{enumerate}
                  \item 若 $(X, Y)$ 是连续型的,则 $X$ 与 $Y$ 相互独立的充要条件是 $f(x, y) = f_X(x) \cdot f_Y(y)$;
                  \item 若 $(X, Y)$ 是离散型的,则 $X$ 与 $Y$ 相互独立的充要条件是
                        $$P\{X = x_i, Y = y_j\} = P\{X = x_i\} \cdot P\{Y = y_j\} \, .$$
            \end{enumerate}

      \item 用数字特征判不相关

            随机变量 $X$ 与 $Y$ 不相关,意指 $X$ 与 $Y$ 之间不存在线性相依性,即 $\rho_{XY} = 0$,其充要条件是
            $$\rho_{XY} = 0 \Leftrightarrow \text{Cov}(X, Y) = 0 \Leftrightarrow E(XY) = EXEY \Leftrightarrow D(X \pm Y) = DX + DY \, .$$
      \item 步骤

            先计算 $\text{Cov}(X, Y)$,然后按下列步骤进行判断或再计算:
            $$\text{Cov}(X, Y) = E(XY) - EXEY \begin{cases} \neq 0 \Leftrightarrow X \text{ 与 } Y \text{ 相关} \Rightarrow X \text{ 与 } Y \text{ 不独立} \\ = 0 \Leftrightarrow X \text{ 与 } Y \text{ 不相关} \text{,通过分布推断} \begin{cases} X, Y \text{ 独立} \\ X, Y \text{ 不独立} \end{cases} \end{cases}$$

      \item 重要结论
            \begin{enumerate}
                  \item 如果 $X$ 与 $Y$ 独立,则 $X, Y$ 不相关,反之不然.
                  \item 如果 $X$ 与 $Y$ 相关,则 $X, Y$ 不独立.
                  \item 如果 $(X, Y)$ 服从二维正态分布,则 $X, Y$ 独立 $\Leftrightarrow X, Y$ 不相关.
                  \item 如果 $X$ 与 $Y$ 均服从 $0-1$ 分布,则 $X, Y$ 独立 $\Leftrightarrow X, Y$ 不相关.
            \end{enumerate}
\end{enumerate}

\section{切比雪夫不等式}
设随机变量$X$的数学期望与方差均存在,则对任意$\varepsilon>0$ ,
$$P\big\{\mid X-EX\mid\geqslant\varepsilon\big\}\leqslant\frac{DX}{\varepsilon^{2}}\:\text{或}\:P\{\big|X-EX\big|<\varepsilon\}\geqslant1-\frac{DX}{\varepsilon^{2}}.$$