\GAIchapter{数字特征}

\textbf{考点综述:}
掌握数学期望、方差、协方差、相关系数的定义与计算;
判别独立与不相关的区别;
能用切比雪夫不等式进行概率估计。


\section{数学期望(Expectation)}
数学期望即“取值 × 概率”的加权平均值。

\subsection*{1. 一维随机变量}
\begin{enumerate}
      \item 离散型:$X\sim p_i$
            $$EX=\sum_i x_i p_i.$$
      \item 连续型:$X\sim f(x)$
            $$EX=\int_{-\infty}^{+\infty} x f(x)\,dx.$$
\end{enumerate}

\subsection*{2. 一维函数 $g(X)$}
\begin{enumerate}
      \item 离散型:$EY=\sum_i g(x_i)p_i.$
      \item 连续型:$EY=\int_{-\infty}^{+\infty} g(x)f(x)\,dx.$
\end{enumerate}

\subsection*{3. 二维函数 $g(X,Y)$}
\begin{enumerate}
      \item 离散型:$E[g(X,Y)]=\sum_i\sum_j g(x_i,y_j)p_{ij}.$
      \item 连续型:$E[g(X,Y)]=\iint g(x,y)f(x,y)\,dx\,dy.$
\end{enumerate}

\subsection*{4. 最值随机变量}
若 $X_i(i=1,\dots,n)$ 独立同分布,分布函数 $F(x)$,密度 $f(x)$:
\begin{align*}
      F_Y(y) & =1-[1-F(y)]^n, & f_Y(y) & =n[1-F(y)]^{n-1}f(y), \\
      F_Z(z) & =[F(z)]^n,     & f_Z(z) & =n[F(z)]^{n-1}f(z),
\end{align*}
$$EY=\int yf_Y(y)dy,\quad EZ=\int zf_Z(z)dz.$$

常用代换:
$$
      \max\{X,Y\}=\tfrac{X+Y+|X-Y|}{2},\quad
      \min\{X,Y\}=\tfrac{X+Y-|X-Y|}{2}.
$$

\subsection*{5. 分解与性质}
\begin{enumerate}
      \item 线性性:$E(aX+bY)=aEX+bEY$.
      \item 独立性:若 $X,Y$ 独立,则 $E(XY)=EX\cdot EY$.
      \item 可加性:$E\!\left(\sum_i X_i\right)=\sum_i EX_i$.
\end{enumerate}


\section{方差(Variance)}

\subsection*{1. 定义与计算公式}
$$DX=E[(X-EX)^2]=E(X^2)-(EX)^2.$$

\begin{itemize}
      \item 离散型:$DX=\sum_i (x_i-EX)^2p_i.$
      \item 连续型:$DX=\int (x-EX)^2f(x)\,dx.$
\end{itemize}

\subsection*{2. 常见性质}
\begin{enumerate}
      \item $DX\ge0$,且 $DX=0\Leftrightarrow X=a(常数)$;
      \item $D(aX+b)=a^2DX$;
      \item $D(X\pm Y)=DX+DY\pm2\operatorname{Cov}(X,Y)$;
      \item 若 $X_i$ 独立:$D\!\left(\sum_i X_i\right)=\sum_i DX_i$。
\end{enumerate}

\subsection*{3. 常见分布的 $EX,DX$}
\begin{center}
      \begin{tabular}{lcc}
            \toprule
            分布                & $EX$                 & $DX$                   \\
            \midrule
            $B(1,p)$          & $p$                  & $p(1-p)$               \\
            $B(n,p)$          & $np$                 & $np(1-p)$              \\
            $P(\lambda)$      & $\lambda$            & $\lambda$              \\
            $G(p)$            & $\dfrac1p$           & $\dfrac{1-p}{p^2}$     \\
            $U(a,b)$          & $\dfrac{a+b}{2}$     & $\dfrac{(b-a)^2}{12}$  \\
            $E(\lambda)$      & $\dfrac{1}{\lambda}$ & $\dfrac{1}{\lambda^2}$ \\
            $N(\mu,\sigma^2)$ & $\mu$                & $\sigma^2$             \\
            $\chi^2(n)$       & $n$                  & $2n$                   \\
            \bottomrule
      \end{tabular}
\end{center}


\section{协方差与相关系数}

\subsection*{1. 协方差}
\begin{align*}
      \mathrm{Cov}(X,Y)
       & =E[(X-EX)(Y-EY)] \\
       & =E(XY)-EXEY.
\end{align*}

\subsection*{2. 相关系数}
$$
      \rho_{XY}=\frac{\mathrm{Cov}(X,Y)}{\sqrt{DX\,DY}}.
$$
\begin{itemize}
      \item $|\rho_{XY}|\le1$;
      \item $\rho=0$:不相关;
      \item $\rho=\pm1$:完全线性相关;
      \item $\rho=1\Rightarrow Y=aX+b,a>0$;$\rho=-1\Rightarrow Y=aX+b,a<0$。
\end{itemize}

\subsection*{3. 常用性质}
\begin{enumerate}
      \item $\mathrm{Cov}(X,Y)=\mathrm{Cov}(Y,X)$;
      \item $\mathrm{Cov}(aX,bY)=ab\,\mathrm{Cov}(X,Y)$;
      \item $\mathrm{Cov}(X_1+X_2,Y)=\mathrm{Cov}(X_1,Y)+\mathrm{Cov}(X_2,Y)$;
      \item 若 $X,Y$ 独立,则 $\mathrm{Cov}(X,Y)=0$;
      \item $(X,Y)\sim N(\mu_1,\mu_2;\sigma_1^2,\sigma_2^2;\rho)$ 时,
            独立 $\Leftrightarrow$ 不相关。
\end{enumerate}


\section{独立性与不相关性}

\subsection*{1. 独立性判定}
\begin{itemize}
      \item 连续型:$f(x,y)=f_X(x)f_Y(y)$;
      \item 离散型:$p_{ij}=p_i p_j$。
\end{itemize}

\subsection*{2. 不相关性判定}
$$\rho_{XY}=0\Leftrightarrow \mathrm{Cov}(X,Y)=0\Leftrightarrow E(XY)=EXEY.$$

\subsection*{3. 关系总结}
\begin{itemize}
      \item 独立 $\Rightarrow$ 不相关;
      \item 不相关 $\centernot\Rightarrow$ 独立;
      \item 二维正态分布中:独立 $\Leftrightarrow$ 不相关;
      \item 若 $X,Y$ 为 $0$-$1$ 分布:独立 $\Leftrightarrow$ 不相关。
\end{itemize}


\section{切比雪夫不等式}
若随机变量 $X$ 的数学期望与方差存在,则对任意 $\varepsilon>0$:
$$
      P\{|X-EX|\ge\varepsilon\}\le\frac{DX}{\varepsilon^2},\quad
      P\{|X-EX|<\varepsilon\}\ge1-\frac{DX}{\varepsilon^2}.
$$

\textbf{常考理解:}
方差越小,$X$ 越集中于期望附近;
切比雪夫不等式用于估计偏离概率的上界。


\section*{小结}
\begin{itemize}
      \item \textbf{计算:} 掌握 $E[g(X,Y)]$ 与 $D(aX+bY)$。
      \item \textbf{判断:} 独立看分布,相关看 $\mathrm{Cov}$。
      \item \textbf{记忆:}
            $$\mathrm{Cov}(X,Y)=E(XY)-EXEY,\quad D(X\pm Y)=DX+DY\pm2\mathrm{Cov}(X,Y).$$
      \item \textbf{技巧:} 用标准化变量 $U=\dfrac{X-EX}{\sqrt{DX}}$ 可快速处理。
\end{itemize}