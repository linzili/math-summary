\chapter{统计量及其分布}
\section{统计量及其数字特征}
设 $X_{1}, X_{2}, \cdots, X_{n}$ 是来自总体 $X$ 的简单随机样本,则
\begin{enumerate}
    \item 样本均值 $\bar{X}=\frac{1}{n} \sum_{i=1}^{n} X_{i}$.
    \item 样本方差 $S^{2}=\frac{1}{n-1} \sum_{i=1}^{n}\left(X_{i}-\bar{X}\right)^{2}=\frac{1}{n-1}\left(\sum_{i=1}^{n} X_{i}^{2}-n \bar{X}^{2}\right)$.

          样本标准差 $S=\sqrt{\frac{1}{n-1} \sum_{i=1}^{n}\left(X_{i}-\bar{X}\right)^{2}}$.
    \item 样本 $k$ 阶原点矩 $A_{k}=\frac{1}{n} \sum_{i=1}^{n} X_{i}^{k}(k=1,2, \cdots)$.
    \item 样本 $k$ 阶中心矩 $B_{k}=\frac{1}{n} \sum_{i=1}^{n}\left(X_{i}-\bar{X}\right)^{k}(k=2,3, \cdots)$.
    \item 顺序统计量

          将样本 $X_{1}, X_{2}, \cdots, X_{n}$ 的 $n$ 个观测量按其取值从小到大的顺序排列,得
          $$X_{(1)} \leqslant X_{(2)} \leqslant \cdots \leqslant X_{(n)}.$$
          随机变量 $X_{(k)}(k=1,2, \cdots, n)$ 称作第 $k$ 顺序统计量,其中 $X_{(1)}$ 是最小观测量, $X_{(n)}$ 是最大观测量,即
          $$X_{(1)}=\min \left\{X_{1}, X_{2}, \cdots, X_{n}\right\}, \quad X_{(n)}=\max \left\{X_{1}, X_{2}, \cdots, X_{n}\right\}.$$
\end{enumerate}

\section{判别统计量的分布}
定义:统计量的分布称为抽样分布

\begin{enumerate}
    \item 正态分布
          \begin{enumerate}
              \item 概念

                    如果 $X$ 的概率密度为
                    $$f(x) = \frac{1}{\sqrt{2 \pi} \sigma} \mathrm{e}^{-\frac{1}{2} \left( \frac{x - \mu}{\sigma} \right)^2} \quad (-\infty < x < +\infty),$$
                    其中 $-\infty < \mu < +\infty$, $\sigma > 0$, 则称 $X$ 服从参数为 $(\mu, \sigma^2)$ 的正态分布或称 $X$ 为正态变量, 记为 $X \sim N(\mu, \sigma^2)$.
              \item 上$\alpha$分位数

                    若 $X \sim N(0, 1)$, $P\{X > \mu_\alpha\} = \alpha$ ( $0 < \alpha < 1$ ), 则称 $\mu_\alpha$ 为标准正态分布的上 $\alpha$ 分位数
              \item 性质

                    $f(x)$ 的图形关于直线 $x=\mu$对称,即$f(\mu-x)=f(\mu+x)$,并在$x=\mu$处有唯一最大值
                    $$f(\mu)=\frac{1}{\sqrt{2\pi}\sigma}.$$
                    通常称$\mu=0$ , $\sigma=1$时的正态分布$N(0,1)$为标准正态分布,记标准正态分布的概率密度为
                    $\varphi(x)=\frac{1}{\sqrt{2\pi}}\mathrm{e}^{-\frac{1}{2}x^{2}}$,分布函数为$Q(x)=\frac1{\sqrt{2\pi}}\int_{-\infty}^{x}\mathrm{e}^{-\frac{t^{2}}{2}}dt$ .显然$\varphi(x)$为偶函数,且有
                    $$\Phi(0)=\frac{1}{2},\Phi(-x)=1-\Phi(x).$$
          \end{enumerate}
    \item $\chi^2$分布
          \begin{enumerate}
              \item 概念

                    若随机变量 $X_{1}, X_{2}, \cdots, X_{n}$ 相互独立,且都服从标准正态分布,则随机变量 $X = \sum_{i=1}^{n} X_{i}^{2}$ 服从自由度为 $n$ 的 $\chi^{2}$ 分布,记为 $X \sim \chi^{2}(n)$.
              \item 上$\alpha$分位数

                    对给定的 $\alpha (0 < \alpha < 1)$,称满足
                    $$P\{\chi^{2} > \chi_{\alpha}^{2}(n)\} = \int_{\chi_{\alpha}^{2}(n)}^{+\infty} f(x) \, \mathrm{d}x = \alpha$$
                    的 $\chi_{\alpha}^{2}(n)$ 为 $\chi^{2}(n)$ 分布的上 $\alpha$ 分位数(见图). 对于不同的 $\alpha, n$,$\chi^{2}(n)$ 分布上 $\alpha$ 分位数可通过查表求得.
              \item 性质
                    \begin{enumerate}
                        \item  若 $X_{1} \sim \chi^{2}(n_{1})$,$X_{2} \sim \chi^{2}(n_{2})$,$X_{1}$ 与 $X_{2}$ 相互独立,则
                              $$
                                  X_{1} + X_{2} \sim \chi^{2}(n_{1} + n_{2}).$$
                              此结论可推广至有限多个随机变量的和.
                        \item $若X\sim\chi^{2}(n)$,则$EX=n,DX=2n.$
                    \end{enumerate}
          \end{enumerate}
    \item $t$ 分布
          \begin{enumerate}
              \item 概念
                    设随机变量 $X \sim N(0,1)$, $Y \sim \chi^2(n)$, $X$ 与 $Y$ 相互独立, 则随机变量 $t = \frac{X}{\sqrt{Y/n}}$ 服从自由度为 $n$ 的 $t$ 分布, 记为 $t \sim t(n)$.
              \item 上$\alpha$分位数

                    对给定的 $\alpha(0 < \alpha < 1)$, 称满足
                    $$P\{t > t_{\alpha}(n)\} = \alpha$$
                    的 $t_{\alpha}(n)$ 为 $t(n)$ 分布的上 $\alpha$ 分位数.
              \item 性质
                    \begin{enumerate}
                        \item  $t$ 分布概率密度 $f(x)$ 的图形关于 $x = 0$ 对称, 因此
                              $$Et = 0 \quad (n \geqslant 2).$$
                        \item 由 $t$ 分布概率密度 $f(x)$ 图形的对称性, 知 $P\{t > -t_{\alpha}(n)\} = P\{t > t_{1-\alpha}(n)\}$, 故 $t_{1-\alpha}(n) = -t_{\alpha}(n)$. 当 $\alpha$ 值在表中没有时, 可用此式求得上 $\alpha$ 分位数.
                    \end{enumerate}
          \end{enumerate}
    \item $F$ 分布
          \begin{enumerate}
              \item 概念

                    设随机变量 $X \sim \chi^2(n_1)$,$Y \sim \chi^2(n_2)$,且 $X$ 与 $Y$ 相互独立,则 $F = \frac{X / n_1}{Y / n_2}$ 服从自由度为 $(n_1, n_2)$ 的 $F$ 分布,记为 $F \sim F(n_1, n_2)$,其中 $n_1$ 称为第一自由度,$n_2$ 称为第二自由度.$F$ 分布的概率密度 $f(x)$ 的图形.
              \item 上$\alpha$分位数

                    对给定的 $\alpha (0 < \alpha < 1)$,称满足
                    $$P\{F > F_\alpha(n_1, n_2)\} = \alpha$$
                    的 $F_\alpha(n_1, n_2)$ 为 $F(n_1, n_2)$ 分布的上 $\alpha$ 分位数.
              \item 性质
                    \begin{enumerate}
                        \item 若 $F \sim F(n_1, n_2)$,则 $\frac{1}{F} \sim F(n_2, n_1)$.
                        \item $F_{1-\alpha}(n_1, n_2) = \frac{1}{F_\alpha(n_2, n_1)}$.常用来求 $F$ 分布表中未列出的上 $\alpha$ 分位数,显然,有些特殊值可直接得出,如 $1-\alpha = \alpha$,$n_1 = n_2 = n$ 时,有 $F_{0.5}(n, n) = \frac{1}{F_{0.5}(n, n)}$,且 $F_{0.5}(n, n) > 0$,故 $F_{0.5}(n, n) = 1$.
                        \item 若 $t \sim t(n)$,则 $t^2 \sim F(1, n)$.
                    \end{enumerate}
          \end{enumerate}
\end{enumerate}
\section{用正态总体下的常用结论判别分布、计算概率}
设 $X_{1}, X_{2}, \cdots, X_{n}$ 是取自正态总体 $N(\mu, \sigma^{2})$ 的一个样本, $\bar{X}$ , $S^{2}$ 分别是样本均值和样本方差,则

\begin{enumerate}
    \item $\bar{X} \sim N\left(\mu, \frac{\sigma^{2}}{n}\right)$ ,即 $\frac{\bar{X}-\mu}{\frac{\sigma}{\sqrt{n}}} = \frac{\sqrt{n}(\bar{X}-\mu)}{\sigma} \sim N(0,1)$
    \item $\frac{1}{\sigma^{2}} \sum_{i=1}^{n}(X_{i}-\mu)^{2} \sim \chi^{2}(n)$ ;
    \item $\frac{(n-1)S^{2}}{\sigma^{2}} = \sum_{i=1}^{n}\left(\frac{X_{i}-\bar{X}}{\sigma}\right)^{2} \sim \chi^{2}(n-1)$ ( $\mu$ 未知,在 “2.” 中用 $\bar{X}$ 替代 $\mu$ );
    \item $\bar{X}$ 与 $S^{2}$ 相互独立, $\frac{\sqrt{n}(\bar{X}-\mu)}{S} \sim t(n-1)$ ( $\sigma$ 未知,在 “1.” 中用 $S$ 替代 $\sigma$ ). 进一步有
          $$\frac{n(\bar{X}-\mu)^{2}}{S^{2}} \sim F(1, n-1).$$
\end{enumerate}

