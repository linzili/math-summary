\GAIchapter{统计量及其分布}

\textbf{考点综述:}
掌握常见统计量及其分布($N$、$\chi^2$、$t$、$F$ 分布)的定义、性质与相互关系;
会判别统计量的分布类型并据此计算概率。


\section{统计量及其数字特征}

设 $X_1, X_2, \cdots, X_n$ 是来自总体 $X$ 的简单随机样本。

\begin{enumerate}
      \item 样本均值:\quad $\bar{X} = \dfrac{1}{n} \sum_{i=1}^{n} X_i.$
      \item 样本方差:
            $$
                  S^2 = \frac{1}{n-1}\sum_{i=1}^{n}(X_i - \bar{X})^2
                  = \frac{1}{n-1}\left(\sum_{i=1}^{n} X_i^2 - n\bar{X}^2\right),
                  \quad S = \sqrt{S^2}.
            $$
      \item 样本 $k$ 阶原点矩:\quad $A_k = \dfrac{1}{n} \sum_{i=1}^{n} X_i^k.$
      \item 样本 $k$ 阶中心矩:\quad $B_k = \dfrac{1}{n} \sum_{i=1}^{n}(X_i - \bar{X})^k.$
      \item 顺序统计量:\quad
            将样本按大小排列:
            $$
                  X_{(1)} \le X_{(2)} \le \cdots \le X_{(n)},
            $$
            称 $X_{(k)}$ 为第 $k$ 顺序统计量。
            特别地,
            $$
                  X_{(1)} = \min\{X_1, \cdots, X_n\}, \quad
                  X_{(n)} = \max\{X_1, \cdots, X_n\}.
            $$
\end{enumerate}


\section{统计量的抽样分布}

\textbf{定义:}
统计量的分布称为\textbf{抽样分布}。


\subsection*{1. 正态分布 $N(\mu,\sigma^2)$}

\begin{enumerate}
      \item 概念:
            若随机变量 $X$ 的密度为
            $$
                  f(x) = \frac{1}{\sqrt{2\pi}\sigma}e^{-\frac{1}{2}\left(\frac{x-\mu}{\sigma}\right)^2},
            $$
            则称 $X \sim N(\mu,\sigma^2)$。
      \item 上 $\alpha$ 分位数:
            若 $X\sim N(0,1)$,有 $P\{X > \mu_\alpha\} = \alpha$,则 $\mu_\alpha$ 称为上 $\alpha$ 分位数。
      \item 性质:
            \begin{itemize}
                  \item 关于 $x=\mu$ 对称,峰值 $f(\mu)=\dfrac{1}{\sqrt{2\pi}\sigma}$;
                  \item 标准正态分布:$\varphi(x)=\dfrac{1}{\sqrt{2\pi}}e^{-x^2/2}$;
                  \item 分布函数 $\Phi(x)=\dfrac{1}{\sqrt{2\pi}}\int_{-\infty}^{x} e^{-t^2/2}dt$,
                        且 $\Phi(0)=\frac{1}{2},\ \Phi(-x)=1-\Phi(x)$。
            \end{itemize}
\end{enumerate}


\subsection*{2. $\chi^2$ 分布}

\begin{enumerate}
      \item 概念:
            若 $X_i \sim N(0,1)$ 且相互独立,则
            $$
                  X = \sum_{i=1}^{n} X_i^2 \sim \chi^2(n),
            $$
            称自由度为 $n$。
      \item 上 $\alpha$ 分位数:
            $$P\{\chi^2 > \chi^2_\alpha(n)\} = \alpha.$$
      \item 性质:
            \begin{itemize}
                  \item 若 $X_1\sim\chi^2(n_1)$, $X_2\sim\chi^2(n_2)$ 独立,则 $X_1+X_2\sim\chi^2(n_1+n_2)$;
                  \item $E X = n,\quad D X = 2n$。
            \end{itemize}
\end{enumerate}


\subsection*{3. $t$ 分布}

\begin{enumerate}
      \item 概念:
            若 $X\sim N(0,1)$, $Y\sim\chi^2(n)$ 且独立,则
            $$t=\frac{X}{\sqrt{Y/n}}\sim t(n).$$
      \item 上 $\alpha$ 分位数:
            $$P\{t>t_\alpha(n)\}=\alpha.$$
      \item 性质:
            \begin{itemize}
                  \item 关于 0 对称,$E t = 0$;
                  \item $t_{1-\alpha}(n) = -t_\alpha(n)$;
                  \item 当自由度 $n\to\infty$ 时,$t(n)\to N(0,1)$。
            \end{itemize}
\end{enumerate}


\subsection*{4. $F$ 分布}

\begin{enumerate}
      \item 概念:
            若 $X\sim\chi^2(n_1)$,$Y\sim\chi^2(n_2)$ 且独立,则
            $$
                  F=\frac{X/n_1}{Y/n_2}\sim F(n_1,n_2).
            $$
      \item 上 $\alpha$ 分位数:
            $$P\{F>F_\alpha(n_1,n_2)\}=\alpha.$$
      \item 性质:
            \begin{itemize}
                  \item $\dfrac{1}{F}\sim F(n_2,n_1)$;
                  \item $F_{1-\alpha}(n_1,n_2)=\dfrac{1}{F_\alpha(n_2,n_1)}$;
                  \item 若 $t\sim t(n)$,则 $t^2\sim F(1,n)$。
            \end{itemize}
\end{enumerate}


\section{正态总体下的常用结论}

设 $X_1,\cdots,X_n$ 来自总体 $N(\mu,\sigma^2)$,
样本均值 $\bar{X}$,样本方差 $S^2$。

\begin{enumerate}
      \item $\bar{X}\sim N\!\left(\mu,\dfrac{\sigma^2}{n}\right)$,
            即
            $$\frac{\sqrt{n}(\bar{X}-\mu)}{\sigma}\sim N(0,1).$$
      \item $\dfrac{1}{\sigma^2}\sum_{i=1}^{n}(X_i-\mu)^2 \sim \chi^2(n).$
      \item $\dfrac{(n-1)S^2}{\sigma^2}=\sum_{i=1}^{n}\left(\dfrac{X_i-\bar{X}}{\sigma}\right)^2 \sim \chi^2(n-1).$
      \item $\bar{X}$ 与 $S^2$ 相互独立,且
            $$\frac{\sqrt{n}(\bar{X}-\mu)}{S}\sim t(n-1),\quad \frac{n(\bar{X}-\mu)^2}{S^2}\sim F(1,n-1).$$
\end{enumerate}


\section*{小结:四种分布间的关系}
\[
      \boxed{
            \begin{aligned}
                  X_i                                & \sim N(0,1) \Rightarrow \sum X_i^2 \sim \chi^2(n); \\[3pt]
                  X\sim N(0,1),~Y\sim\chi^2(n)       & \Rightarrow \frac{X}{\sqrt{Y/n}}\sim t(n);         \\[3pt]
                  X\sim\chi^2(n_1),~Y\sim\chi^2(n_2) & \Rightarrow \frac{X/n_1}{Y/n_2}\sim F(n_1,n_2);    \\[3pt]
                  t^2(n)                             & \sim F(1,n).
            \end{aligned}
      }
\]