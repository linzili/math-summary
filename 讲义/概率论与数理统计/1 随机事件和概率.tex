\GAIchapter{随机事件和概率}

\section{用对立思想求事件概率}

\begin{enumerate}
    \item 对立关系(长杠变短杠,开口换方向):
          \[
              \overline{A\cup B}=\overline{A}\cap\overline{B},\qquad
              \overline{AB}=\overline{A}\cup\overline{B}.
          \]
    \item 概率对立关系:
          \[
              P(A)=1-P(\bar A).
          \]
\end{enumerate}

\section{用互斥思想求事件概率}

\begin{enumerate}
    \item 基本分拆(互斥化):
          \[
              A\cup B = A\cup(\overline{A}B)=B\cup(A\overline{B})
              =\overline{A}\,\overline{B} \cup AB \cup \overline{A}B.
          \]

    \item 若 $B_1,B_2,B_3$ 为完备事件组,则
          \[
              A = AB_1 \cup AB_2 \cup AB_3.
          \]

    \item 去交求差:
          \[
              P(A\overline{B}) = P(A-B) = P(A)-P(AB).
          \]

    \item 二事件加法公式:
          \[
              P(A+B) = P(A)+P(B)-P(AB).
          \]

    \item 三事件加法公式:
          \[
              P(A+B+C)=P(A)+P(B)+P(C)
              -P(AB)-P(AC)-P(BC) + P(ABC).
          \]

    \item 若 $A_1,A_2,\dots,A_n$ 两两互斥,则:
          \[
              P\Big(\bigcup_{i=1}^n A_i\Big)=\sum_{i=1}^n P(A_i).
          \]
\end{enumerate}

\section{用条件思想求事件概率}

\begin{enumerate}
    \item 条件概率定义($P(B)>0$):
          \[
              P(A|B)=\frac{P(AB)}{P(B)}.
          \]

    \item 交集概率的多种表达:
          \[
              \begin{aligned}
                  P(AB) & = P(B)P(A|B) \qquad (P(B)>0), \\
                        & = P(A)P(B|A) \qquad (P(A)>0), \\
                        & = P(A)+P(B)-P(A+B),           \\
                        & = P(A)-P(A\bar B).
              \end{aligned}
          \]

    \item 全概率公式(完备事件组 $A_1,\dots,A_n$, $P(A_i)>0$):
          \[
              P(B)=\sum_{i=1}^n P(A_i)P(B|A_i).
          \]

    \item 贝叶斯公式(执果索因):
          \[
              P(A_j|B)
              = \frac{P(A_jB)}{P(B)}
              = \frac{P(A_j)P(B|A_j)}
              {\sum_{i=1}^n P(A_i)P(B|A_i)},
              \quad j=1,2,\dots,n.
          \]
\end{enumerate}
\section{用单调性处理事件概率}

\begin{enumerate}
    \item $0 \le P(A) \le 1$。
    \item 若 $A \subseteq B$,则 $P(A) \le P(B)$。
    \item 由于 $AB \subseteq A \subseteq A+B$,故
          \[
              P(AB) \le P(A) \le P(A+B).
          \]
\end{enumerate}


\section{用最值关系式处理事件概率}

\begin{enumerate}
    \item $\{\max(X,Y)\le a\} = \{X\le a\}\cap\{Y\le a\}$;
    \item $\{\max(X,Y)>a\} = \{X>a\}\cup\{Y>a\}$;
    \item $\{\min(X,Y)\le a\} = \{X\le a\}\cup\{Y\le a\}$;
    \item $\{\min(X,Y)>a\} = \{X>a\}\cap\{Y>a\}$;
    \item $\{\max(X,Y)\le a\} \subseteq \{\min(X,Y)\le a\}$;
    \item $\{\min(X,Y)>a\} \subseteq \{\max(X,Y)>a\}$;
    \item $\max(X,Y)=\dfrac{X+Y+|X-Y|}{2}$;
    \item $\min(X,Y)=\dfrac{X+Y-|X-Y|}{2}$;
    \item $\max(X,Y)+\min(X,Y)=X+Y$;
    \item $\max(X,Y)-\min(X,Y)=|X-Y|$;
    \item $\max(X,Y)\cdot \min(X,Y)=XY$。
\end{enumerate}


\section{用独立、有利或抑止处理事件概率}

\subsection*{(1)定义}
设事件 $A,B$:
\begin{enumerate}
    \item 若 $P(AB)=P(A)P(B)$,称 $A,B$ \textbf{相互独立};
    \item 若 $P(AB)>P(A)P(B)$,称 $A,B$ \textbf{相互有利};
    \item 若 $P(AB)<P(A)P(B)$,称 $A,B$ \textbf{相互抑止}。
\end{enumerate}

\subsection*{(2)重要结论}

\begin{enumerate}
    \item 若 $A_1,A_2,\dots,A_n$ 相互独立,则
          \[
              P(A_1A_2\cdots A_n)=\prod_{i=1}^n P(A_i).
          \]

    \item 若 $A_1,\dots,A_n$ 相互独立,则
          \[
              P\Big(\bigcup_{i=1}^n A_i\Big)
              =1-P\Big(\bigcap_{i=1}^n \overline{A_i}\Big)
              =1-\prod_{i=1}^n P(\overline{A_i})
              =1-\prod_{i=1}^n (1-P(A_i)).
          \]

    \item 下列四个等价:
          \[
              A\perp B
              \;\Longleftrightarrow\;
              A\perp\overline{B}
              \;\Longleftrightarrow\;
              \overline{A}\perp B
              \;\Longleftrightarrow\;
              \overline{A}\perp\overline{B}.
          \]

    \item 下列四个等价:
          \[
              A\text{ 与 } B \text{有利}
              \Longleftrightarrow
              A \text{ 与 }\overline{B}\text{抑止}
              \Longleftrightarrow
              \overline{A}\text{ 与 }B\text{抑止}
              \Longleftrightarrow
              \overline{A}\text{ 与 }\overline{B}\text{有利}.
          \]

    \item 若 $A,B,C,D$ 相互独立,则
          \[
              AB \perp CD,\qquad A \perp (BC-D),
          \]
          即独立事件组运算后仍保持独立。

    \item 若 $P(A)=0$ 或 $P(A)=1$,则 $A$ 与任意事件 $B$ 独立。

    \item 若 $0<P(A)<1$ 且 $A,B$ \textbf{互斥} 或 \textbf{存在包含关系},则 $A,B$ \textbf{必不独立}。
\end{enumerate}