\GAIchapter{大数定律与中心极限定理}

\textbf{考点综述:}
掌握随机变量的依概率收敛、三种大数定律及中心极限定理的内容与应用;
会判断收敛类型、求极限值或计算概率。


\section{依概率收敛(Convergence in Probability)}

\textbf{定义:}
设随机变量 $X$ 与序列 $\{X_n\}$,若对任意 $\varepsilon>0$:
$$
      \lim_{n\to\infty}P(|X_n-X|\ge\varepsilon)=0
      \quad\text{或}\quad
      \lim_{n\to\infty}P(|X_n-X|<\varepsilon)=1,
$$
则称 $\{X_n\}$ 依概率收敛于 $X$,记为
$$
      X_n\xrightarrow{P}X\quad(n\to\infty).
$$

\textbf{理解:} 随着 $n$ 增大,$X_n$ 偏离 $X$ 的概率趋于 0。


\section{大数定律(Law of Large Numbers)}

\textbf{核心思想:}
大量独立随机变量的平均值趋向其期望值。

\subsection*{1. 切比雪夫大数定律}
若 $\{X_i\}$ 相互独立,且 $DX_i$ 存在并一致有上界(即 $\exists C>0$,使 $DX_i\le C$),则:
$$
      \frac1n\sum_{i=1}^{n}X_i\xrightarrow{P}\frac1n\sum_{i=1}^{n}EX_i.
$$

\subsection*{2. 伯努利大数定律}
设 $\mu_n$ 为 $n$ 次伯努利试验中事件 $A$ 发生次数,每次成功概率为 $p$,则:
$$
      \frac{\mu_n}{n}\xrightarrow{P}p,
$$
即
$$
      \forall \varepsilon>0,\quad
      \lim_{n\to\infty}P\left(\Big|\frac{\mu_n}{n}-p\Big|<\varepsilon\right)=1.
$$

\subsection*{3. 辛钦大数定律}
若 $\{X_i\}$ 为独立同分布序列,且 $E X_i=\mu$ 存在,则:
$$
      \frac1n\sum_{i=1}^{n}X_i\xrightarrow{P}\mu.
$$

\subsection*{4. 统一结论}
在适当条件下,大数定律的结论形式一致:
$$
      \frac1n\sum_{i=1}^{n}X_i\xrightarrow{P}E\!\left(\frac1n\sum_{i=1}^{n}X_i\right).
$$

\textbf{记忆口诀:}
“方差有界用切比,伯努利事件趋概率,同分布者用辛钦。”


\section{中心极限定理(Central Limit Theorem)}

\textbf{核心思想:}
大量独立同分布随机变量的标准化和,近似服从标准正态分布。

\subsection*{1. 列维–林德伯格定理(一般形式)}
若 $\{X_i\}$ 独立同分布,且 $E X_i=\mu, D X_i=\sigma^2>0$,则对任意实数 $x$:
$$
      \lim_{n\to\infty}P\!\left(\frac{\sum_{i=1}^{n}X_i-n\mu}{\sqrt{n}\sigma}\le x\right)
      =\Phi(x)=\frac{1}{\sqrt{2\pi}}\int_{-\infty}^{x}e^{-t^2/2}\,dt.
$$

\subsection*{2. 棣莫弗–拉普拉斯定理(伯努利情形)}
若 $Y_n\sim B(n,p)$,则对任意实数 $x$:
$$
      \lim_{n\to\infty}P\!\left(\frac{Y_n-np}{\sqrt{np(1-p)}}\le x\right)
      =\Phi(x).
$$
\textbf{说明:} 二项分布在 $n$ 大时近似 $N(np,np(1-p))$。

\subsection*{3. 一般结论}
若 $X_i$ 独立同分布于某分布,$E X_i=\mu$,$D X_i=\sigma^2$,则当 $n\to\infty$:
$$
      \sum_{i=1}^{n}X_i\sim N(n\mu,n\sigma^2),
      \qquad
      \frac{\sum_{i=1}^{n}X_i-n\mu}{\sqrt{n}\sigma}\sim N(0,1),
$$
即
$$
      \lim_{n\to\infty}P\!\left(\frac{\sum_{i=1}^{n}X_i-n\mu}{\sqrt{n}\sigma}\le x\right)=\Phi(x).
$$

\textbf{记忆口诀:}
“同分布加独立,标准化后正态起。”


\section*{小结}
\begin{itemize}
      \item \textbf{依概率收敛:} $X_n\xrightarrow{P}X \Rightarrow P(|X_n-X|\ge\varepsilon)\to0$。
      \item \textbf{大数定律:} 平均值趋向期望值。
      \item \textbf{中心极限定理:} 和的标准化近似正态分布。
      \item \textbf{常见应用:} 二项分布$\to$正态分布近似求概率。
\end{itemize}