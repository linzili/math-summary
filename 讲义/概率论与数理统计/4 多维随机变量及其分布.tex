\chapter{多维随机变量及其分布}

\section{离散型问题}
一般不会考

\section{连续型问题}
\begin{enumerate}
    \item 二维均匀分布

          如果$(X,Y)$的概率密度为
          $$f(x,y)=\begin{cases}\dfrac{1}{S_D},&(x,y)\in D,\\0,&\text{其他,}\end{cases}$$
          其中$S_{_{D}}$为区域$D$ 的面积,则称$(X,Y)$在平面有界区域$D$ 上服从均匀分布.

    \item 二维正态分布

          如果(X,Y)的概率密度为
          $$f(x,y)=\frac{1}{2\pi\sigma_1\sigma_2\sqrt{1-\rho^2}}\exp\left\{-\frac{1}{2(1-\rho^2)}\left[\left(\frac{x-\mu_1}{\sigma_1}\right)^2-2\rho\left(\frac{x-\mu_1}{\sigma_1}\right)\left(\frac{y-\mu_2}{\sigma_2}\right)+\left(\frac{y-\mu_2}{\sigma_2}\right)^2\right]\right\},$$

          其中$\mu_1\in R$,$\mu_2\in R$,$\sigma_1>0$,$\sigma_2>0$,$-1<\rho<1$,则称(X,Y)服从参数为$\mu_1$,$\mu_2$,$\sigma_1^2$,$\sigma_2^2$,$\rho$的二维正态分布,记为$(X,Y)\sim N(\mu_1,\mu_2;\sigma_1^2,\sigma_2^2;\rho)$。

\end{enumerate}
\section{求边缘分布、条件分布与独立性问题}
\begin{enumerate}
    \item 边缘分布
          \begin{enumerate}
              \item 求$F_X(x),F_Y(y)$.
                    $$F_{X}(x)=F(x,+\infty) ,\quad F_{Y}(y)=F(+\infty,y).$$
              \item 求$p_{i\cdot},p_{\cdot j}$.
                    $$p_{i\cdot}=\sum_{j}p_{ij},\quad p_{\cdot j}=\sum_{i}p_{ij}.$$
              \item 求$f_X(x),f_Y(y)$.
                    $$f_{X}(x)=\int_{-\infty}^{+\infty}f(x,y)\mathrm{d}y=\int_{-\infty}^{+\infty}f_{Y}(y)f_{X|Y}(x\mid y)\mathrm{d}y,$$
                    $$f_{Y}(y)=\int_{-\infty}^{+\infty}f(x,y)\mathrm{d}x=\int_{-\infty}^{+\infty}f_{X}(x)f_{Y|X}(y\mid x)\mathrm{d}x.$$
          \end{enumerate}
    \item 条件分布
          \begin{enumerate}
              \item 求 $F(x\mid y_j),F(y\mid x_i).$
                    $$F(x\mid y_{j})=\sum_{x_{i}\leq x}P\{X=x_{i}\mid Y=y_{j}\},$$
                    $$F(y\mid x_{i})=\sum_{y_{j}\leq y}P\{Y=y_{j}\mid X=x_{i}\}.$$
              \item 求 $F(x\mid y),F(y\mid x).$
                    $$F(x\mid y)=\int_{-\infty}^{x}f(u\mid y)\mathrm{d}u=\int_{-\infty}^{x}\frac{f(u,y)}{f_{Y}(y)}\mathrm{d}u,$$
                    $$F(y\mid x)=\int_{-\infty}^{y}f(v\mid x)\mathrm{d}v=\int_{-\infty}^{y}\frac{f(x,v)}{f_{X}(x)}\mathrm{d}v.$$
              \item 求$P\{ Y= y_{j}\mid X= x_{i}\}$ , $P\{ X= x_{i}\mid Y= y_{j}\}$ .
                    $$P\{Y=y_{j}\mid X=x_{i}\}=\frac{P\{X=x_{i},Y=y_{j}\}}{P\{X=x_{i}\}}=\frac{p_{ij}}{p_{i}.}\:,$$
                    $$P\{X=x_{i}\mid Y=y_{j}\}=\frac{P\Big\{X=x_{i},Y=y_{j}\Big\}}{P\Big\{Y=y_{j}\Big\}}=\frac{p_{ij}}{p_{\cdot j}}\:.$$
              \item 求$f_{Y\mid X}(y\mid x),f_{X\mid Y}(x\mid y).$
                    $$f_{Y\mid X}(y\mid x)=\frac{f(x,y)}{f_{X}(x)},\quad f_{X|Y}(x\mid y)=\frac{f(x,y)}{f_{Y}(y)}$$
          \end{enumerate}
    \item 判独立
          \begin{enumerate}
              \item $X$与$Y$相互独立 $\Leftrightarrow$ 对任意$x,y$,$F(x,y)=F_X(x)\cdot F_Y(y)$.

                    $X,Y$不独立 $\Leftrightarrow$ 存在$x_0,y_0$,使$A=\{X\leqslant x_0\}$与$B=\{Y\leqslant y_0\}$不独立,即$F(x_0,y_0)\neq F_X(x_0)\cdot F_Y(y_0)$.

                    因此,证明不独立的常用方法:找$x_0,y_0$,使$0<P\{X\leqslant x_0\}$,$P\{Y\leqslant y_0\}<1$,$\{X\leqslant x_0\}\subseteq\{Y\leqslant y_0\}$或$\{Y\leqslant y_0\}\subseteq\{X\leqslant x_0\}$或$\{X\leqslant x_0,Y\leqslant y_0\}=\varnothing$.
              \item 若$(X,Y)$为二维离散型随机变量,$X$与$Y$相互独立 $\Leftrightarrow$ 对任意$i,j$,$p_{ij}=p_i\cdot p_j$.
              \item 若$(X,Y)$为二维连续型随机变量,$X$与$Y$相互独立 $\Leftrightarrow$ 对任意$x,y$,$f(x,y)=f_X(x)f_Y(y)$.
          \end{enumerate}
\end{enumerate}

\section{用分布求概率及反问题}
\begin{enumerate}
    \item $(X,Y)\sim p_{ij}$,则$P\{ ( X, Y) \in D\} = \sum_{( x_{i}, y_{j}) \in D}p_{ij}$.
    \item $(X,Y)\sim f(x,y)$,则$P\{(X,Y)\in D\}=\iint_{D}f(x,y)$d$x$d$y$ .
    \item $(X,Y)$为混合型,则用全概率公式.
    \item 反问题:已知概率反求参数.
\end{enumerate}