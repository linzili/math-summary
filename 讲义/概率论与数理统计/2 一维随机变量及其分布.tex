\chapter{一维随机变量及其分布}

\section{判分布}
\DOne + \DTwoTwo
\begin{enumerate}
    \item 判分布函数
          \begin{enumerate}
              \item 充要条件

                    $F(x)$ 是分布函数 $\Leftrightarrow F(x)$ 是 $x$ 的单调不减且右连续的函数,且 $F(-\infty)=0$,$F(+\infty)=1$.
              \item 分布函数形式大观.
                    \begin{enumerate}
                        \item 设 $F_i(x)$ 是分布函数,$\lambda_i>0$,$\sum_{i=1}^{n}\lambda_i=1$,则 $\sum_{i=1}^{n}\lambda_iF_i(x)$ 是分布函数.特别地,算术平均值 $\frac{F_1(x)+F_2(x)}{2}$ 是分布函数.
                        \item 设 $F(x)$ 是分布函数,则 $F(x)$ 在 $[x,x+T]$ $(T>0)$ 上的均值 $\frac{1}{T}\int_{x}^{x+T}F(t)dt$ 是分布函数.可见,线性组合 $\sum_{i=1}^{n}\lambda_iF_i(x)$ 及其连续形式均仍是分布函数.
                        \item 几何平均值 $\sqrt{F_1(x)F_2(x)}$ 是分布函数.
                        \item $[F(x)]^n$,$1-[1-F(x)]^n$ 是分布函数.
                    \end{enumerate}
          \end{enumerate}
    \item 判分布律的充要条件

          $\{p_i\}$ 是概率分布 $\Leftrightarrow p_i\geqslant 0$,且 $\sum_{i}p_i=1$.
    \item 判概率密度
          \begin{enumerate}
              \item 充要条件

                    $f(x)$ 是概率密度 $\Leftrightarrow f(x) \geqslant 0$,且 $\int_{-\infty}^{+\infty} f(x) \mathrm{d}x = 1$.
              \item 概率密度形式大观
                    \begin{enumerate}
                        \item 设 $f(x)$ 为概率密度,$\lambda_i > 0$,$\sum_{i=1}^{n} \lambda_i = 1$,则 $\sum_{i=1}^{n} \lambda_i f_i(x)$ 是概率密度.特别地,$\frac{1}{2}[f_1(x) + f_2(x)]$ 是概率密度.
                        \item 设 $f(x)$ 为概率密度,则 $f(x)$ 在 $[x, x+T]$($T > 0$)上的均值 $\frac{1}{T} \int_{x}^{x+T} f(t) \mathrm{d}t$ 是概率密度.
                        \item 设 $X_i$ 的分布函数为 $F_i(x)$,概率密度为 $f_i(x)$,则 $\frac{2}{n}\sum_{i=1}^{n}F_i(x)f_i(x)$ 是概率密度.
                        \item 设 $X_i$ 的分布函数为 $F_i(x)$,概率密度为 $f_i(x)$,则 $f_1(x)F_2(x)\cdots F_n(x) + F_1(x)f_2(x)\cdots F_n(x) + \cdots + F_1(x)F_2(x)\cdots f_n(x)$ 是概率密度.
                        \item 设 $F(x)$ 是分布函数,$f(x)$ 是对应的概率密度,则 $n[F(x)]^{n-1}f(x)$,$n[1-F(x)]^{n-1}f(x)$ 是概率密度.
                    \end{enumerate}
          \end{enumerate}
    \item 反问题
          $$\begin{cases}F(-\infty)=0,\\F(+\infty)=1,\\\sum_{i}p_{i}=1,\\\int_{-\infty}^{+\infty}f(x)\mathrm{d}x=1 \end{cases}$$
          建方程,求参数
\end{enumerate}
\section{用分布}
\DOne

\begin{enumerate}
    \item 离散型分布
          \begin{enumerate}
              \item 0-1分布.

                    $X \sim B(1,p)$, $X$ (伯努利计数变量) $\sim \begin{pmatrix} 1 & 0 \\ p & 1-p \end{pmatrix}$.
              \item 二项分布.

                    $X \sim B(n,p) \begin{cases} n\text{次试验相互独立;} \\ P(A) = p; \\ \text{只有}A, \overline{A}\text{两种结果}. \end{cases}$
                    记 $X$ 为 $A$ 发生的次数, 则$P\{X = k\} = C_n^k p^k (1-p)^{n-k}$, $k = 0, 1, 2, \cdots, n$,$EX = np$, $DX = np(1-p)$.

                    二项分布 $X \sim B(n,p)$ 还具有如下性质:
                    \begin{enumerate}
                        \item $Y = n - X$ 服从二项分布 $B(n,q)$, 其中 $q = 1 - p$.
                        \item 对固定的 $n$ 和 $p$,随着 $k$ 的增大,$P\{X=k\}$ 先上升到最大值而后下降
                              \begin{enumerate}
                                  \item 当 $(n+1)p$ 为整数时,$P\{X=(n+1)p\} = P\{X=(n+1)p-1\}$ 最大.
                                  \item 当 $(n+1)p$ 不为整数时,$P\{X=\left[(n+1)p\right]\}$ 最大,其中 $\left[(n+1)p\right]$ 表示 $(n+1)p$ 的整数部分.
                              \end{enumerate}
                    \end{enumerate}
              \item 离散型首次冲击分布(几何分布).

                    在伯努利试验序列中 $P(A)=p,P(\bar{A})=1-p$, 首次出现 $A$ 即停止(即首次冲击即失效).令$X$为试
                    验次数,则$P\{X=k\}=p(1-p)^{k-1},k=1,2,\cdots$,其中$P\left\{X=1\right\}$最大,且$EX=\frac1p,DX=\frac{1-p}{p^{2}}.$
              \item 超几何分布
                    $N$件产品中有$M$件正品, 从中无放回地随机抽取$n$件, 则取到$k$个正品的概率为

                    $P\{X=k\}=\frac{\mathrm{C}_{M}^{k} \mathrm{C}_{N-M}^{n-k}}{\mathrm{C}_{N}^{n}}$, $k$为整数, $\max \{0, n-N+M\} \leqslant k \leqslant \min \{n, M\}$, 且 $E X=\frac{n M}{N}$.
              \item 泊松分布

                    泊松分布是指某单位时间段, 某场合下, 源源不断的随机质点流的个数, 也常用于描述稀有事件的概率.

                    $$P\{X=k\}=\frac{\lambda^{k}}{k !} \mathrm{e}^{-\lambda}(k=0,1, \cdots ; \lambda>0),$$

                    $\lambda$ 表示强度 $(E X=\lambda)$, 且 $P\{X=[\lambda]\}$ 最大, 其中 $[\lambda]$ 表示对 $\lambda$ 取整.
          \end{enumerate}

    \item 连续型分布
          \begin{enumerate}
              \item 均匀分布$U(a,b)$.
                    如果随机变量X的概率密度和分布函数分别为
                    $$f(x)=\begin{cases}
                            \frac{1}{b-a}, & a<x<b,     \\
                            0,             & \text{其他},
                        \end{cases}$$
                    $$F(x)=\begin{cases}
                            0,               & x<a,            \\
                            \frac{x-a}{b-a}, & a\leqslant x<b, \\
                            1,               & x\geqslant b,
                        \end{cases}$$
                    则称X在区间(a,b)上服从均匀分布,记为$X\sim U(a,b)$.
              \item 连续型首次冲击分布(指数分布).

                    设随机质点流的计数过程为 $\{N_{t}\}(t\geqslant 0)$,$N_{t}$ 服从参数为 $\lambda t$ 的泊松分布.令 $T_{1}$ 表示第 1 个质点到来的时刻,则当 $t>0$ 时,令 $A=\{T_{1}>t\}$ 表示第 1 个质点在时刻 $t$ 之后到来,$B=\{N_{t}=0\}$ 表示在 $[0,t]$ 时间上有 0 个质点到来,即 $A$ 与 $B$ 是相等事件,故 $P(A)=P(B)$,即
                    $$P\{T_{1}>t\}=P\{N_{t}=0\}=\frac{(\lambda t)^{0}}{0!}e^{-\lambda t}=e^{-\lambda t},$$
                    于是
                    $$
                        F_{T_{1}}(t)=P\{T_{1}\leqslant t\}=1-e^{-\lambda t}, t>0,$$
                    即 $T_{1}$ 服从参数为 $\lambda$ 的指数分布。
                    如果 $X$ 的概率密度和分布函数分别为
                    $$f(x)=\begin{cases}
                            \lambda e^{-\lambda x}, & x\geqslant 0, (\lambda>0), \\
                            0,                      & \text{其他}
                        \end{cases}, F(x)=\begin{cases}
                            1-e^{-\lambda x}, & x\geqslant 0, (\lambda>0), \\
                            0,                & x<0
                        \end{cases},$$
                    则称 $X$ 服从参数为 $\lambda$ 的指数分布,记为 $X\sim E(\lambda)$.
                    \begin{note}{}{}
                        \begin{enumerate}
                            \item 当$t,s > 0$时, $P\{X\geq t+s|X\geq t\}=P\{X\geq s\}$称为指数分布的无记忆性.
                            \item $EX=\frac{1}{\lambda}$称为平均寿命, 也称为平均等待时间, $\lambda$称为失效频率, 它是一个常数, 只有失效频率不变, 元件无损耗, 才有无记忆性.
                            \item 特别地, 当λ=$\frac{1}{2}$, 即X~f(x)=$\begin{cases}\frac{1}{2}e^{-\frac{x}{2}}, & x≥0\\0, & x<0\end{cases}$时, 也称$X$服从自由度为$2$的$\chi^2$ 分布, 故$E(\frac{1}{2})$与$\chi^2(2)$ 是同一分布.
                            \item 若$X\sim E(1)$, 则$2X~E(\frac{1}{2})$, $2X\sim \chi^2(2)$ .
                            \item 若$X\sim E(\lambda)$, 则$2\lambda X~E(\frac{1}{2})$, $2\lambda X \sim\chi^2(2)$ .
                        \end{enumerate}
                    \end{note}
              \item 自由度为$1$的$t$分布(标准柯西分布).

                    若
                    $$X\sim f(x)=\frac{1}{\pi(1+x^{2})}\:,\:-\infty<x<+\infty\:,$$
                    则称$X$服从自由度为1的$t$分布(标准柯西分布),即$X\sim t(1)$这是用于描述受迫共振的一种分布.
              \item 正态分布

                    若$X\sim f(x)=\frac{1}{\sqrt{2\pi}\sigma}\mathrm{e}^{-\frac{(x-\mu)^2}{2\sigma^2}}$,$-\infty<x<+\infty$,其中$-\infty<\mu<+\infty$,$\sigma>0$,则称$X$服从参数为$(\mu,\sigma^2)$的正态分布,记为$X\sim N(\mu,\sigma^2)$.
                    \begin{note}{}{}
                        \begin{enumerate}
                            \item  $\mu=0$,$\sigma=1$时的正态分布为标准正态分布,记为$X\sim N(0,1)$.
                                  $$X\sim\varphi(x)=\frac{1}{\sqrt{2\pi}}\mathrm{e}^{-\frac{x^2}{2}}, \quad \Phi(x)=\int_{-\infty}^x\frac{1}{\sqrt{2\pi}}\mathrm{e}^{-\frac{t^2}{2}}\mathrm{d}t,$$
                                  且
                                  $$Y=\mid X\mid\sim f_{\gamma}(y)=\begin{cases}\dfrac{2}{\sqrt{2\pi}}\mathrm{e}^{-\frac{y^{2}}{2}},&y>0,\\0,&y\leqslant0\end{cases}=\begin{cases}2\varphi(y),&y>0,\\0,&y\leqslant0.\end{cases}$$
                            \item 计算公式与重要数据.

                                  若$X\sim N(0,1)$,则有
                                  $$
                                      \Phi(-x)=1-\Phi(x); \Phi(0)=\frac{1}{2};$$
                                  $$P\{|X|\leqslant a\}=2\Phi(a)-1(a>0).$$

                            \item 标准化.

                                  若$X\sim N(\mu,\sigma^2)$,则
                                  $$
                                      \frac{X-\mu}{\sigma}\sim N(0,1),$$
                                  且有
                                  $$F(x)=P\{X\leqslant x\}=\Phi\left(\frac{x-\mu}{\sigma}\right),$$
                                  $$
                                      P\{a\leqslant X\leqslant b\}=\Phi\left(\frac{b-\mu}{\sigma}\right)-\Phi\left(\frac{a-\mu}{\sigma}\right),$$
                                  $$P\{\mu-k\sigma\leqslant X\leqslant\mu+k\sigma\}=2\Phi(k)-1(k>0).$$
                            \item 含参数的概率密度的结构.

                                  设函数$f(x)=k\mathrm{e}^{-(ax^2+bx+c)}$, $x\in(-\infty,+\infty)(a>0)$,则
                                  $$
                                      ax^2+bx+c=a\left[\left(x+\frac{b}{2a}\right)^2+\frac{4ac-b^2}{4a^2}\right],$$
                                  且$k=\sqrt{\frac{a}{\pi}}\mathrm{e}^{\frac{4ac-b^2}{4a}}$,如$f(x)=k\mathrm{e}^{-\left(\frac{x^2}{4}+\frac{x}{2}+\frac{1}{4}\right)}$,则
                                  $$\frac{x^2}{4}+\frac{x}{2}+\frac{1}{4}=\frac{1}{4}\left[\left(x+\frac{1}{2}\right)^2+\frac{4\cdot\frac{1}{4}\cdot\frac{1}{4}-\left(\frac{1}{2}\right)^2}{4\cdot\left(\frac{1}{4}\right)^2}\right]$$
                                  $$
                                      =\frac{1}{4}(x+1)^2,$$
                                  且$k=\sqrt{\frac{1}{4\pi}}\mathrm{e}^0=\frac{1}{2\sqrt{\pi}}$.
                        \end{enumerate}
                    \end{note}
          \end{enumerate}
    \item 利用分布求概率及反问题
          \begin{enumerate}
              \item  $X \sim F(x)$,则
                    \begin{enumerate}
                        \item  $P\{X \leqslant a\} = F(a)$;
                        \item  $P\{X < a\} = F(a-0)$;
                        \item  $P\{X = a\} = P\{X \leqslant a\} - P\{X < a\} = F(a) - F(a-0)$;
                        \item  $P\{a < X < b\} = P\{X < b\} - P\{X \leqslant a\} = F(b-0) - F(a)$;
                        \item  $P\{a \leqslant X \leqslant b\} = P\{X \leqslant b\} - P\{X < a\} = F(b) - F(a-0)$.
                    \end{enumerate}
              \item $X \sim p_{i}$,则
                    $$P\{X \in I\} = \sum_{x_{i} \in I} P\{X = x_{i}\}$$
              \item $X \sim f(x)$,则
                    $$P\{X \in I\} = \int_{I} f(x) \, dx$$
              \item 反问题:已知概率反求参数.
          \end{enumerate}
\end{enumerate}

\section{求分布}
\DOne

根据题设条件,建立$F(x) = P\{X\leq x\}$并计算此概率.