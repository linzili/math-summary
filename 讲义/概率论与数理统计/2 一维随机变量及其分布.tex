\GAIchapter{一维随机变量及其分布}

\section{判分布}
\DOne + \DTwoTwo

\begin{enumerate}
      \item \textbf{分布函数判定}
            \begin{enumerate}
                  \item \textbf{充要条件:}
                        \[
                              F(x)\text{ 是分布函数 } \Leftrightarrow
                              \begin{cases}
                                    F(x) \text{ 单调不减且右连续}, \\
                                    F(-\infty)=0,\quad F(+\infty)=1.
                              \end{cases}
                        \]

                  \item \textbf{典型形式:}
                        \begin{enumerate}
                              \item 若 $F_i(x)$ 为分布函数,$\lambda_i>0,\ \sum\lambda_i=1$,则 $\sum\lambda_iF_i(x)$ 仍为分布函数;
                              \item 若 $F(x)$ 为分布函数,则其平移均值
                                    \[
                                          \frac{1}{T}\int_x^{x+T}F(t)\,dt
                                    \]
                                    仍为分布函数;
                              \item 几何平均 $\sqrt{F_1(x)F_2(x)}$ 为分布函数;
                              \item $[F(x)]^n$ 与 $1-[1-F(x)]^n$ 亦为分布函数。
                        \end{enumerate}
            \end{enumerate}

      \item \textbf{分布律判定}
            \[
                  \{p_i\}\text{ 为概率分布 } \Leftrightarrow p_i\ge0,\ \sum_i p_i=1.
            \]

      \item \textbf{概率密度判定}
            \begin{enumerate}
                  \item \textbf{充要条件:}
                        \[
                              f(x)\text{ 为概率密度 } \Leftrightarrow
                              f(x)\ge0,\quad
                              \int_{-\infty}^{+\infty}f(x)\,dx=1.
                        \]
                  \item \textbf{常见构造形式:}
                        \begin{enumerate}
                              \item 若 $f_i(x)$ 为概率密度,$\sum\lambda_i=1$,则 $\sum\lambda_i f_i(x)$ 仍为密度;
                              \item 平均形式 $\dfrac{1}{T}\!\int_x^{x+T}\!f(t)\,dt$;
                              \item 混合型:$\dfrac{2}{n}\sum F_i(x)f_i(x)$;
                              \item 累积分布组合:
                                    \[
                                          f_1F_2\cdots F_n + F_1f_2\cdots F_n + \cdots + F_1F_2\cdots f_n;
                                    \]
                              \item 若 $F,f$ 对应,则 $n[F(x)]^{n-1}f(x)$、$n[1-F(x)]^{n-1}f(x)$ 皆为密度。
                        \end{enumerate}
            \end{enumerate}

      \item \textbf{反问题:建立方程求参数}
            \[
                  \begin{cases}
                        F(-\infty)=0, \\
                        F(+\infty)=1, \\
                        \sum p_i = 1, \\
                        \int_{-\infty}^{+\infty} f(x)\,dx = 1.
                  \end{cases}
            \]
\end{enumerate}


\section{用分布}
\DOne

\subsection{离散型分布}
\begin{enumerate}
      \item \textbf{0-1分布(伯努利试验)}
            \[
                  X\sim B(1,p),\quad P(X=1)=p,\ P(X=0)=1-p.
            \]

      \item \textbf{二项分布}
            \[
                  X\sim B(n,p),\quad P(X=k)=C_n^k p^k(1-p)^{n-k},
            \]
            \[
                  E X = np,\quad D X = np(1-p).
            \]
            性质:
            \begin{itemize}
                  \item $Y=n-X\sim B(n,1-p)$;
                  \item 概率$P(X=k)$先增后减,最大值在$k=\lfloor (n+1)p\rfloor$附近。
            \end{itemize}

      \item \textbf{几何分布(首次成功试验次数)}
            \[
                  P(X=k)=p(1-p)^{k-1},\quad k=1,2,\dots;\quad E X=\tfrac{1}{p},\ D X=\tfrac{1-p}{p^2}.
            \]

      \item \textbf{超几何分布}
            \[
                  P(X=k)=\frac{C_M^kC_{N-M}^{n-k}}{C_N^n},\quad E X=\frac{nM}{N}.
            \]

      \item \textbf{泊松分布}
            \[
                  P(X=k)=\frac{\lambda^k e^{-\lambda}}{k!},\quad E X = D X = \lambda.
            \]
            用于描述稀有事件次数。
\end{enumerate}


\subsection{连续型分布}

\begin{enumerate}
      \item \textbf{均匀分布} $U(a,b)$
            \[
                  f(x)=
                  \begin{cases}
                        \dfrac{1}{b-a}, & a<x<b,     \\
                        0,              & \text{其他},
                  \end{cases}\quad
                  F(x)=
                  \begin{cases}
                        0,                & x<a,      \\
                        \dfrac{x-a}{b-a}, & a\le x<b, \\
                        1,                & x\ge b.
                  \end{cases}
            \]
            $E X=\tfrac{a+b}{2},\quad D X=\tfrac{(b-a)^2}{12}.$

      \item \textbf{指数分布(连续型首次冲击)}
            \[
                  f(x)=
                  \begin{cases}
                        \lambda e^{-\lambda x}, & x\ge0, \\
                        0,                      & x<0,
                  \end{cases}\quad
                  F(x)=1-e^{-\lambda x}.
            \]
            \[
                  E X=\tfrac{1}{\lambda},\quad D X=\tfrac{1}{\lambda^2}.
            \]
            无记忆性:$P(X>t+s\mid X>t)=P(X>s)$。

            \begin{note}{}{}
                  特殊关系:
                  \[
                        E\!\left(\tfrac12\right)\equiv\chi^2(2),\quad
                        2\lambda X\sim\chi^2(2).
                  \]
            \end{note}

      \item \textbf{自由度为1的$t$分布(标准柯西分布)}
            \[
                  f(x)=\frac{1}{\pi(1+x^2)},\quad X\sim t(1).
            \]

      \item \textbf{正态分布}
            \[
                  f(x)=\frac{1}{\sqrt{2\pi}\sigma}e^{-\frac{(x-\mu)^2}{2\sigma^2}},\quad
                  X\sim N(\mu,\sigma^2).
            \]
            \begin{note}{}{}
                  \begin{enumerate}
                        \item 标准正态:$N(0,1)$,
                              \[
                                    \varphi(x)=\frac{1}{\sqrt{2\pi}}e^{-x^2/2},\quad
                                    \Phi(x)=\int_{-\infty}^x\varphi(t)\,dt.
                              \]
                        \item 性质:
                              \[
                                    \Phi(-x)=1-\Phi(x),\quad
                                    P(|X|\le a)=2\Phi(a)-1.
                              \]
                        \item 标准化:
                              \[
                                    Z=\frac{X-\mu}{\sigma}\sim N(0,1),
                                    \quad F(x)=\Phi\!\left(\frac{x-\mu}{\sigma}\right).
                              \]
                        \item 含参数密度$f(x)=ke^{-(ax^2+bx+c)}$,
                              其规范化常数$k=\sqrt{\frac{a}{\pi}}e^{\frac{4ac-b^2}{4a}}$。
                  \end{enumerate}
            \end{note}
\end{enumerate}


\section{利用分布求概率}
\DOne

\begin{enumerate}
      \item 若 $X\sim F(x)$,则
            \[
                  P(X\le a)=F(a),\quad
                  P(a<X<b)=F(b-0)-F(a),\quad
                  P(X=a)=F(a)-F(a-0).
            \]
      \item 若 $X\sim p_i$,则
            \[
                  P(X\in I)=\sum_{x_i\in I}p_i.
            \]
      \item 若 $X\sim f(x)$,则
            \[
                  P(X\in I)=\int_I f(x)\,dx.
            \]
      \item \textbf{反问题:}已知概率表达式,建立方程求参数。
\end{enumerate}


\section{求分布}
\DOne

根据题意建立
\[
      F(x)=P\{X\le x\},
\]
利用定义及已知条件计算并验证其合法性。