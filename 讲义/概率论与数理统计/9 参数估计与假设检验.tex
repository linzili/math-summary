\GAIchapter{参数估计与假设检验}

本章主要考查:参数点估计、区间估计、假设检验(含拒绝域与两类错误)。
重点:掌握矩估计与极大似然估计的步骤,区分各检验条件与统计量类型。


\section{参数的点估计与评价}

\subsection*{1. 矩估计法}

\begin{enumerate}
      \item \textbf{单参数:}
            建立矩方程
            \[
                  \bar{X} = EX
                  \quad \text{或} \quad
                  \frac{1}{n}\sum_{i=1}^{n}X_i^2 = E(X^2),
            \]
            解得参数估计 $\hat{\theta}$。
      \item \textbf{双参数:}
            由两阶矩建立方程组
            \[
                  \begin{cases}
                        \bar{X} = EX, \\[3pt]
                        \frac{1}{n}\sum X_i^2 = E(X^2),
                  \end{cases}
            \]
            解得 $\hat{\theta}_1, \hat{\theta}_2$。
\end{enumerate}

\subsection*{2. 极大似然估计(MLE)}

\begin{enumerate}
      \item 写似然函数:
            \[
                  L(\theta) =
                  \begin{cases}
                        \prod p(x_i;\theta), & \text{离散型}; \\[3pt]
                        \prod f(x_i;\theta), & \text{连续型}.
                  \end{cases}
            \]
      \item 求导取极值:
            \[
                  \frac{\mathrm{d}L}{\mathrm{d}\theta}=0
                  \quad \text{或} \quad
                  \frac{\mathrm{d}(\ln L)}{\mathrm{d}\theta}=0,
            \]
            解得 $\hat{\theta}$。若 $L$ 单调,用定义法确定。
      \item \textbf{不变性:}
            若 $\hat{\theta}$ 为 $\theta$ 的 MLE,且 $u=u(\theta)$ 单调,则 $\hat{u}=u(\hat{\theta})$ 为 $u(\theta)$ 的 MLE。
\end{enumerate}

\subsection*{3. 估计量的评价标准}

\begin{enumerate}
      \item \textbf{无偏性:} $E\hat{\theta}=\theta$。
      \item \textbf{有效性:} 若 $\hat{\theta}_1,\hat{\theta}_2$ 均无偏,且 $D\hat{\theta}_1 < D\hat{\theta}_2$,则 $\hat{\theta}_1$ 更有效。
      \item \textbf{一致性:}
            \[
                  \hat{\theta}\xrightarrow{P}\theta
                  \Leftrightarrow
                  \lim_{n\to\infty}P(|\hat{\theta}-\theta|<\varepsilon)=1.
            \]
\end{enumerate}

\section{区间估计}

\subsection*{1. 概念}

若存在统计量 $\hat{\theta}_1,\hat{\theta}_2$ 满足
\[
      P\{\hat{\theta}_1<\theta<\hat{\theta}_2\}=1-\alpha,
\]
则称 $(\hat{\theta}_1,\hat{\theta}_2)$ 为参数 $\theta$ 的置信区间,置信度为 $1-\alpha$。
若 $P\{\theta<\hat{\theta}_1\}=P\{\theta>\hat{\theta}_2\}=\frac{\alpha}{2}$,称为\textbf{等尾置信区间}。

\subsection*{2. 正态总体下的常用区间}

设 $X_1,\cdots,X_n\sim N(\mu,\sigma^2)$,样本均值 $\bar{X}$、样本方差 $S^2$。

| 参数 | 已知条件 | 置信水平 $1-\alpha$ 区间 | 分布类型 |
|:----:|:----------|:---------------------------|:----------|
| $\mu$ | $\sigma^2$ 已知 | $\left(\bar{X}-\dfrac{\sigma}{\sqrt{n}}z_{\frac{\alpha}{2}},\ \bar{X}+\dfrac{\sigma}{\sqrt{n}}z_{\frac{\alpha}{2}}\right)$ | $N(0,1)$ |
| $\mu$ | $\sigma^2$ 未知 | $\left(\bar{X}-\dfrac{S}{\sqrt{n}}t_{\frac{\alpha}{2}}(n-1),\ \bar{X}+\dfrac{S}{\sqrt{n}}t_{\frac{\alpha}{2}}(n-1)\right)$ | $t(n-1)$ |
| $\sigma^2$ | $\mu$ 已知 | $\left(\dfrac{\sum (X_i-\mu)^2}{\chi^2_{\frac{\alpha}{2}}(n)},\ \dfrac{\sum (X_i-\mu)^2}{\chi^2_{1-\frac{\alpha}{2}}(n)}\right)$ | $\chi^2(n)$ |
| $\sigma^2$ | $\mu$ 未知 | $\left(\dfrac{(n-1)S^2}{\chi^2_{\frac{\alpha}{2}}(n-1)},\ \dfrac{(n-1)S^2}{\chi^2_{1-\frac{\alpha}{2}}(n-1)}\right)$ | $\chi^2(n-1)$ |

\section{假设检验}

\subsection*{1. 基本概念}

\begin{itemize}
      \item 原假设(零假设)$H_0$:默认成立;
      \item 备择假设 $H_1$:与 $H_0$ 对立;
      \item 显著性水平 $\alpha$:犯第一类错误(弃真)的最大容许概率;
      \item 依据“小概率原理”:若观测结果落入小概率区间,则拒绝 $H_0$。
\end{itemize}

\subsection*{2. 正态总体下的常用检验(单样本)}

设总体 $X\sim N(\mu,\sigma^2)$,样本均值 $\bar{X}$、样本标准差 $S$。

\begin{table}[H]
      \centering
      \caption{正态总体均值检验汇总表}
      \renewcommand{\arraystretch}{1.3} % 表格行距
      \begin{tabular}{|c|c|c|c|}
            \hline
            \textbf{检验类型} & \textbf{已知条件} & \textbf{检验统计量}                             & \textbf{拒绝域(双侧检验)}              \\ \hline
            $\mu$ 检验      & $\sigma^2$ 已知 & $Z=\dfrac{\bar{X}-\mu_0}{\sigma/\sqrt{n}}$ & $|Z|>z_{\frac{\alpha}{2}}$      \\ \hline
            $\mu$ 检验      & $\sigma^2$ 未知 & $T=\dfrac{\bar{X}-\mu_0}{S/\sqrt{n}}$      & $|T|>t_{\frac{\alpha}{2}}(n-1)$ \\ \hline
      \end{tabular}
\end{table}

\begin{table}[H]
      \centering
      \caption{单侧检验形式}
      \renewcommand{\arraystretch}{1.3}
      \begin{tabular}{|c|c|}
            \hline
            \textbf{备择假设 $H_1$} & \textbf{拒绝域}                       \\ \hline
            $\mu>\mu_0$         & $Z>z_\alpha$ 或 $T>t_\alpha(n-1)$   \\ \hline
            $\mu<\mu_0$         & $Z<-z_\alpha$ 或 $T<-t_\alpha(n-1)$ \\ \hline
      \end{tabular}
\end{table}

\section{两类错误}

\begin{itemize}
      \item \textbf{第一类错误(弃真)}:$H_0$ 为真却被拒绝,概率 $\alpha=P(\text{拒绝 }H_0|H_0\text{ 真})$;
      \item \textbf{第二类错误(取伪)}:$H_0$ 为假却被接受,概率 $\beta=P(\text{接受 }H_0|H_0\text{ 假})$。
\end{itemize}

常用关系:显著性水平 $\alpha$ 固定时,样本容量 $n$ 增大 $\Rightarrow$ 检验功效($1-\beta$)提高。

\section*{小结:参数估计与假设检验思路}

\[
      \boxed{
            \begin{aligned}
                   & \text{估计:}\quad \text{先点估计(矩、MLE)→ 再区间估计(置信区间);}   \\[3pt]
                   & \text{检验:}\quad \text{先设假设 → 选统计量 → 确定拒绝域 → 比较判断。}
            \end{aligned}
      }
\]