\chapter{参数估计与假设检验}
\section{求点估计、作评价}
\begin{enumerate}
      \item 矩估计
            \begin{enumerate}
                  \item 对于一个参数$\begin{cases}\text{a.用一阶矩建立方程:令}\bar{X}=EX;\\\\\text{b.若“a”不能用,用二阶矩建立方程:令}\frac1n\sum_{i=1}^nX_i^2=E(X^2)\:.\end{cases}$

                        一个方程解出一个参数即可作为矩估计.
                  \item 对于两个参数,用一阶矩与二阶矩建立两个方程,即$\bar{X}=EX$ 与$\frac1n\sum_{i=1}^nX_i^2=E(X^2)$,两个方程解出两
                        个参数即可作为矩估计.
            \end{enumerate}
      \item 最大似然估计

            对未知参数$\theta$进行估计时,在该参数可能的取值范围$I$内选取,使“样本获此观测值$x_1,x_2,\cdotp\cdotp\cdotp,x_n$”的概率最大的参数值$\hat{\theta}$作为$\theta$的估计,这样选定的$\hat{\theta}$最有利于$x_1,x_2,\cdotp\cdotp\cdotp,x_n$的出现,即“参数$\theta$为多少时,观测值出现的概率最大”.
            \begin{enumerate}
                  \item 写似然函数$L(x_{1},x_{2},\cdots,x_{n};\theta)=\left\{\begin{array}{l}{\prod_{i=1}^{n}p(x_{i};\theta)\ (\text{离散型;})}\\{\prod_{i=1}^{n}f(x_{i};\theta)\ (\text{连续型.})}\end{array}\right.$
                  \item 求参数$\left\{\begin{array}{l}\text{若似然函数有驻点,则令}\frac{dL}{d\theta}=0\text{或}\frac{d(\ln L)}{d\theta}=0\text{,解出}\hat{\theta};\\\text{若似然函数无驻点(单调),则用定义求}\hat{\theta};\\\text{若似然函数为常数,则用定义求}\hat{\theta},\text{此时}\hat{\theta}\text{不唯一}.\end{array}\right.$
                  \item 最大似然估计量的不变性原则.

                        设$\hat{\theta}$是总体分布中未知参数$\theta$的最大似然估计,函数$u=u(\theta)$具有单值的反函数$\theta=\theta(u)$,则$\hat{u}=u(\hat{\theta})$是$u(\theta)$的最大似然估计.
                  \item 双总体的最大似然估计.
            \end{enumerate}
      \item 估计量的评价
            \begin{enumerate}
                  \item 无偏性.

                        对于估计量 $\hat{\theta}$,若 $E\hat{\theta}=\theta$,则称 $\hat{\theta}$ 为 $\theta$ 的无偏估计量.
                  \item 有效性.

                        若 $E\hat{\theta}_{1}=\theta$,$E\hat{\theta}_{2}=\theta$,即 $\hat{\theta}_{1}$,$\hat{\theta}_{2}$ 均是 $\theta$ 的无偏估计量,当 $D\hat{\theta}_{1}<D\hat{\theta}_{2}$ 时,称 $\hat{\theta}_{1}$ 比 $\hat{\theta}_{2}$ 有效.
                  \item 一致性(相合性).(只针对大样本$n\rightarrow\infty$)

                        若 $\hat{\theta}$ 为 $\theta$ 的估计量,则对任意 $\varepsilon>0$,有
                        $$\lim_{n\rightarrow\infty}P\{\left|\hat{\theta}-\theta\right|\geqslant\varepsilon\}=0,$$
                        或
                        $$
                              \lim_{n\rightarrow\infty}P\{\left|\hat{\theta}-\theta\right|<\varepsilon\}=1,$$
                        即当 $\hat{\theta}\stackrel{P}{\longrightarrow}\theta$ 时,称 $\hat{\theta}$ 为 $\theta$ 的一致(或相合)估计.
            \end{enumerate}
\end{enumerate}

\section{作区间估计、假设检验、求两类错误}
\begin{enumerate}
      \item 区间估计
            \begin{enumerate}
                  \item 概念.

                        设$\theta$是总体$X$的分布函数的一个未知参数,对于给定$\alpha(0<\alpha<1)$ ,如果由样本$X_1,X_2,\cdotp\cdotp\cdotp,X_n$确定的
                        两个统计量$\hat{\theta } _1= \hat{\theta } _1( X_1, X_2, \cdotp \cdotp \cdotp , X_n)$ , $\hat{\theta } _2= \hat{\theta } _2( X_1, X_2, \cdotp \cdotp \cdotp , X_n)$,使
                        $$P\{\hat{\theta}_1(X_1,X_2,\cdotp\cdotp\cdotp,X_n)<\theta<\hat{\theta}_2(X_1,X_2,\cdotp\cdotp\cdotp,X_n)\}=1-\alpha\:,$$
                        则称随机区间($\hat{\theta}_{\mathrm{l}},\hat{\theta}_{\mathrm{2}})$是$\theta$的置信度为 $1-\alpha$的置信区间,$\hat{\theta}_{\mathrm{l}}$ 和$\hat{\theta}_{\mathrm{2}}$分别称为$\theta$的置信度为 $1-\alpha$的双侧置信区间的置信下限和置信上限,$1-\alpha$称为置信度或置信水平,$\alpha$称为显著性水平.如果$P\{\theta<\hat{\theta}_1\}=P\{\theta>\hat{\theta}_2\}=\frac\alpha2,则称$
                        这种置信区间为等尾置信区间.
                  \item 单个正态总体均值和方差的置信区间.

                        设 $X \sim N(\mu, \sigma^2)$,从总体 $X$ 中抽取样本 $X_1, X_2, \cdots, X_n$,样本均值为 $\bar{X}$,样本方差为 $S^2$.
                        \begin{enumerate}
                              \item  $\sigma^2$ 已知,$\mu$ 的置信水平是 $1-\alpha$ 的置信区间为
                                    $$\left( \bar{X} - \frac{\sigma}{\sqrt{n}} z_{\frac{\alpha}{2}}, \bar{X} + \frac{\sigma}{\sqrt{n}} z_{\frac{\alpha}{2}} \right).$$
                              \item $\sigma^2$ 未知,$\mu$ 的置信水平是 $1-\alpha$ 的置信区间为
                                    $$\left( \bar{X} - \frac{S}{\sqrt{n}} t_{\frac{\alpha}{2}}(n-1), \bar{X} + \frac{S}{\sqrt{n}} t_{\frac{\alpha}{2}}(n-1) \right).$$
                              \item $\mu$ 已知,$\sigma^2$ 的置信水平是 $1-\alpha$ 的置信区间为
                                    $$\left( \frac{\sum_{i=1}^{n} (X_i - \mu)^2}{\chi^2_{\frac{\alpha}{2}}(n)}, \frac{\sum_{i=1}^{n} (X_i - \mu)^2}{\chi^2_{1-\frac{\alpha}{2}}(n)} \right).$$
                              \item $\mu$ 未知,$\sigma^2$ 的置信水平是 $1-\alpha$ 的置信区间为
                                    $$\left( \frac{(n-1)S^2}{\chi^2_{\frac{\alpha}{2}}(n-1)}, \frac{(n-1)S^2}{\chi^2_{1-\frac{\alpha}{2}}(n-1)} \right).$$
                        \end{enumerate}

            \end{enumerate}
      \item 假设检验
            \begin{enumerate}
                  \item 概念.

                        关于总体(分布中的未知参数,分布的类型、特征、相关性、独立性$\cdots\cdots)$ 的每一种论断(“看法”) 称为统计假设,然后根据样本观察数据或试验结果所提供的信息去推断(检验)这个“看法”(即假设) 是否成立,这类统计推断问题称为假设检验.
                  \item 原假设与备择假设.

                        常常把没有充分理由不能轻易否定的假设取为原假设(基本假设或零假设),记为$H_{0}$,将其否定的
                        陈述(假设)称为对立假设或备择假设,记为$H_{\mathrm{l}}.$
                  \item 小概率原理与显著性水平.
                        \begin{enumerate}
                              \item 小概率原理

                                    对假设进行检验的基本思想是采用某种带有概率性质的反证法.这种方法的依据是小概率原理—— 概率很接近于0 的事件在一次试验或观察中认为备择假设不会发生.若小概率事件发生了,则拒绝原假设.
                              \item 显著性水平$\alpha$

                                    小概率事件中“小概率”的值没有统一规定,通常是根据实际问题的要求,规定一个界限$\alpha(0<\alpha<1)$,当一个事件的概率不大于 $\alpha$时,即认为它是小概率事件.在假设检验问题中,$\alpha$称为显著性水平,通常取$\alpha=0.1,0.05,0.01$等.
                        \end{enumerate}
                  \item 正态总体下的六大检验及拒绝域.
                        \begin{enumerate}
                              \item $\sigma^2$已知,$\mu$未知.$H_0:\mu=\mu_0,H_1:\mu\neq\mu_0$,则拒绝域为$\left(-\infty,\mu_{0}-\frac{\sigma}{\sqrt{n}}z_{\frac{\alpha}{2}}\right]\cup\left[\mu_{0}+\frac{\sigma}{\sqrt{n}}z_{\frac{\alpha}{2}},+\infty\right)$.
                              \item $\sigma^2$未知,$\mu$未知.$H_0:\mu=\mu_0,H_1:\mu\neq\mu_0$,则拒绝域为
                                    $$\left(-\infty,\mu_0-\frac{S}{\sqrt{n}}t_{\frac{\alpha}{2}}(n-1)\right]\cup\left[\mu_0+\frac{S}{\sqrt{n}}t_{\frac{\alpha}{2}}(n-1),+\infty\right).$$
                              \item $\sigma^{2}$已知,$\mu$未知.$H_{0}$:$\mu\leqslant\mu_{0}$(或写$\mu=\mu_{0}$),$H_{1}$:$\mu>\mu_{0}$,则拒绝域为$\left[\mu_{0}+\frac{\sigma}{\sqrt{n}}z_{\alpha},+\infty\right)$.
                              \item $\sigma^{2}$已知,$\mu$未知.$H_{0}$:$\mu\geqslant\mu_{0}$(或写$\mu=\mu_{0}$),$H_{1}$:$\mu<\mu_{0}$,则拒绝域为$\left[-\infty,\mu_{0}-\frac{\sigma}{\sqrt{n}}z_{\alpha}\right)$.
                              \item $\sigma^{2}$未知,$\mu$未知.$H_{0}$:$\mu\leqslant\mu_{0}$(或写$\mu=\mu_{0}$),$H_{1}$:$\mu>\mu_{0}$, 则拒绝域为$\left[\mu_{0}+\frac{S}{\sqrt{n}}t_{\alpha}(n-1),+\infty\right)$.
                              \item $\sigma^{2}$未知,$\mu$未知.$H_{0}$:$\mu\geqslant\mu_{0}$(或写$\mu=\mu_{0}$),$H_{1}$:$\mu<\mu_{0}$,则拒绝域为$\left(-\infty,\mu_{0}-\frac{S}{\sqrt{n}}t_{\alpha}(n-1)\right]$.
                        \end{enumerate}
            \end{enumerate}
      \item 两类错误

            第一类错误(“弃真”):若$H_0$为真,按检验法则,否定$H_{0}$,此时犯了“弃真”的错误,这种错误称为第一类错误,犯第一类错误的概率为 $\alpha=P\{$拒绝$H_0|H_0$为真 $\}.$

            第二类错误(“取伪”):若$H_0$不真,按检验法则,接受$H_0$,此时犯了“取伪”的错误,这种错误称为第二类错误,犯第二类错误的概率为 $\beta=P\{$ 接受$H_0|H_0$为假 $\}.$
\end{enumerate}


